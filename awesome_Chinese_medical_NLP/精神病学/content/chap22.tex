\chapter{精神疾病和法律相关问题}

\section{司法精神病学概述}

\subsection{司法精神病学概念}

司法精神病学(forensic
psychiatry)是建立在临床精神病学和法学两大基础上的交叉学科,是应用临床精神病学知识研究和解决精神疾病患者在法律方面的有关问题的学科。

而广义的司法精神病学包括更广泛的内容,除涉及精神疾病患者相关的法律问题,如各种法律能力、精神损伤程度、劳动能力、伤残等级等,还涉及精神疾病患者危险行为的预测和预防;精神疾病患者各种权益的法律保障;有危害行为的精神疾病患者的治疗监护;精神病学临床实践中相关的伦理和法律问题;精神卫生立法等问题。

总之,随着人们对精神疾病患者有关的法律、社会学、犯罪学、心理学、伦理学和行为学等问题的不断深入研究,司法精神病学的内涵逐渐扩大。司法精神病学已成为精神医学中一个重要的分支学科。

\subsection{精神疾病与法律关系}

精神疾病患者是我们社会生活中的一个群体,在社会生活中与正常人一样与法律有着十分密切的关系,但他们因各种精神疾病使其大脑功能发生程度不同的障碍,故又是一个特殊的群体。一旦他们的行为涉及某种法律关系时,就需要对他们的各种行为与法律的关系作出评定。

主要表现在刑事方面,一个精神疾病患者,当他实施了我国刑法所禁止的危害行为后,就需要对其在实施危害行为时的责任能力作出评定,以明确其相应的刑事责任能力,以及受审能力、服刑能力、作证能力;在民事方面,当精神疾病患者实施某种民事行为时,要对其实施该行为时的民事行为能力作出评定,以明确其民事行为的有效性,或目前的民事行为能力状态。当精神疾病患者参与刑事或民事诉讼时,要对其诉讼能力作出评定;而在刑事或民事案件中,当其人身、财产等合法权益遭受侵害时,要对其自我防卫、保护能力作出评定,或对精神损伤与致伤因素间的关系作出评定。

\subsubsection{精神病人的刑事法律能力}

1.责任能力

(1)责任能力的概念:刑事责任能力,即责任能力,是指行为人了解自己行为的性质、意义和后果,并自觉地控制自己行为和对自己行为负责的能力。简单地说,就是能够辨认和控制自己行为的能力。它是我国犯罪构成理论中,犯罪主体成立的必要条件之一,即达到一定的责任年龄,且生理和智力发育正常,就具有相应的辨认和控制自己行为的能力,亦就具有刑事责任能力。

(2)刑事责任能力的种类:刑事责任能力的种类,不同国家依据其刑法有二分法(即有责任能力与无责任能力)和三分法(即有责任能力、限制责任能力和无责任能力)。依据我国刑法规定为三分法,即有责任能力、限制责任能力和无责任能力。

(3)刑事责任能力评定:我国《刑法》第18条是刑事责任能力评定的法律依据。第18条包含了两个重要的内容,即医学要件和法学要件。

医学要件是指行为人是精神病人,即患有某精神疾病。由于精神疾病使其精神功能发生障碍,有可能导致其实施危害行为。因此,医学要件是评定行为人在实施危害行为时责任能力状态的前提和客观依据。

法学要件亦称心理学标准,是指行为人在实施危害行为时,是否由于精神疾病使其丧失或削弱了辨认能力和控制能力。因此,在医学要件确定后,法学要件是确定其责任能力状态的分析依据。

所以精神病人实施《刑法》所禁止的危害行为时的责任能力评定原则是:以医学要件为基础和前提,以法学要件为依据评定责任能力状态。

2.其他刑事法律能力 在精神病人涉及的刑事法律能力中,除主要具有刑事责任能力外,还包括受审能力、服刑能力和性自卫能力等。

(1)受审能力:受审能力是指刑事案件的犯罪嫌疑人、被告人在刑事诉讼中,对法律赋予自己的权利、义务和刑事诉讼的意义的认识理解以及接受刑事审判的能力。如有权拒绝回答与案件无关的问题,了解对他起诉的目的和性质等。

受审能力评定时,应分析所患精神疾病是否影响了其对起诉的目的和性质的理解;能否理解自己的情况与目前诉讼的关系:有无能力与律师合作、商量,或协助辩护人为其辩护;对诉讼过程中所提问题能否做出相应的回答;能否理解可能的审判结果和惩罚等。若其能对上述问题得出肯定的结论,则其具有接受刑事审判和判决能力,有受审能力;否则为无受审能力。

(2)服刑能力:服刑能力是指已判决或服刑人员能够理解和承受法庭对其刑罚的能力。如对判刑的意义和服刑的理解,对自己的身份和未来前途的认识,对自己当前应遵守的行为规范的认识等。

(3)性自卫能力:性自卫能力是指被害人对两性行为的社会意义、性质及其后果的理解能力。

女性精神病人因其疾病的影响使其辨别是非的能力受到损害,意志行为能力削弱或缺乏,或本能欲望的亢进,而遭到他人性侵害,便可能做出相应的反抗行为,甚至主动追逐异性,其实质是丧失了性自卫能力。

\subsubsection{精神病人的民事法律能力}

1.民事行为能力

(1)民事行为能力的概念:民事行为能力(civil
capability)简称行为能力,是指公民能够通过自己的行为,取得民事权利和承担民事义务,从而设立、变更或终止法律关系的资格,亦即一个人的行为能否发生民事法律效力的资格。

(2)民事行为能力的种类:法律赋予公民民事行为能力是以意思表示能力为基础的。即公民的认识能力和判断能力。认识能力是指对人和事物的分析能力,即能够辨认自己行为的能力。公民的民事行为能力分为三种,即完全民事行为能力、限制民事行为能力和无民事行为能力。

(3)精神病人的民事行为能力:精神病人在疾病的进程中,由于他们精神功能存在障碍,对其意思表示具有不同程度的影响,法律为了维护他们的利益和社会的正常经济秩序,作了专门的规定。

我国《民法通则》第十三条规定:“不能辨认自己行为的精神病人是无民事行为能力人,由他的法定代理人代理。”第十三条第二款规定:“不能完全辨认自己行为的精神病人是限制民事行为能力人,可以进行与他的精神健康状况相适应的民事活动;其他民事活动由他的法定代理人代理,或者征得他的法定代理人的同意。”

(4)民事行为能力评定:精神病人的民事行为能力评定,亦需遵循医学和法学两个条件。

首先应满足医学条件,即被鉴定人患有精神疾病,并要确定其精神疾病性质、疾病的不同阶段及严重程度、可能的预后等。而法学要件则是被鉴定人的意思表示,即是否具有独立地判断是非和理智地处理自己的事务的能力。其评定分为:①宣告行为能力评定:宣告民事行为能力是指精神疾病患者尚未涉及某一具体民事行为时,经其利害关系人申请,经人民法院受理、委托,对其行为能力进行评定,并经人民法院判决认定宣告。②民事行为时的行为能力评定:精神病人民事行为时的行为能力是指精神病人针对某一民事行为时的行为能力。

2.精神损伤 近十年来,在司法精神疾病鉴定实践中,有关精神疾病患者的人身损害赔偿案逐年增加,已成为司法精神病鉴定中一项重要的内容和研究课题。

(1)精神损伤的概念:精神损伤(mental
damage)是指个体遭受外来物理、化学、生物或心理等因素作用后,大脑功能活动发生障碍,出现认知、情感、意志和行为等方面的精神功能紊乱或缺乏。即精神损伤是遭受外界致害因素作用后出现的精神功能的障碍,其致害因素不仅指外界因素造成了脑器质性伤害,还包括心理刺激因素的作用,导致大脑功能紊乱。

(2)精神损伤的评定:在涉及人身损害赔偿案中,受害人的精神损伤与某一生活事件的关系是精神损伤评定的核心问题,它包括了其精神损伤的性质、严重程度及其预后,以及该精神损伤与生活事件的关系。

(3)精神损伤与生活事件关系评定:现阶段,对于精神损伤与生活事件的关系以及精神损伤程度的评定尚缺乏统一的标准和相应的规范,因此在司法精神疾病鉴定中关于精神损伤与生活事件的关系有着许多不同的描述,有以因果关系描述为直接因果、间接因果和无因果关系;有以相关关系描述为直接相关、间接相关和无关。而不同的描述可能导致不同的司法审判结果,即产生不同的民事赔偿责任。而在鉴定实践中,并不是所有的精神障碍的发生与生活事件的关系都能用因果关系来描述。

目前,较易达成共识的有:①脑器质性精神障碍与其致害因素评定为直接因果关系;②反应性精神病与其生活事件评定为直接因果关系;③内源性精神病通常不用因果关系描述而将生活事件描述为诱发因素;④癔症与其生活事件一般亦描述了诱发因素。

\subsection{司法精神病鉴定}

司法精神病鉴定是指鉴定人受司法机关的委托,运用精神病学专业知识,对被鉴定人的精神状态及相关的法律能力等作出评定的过程。

我国司法精神病鉴定工作的组织和实施是依据相关的法律、法规而进行的。具体有:依1989年最高人民法院、最高人民检察院、公安部、司法部和卫生部颁布实施的《精神疾病司法鉴定暂行规定》;依1996年《刑事诉讼法》第120条规定:对人身伤害的医学鉴定有争议需要重新鉴定或者对精神病的医学鉴定,由省级人民政府指定的医院进行。

2005年2月28日全国人大常委会颁布了《关于司法鉴定管理问题的决定》,确立了建立统一的司法鉴定管理体制的基本目标和基本框架。司法部成立了司法鉴定管理局负责全国司法鉴定的领导和管理工作。2005年9月30日颁布实施了《司法鉴定机构登记管理办法》和《司法鉴定人登记管理办法》。2007年10月1日起实施了《司法鉴定程序通则》。

1.司法精神病鉴定的任务 司法精神病鉴定时,首先是确定被鉴定人的精神状态是否正常,是否患精神疾病,以及患何种精神疾病;其次是根据被鉴定人精神疾病对其相关法律能力的影响程度,确定其法律能力。具体包括:①刑事案件相关的司法精神病鉴定任务:确定被鉴定人是否有精神病,患何种精神疾病,实施危害行为时的精神状态,精神疾病与所实施的危害行为之间的关系,以确定其实施危害行为时的责任能力、受审能力、服刑能力、劳动教养能力和受处罚能力。②民事案件相关的司法精神疾病鉴定任务:确定被鉴定人是否患有精神疾病,患何种精神疾病,其精神疾病对其在进行或可能进行的民事活动时意思表达能力的影响判定其民事行为能力、诉讼能力;确定被鉴定人是否患有精神疾病,患何种精神疾病、精神疾病的性质及严重程度,以及其精神疾病与生活事件的关系,为民事赔偿提供依据。③其他鉴定任务:确定各类案件的有关证人的精神状态及作证能力;确定各类案件的受害人的自我防卫或自我保护能力。

2.司法精神病鉴定的方式 鉴定的方式主要有门诊鉴定、住院鉴定、院外鉴定和缺席鉴定。也可分为直接鉴定和间接鉴定。直接鉴定是指鉴定人与被鉴定人直接接触,进行详细必要的精神检查,并结合送鉴材料和必要的调查材料确定被鉴定人的精神状态及相关的法律能力。间接鉴定即鉴定人无法与被鉴定人直接接触。

3.司法精神病鉴定程序 司法精神病鉴定的活动,作为诉讼程序的一个组成部分,应按一定的程序有序地进行。它包括:①鉴定委托:由公安机关、检察机关、人民法院、司法机关和其他办案机关以及其他单位向鉴定机构出具鉴定委托书,并明确鉴定目的和要求。②提交资料:委托鉴定机关向鉴定机构提供被鉴定人的有关资料,包括案情资料和相关医学资料,如全部案情的卷宗资料、有关被鉴定人既往疾病救治、病历、检查报告等。③阅读资料:在接触被鉴定人前,每一鉴定人必须认真细致地阅读全部送鉴资料,充分了解被鉴定人的有关情况,为与被鉴定人接触,进行精神检查提供充分的保障。④鉴定检查讨论:由全体鉴定人员共同对被鉴定人进行精神检查、神经系统检查以及相关的理化检查。鉴定人员就被鉴定人的精神状态及相关的法律能力进行充分地讨论,得出鉴定结论。若需进一步调查,待调查后讨论得出鉴定结论。⑤出具鉴定报告:经全体鉴定人员讨论得出鉴定结论后,形成鉴定报告,经鉴定机构盖章和鉴定人签名后,提交给送鉴机构。

4.精神疾病司法鉴定书 鉴定书是鉴定人用以记载其鉴定结论的文书,是鉴定人提供给司法机关或送鉴机构的专家证言。

精神疾病司法鉴定书的内容一般应包括:①一般资料:鉴定书编号、被鉴定人姓名、性别、年龄、婚姻、民族、文化程度、职业、家庭住址等;②委托鉴定机关名称;③鉴定日期;④鉴定场所;⑤鉴定目的和要求;⑥案由及案情摘要;⑦调查和有关证据材料;⑧精神检查及其他检查所见;⑨分析意见⑩鉴定结论、鉴定人签名、鉴定单位盖章。

\section{各种精神疾病的司法鉴定}

\textbf{精神分裂症}

精神分裂症是精神疾病司法鉴定工作中最为常见的精神疾病,占所有鉴定案件的1/3~1/2。精神分裂症患者的思维、情感和意志活动的严重障碍,特别是其思维障碍较为突出,因而在日常生活中常涉及各种法律问题。

\subsection{刑事法律能力}

在精神病理的影响下与周围环境产生各种冲突,出现各种危害行为。因而涉及某些法律关系。如责任能力、受审能力和服刑能力等,其中以实施危害行为时的责任能力问题最多见。

1.精神分裂症患者与危害行为 以凶杀行为最多见,占精神分裂症危害行为鉴定案例的1/3~1/2,凶杀和伤害行为超过1/2。以精神分裂症偏执型最多见,常在妄想、幻觉的直接支配下所为,较多见的是关系妄想、被害妄想、嫉妒妄想和命令性幻听的影响,对周围的人发生突然的攻击行为。如一偏执型精神分裂患者存在被害妄想和关系妄想,一日在门卫值勤时,忽听到“干掉他们,干掉他们”,即持刀连续伤害9人,造成2人死亡、4人重伤、3人轻伤。强奸、猥亵等性侵犯行为也是一种较为常见的危害行为,以青春型、慢性或残留性精神分裂症多见,亦也可见于偏执型精神分裂症。青春型患者除思维紊乱、内容荒诞、行为幼稚外常有较丰富的性色彩,易导致流氓猥亵行为甚至强奸行为。盗窃、抢劫、贪污等侵犯财产行为也是精神分裂症患者常见的一类危害行为,特别是盗窃行为较为常见。多见于慢性精神分裂症患者,也可见于偏执型或其他型精神分裂症、慢性精神分裂症的盗窃行为,常仅为满足饥饱等基本需要,扰乱社会治安等其他危害行为常见于精神分裂症偏执型、青春型、残留型或慢性精神分裂症。

2.危害行为与责任能力 精神分裂症患者实施危害行为时的责任能力评定的总的法律依据是刑法第18条,即根据其实施危害行为时疾病对其辨认和控制能力的影响,评定其作案时的责任能力状态。在司法精神疾病鉴定实践中,对精神分裂症患者危害行为时的责任能力评定过程,不同学者之间有时会产生较大分歧。分歧的主要原因是掌握医学标准与法学标准的着重点不同。其次是精神分裂症为一具有思维、情感和意志行为严重障碍且不协调;并具有人格甚至基本特征改变,且病程、转归非常复杂多样化的精神病。因此,精神分裂症患者实施危害行为时责任能力评定要在明确精神分裂诊断,并判明其实施危害行为时疾病所处的疾病阶段以及疾病的严重程度,综合地分析对其辨认能力和控制能力的影响,作出责任能力评定。

\subsection{民事法律能力}

精神分裂症患者因涉及其民事法律能力问题的案例近十年来呈明显的增加趋势。常见的案例涉及患者的婚姻能力,如离婚案件中,患者是否有能力参与离婚诉讼;财产处置及继承能力,如患者是否有能力处置自己的房产或继承其他人的财产等;遗嘱能力,如患者生前所立遗嘱或现在所立遗嘱是否有效;劳动合同能力,如患者自己提出辞职申请,且被单位采纳辞退,写辞职申请时的行为能力如何等。这些都归属于患者的民事行为能力范畴。

1.民事行为能力评定原则 精神分裂症患者,由于受疾病影响,其正确判断事物的能力可能受到不同程度的影响,使其在民事行为中正确地表达自己的意思,并理智地处理自己事务的能力受损,即影响到其正确表达自己的意思。因此对精神分裂症患者行为能力评定的总体原则是:结合患者精神分裂症疾病的不同疾病阶段及严重程度,看其是否具有独立地判断是非和理智地处理自己事务的能力,分别评为有行为能力、限制行为能力和无行为能力。

2.宣告民事行为能力 这是指在精神分裂症患者尚未涉及某一具体民事行为时,经其利害关系人申请,经法院受理、委托,对其行为能力进行评定,并经法院判决认定宣告。对精神分裂症患者该类行为能力的评定原则是:根据该患者现时精神分裂症所处的阶段、疾病的严重程度、疾病对其一般意志行为可能产生的影响的一种推定式的行为能力评定。在评定时对该被鉴定人所患精神分裂症在今后相当一段时期疾病的可能发展状态作出充分的估计,注意保护精神分裂症病人的合法民事权益。一般说来:①处于疾病发展阶段或严重阶段评定为无行为能力或限制行为能力;②疾病处于缓解不全期阶段(或不完全缓解阶段)评定为限制行为能力;③疾病处于完全缓解阶段为完全行为能力。

3.民事行为时的行为能力 在精神分裂症患者民事行为能力评定中,大部分属于此类,它包括:精神分裂症患者已经实施完成的某一民事行为时的行为能力,如生前或现已立的医嘱或已完成的财产公证、已签约的合同或已提交的辞职报告等;即将进行的某一民事行为能力,如离婚诉讼、出庭作证、财产分割或处置等。此类行为能力评定的特点是争对某一明确的具体的民事行为时的行为能力评定,因此评定时重点是考察患者对这一具体的民事行为是否具有真实的意思表达,即对该事物的判断、理解、处置能力。

\subsection{其他相关法律问题}

1.性保护能力 女性精神分裂症患者,在社会生活中时有受到不法分子的性侵害行为。对精神分裂症患者的性保护能力的鉴定,在性保护能力鉴定中占第二位,仅次于精神发育迟滞患者。

女性精神分裂症患者受性侵犯性保护能力的评定,要结合患者精神分裂症病情的严重程度,和对该性行为的实质性辨认能力结合评定。一般说:①精神分裂症处于疾病的发展阶段或严重阶段,评定为无性保护能力;②精神分裂症处于不完全缓解期或缓解不完全阶段,要结合性行为事件的过程及患者对该性行为的实质性辨认能力确定其性保护能力,可评定为无性保护能力、性保护能力削弱或有性保护能力;③精神分裂症处于完全缓解期,对性行为有辨认能力时评定为有性保护能力。

2.精神损伤 精神分裂症患者人身损害赔偿案,近年来在司法精神病鉴定实践中逐年增加。在现阶段对精神分裂症与生活事件,即心理刺激因素对疾病产生的作用着手分析,故以诱发因素来描述生活事件与精神分裂症的关系较为合适。

具体评定原则为:①明确查清生活事件即心理刺激前被鉴定人是否完全正常。因多数精神分裂症患者是缓慢、隐袭起病,开始可能表现为个性改变、学习、工作能力下降,甚至思维上有明确的精神病性症状,不易被当事人觉察。若生活事件前确实完全正常而且该生活事件与该患者精神分裂症的发病有密切的时间联系,可评定为该生活事件是其精神分裂症发病的诱发因素。若生活事件发生时,被鉴定人已处于精神分裂症的病程中,要确定该生活事件是否加重了精神分裂症疾病,除要查明该生活事件与精神分裂症病情加重有密切的时间联系,还必须确定其加重的疾病症状内容与生活事件有密切的联系,即有时间的关联性和内容的关联性,方可评定为该生活事件加速了被鉴定人原有精神分裂症的发展;否则评定为无关。②注意事项:评定中要注意区分生活事件的心理刺激因素的强弱:有时是在受到明显而强烈的心理刺激后出现精神分裂症,有些刺激因素并不强烈,仅为一般性的,属人们经常遇到的心理刺激因素;有一些看似心理刺激因素的生活事件其实是患者病态行为的结果,是患者对于环境适应不良的结果。心理刺激与起病时间的距离:有些患者是在明确的心理刺激因素作用下起病,其起病与该生活事件有明确的时间关联性;有一些虽有明确的心理刺激因素,但距离患者起病时间较远,其生活事件与起病缺乏明确的时间关联。一因还是多因:在鉴定中要注意对心理刺激因素进行具体分析,有些是某单心理刺激因素与精神分裂症的起病的关系;有些是同时几个互不相关的心理刺激因素与精神分裂症起病的关系;还有一些是同时几个互为因果关系与精神分裂症起病的关系。

\textbf{心境障碍}

心境障碍的患病率近十多年来呈增加的倾向,特别是抑郁发作的增加更为明显。心境障碍也成为精神疾病司法鉴定工作中较常见的一种精神疾病,占整个鉴定案件的5%~10%,仅次于精神分裂症和精神发育迟滞,位于第三位。

\subsection{刑事法律能力}

心境障碍虽以情绪的高涨或低落为其特征,但受病态情绪的影响,也同样产生相应的认知障碍,而与周围环境产生各种冲突,出现各种危害行为。因而涉及某些法律关系,如责任能力、受审能力和服刑能力等,其中以实施危害行为时的责任能力问题最多见。

1.心境障碍与危害行为 心境障碍在疾病过程中出现的危害行为依据不同的发作,即躁狂发作或抑郁发作,而有所不同。躁狂发作时危害行为较抑郁发作少见。躁狂发作的危害行为类型主要有调戏、猥亵行为、扰乱社会、治安行为和轻伤害行为。而因躁狂发作出现严重的杀人、强奸、抢劫等行为较少见。

有些躁狂症患者表现为激惹性明显增高,易于激惹,而导致与周围人发生冲突或滋生事端,或发生扰乱社会治安的行为。有些患者举止轻佻,追逐异性,性欲亢进,行为放荡,而出现流氓猥亵行为,或嫖娼行为。有些患者表现在经济上慷慨大方,随意施舍,甚至挥霍无度。有些严重的急性躁狂和谵妄性躁狂患者,可有一定程度的意识障碍,甚至可出现一过性的错觉、幻觉和妄想,而出现冲动、伤害行为。

抑郁发作时出现的危害行为明显较躁狂发作时多见,且危害行为的危害性也较大。抑郁发作时的危害行为以凶杀行为最为多见,包括“扩大性自杀”、“间接自杀”和“激越性杀人”等,还可出现偷窃行为,纵火、抢劫行为。

“扩大性自杀”是抑郁发作时杀人的经典范例,即患者在严重的情绪低落的状态下,感觉困难重重,一筹莫展,陷入绝境,而产生强烈的自杀企图,并决意自杀摆脱痛苦,但想到自己的亲人也处在重重困难之中,为免除亲人的痛楚和不幸的遭遇,常将自己的配偶或儿女杀死后自杀,也称为“怜悯性杀亲”或“家族性自杀”等。间接自杀常是在抑郁发作时,情绪极度低落时,产生自杀观念。而以往数次自杀不成功,欲通过杀人的行为使其被判死刑达到自杀的目的,也称为“曲线自杀”。有些抑郁发作患者在严重的情绪低落下,对外界的刺激产生严重的负性认知,出现关系、被害或嫉妒妄想或偏执观念,并在这些精神病性症状的影响下可出现杀人行为。另一类较常见的抑郁发作时杀人,是患者一方面情绪极度低落,一方面又极度地情绪恶劣,焦虑不安,情绪易激惹,呈激越状态,因周围环境中一点小的刺激而出现突然的冲动杀人行为。

有些抑郁发作患者在发作时出现偷窃行为。国外报道主要是一些女性患者发生于超市的偷窃行为。近年随着超市在我国的普遍出现,该类案例也有所见。主要是因为抑郁发作时患者在情绪低落时,注意力涣散,在超市购买时的一种漫不经心的行为,随手将物品放入自己的衣袋中。

2.危害行为与责任能力 心境障碍患者实施危害行为的责任能力评定的法律依据仍是刑法第18条,即根据其实施危害行为时的疾病对其辨认和控制能力的影响评定其作案时的责任能力状态。

对轻性躁狂症和轻性抑郁症患者在疾病期间实施危害行为时辨认能力受损不明显,控制能力明显削弱,一般评定为部分责任能力;重性心境障碍包括躁狂发作、抑郁发作和谵妄性躁狂。其辨认和控制能力也常受到较严重的影响,结合其具体实施危害行为时的辨认和控制能力一般评定为无责任能力或限制责任能力;伴精神病性症状的心境障碍,在抑郁发作或躁狂发作的同时伴有精神病性症状时,患者严重的情绪障碍与认知障碍相互影响,较易与周围环境产生冲突。对其实施危害行为时的辨认或控制能力丧失评定为无责任能力。

\subsection{民事法律能力}

心境障碍患者涉及民事法律能力问题常见的案例涉及患者的婚姻能力,如离婚案件中患者是否有行为能力是与离婚诉讼;合同能力,如患者有能力与别人订合同;财产处置及继承能力,如患者是否有能力处置自己的财产或继承他人的财产等。这些都归属于患者的民事行为能力范畴。

1.民事行为能力评定原则 心境障碍患者是以情感和心境改变为突出特征,而情感和心境的改变很明显地影响到患者的意志和行为,使其在民事行为中正确地表达自己意思,并理智地处理自己事务的能力受到不同程度的影响,即影响到其正确表达自己的意思。因此对心境障碍患者行为能力评定总的原则是:结合根据心境障碍患者病情严重程度,看其是否具有独立地判断是非和理智地处理自己事务的能力,分别评为有行为能力、限制行为能力和无行为能力,同时要区分是宣告民事行为能力还是某一行为当时的行为能力。

2.宣告民事行为能力 这是指心境障碍患者尚未涉及某一具体民事行为时,经其利害关系人申请,经法院受理委托,对其行为能力进行评定,经法院认定宣告。这是对心境障碍患者行为能力的一种广义的评定。考虑到心境障碍是一种发作性精神疾病,有正常的间歇期这种特殊性,因此无特殊的需求和必要性,一般不易对心境障碍患者进行宣告民事行为能力评定。在精神疾病司法鉴定实被中对心境障碍患者进行宣告民事行为能力一般适用于慢性心境障碍或持续性心境障碍。因这些心境障碍病程持续较长,多数缺乏明显的缓解期或缓解期比较短暂。鉴定中可以根据疾病严重程度可能对其意志行为和意思表达能力的影响进行推定式的行为能力评定。

3.民事行为时的行为能力 在心境障碍患者民事行为能力评定中绝大部分属于此类,它包括:①心境障碍患者已经实施完成的某一民事行为时的行为能力,如生前或现在已立的遗嘱、已完成的财产公证、已签约的合同或已提交的辞职报告等;②已明确的即将进行的某一民事行为时的行为能力,如离婚诉讼、财产分割或处置等。此类行为能力评定特点是针对某一明确的具体的民事行为时的行为能力评定,因此评定原则是结合心境障碍患者的病情重点考察对这一具体民事行为是否具有真实的意思表达,即对该事物的判断、理解和处理能力。

\subsection{其他相关法律问题}

1.性保护能力 女性心境障碍患者,在社会生活中有时会受到不法分子的性侵害行为。特别女性患者在轻性躁狂发作时,常伴有性欲亢进,患者常浓妆艳抹、花枝招展、举止轻浮,好接近男性,此时更易受到性侵害。女性心境障碍患者受到性侵害时性保护能力的评定,要结合患者心境障碍的严重程度和对该性行为的实质性辨认能力综合评定。一般地说,①重性心境障碍评定为无性保护能力;②轻性心境障碍、环性心境障碍和恶劣心境患者,要结合性行为事件的过程及患者对该性行为的实质性辨认能力确定其性保护能力,可评定为无性保护能力、性保护能力削弱和有性保护能力;③心境障碍缓解期,对性行为有辨认能力时评定为有性保护能力。

2.精神损伤 心境障碍患者人身损害赔偿案近年来在司法精神病鉴定中逐年增多。在鉴定实践中患者由于打架纠纷、被处罚、惊吓或交通事故后出现心境障碍,而导致一些民事纠纷。因心境障碍与精神分裂症一样目前对其病因的共识是归因于内因性精神疾病,所以对于有关心境障碍与生活事件的关系的精神损伤的描述同精神分裂症,以诱发因素描述为妥。

具体评定原则:明确生活事件即心理刺激前被鉴定人精神状态是否完全正常。若生活事件前确实精神状态完全正常,而且该生活事件与该患者心境障碍的发病有密切的时间联系,可评定为该生活事件是其心境障碍发病的诱发因素;若生活事件发生时,被鉴定人已处于心境障碍病程中,要确定该生活事件是否加重了被鉴定人心境障碍的疾病严重程度,除要查明生活事件与心境障碍病情加重有密切的时间联系,而且必须确定其加重的疾病症状的内容与生活事件有密切的联系,即有时间的关联性和内容的关联性,方可评定为该生活事件加重了被鉴定人原有心境障碍的发展,否则评定为无关。

\textbf{心因性精神障碍}

心因性精神障碍是指一类其起病及临床表现与心理社会因素密切相关的精神障碍,包括数种精神疾病,如急性应激障碍、反应性精神病、创伤后应激障碍、适应障碍以及与文化相关的精神障碍等。

在司法精神病鉴定实践中,涉及刑事责任能力评定的案件时有可见。近十年来涉及心因性精神障碍的精神损伤案例逐年增加,已成为司法精神病鉴定中十分重要的内容。

\subsection{刑事责任能力}

因心因性精神障碍而出现危害行为主要见于反应性精神病、旅途性精神障碍、气功所致精神障碍、迷信与巫术所致精神障碍。

1.心因性精神障碍与危害行为 以凶杀和伤害最为多见,其次见于性侵害行为,较罕见有关财产侵害行为。该类精神障碍常起病较急,可有程度不同的意识障碍。一般都具有明显、片断的幻觉、妄想,且多为被害妄想。情绪高度紧张、恐惧,常在幻觉、妄想直接影响下出现冲动伤人行为。

2.危害行为和责任能力 该类精神障碍患者有明显的幻觉、妄想,常是在幻觉、妄想的直接影响下发生伤人等危害行为,或有些患者伴有一定的意识障碍,其危害行为常具有自动症的性质。因此他们对自己的危害行为丧失了辨认或控制能力,一般评定为无责任能力。

\subsection{民事行为能力}

由于该类精神障碍的特点是起病较急,一般病程较短(除创伤后应激障碍),且预后较好,故一般不易进行宣告民事行为能力的评定。对已经发生或近期内即将发生的民事行为能力评定,应根据疾病严重程度对其真实意思表示能力的影响程度来评定。在司法精神病鉴定实践中,涉及该类精神障碍患者民事行为能力评定的案例较少见,主要涉及的是少数迁延不愈的创伤后应激障碍患者。

\subsection{精神损伤}

心因性精神障碍的司法精神病鉴定中最常见的是精神损伤的评定,其中以创伤后应激障碍的精神损伤评定最为常见,其次是反应性精神病的精神损伤评定。

一般来说反应性精神病,因是在遭受强烈精神刺激后发生的精神病的状态,如被强奸、突然被殴打、亲人被害等,患者病前心理素质常较健全,故伤害因素与该病之间关系十分明显,通常评定为伤害因素与此病的发生有直接因果关系。

而创伤后应激障碍的发生与生活事件之间的关系评定则较为复杂,也是精神损伤评定中的难点之一。因为作为心理刺激因素的生活事件的刺激强度大小不一,创伤后应激障碍的发生与个性心理素质有较密切的关系,所以在评定生活事件与创伤后应激障碍发生的关系时,要根据刺激的强度、个体的心理素质以及当时的躯体状况,综合分析其生活事件与该病发生的关系。具体关系的描述可有生活事件与该病的发生有关诱发因素,或间接因果关系。

\textbf{精神发育迟滞}

精神发育迟滞在精神疾病司法鉴定中仅次于精神分裂症,占20%~30%,并以轻、中度精神发育迟滞者较多。

\subsection{刑事责任能力}

精神发育迟滞患者,因其智力发育障碍,其自我控制能力较差,社会道德、法制观念薄弱,以及工作能力、社会适应能力低下,而本能相对亢进等出现各种危害行为。

1.精神发育迟滞与危害行为 精神发育迟滞患者的危害行为以盗窃行为、性侵犯行为和纵火行为多见,其危害行为特点:①动机简单:作案动机常十分幼稚单纯,如盗窃行为常是为满足基本的饥饿或为一件小事,而对他人不满采取报复行为,对后果缺乏预见,动机与后果明显不相称。②手段、方法笨拙:其作案常无预谋,受本能支配,如性本能相对亢进,常出现性侵犯行为,而不选择时间、地点,多为强奸未遂。③多单独作案:由于其智力低下,难以与他人交往,多单独活动,但易被别人利用唆使作用,在团伙作案中常是从属地位。

2.危害行为与责任能力 精神发育迟滞患者危害行为时责任能力评定应结合智商、学习能力、生活、工作、社会适应能力和危害行为的动机,分析其危害行为时的辨认和控制能力,评定其责任能力。

\subsection{民事行为能力}

精神发育迟滞患者因其智力发育障碍,其意思表示能力常受到不同程度的影响,因此,其民事行为能力评定,应根据疾病严重程度对其真实意思表示能力的影响程度来评定。

\subsection{其他相关法律问题}

女性精神发育迟滞患者常易遭受不法分子的性侵犯。患者对性行为理解、认识和控制能力是评定其性防卫能力的主要依据,即患者是否理解遭侵害的行为的性质及后果,以及可能给自己造成的生理、心理的伤害,特别要结合被性侵害时的实际情况考察其自我保护能力。

\textbf{脑器质性精神障碍、躯体疾病及精神活性物质所致精神障碍}

脑器质性精神障碍、躯体疾病及精神病性物质所致精神障碍可分为急性和慢性精神障碍。急性精神障碍主要表现意识障碍,意识障碍程度轻重不一,常可出现幻觉、妄想,甚至谵妄状态;慢性的精神障碍表现为人格改变智能障碍及精神病性状态。

\subsection{刑事责任能力}

\subsubsection{急性精神障碍}

1.急性精神障碍与危害行为 在该类精神疾病急性精神障碍的意识障碍、幻觉和妄想影响下,患者常会对周围环境产生突然的攻击行为而伤人、毁物。较为多见的为癫痫
性精神障碍。在癫痫
发作性精神障碍中,如癫痫
精神运动性发作、癫痫
性发作性情绪障碍(又称病理性心境恶劣)时,常见严重的攻击行为,包括凶杀和伤害行为,其攻击行为常具有突然性,无明显的动机等。其次较为常见的是酒精所致精神障碍。少数的脑动脉硬化症、老年性痴呆可出现性侵犯行为,常是在脑部病变后出现性功能的相对亢进,以及社会道德观念的衰退,与其病前行为判若两人。因颅脑外伤后急性精神障碍出现危害行为的在司法精神病鉴定中较为少见。

2.危害行为与责任能力 这类精神疾病的急性精神障碍时,通常有明显的意识障碍,片断的幻觉、妄想,故在意识障碍、幻觉、妄想的直接影响下丧失了辨认和控制能力,一般评定为无责任能力。但普通醉酒时虽有意识障碍,根据我国《刑法》规定仍然有完全责任能力。而复杂性醉酒一般评定为限制责任能力。

\subsubsection{慢性精神障碍}

在该类精神疾病的慢性精神障碍影响下发生危害行为的主要见于人格改变及精神病性状态。

1.慢性精神障碍与危害行为 该类精神疾病所致的人格改变,常表现为激惹性明显增加。常因一点小事而出现明显的攻击行为,其特点是动机与结果的严重程度明显不相称;有些表现为极端的自私记仇,因一次矛盾而长期耿耿于怀,而为数月或数年前的矛盾而出现明显的攻击行为;有些人格改变表现为精神活动减少,缺乏能动性、社会道德伦理的衰退,而出现反复的盗窃行为,如严格的额叶颅脑损伤后的人格改变。

该类精神疾病所致的慢性精神病性状态,常在幻觉、妄想的影响下出现各种危害行为,主要有凶杀、伤害和性侵害行为。如酒精中毒幻觉症患者,在大量生动的幻觉影响下对周围环境产生明显的攻击行为,其特点是行为紊乱十分明显。

2.危害行为与责任能力 脑器质性疾病所致的人格改变患者实施危害行为时,因受人格改变的影响,其控制能力明显削弱,或辨认能力明显削弱,一般评定为限制责任能力。而酒精所致人格改变结合作案过程可评定为有责任能力或限制责任能力。在该类人格改变患者责任能力评定时,要掌握充分的人格改变的依据,看其人格改变是否具有普遍性,及严重程度等。

该类精神疾病的慢性精神病性状态在幻觉、妄想直接影响下实施危害行为的一般评定为无责任能力。

\subsection{民事行为能力}

该类精神疾病急性精神障碍时,常意识障碍明显,一般病程较短,故一般不易进行宣告民事行为能力评定。

该类疾病所致的慢性精神病性状态患者的民事行为能力评定,通常受其精神病性症状的影响,其辨别事物、自我保护能力明显受损,较难形成真实的意思表示,一般评定为无行为能力。

\subsection{其他相关法律问题}

该类精神疾病的精神损伤评定主要见于颅脑外伤所致的精神障碍,包括急性和慢性精神障碍的精神损伤评定。

一般情况下,颅脑损伤所致精神障碍与致害因素之间的关系评定为直接因素关系。

\textbf{人格障碍及性心理障碍}

人格障碍是人格特征显著偏离正常,表现特有的行为模式,造成对环境适应不良。在司法精神病鉴定中时有所见,较多见为反社会人格障碍。

\subsection{刑事责任能力}

1.危害行为 反社会人格障碍患者通常在少年时期即有各种品行障碍,如逃学、斗殴、抽烟、喝酒、虐待小动物等,他们常性情冷酷,冲动性较明显,自我控制力差,成年后常可出现各种危害社会的犯罪行为,如盗窃、凶杀、伤害、妨碍治安、诈骗等;冲动型人格障碍(又称爆发性或攻击型人格障碍)者,耐受力极差,情绪易激惹,往往在遭刺激后,失去控制能力,出现强烈的冲动行为,毁物伤人等。

2.危害行为与责任能力 人格障碍者实施危害行为时,无辨认能力障碍,一般评定为具有完全责任能力;而冲动型人格障碍者,结合具体的案情及平时人际关系、品行及脑电图等,可以考虑评定为限制责任能力。

性心理障碍表现性指向、性偏好及性身份障碍,一般为有责任能力,其中恋物癖、露阴癖及窥阴癖可结合作案情况评定为限制责任能力。

\subsection{民事行为能力}

在司法精神病鉴定实践中,涉及人格障碍者行为能力鉴定主要见于偏执性人格障碍,因其偏执性人格的特点常与环境纠纷出现反复的诉讼行为,结合其诉讼行为内容与人格障碍之间的关系,可评定为限制行为能力。