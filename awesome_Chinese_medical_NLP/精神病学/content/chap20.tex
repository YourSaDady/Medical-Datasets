\chapter{精神障碍的预防与康复}

\section{精神障碍的预防}

尽管精神障碍的确切病因和发病机制大多未明,但预防工作仍大有可为,且越来越得到政府部门和医学专家的重视和关注。特别对一些病因明确的精神障碍或心理社会因素较突出者,通过早期干预,能使患者尽快康复或不出现严重的精神障碍。要将精神障碍的预防工作做到位,仍需不断努力,特别是加强对各类精神障碍的病因和发病机制的探索,只有在明确病因的基础上,才能从源头去预防。目前各国对精神障碍的预防工作主要以Caplan(1964)提出的精神障碍三级预防模式为基础。

\subsection{一级预防}

一级预防即病因预防,对于精神障碍而言较为困难,因为大多数精神疾病的病因仍不清楚。但是我们仍要采取积极主动的预防措施,尤其是对一些病因已经明确的器质性精神障碍。

1.对病因较明确的疾病 如器质性精神障碍、精神活性物质所致精神障碍、精神发育迟滞等,应加强对原发疾病的治疗;加大宣传,使公众了解毒品和酒精对人体的伤害;加强环保意识,改善工作环境,减少有害物质对人体的侵入等。这些都需要得到政府各级部门和全社会的重视。

2.对一些可能与遗传相关的精神疾病,积极开展遗传咨询,必要时采用遗传学检测,尽早发现并处理与遗传相关疾病,如苯丙酮尿症,早期可通过限制苯丙氨酸摄入和饮食治疗,即能避免影响患者的智能发育。

3.开展精神卫生宣教普及,让人们认识到心理社会压力对健康的影响,怎样预防和化解心理压力的影响,或尽早寻求心理支持,普及心理咨询和治疗,提高全民的精神卫生意识。

4.提高全民心理素质,从小抓起,这需要家庭、学校和社会的共同努力,促进儿童心理的全面发展,塑造健全人格。

5.加强精神疾病的病因学研究,定期开展流行病学调查,为源头预防提供可靠依据。

\subsection{二级预防}

二级预防,即对精神障碍的“三早”:早期发现,早期诊断,早期治疗。这也是目前最为行之有效的精神障碍预防措施。通过“三早”,可以帮助患者达到良好的康复。由于许多患者潜隐起病,早期往往不被家人重视,待到医院就诊,已延误最佳治疗时机。社会的歧视和偏见,也是妨碍患者及时就诊的重要因素,对于良好康复带来不利影响。

1.早期发现 需要全民关注精神卫生事业,普及精神卫生知识,消除社会歧视和偏见。精神障碍同样也是影响人类健康的一种疾病,一旦发现原有的行为模式改变,工作和学习效率无明显原因下降,情绪持续不稳定,人际关系紧张等,应及时到精神卫生机构就诊。

2.早期诊断 对发现有精神异常,或原有生活模式、习惯发生改变时,应及早到专科门诊就诊,以便得到及时诊断,而不延误最佳治疗时机。这需要公众消除偏见,克服侥幸心理,改变对出现的问题漠不关心的态度。

3.早期治疗 要相信科学,不要延误最佳治疗时机。一旦明确诊断,应及时有效地采取治疗措施。无论门诊或住院治疗,均应做到足量足程,且在精神科医师指导下进行。维持治疗,对防止精神病复发也非常重要。

\subsection{三级预防}

三级预防主要减轻功能残疾或延缓残疾发生,保持患者原有的或部分的功能,提高患者及家庭的生活质量,减轻社会负担。

1.防止疾病反复或恶化 这需要得到家庭和患者的配合,在医生指导下,坚持巩固或维持治疗,防止疾病反复。患者必须要有长期治疗的心理准备,而家庭应做好督促、检查工作,防止疾病反复发作而转为慢性,导致精神残疾出现。

2.督促患者参加各种有益活动和康复训练,防止残疾出现。

3.政府、社会和家庭都来关爱和帮助精神障碍患者,帮助他们树立战胜疾病的信念,创造良好的康复环境,提高生活质量。

4.根据个体差异,制订个体化的康复训练计划。

\section{精神康复的概念与任务}

精神康复(psychiatric
rehabilitation)是康复医学的一个分支,是把精神疾病与康复有机地结合起来。世界卫生组织(WHO)于1969年提出了康复的定义:“康复是指综合性地与协调性地应用医学的、教育的、社会的、职业的和其他一切可能的措施,对残疾者进行反复训练,减轻致残因素造成的后果,使伤者、病者和残疾人尽快和最大限度地恢复与改善其已经丧失或削弱的各方面功能,以尽量提高其活动能力,改善生活自理能力,促使其重新参加社会活动并提高生活质量。”随着精神医学的不断发展和各级政府和社会的重视,特别是精神疾病的治疗学和社区服务的发展,为精神康复提供了可靠的基础和保障,使这门学科越来越受到重视,得到进一步的发展。

精神康复有狭义和广义的理解。狭义的概念是最初形成的,精神病患者通过住院和门诊的各种治疗,症状消失,自知力恢复,患者达到临床康复。这也是临床医生和家属认可的目标。实践工作表明,这远远不够,并且有可能使部分患者脱离社会,表现为退缩,不能适应家庭和社会生活,成为残疾人。对精神康复的广义理解,越来越为临床医生,心理、社会工作者,患者及家属认可,既要达到临床康复,同时要恢复患者原有的家庭、社会功能,达到完整意义上的精神康复。

精神康复的任务,即采取各种治疗和训练方法,消除和减少精神残疾的发生,使绝大多数患者恢复其原有的社会功能和家庭功能,回归社会,得到全面康复。这项任务非常艰巨和长远,需要得到各级政府、医院、社会和每一个人的理解和支持,在人、财、物和政策上提供保障。

\section{精神障碍的医院康复}

根据各级各类精神病院的功能不同,开展的院内康复可以不同。设在三级甲等综合性医院的精神科病房,以收治急性、危重精神疾病患者为主,通过各种治疗,使患者达到临床痊愈后,可以转入社区进行康复训练。地市级以上精神病专科医院,在治疗急性、危重患者的同时,有条件的要开展精神康复工作。对县市级精神病院应逐渐转入社区康复工作。对慢性收容性精神病院,应以康复治疗为主,开展对患者的康复训练,使患者尽快适应社会和家庭的工作、学习和生活,减少或降低精神残疾的发生率。

\subsection{医院康复的工作内容}

1.精神卫生机构收治急性患者,其精神康复工作的内容是:

(1)训练患者的心理社会承受能力及对治疗的依从性,在急性症状得到控制后,要使患者正确地面对自己的疾病,面对家庭及社会,从心态上做好准备。

(2)恢复患者原有的家庭和社会方面的行为技能,包括生活、学习、工作能力及社会交往能力等训练,使他们尽快适应回归社会。

(3)对急性患者采用封闭式治疗后,可采取开放或半开放式管理,提供有利于患者进一步康复的合适条件。

(4)健全医院的康复管理体制与相关制度,在场地、人员、设施上做好配备,建立良好的医患关系,增强患者战胜疾病的信心,培养患者独立自主处理问题的能力。

(5)对于康复效果,可采用评定工具或记录表格进行评估,有利于康复工作不断改进。

2.精神卫生机构收治慢性患者,主要以防止出现精神残疾为目的进行康复训练,以提高生活自理能力和改善生活质量。通过工娱治疗、音乐治疗、绘画、书法、手工劳动等技能训练,延缓患者的精神衰退,提高患者与家人、社会交往的能力,促使患者回归正常生活。

\subsection{医院康复的训练措施}

根据精神疾病患者病种、病情、病前社会角色和文化程度等方面的不同,制定不同的康复训练内容并采取相应方法,使他们尽快适应相应的康复训练,尽快回归社会。具体康复有以下方面:

\subsubsection{心理康复}

心理康复对首次发病患者十分重要。这些患者往往在精神症状得到缓解后无法面对现实,出现自卑,采取回避社会的态度,严重者甚至可能出现自杀行为,故对这部分患者的心理康复非常重要。在药物治疗基础上,开展心理治疗和康复训练,帮助患者认识到精神疾病也是一种疾病,是能够战胜的,只要积极与医护人员、家庭及社会配合,是能够恢复正常并回归社会的。

\subsubsection{发挥请假出院的康复作用}

请假出院是精神科治疗的一项特殊措施,对患者适应社会和家庭生活起着重要的作用。因此,患者精神症状一旦得到控制,且条件允许,应允许患者请假出院(但需得到监护人的保证),让患者尽早接触正常的生活和社会交往。有些患者可在请假出院期间,恢复学习和工作,也有利于观察和检验他们社会功能的恢复情况。

\subsubsection{行为技能的训练}

1.日常生活与活动技能训练 对于慢性衰退患者非常重要。慢性患者往往生活不能完全自理,退缩、被动,训练的重点是培养良好的个人卫生习惯和生活自理能力。通过定时、定期持之以恒的训练,鼓励和帮助患者建立起良好的生活模式。

2.简单作业训练 可帮助患者提高动手能力,促使他们多活动。这种作业技术要求不高,操作简便,适合大多数精神病患者,如折叠纸盒、包装小玩具、整理文件、抄写文书等。可以根据患者不同的职业、文化程度,相应安排种菜、养殖等工作。

3.工艺制作训练 有助于提高患者思维能力和操作能力的协调性,培养他们的兴趣爱好和创造力。要求和起点相对较高,需配备具有一定专业背景的工艺师来指导和带教。该训练有助于稳定患者情绪,转移他们的病态体验。训练方法有玩具及装饰品的制作、各种编织、服装裁剪与缝制,以及工艺品的制作等。

\subsubsection{文体娱乐活动训练}

文体娱乐活动适用于绝大多数精神疾病患者的训练,患者在其中容易找到生活的乐趣,促进身心健康,消除或减轻药物可能带来的不良反应。这些活动丰富多样,患者可根据自身特点选择最合适的活动。训练方法有音体治疗、歌咏、舞蹈、乐器演奏、健美、形体等训练,以及参与性的体育活动,如体操、球类、八十分扑克牌、桥牌等比赛,这些训练要根据患者的体能来确定。可开展书法、绘画等活动,熏陶患者的情操;可定期组织患者郊游,参观文化古迹和参加一些公益性社会活动,让患者能多接触社会。在这些活动中要注意培养患者的团队协作精神,学会沟通和理解。

\section{精神障碍的社区康复}

精神障碍的社区康复实际是医院住院或门诊治疗后康复的延续,使患者最终能恢复病前的社会功能和承担自己应有的责任,或减缓精神残疾的发生,或最大限度地保障残疾人的各项权益,此为社区康复的最终目的,也是全程治疗的一个重要部分。社区(community)是指具有一定的地理区域,由若干群体式组织自然汇聚而成,并且在社会生活相互依赖和关联中形成的一个大的集体。社区的特点:具有一定地域或空间;具有一定生产关系和社会关系为基础的人群;在特定的时代背景下具有一定的行为规范和生活方式;在心理和情感具有潜在的亲和力,即本土观念。社区康复(community-based
rehabilitation),即以社区为基础的康复,需要得到各级政府、社会团体和各界人士的支持和关爱,只有这样社区康复才能顺利开展。

\subsection{精神障碍社区康复的体系}

精神障碍社区康复在发达国家和地区已形成了一套完整的体系,在急性期通过短期住院治疗,后转向社区治疗,使患者能尽快回归社会。我国自20世纪60年代起,在一些较发达地区,即开展了这方面工作,且取得许多成功经验,形成了一些地区性的康复模式。但就全国范围而言,尚未形成一个国家性的行之有效的康复网络系统。我国政府和各级地方政府对该项工作越来越重视,这对促进精神障碍的社区康复具有重要意义。该项工作不但需要具备精神医学、心理学、流行病学和社会学等方面知识的人员参与,更需要得到政府的领导和财政支持以及社会各部门的密切配合,切实关心和保障精神疾病患者的切身利益。

1.成立相应的精神障碍防治康复工作领导小组,由政府牵头,由相关各部门和医学专家等组成,负责本地及指导下一级的工作。

2.依靠当地精神卫生资源,建立社区基层卫生保健网络,由经过精神卫生知识培训的基层医护人员、家庭、社会工作者、社区等共同参与。

3.根据患者的病情和不同需求,设立相应的康复场所,为患者提供必要的医疗、护理、活动和工作场所。

\subsection{精神障碍社区康复的形式}

1.在社区建立精神专科门诊的网点,定期由精神专科医师、社会工作者、心理学家及基层医护人员参加的门诊。门诊与家庭病床相结合开展工作,对社区患者分别归类记录在案。工作内容为:①负责本社区精神障碍患者康复期的维持治疗和记录,以及不愿住院的患者治疗和病情变化的记录,及时提出干预对策。②定期进行家庭访视,指导家庭和志愿者。③在社区开展精神障碍防治宣教工作,以便及早发现新出现病例,予以早期诊断和治疗。④对本社区的精神障碍患者记录、归档,得到完整的流行病学资料。⑤配备社会工作者、心理学家对患者开展心理康复训练和社会家庭职业功能指导培训。⑥对患病期间出现违反社会治安的患者要重点定期随访。

2.过渡公寓 是国外开展较多的一种形式,为患者出院回归家庭时的一种过渡形式。出院患者住进有精神科医师、护理人员参与管理的公寓,使患者有一适应过程,包括白天可以外出参加其他康复训练或职业训练,晚上回公寓住宿,治疗仍由医护人员指导,可按自己的需求安排生活,一旦适应家庭生活,即可重返家庭。

3.工疗站 是国内许多城市开展较多的一种康复形式,主要以职业培训为主,在大的厂矿内,对本单位的精神障碍患者进行集中职业康复,并且得到医疗上的支持,使他们在保证维持治疗的基础上,发挥原有的劳动技能。此外由社区和民政福利部门主办的福利性企业,可以解决部分没有职业的,相对比较稳定的患者,解决他们生活上的困难,提高其生活质量,有利于病情的稳定和减少精神残疾的发生,可以同时辅以文娱体育活动和心理治疗。

4.日间医院 对一部分由于没有完全恢复正常的患者,家庭白天无人照料的,可住进日间医院,继续接受治疗和其他各方面的康复训练及心理治疗,晚上和亲人团聚,增进亲人之间的交往和感情,防止患者脱离社会。

5.家庭病床和群众性看护小组两者可以结合,专业人员和经培训的志愿人员可以定期对社区或负责的小组内患者定期随访,掌握他们治疗的情况以及病情变化。帮助患者提高自我解决问题的能力,指导家属对患者进行治疗、护理和照料。一旦发现病情恶化或可能出现自伤、自杀或违反社会治安行为时,及时联系送医院治疗,指导患者面对工作或生活中可能遇到的困难并寻求解决方法。

6.长期看护 适用于慢性患者,生活自理能力较差者,治疗效果不理想且明显影响社会治安的患者,需要长期照料的患者。

7.成立相应的联系会 这是民间自发组成的一种自助式团体,可以是由家属组成的联谊会,或同种疾病患者的联系会。在国外常见的有酒依赖者匿名戒酒会(alcoholic
anonymous,
AA)等,他们在照料或康复过程中可以相互交流、支持,提高抗病的能力和信心,同时可接受专业人士的指导,定期授课和集体心理治疗等。国内目前也有这种发展趋势。