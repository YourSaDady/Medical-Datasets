\chapter{绪论}

\section{精神病学的概念与任务}

精神病学(psychiatry)是临床医学的一个分支,是以研究各种精神疾病的病因、发病机制、临床表现、疾病的发生发展规律以及治疗和预防为目的的一门学科。

由于精神疾病本身的特点和复杂性,精神病学既为医学的分支,又往往涉及很多其他方面,如社会、文化、伦理、经济等等问题。20世纪50年代以来,由于医学模式的改变,传统的精神病学概念遇到了挑战,逐渐被新的、范围更广泛、内容更丰富的精神卫生(mental-health)所取代。

自然科学的发展引起生物学技术革新,使许多疾病的发生、发展从生物学角度得到较满意的解释,并找到了不少有效的治疗方法,生物医学模式便成为现代医学的标志。但在半个世纪的实践中,暴露出生物医学模式的缺陷,即疾病被认为完全可用偏离正常的、可测量的生物变量来说明,没有考虑社会心理和行为方面的作用。为此,医学家们提出了生物医学模式应向新的生物-心理-社会医学模式改变。新模式强调医学的对象是完整的、社会的“人”。“人”是生活在一定自然、社会、文化环境中,具有复杂心理活动的生物,“人”可看作是由许多连续的功能平面(系统、器官、细胞、亚细胞、分子\ldots{}\ldots{})构成,并向外部世界开放的系统。社会环境的各种刺激,通过人的心理活动,后者又通过各种生物学的中介机制来影响机体各个平面的功能状态。这种医学模式整体观的新发展,反映在精神医学方面尤为突出。也由于这种认识,20世纪70年代以来世界卫生组织(WHO)宪章序言中提出了健康的新概念:“健康不仅是指没有疾病或残缺,而应包括躯体、心理和社会功能的完好状态。”与之相应,便提出了如何保障精神健康的内容。精神卫生这一术语从此在国际和国内广泛应用。广义的精神卫生的含义较精神病学更广,即不仅研究各种精神疾病的发生、发展规律,而且要探讨如何保障和促进人群心理健康,以减少和预防各种心理或行为问题的发生,这就逐渐取代了传统狭义的精神病学的概念。

\section{精神病学发展的概况}

古希腊医学家希波克拉底(Hippocrates,公元前460---公元前377)被认为是医学奠基人,也被称为精神病学之父。他认为脑是思维活动的器官,提出精神病的体液病理学说。他认为人体存在4种基本体液,即血、黏液、黄疸汁和黑胆汁。四种体液如果正常混合起来则健康,如果其中某一种过多或过少,或者它们之间的相互关系失常,人就生病。他认为抑郁症是由于黑胆汁过多,进入脑内破坏脑的活动的缘故。这一时期对精神疾病进行了初步分类,并对某些精神疾病的原因有了初步设想。到了中世纪(公元476年至17世纪),由于医学被神学和宗教所掌握,精神病患者被视为魔鬼附体,采用拷问、烙烧、坑害等苦刑来处罚,使精神病患者处于十分悲惨的境地,精神病学不但没有发展反而后退。18世纪法国大革命的胜利对精神病学产生了很大影响,对西欧精神病学来说也是一个转折点。从这一时期开始,精神病被看作是一种需要治疗的疾病,精神病患者被看作是社会的成员。比奈尔(Pinel)是第一个担任“疯人院”院长的人,他去掉了精神病患者身上的铁链和枷锁,把他们从不见天日终身囚禁的监狱生活中解放出来,将“疯人院”变成了医院,进行了有历史意义的革命,为后来的精神病学的发展奠定了基础。到19世纪中叶,随着自然科学的发展以及临床资料的积累,Griesinger于1884年指出了精神病是由于脑病变所致,精神病学从此进入现代精神病学的发展阶段,其代表人物是德国精神病学家克雷丕林(Kraepelin)。他以临床观察为基础,以病因学为根据,提出了临床分类学原则。他认为精神病是一个有客观规律的生物学过程,可以分为数类,每一类都有自己的病因、特征性躯体和精神症状、典型的病程经过和病理解剖所见以及与疾病本质相关的转归和结局。他的思想推动了精神病学理论的发展,为精神疾病分类学打下了基础,并使精神病学的理论进入自然疾病单元的研究。克雷丕林被认为是现代精神病学医学模式的奠基人。20世纪以来,许多精神病学家从神经解剖、神经生理、神经生化和心理学等不同角度,对精神疾病的病因、发病机制、诊断和治疗进行了大量的研究和探讨。另一方面,社会学科特别是人类学、社会学和社会心理学参与了精神病学的实践和研究,使社会文化、社会心理因素对精神疾病、心理和行为问题的发生、发展的影响日益受到重视,并相继形成了学术观点不同的学派,如生物学派、心理动力学派、行为学派及社会学派等等。当代医学家提出了生物心理社会医学模式,认为应该从生物学、心理学和社会学三个方面,而不能仅仅从生物学单方面研究人类的健康和疾病问题以及社会的医疗保健措施,包括精神疾病和精神卫生问题的医疗保健措施。

我国精神病学发展较迟。新中国成立前全国精神病医疗机构不到10所,床位不足1000张。中华人民共和国成立后,我国精神病学进入了一个新的历史时期。1958年精神病医疗机构增加到70所,床位11000余张。1978年以后各种类型精神病医疗机构已达500所。这些机构包括综合医院的精神科病房、精神病院、精神病疗养院、精神病收容所等。与新中国成立前相比,不仅医疗机构和床位数量大量增加,尤为重要的是医疗设备和技术水平有显著提高。国内一些重点医学院校迅即建立了精神病学教研组,在教学计划中把精神病学列为临床必修课。20世纪80年代以后,在部分医学院校成立了精神卫生专业,建立硕士点、博士点,已培养出许多高质量的专业人才。

1951年出版了粟宗华著的《精神病学概论》,这是新中国成立后的第一部精神病学著作。随后,1960年由南京神经精神病防治院编的《精神病学》出版。1961年四川医学院编的《精神病学》为我国正式出版的第一部高等医学院校精神病学教材。至今,不仅有全国统一的精神病学教材,还出版了许多高质量的精神病学参考书及专著。

我国精神病学的科学研究基础比较薄弱,20世纪80年代以前科研工作主要是对常见的精神疾病进行临床观察和总结,积累本国资料,20世纪80年代以后逐步开展了基础理论研究,尤其是生物精神病学研究工作在深度和广度上均有较迅速的发展。在临床精神药理研究方面,广泛地开展了血药浓度的测定和药代动力学研究,精神药理机制的研究已具有相当的进展;在神经生化方面,研究内容从与神经递质有关的酶的代谢提高到受体水平,研究范围扩大到神经内分泌、肽类、免疫功能、微量元素以及氧自由基的测定;分子遗传学研究也有了可喜的开始;心理社会因素、应激和健康的研究引起医学界广泛的兴趣,研究病种也不断扩大。

1982年第一次在全国范围内使用统一的国际通用筛选工具和诊断标准,进行了12个地区精神疾病流行病学协作调查,取得国内精神疾病流行病学较全面的资料。为了加强国际学术交流,提高临床和实验室的研究水平,我国先后制定了《中国精神疾病的分类方案和诊断标准》,如:CCMD-1(1986年)、CCMD-2(1989年)、CCMD-2-R(1994年)、CCMD-3(2001年)。这些均为临床医生不可缺少的诊断工具。与此同时,精神病学学科的建设,根据临床工作的需要,而又分为临床精神病学、儿童精神病学、老年精神病学、司法精神病学、精神病流行病学、社会精神病学、社区精神病学、成瘾精神病学、跨文化精神病学及联络-会诊精神病学等分支,使精神病学得到全面发展,并且研究的范围已扩大到各种心理卫生及行为问题和保障人群心理健康等。

\section{精神病学与其他学科的关系}

\subsection{其他临床学科}

人的机体是一个整体。中枢神经系统,特别是大脑,在协调、筛选和整合来自机体内外环境的各种刺激中起着主导的作用。大脑活动和机体其他系统活动是密不可分的,且受到机体内外环境因素的制约。因此,精神病学与临床其他学科的关系是十分密切的。各种躯体疾病,如脏器、内分泌、结缔组织、营养代谢等疾病均可导致脑功能的变化而引起精神症状,即所谓的躯体疾病所致精神障碍;而持久的心理社会应激、强烈的情绪体验,使机体某些功能出现持续性紊乱,甚至出现组织结构上的异常改变或削弱机体的抵抗力,导致各种心理生理障碍,甚至心理生理疾病(心身疾病),如神经性皮炎、支气管哮喘、冠心病、高血压、消化性溃疡等均属于心身疾病,为此联络-会诊精神病学应运而生,特别在综合性医院其他躯体性疾病引起的精神障碍或由于持久的心理因素而致的严重躯体疾病的诊断、治疗和研究,解决了其他临床学科无法解决的问题。此外,精神疾病往往可以出现各种各样的躯体症状,如:惊恐发作的患者常因心慌气短而首先在内科就诊;抑郁症患者可因消化症状、闭经或躯体不适而去内科、妇产科求治。精神科与神经科的关系就更为密切。中枢神经系统病变时,临床上可以表现为低级神经活动障碍,如感觉、运动功能障碍,也可表现为高级神经活动障碍,如幻觉、妄想等。一般来讲,前者属于神经科诊治范畴,后者属于精神科诊治范畴。两者可以在同一种疾病的不同阶段出现或同时出现,如病毒性脑炎、癫痫
、脑外伤、老年性痴呆等既可以有低级神经活动障碍,又可出现高级神经活动障碍,临床处理时常常需要神经、精神两个科的共同诊治。因此,一个精神科医生必须掌握临床其他各科的知识,才能对精神和躯体的疾病有一个整体的全面的了解,从而做出正确的诊断和治疗。

\subsection{基础医学}

精神病学是临床医学的一个分支,它的发展有赖于基础医学,尤其是神经科学的发展。神经科学是由神经解剖、神经生理、神经生化、神经药理和神经心理组成的一门综合性学科。这些学科的发展以及近十年来分子生物学的巨大成就和新技术的应用,使神经科学有了十分迅速的发展,科学家可以深入到神经细胞膜、受体、氨基酸和分子水平研究脑功能和药物作用的机制,使精神疾病生物基础的研究进入了一个新阶段。如:近几十年来神经生化的研究探讨了中枢多巴胺、去甲肾上腺素、5-羟色胺和γ-氨基丁酸与精神分裂症、情感性精神障碍以及与神经症的关系,加深了对精神疾病生物学基础的理解,从而推动了精神药理学、分子遗传学的发展,还为精神疾病的治疗提供了更好的药物,使精神疾病的治疗水平有了较大的提高。此外,神经科学研究的进展,同样为精神病的治疗和康复提供了一定的科学依据,神经可塑性的研究虽然表明中枢神经系统细胞死亡后不能再生,但对神经细胞的轴突、树突及突触连接上的研究表明,通过学习、训练和治疗等措施能使之发生改变,如海马中轴突长芽并导致功能恢复已被证实。功能影像学研究发现精神分裂症表现轴突和树突等减少,而致病人脑功能活动下降,出现相应的精神症状等,而近10年来发现的新型抗精神病药物具有神经营养作用,可以增强轴突和树突的形态学功能,而提高病人的脑活动功能,改善病人的认知功能和阴性症状,达到治疗作用。所以基础医学的发展能更好为临床医学服务。

\subsection{心理学}

情绪、心理活动影响机体功能和心身健康,早已为人们所重视。由于精神疾病表现为精神活动的障碍,要认识这些异常的精神现象,必须知道正常精神现象的有关科学知识,普通心理学便是研究正常心理现象的科学。这方面的知识、概念和术语有助于对精神疾病的精神症状和临床诊断进行分析和判断,因此普通心理知识是精神科医师必须掌握的基础知识。心理学的研究又推动了心身疾病的研究。心身疾病(psychosomatic
diseases,心理生理疾病)是一组与精神紧张有关的躯体疾病,它具有器质性病变的表现,或确定的病理生理过程所致的临床症状,心理社会因素在该病的发生、发展、治疗和预后中有相对重要的作用。临床心理学探讨了心理因素特别是情绪因素在疾病发生中的作用,可提高对神经症、某些心因性和器质性精神病的认识。临床心理学中的各种心理测验,通过对患者进行检查,可为临床诊断提供辅助性依据。心理治疗方法和技术适用于许多精神疾病的治疗,大大提高了单纯药物治疗的效果,从而对精神疾病的治疗与预防起了积极的推动作用。由此可见,在精神疾病的检查、诊断、治疗和预防工作中,具备心理学知识是十分必要的。

\subsection{社会学}

人类的思想和方法、风俗习惯、行为举止以及人际交往等,都具有一定的社会根源,并和特定的文化背景相关联。这些因素均可影响到精神疾病的发生、发展和转归。因此,有关社会学和人类学的知识,有助于理解和认识这些因素在精神疾病的发生、发展和转归中所起的作用,有助于人们从生物-心理-社会医学模式研究和探讨精神疾病的发生原因、治疗和预防干预措施以及对理论研究和临床实践都有着十分深远的意义。另外,当精神疾病涉及刑事、民事和刑事诉讼、民事诉讼时,需进行司法精神病学鉴定,确定被鉴定人是否患有精神疾病以及是否不能辨认或不能控制。司法精神病学鉴定的结论也属于诉讼证据的一种,因此从事这项工作的医师也应具有法学知识。

\section{今后的任务}

根据WHO的统计,非传染性疾病的比重日益增加,其中精神疾病的总负担占全部疾病负担的1/4,在10种造成社会最沉重负担的疾病中,精神疾病占4种。随着社会物质文明与精神文明的提高,人们对健康的需求不断增长,尤其是对心理健康的认识和需要更加突出。因此,精神病学与精神卫生将越来越受到人们的重视,在新世纪将会有较大的发展。

精神病学是临床医学的一个分支,它的发展有赖于基础科学的发展。随着分子生物学技术的持续发展和人类基因组计划的完成,从分子生物学水平探索精神疾病的病因将是未来研究工作的重点。精神疾病的相关基因可望被陆续克隆,这将对精神病学的发展产生巨大的促进作用。但是,精神活动毕竟是人类最复杂的功能,人们对它的了解甚少,这一直是制约精神病学发展的根本原因。令人振奋的是,继人类基因组计划完成之后,又一个全球性研究工程------人类神经组计划已经拉开序幕。这是一个揭开脑的奥秘的工程,它将有助于对精神疾病的病因、病理生理及发病机制的阐明,对精神病学学科的发展和完善将产生不可估量的推动作用,有望实现精神医学发展史上一个质的飞跃。

迄今,人类许多疾病的病因尚不明了,精神疾病也不例外,无法实现针对病因的治疗。因此,对于大多数精神疾病采用对症治疗,仍然是将来相当长一段时间内临床工作的重点。过去那种以控制症状、改善病残、延续生命为目的的治疗水平,已远远不能适应当今社会发展和人们对健康的需求。目前,以早期发现、早期治疗、综合干预、改善生命质量为核心的新型治疗模式,已经受到普遍关注和重视。精神药理学的进步,促使疗效更好、不良反应更少的新型精神药物不断出现,加上生物心理社会医学模式的广泛应用,将改写疾病的治疗指南,不但能提高疾病的治疗效果,更重要的是能提高患者的生活质量和社会功能,最终实现改善预后、降低社会精神疾病的总负担,从而产生巨大的社会效益,也使精神科的服务水平有一个较大的提高。

2001年WHO报告的主题是“精神卫生:新的认识、新的希望”,希望提高社会对精神卫生的重要性和精神障碍所致负担的认识,使人们正确了解精神障碍对人类、社会及经济的影响,消除对精神障碍的偏见和歧视。目前我国政府已把精神卫生事业纳入公共卫生,加大了政府对精神卫生事业的投入,并且得到政府和社会越来越多的重视,不但在医院硬件上的投入,而且在人才培养和科研开展上均得到大力支持。开展新型的医疗模式,进行社区的精神病的管理、预防、医疗、保健、康复和健康教育等为一体,使全社会都来关心精神卫生事业,这将为我国精神卫生事业开创新的发展机遇。
