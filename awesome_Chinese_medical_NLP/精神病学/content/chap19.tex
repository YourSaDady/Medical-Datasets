\chapter{心 理 治 疗}

\section{概  述}

\subsection{基本概念}

心理治疗(psychotherapy)是一组治疗方法的总称,在这种治疗中,由经过训练的专业人员根据自己的理论定向,与病人建立某种特殊职业关系,向病人灌输某种理论和生活态度,同时采用某些特殊心理学技术或程序,帮助病人解决某些心理障碍,达到消除、减轻或防止症状,调节紊乱的认知和行为模式,促进正性人格成长或发展。

这个定义包含六个要素:①心理治疗是一组治疗,据不完全统计,目前有各种治疗技术达到800余种;②心理治疗的主体是由接受专门训练的专业人员,包括临床心理学家、精神科医师、心理治疗师、心理咨询师、婚姻和家庭治疗师、精神科护士和临床社会工作者;③采用治疗手段是心理学的,有心理学的理论基础,而且必须是临床实验验证有效的技术;④心理治疗的客体是符合某个诊断系统的病人,处理的是病人的心理问题或由心理因素所致的躯体问题;⑤心理治疗只能在某种特殊职业关系基础上才能生效,这是一种新型的、特殊的、亲密的人际关系;⑥心理治疗的目标有不同的层次,不同学派的目标有所不同,近期目标是消除或减轻症状或防止症状进一步发展,中期目标是调节紊乱的认知和行为模式,最终目标是促进正性人格成长或发展。

\subsection{发展简史}

非正式心理治疗或许在几个世纪前就已存在,我国的中医就隐含许多心理学思想和技术。早在9世纪,Baghdad精神病院Rhazes就发展了以理论为基础的目的定向的心理治疗,然而西方国家直到18世纪还是把精神疾病看作魔鬼附体或躯体疾病,实施驱魔治疗或物理治疗。18世纪,奥地利医生Mesmer开始使用催眠术,后来被法国神经病学家Charcot用于治疗心理障碍。

19世纪末,奥地利医生Freud发现许多心理障碍患者并没有躯体或脑器质性改变,认,为可能与童年期创伤经历和潜意识冲突有关,发展了自由联想、移情解释、梦的分析和心理结构分析等心理治疗技术,创建了精神分析学派,这是第一个科学的心理治疗技术。后来,Freud弟子对他的基本理论加以发展,并提出一些新概念和理论,形成了各自的心理治疗系统,这些治疗系统被统称为精神动力治疗。

在20世纪20年代,出现了行为主义,并成为以后30年的主要流派,他们基于经典条件反射、操作条件反射和社会学习理论创建了许多行为矫正技术,对某些情绪和行为障碍取得较好的效果,J.
Wolpe、H. Eysenck和BF. Skinner等是该学派的主要代表人物。

20世纪50年代,在欧洲出现了存在-人本主义治疗。存在主义哲学关注个人能力发展,强调人生的目的和意义,该学派主要代表人物试图发展一些对生命危机有效的治疗方法,最有影响的治疗体系当属C.
Rogers创建的当事人中心治疗,其他还有Fritz和Perls创建的格式塔治疗、Rosenberg创建的非暴力交流和Berne创建的交流分析。这些人本主义治疗的共同特点是通过支持、真诚和通情性治疗关系促进个体正性整体改变,体现人生意义和自我价值,达到自我实现。

20世纪50年代,在认知主义思潮的影响下,美国心理学家A.
Ellis创建了理性情绪治疗(1995年改称为理性情绪行为治疗),随后AT.
Beck创建了认知治疗,在80年代认知治疗和行为治疗联合产生了认知行为治疗。这种治疗不同于精神动力治疗和人本主义治疗,它们是相对短程的、结构式的和当前定向的,采用积极指导性策略建立协作式的治疗关系,评估和矫正来访者的不合理信念和功能失调。认知和行为治疗“第三浪潮”发展的心理治疗技术还包括接受和许诺治疗、辩证行为治疗。

此外,近30年还出现了叙述治疗和依从治疗等后现代心理治疗、系统治疗、人际治疗、女权治疗、短程治疗和表达治疗等。

\subsection{主要体系}

心理治疗虽然很多,历史上也出现过许多治疗流派,但目前许多治疗流派趋向于整合,目前国内临床上开展的心理治疗可以大致归为五个主要体系。

1.精神动力治疗 精神动力治疗的具体方法很多,但都起源于精神分析。尽管各种动力治疗的理论基础不尽相同,但都认为潜意识心理冲突和童年期创伤体验是心理障碍的根本原因,不过20世纪80年代以后发展起来的动力治疗基本是短程的,以缓解或消除症状为主要目的,不再过度强调人格改变。在技术上,除引用Freud的传统技术外,不同理论定向的动力治疗都有自己独特的技术,同时也采用其他学派的技术。

2.认知行为治疗 认知行为治疗是一组以行为主义的学习理论和认知心理学的有关理论为基础,通过目标定向的系统程序,以改变认知、情绪和行为功能障碍为目标的心理治疗方法。该治疗有几个基本假设:①人的情绪和行为是受认知调节的,不合理的信念、错误的思维和非真实的评价是心理障碍的根本原因;②不合理的信念、错误的思维和非真实的评价是通过不良学习获得的,而且被病理的情绪和不断强化;③不合理的信念、错误的思维和非真实的评价通过再学习、去学习和现实检验是可以改变的。在治疗上以识别和矫正不合理信念的认知技术为主,同时也采用情绪和行为技术。

3.人本主义治疗 代表性治疗技术是Rogers的当事人中心治疗,认为每个人都有自我现实的潜能和动机,治疗目的就是创造一个有利于来访者发展的人文环境,否认决定论、强调主观意义和正性成长,通过无条件接受、积极关怀和设身处地理解等技术构建和谐的治疗关系,激发来访者的内在潜能,达成自我实现的目标。

4.系统治疗 临床上实施的家庭治疗和婚姻咨询属于系统治疗,它是以系统为理论基础的治疗方法,认为人不能独善其身,与周围环境和人有千丝万缕的联系,也就是说人是在特定的关系中成长的。人的心理问题也是个体与特定关系系统相互作用的结果,要治疗个体的心理问题,首先必须改变系统的关系模式和动力学特征。

5.综合治疗 综合治疗是目前心理治疗的发展趋势,许多资深的心理治疗师都有自己的综合治疗方法。综合治疗不是简单的技术杂合,而是不同学派理论和技术的融合和发展所产生的新型治疗系统。

\subsection{基本技术}

心理治疗技术的内容很多,每种治疗都有自己专门的技术,同时也有一些共同的基本技术。每种治疗的专门技术在相应的治疗中介绍,这里只介绍一些基本技术。

1.建立有效的治疗关系 心理治疗关系是一种不批评、不包办代替、不偏倚的新型人际关系;无条件接受、信任、尊重、理解、投情的亲密的人际关系;促进来访者自我理解、增进自信自尊和独立自主、增强责任感,有利于潜能发挥的建设性人际关系。要发展有效的治疗关系,治疗者必须做到:无条件地接受来访者;积极地关心和尊重来访者;设身处地地理解来访者;专注和投情地听来访者的倾诉;准确地表达来访者的内心感受。

2.询问和观察 询问和观察是获取信息和澄清问题的重要手段,同时询问还可以促使来访者自我检查和自我理解。在探索问题时用开放式提问,在澄清问题时用封闭式提问,所有询问要保持中立态度,不带任何批评和指责语气,一切以弄清问题、促进来访者自我理解为宗旨,观察要全面,不能有任何偏见。询问和观察的内容很广,重点观察来访者的情绪和行为反应、谈话方式和行为方式,询问问题的发生发展过程、内心感受、采用的应对策略和对问题成因的解释。

3.用“心”倾听 倾听是治疗者的基本功和获得信息的重要途径,无论在诊断性或治疗性会谈中都是重要的,耐心的倾听能增强来访者的信心,激发来访者的求助动机,投情性倾听有助于建立有效的咨询关系,重点倾听来访者的人生故事、内心感受、心声和期望以及谈话的弦外之音。

4.接受和理解 每个人都希望被别人接受和理解,接受和理解不仅能提高来访者的自信、自尊和价值感,而且能激发来访者的内在潜能。要使来访者感到被接受和理解,治疗者要做到无条件地接受、积极地关怀、设身处地地理解和通情性表达。

5.构建工作假说 在治疗开始前,治疗构建对来访者问题性质和前因后果的工作假说,在治疗过程中可以不断修正,在治疗结束前,治疗者必须明确如下问题:来访者的主要问题是什么,来访者的问题是怎样产生的,使问题持续存在的因素是什么,来访者自己做过哪些努力,问题与来访者的人格有什么关系?

6.促进自我理解 心理治疗目的是帮助来访者自我理解,在治疗过程中,治疗者要帮助来访者做到如下几点:了解自己的需要,在条件允许的情况去满足他;了解自己的优点,树立自信,做力所能及的事;了解自己的人格特点,做自己喜欢做的事;了解问题产生的根源,在现实生活中去克服它。

\subsection{基本过程}

1.初次晤谈 初次晤谈对来访者和治疗者而言都是关键性的,初次晤谈不仅是收集资料,更重要的是引导来访者进入治疗过程,建立起相互依存的治疗关系,要让来访者知道你有能力和愿意帮助他,使来访者知道谈自己的内心感受是适当的和重要的,要让来访者知道治疗者只能帮助来访者解决问题,不能代替来访者解决问题。在初次晤谈中,治疗者的主要工作包括:①建立治疗关系,让来访者放心谈话;②探索问题的性质和引发事件;③了解来访者的期望和治疗动机;④作出处理决定和建立病历档案。

2.评估与诊断 将晤谈、测验和观察所获得的资料加以分析、整理和归纳,对来访者的人格特征、主要问题、接受心理治疗的可能性提出一个整体看法,这便是诊断工作,如:是否面临紧急状况,是否存在重性精神病,自我强度如何,有无改变的动机,是否适合做心理治疗等。

3.解决问题 ①使来访者了解问题的根源,如用技巧性提问使来访者了解问题的前因后果,检验自己的反应模式和内心感受;通过假设和比喻来加深来访者对问题的认识;通过现实检验分清期盼、想象、幻想和现实;用“面质”技术让来访者直接面对事实,进而了解自己的行为后果;用解释来联结来访者未能觉察到两组现象之间的关系。②将过去与现在加以联结,如初次晤谈中,要仔细评估来访者的生活经历;从情绪发展角度评估来访者困惑内容和行为模式;从家庭和社会文化背景考察来访者的早期经历;从过去的生活来看来访者目前的问题;找出过去经历与目前问题的联系,并解释给来访者听。③采用支持、鼓励和教育等技术提升来访者的自尊、自信和自立。④采用接受、探索和解释等技术克服阻抗。⑤处理治疗过程中的一些特殊问题。

4.结束治疗 ①结束的时机和标准:症状消失主要问题或冲突已解决,能独立地处理问题,能认同自己和享受生活。②处理结束期反应:失望、愤怒、失落感、被遗弃感、分离焦虑。③再见和随访。


\section{精 神 分 析}

\subsection{基本概念}

1.精神分析(psychoanalysis) Freud的精神分析学说既是治疗神经症和研究人类心理奥秘的方法,也是解释人类心理功能的重要理论。作为一种治疗方法有其确定的规范,主要采用自由联想技术探索潜意识冲突,用解释(尤其是移情关系的解释)来诱发病人对目前症状与遗忘的潜意识冲突之间关系的理解(领悟),以达到消除症状,促进人格更加成熟。精神分析治疗有四个基本假设:过去的心理创伤,尤其是童年期创伤对目前的心理功能有决定性影响;症状是被压抑在潜意识中的欲望和冲动,通过异常渠道释放的后果;不成熟的防御机制,对现实体验的错误解释和预期性创伤使症状持续存在;暴露潜意识冲突,使之意识化,即可消除症状,通过移情关系处理可使人格更趋成熟。

2.潜意识 Freud把人的心理活动分为三个层次:最上层为意识,指人在清醒状态时对自己的思维、情感和行为的目的和动机能自我觉察到的心理活动领域,即当前心理活动的焦点;中层为潜意识,乃指其心理活动虽不为自己察觉,但经集中注意或努力回忆即可明了其活动的目的和动机;最下层为潜意识,潜意识的心理活动则深深埋藏在不能察觉的境界里,虽经人提醒或自己努力回想,也无法明了活动的目的和动机,只有在特殊情况下,如幻想、梦境、催眠状态或用自由联想时,才会呈现于意识中。潜意识与潜抑有密切关系,其心理活动原本可以意识到,或曾经意识到过,但因所发生的事情过分尴尬、痛苦或难为情,而把部分心理体验,潜抑到意识不到的境界里,并没有消失,仍在影响个体的行为、思维和情感,如遗忘、口误、笔误、神经症症状等都是潜意识心理活动的结果。

3.本能 本能是动物(包括人类)内在的强大的动力或称内驱力,Freud认为人类有两种本能:生存本能和死亡本能。每种本能具有以下特征:①本能代表着基本的生理需要,来源于体内的兴奋;②本能有一种推动力,决定人的行为;③本能的目的是要求满足,这种满足可通过发泄增强了的兴奋而得到;④本能有求得满足的对象。按照Freud的观点,人的思想、情感和行为不可能是“偶然发生的”或“自由意志决定的”,而是由非理性动力、潜意识动机、生物本能和童年早期的某些心理事件所决定的。即使是那些好像是偶然的行为(口误、笔误和遗忘)都有意识上不知道的动机。同样,根据这一原则,神经症的每一症状都有它的意义和起因。不论症状看来多么荒谬,病人本人也不能理解,实际上都有一种潜意识的动机和欲望在起作用。这些意识不到的动机和欲望,可用自由联想法逐渐暴露出来,使病人意识到它们,症状会因此失去存在的意义而消失。

4.人格结构 Freud提出人格由原我、自我和超我三个部分组成。原我是人格中最原始、最模糊而不易把握的部分,运作三个人格系统的心理能量库,因此原我仅能通过反射和欲望达成获得满足;自我是人格的执行机构,介于原我、超我和外界之间的复杂心理组织,具有防御和自治的双重功能,通过选择能降低紧张,带来愉快的适当客体和时机,合理地满足原我的冲动,最终控制原我的大多数心理能量;超我作为人格中的道德机构,用良心阻止能量的异常释放,或通过自我理念引导能量释放。一个原我占优势的人可能有冲动倾向,超我占优势的人可能过分道德化或完美化,自我使个体避免两个极端。一个健康的人,原我、自我或超我必须均衡发展。

\subsection{发展简史}

1880年J.
Breuer在治疗一个癔病少女时发现,当她在催眠状态下,说出与症状有关的痛苦经历后,症状就暂时缓解或消除,称之为谈话疗法。1882年Breuer把这种情况告诉了Freud,给他留下很深的印象,后来在临床实践中试用这种疗法。1890年,Freud与Breuer合作研究癔病,于1895年在《癔病之研究》一书中发表了他们的观察结果。他们认为癔病是创伤性经历被压抑的结果,与经历有关的心理能量被阻滞,不能达到意识层面,而转向异常渠道发泄,以致出现种种症状;在催眠状态下,这些能量从正常渠道发泄出去,症状因此消失,称之为疏泄法。后来在治疗病人露西时,病人拒绝接受催眠,只好向病人提出一些引导性问题,鼓励她回忆有关经历,结果渐渐回忆起创伤性经历,这启发Freud改进治疗方法。1896年后,Freud逐渐放弃催眠疗法,一度采用精神集中法(或称前额法)。在使用这种方法时,又发现有的病人不说出真实的思想,同时按压前额和不断地提问,干扰了病人的思路,遂改用自由联想,加之1901年梦解释技术的发现,使精神分析治疗趋于完善,即弗氏经典精神分析。

由于精神分析技术的突破,精神分析的理论也达到空前发展和完善。1902年开始,维也纳的一些年轻医师对精神分析疗法产生兴趣,聚集在Freud周围,自然形成了“周三晚会”,讨论问题,交换意见。他们中有Adler、Federn、Rank、Rie等。1907年后,各国对精神分析感兴趣的人,如Jung、Eitingon、Binswanger、Abraham、Brill和Jones等陆续到维也纳,参加讨论,以后相关学会相继成立,学术期刊相继创刊,使精神分析学说得到广泛传播。与此同时,精神分析学派内部出现了意见分歧。首先是Adler对婴儿性欲发难,认为儿童心理发展的主要动力是自主冲动,而不是性冲动,于1912年脱离Freud学派,自创个人心理学。自1911年起,Jung也逐渐对Freud观点持不同意见,否认婴儿性欲,认为Libido是普遍的生命能量,在神经症发病方面强调当前激发因素的作用,于1913年正式与Freud学派决裂,自称分析心理学。这些理论观点分歧,也使治疗技术发生相应的改变,对儿童期经历较少强调。

\subsection{主要体系和治疗目标}

1.治疗体系 精神动力定向的治疗方法很多,虽然治疗的最终目的是使人格产生建设性改变,但达到这一目的理论假设和具有方法不完全相同。这些方法大概可分为四类:弗洛伊德精神分析、自我分析、新弗洛伊德式精神分析和分析定向的心理治疗。弗洛伊德精神分析或多或少采用弗洛伊德的基本观点和技术。自我分析保持经典的治疗形式,重点分析自我的适应功能。新弗洛伊德式精神分析包括Horney、Sullivan、Rank、Jung和Adler的方法,采用修改的、更积极的治疗技术。分析定向的心理治疗是精神动力定向治疗中最积极的治疗方法。除上述四种主要类型外,尚有克氏精神分析和交流分析。

2.治疗目的 各种动力性心理治疗的近期目标不完全相同,但它们的最终目标却是相似的。如:降低不合理冲动和欲望的压力,以便使它们能被控制;发展成熟的防御机制,并灵活使用这些防御机制;放松良心的苛刻性,改变价值系统,使病人能适应现实和内在需要。这些目的是令人望而生畏的,因为人格的各成分已逐渐形成一个制约系统,几乎不受外界的影响。

\subsection{特别治疗技术}

精神分析治疗技术是使潜意识材料转换到意识层面,对病人的行为获得理性洞悉,理解症状的意义,通过分析、解释,使病人领悟症状起因和意义,以达到消除症状和人格的改变。主要有五种技术。

1.自由联想 精神分析的主要技术是自由联想,病人靠在躺椅上,分析者坐在他的后面,尽可能避免干扰。要求病人把头脑里浮现的任何感情、思维、记忆不经选择地立即报告出来,不管它们是怎样地痛苦、可笑、不合逻辑或没有意义,也就是说无偏见、无选择地说出想到的一切事情。自由联想是分析者探索潜意识欲望、幻想、冲动和动机的钥匙,这一技术常导致过去经验的重视和被阻抑的强烈情感的释放。分析者借此技术收集潜意识资料,分析理解这些资料间及与症状的关系,然后向病人解释这些资料,指导他们逐渐领悟到原本未意识到的基本动力学。

2.解释 解释是暴露潜意识冲突和干预的重要手段,向病人指出潜意识欲望,说明梦、自由联想、阻抗和治疗性关系中呈现行为的意义,使他对自己一直没有理解的心理事件变为可理解的,把表面看来似乎没有意义的心理事件与可以理解的事件联系起来,以帮助病人对自己的领悟。解释的功用是允许自我吸收新资料,加速暴露潜意识资料的过程。解释必须遵循以下三条规则:①被解释的现象已接近意识层面时,病人还不能看清,但能接受和理解它们,应机智地作出解释;②解释应由表及里、步步深入;③在解释情感或冲突时,最好同时指出阻抗或防御。解释有时要求反复进行多次,才能使来访真正理解。这种用同样内容一而再地对付病人自我的过程,称之为扩通。

3.梦的分析 释梦是发现潜意识材料,使病人领悟未解决问题的主要方法。在治疗中,病人报告他的梦,分析者鼓励他自由地联想显梦中的某些要素,回忆其激发的情感,共同研究显梦的象征意义,发掘梦的隐意,帮助病人理解过去的人际关系模式,梦与目前行为的关系。这种由显梦的分析和自由联想求得梦的隐意的过程,就是释梦。梦是病人所遇困难的反映,释梦是使病人获得对目前行为的意识。

4.阻抗的解析 阻抗是精神分析实践的基本概念,它影响治疗进程,使病人可以不遵守自由联想的规则拒绝把潜意识材料带到意识表面。在分析治疗中,可以发现病人不愿把某些思维、情感和经验报告出来,想方设法借口逃避。因为他们意识到这种被压抑的冲动和情感,就可能产生焦虑。阻抗出现意味着分析已触及病根之所在。另外,阻抗从另一角度为我们提供许多以往生活的重要材料,所以只要分析者的技术精巧,便可将之转化为莫大助益。分析解释它们,克服这些阻抗,乃是分析的基本工作。

5.移情的解析 在经典弗氏分析中,认为病人最终发展为移情性神经症,分析者不仅不加制止,反而鼓励和助长其产生。在移情中,病人表达了被压抑在潜意识中的感情冲突,体验了被压抑的情感,通过统合,适当解释和早年情感修通的过程,病人能改变某些长期稳定的行为模式,重建新的行为模式。操纵这种移情是精神分析的关键技术,分析者需要告诉病人,他的情感不是起源于目前的情境,也和分析者本人无关,只是重演其以往经验而已。请他将这种重演作为回忆的起点,可由此揭示出内心的隐事,如处理成功,不论是爱或恨的感情都可变成治疗的助力。指出它目前的感情仍固着于童年期,这种固着阻碍他的感情成长,使病人领悟到过去情感对目前功能的影响,通过修通治疗关系中类似的情感冲突,就能消除早年关系的影响。

\subsection{适应证和禁忌证}

1.适应证 弗氏认为精神分析主要适用于转换性癔症、强迫性神经症和自恋性神经症,后人研究表明对其他神经症、物质滥用、抑郁症和早期精神分裂症也有一定效果。如果来访者的心理冲突由来已久,并且已深深嵌入其人格中,精神分析可能比精神分析心理治疗更适合,因为其具有很深的分析度。同时要求病人有心理学头脑、人格相对完整和有适当的家庭和社会背景。

2.禁忌证 精神分析不适合那些患严重抑郁症、精神分裂症、人格障碍和器质精神病患者,对攻击性或冲动性障碍、偏执状态和急性危机等均不合适。


\section{行 为 治 疗}

\subsection{定义}

行为治疗(behavioral
therapy)或行为矫正的基本理念是:对具体的、可以观察的、不良的、不适应的或自伤行为,可以通过学习新的、更为合适的行为来加以替代。

\subsection{发展过程}

从孩子抚养到罪犯改造,奖励和惩罚一直被用来影响和纠正个体行为。现代行为治疗起源于20世纪50年代Skinner
B. F.和Joseph
Wolpe的著作。Wolpe用其发展起来的系统脱敏技术来治疗恐惧症来访者。系统脱敏法中来访者被逐级暴露在一个焦虑情境中,直到焦虑反应被消除为止。

Skinner使用了一种他称之为操作性条件反射的行为技术。操作性条件反射的理念是个体在以往行为结果经验的基础上选择自己的行为。如果一个行为与既往积极强化或奖励有关,个体就会选择这种行为而回避惩罚性行为。

到20世纪70年代,行为治疗作为一种治疗方法得到广泛使用。在过去20年中,行为治疗师的注意已转向来访者的认知过程,许多治疗师开始用认知行为治疗中更积极的思维方式替代消极思维,从而来改变来访者的不健康行为。

\subsection{基本治疗技术}

行为治疗或行为矫正的基本假设是情绪问题与任何行为问题一样是一些习得的对环境的反应,因此可以被消除。与心理动力治疗不同,他们不对不良行为背后的潜意识动机加以探讨或分析。换句话说,行为治疗师不对来访者这样做的原因进行探讨,只教他们如何改变行为。

行为治疗最初几次访谈主要用来解释有关行为治疗的基本原则及建立起治疗师与来访者之间积极的工作关系。行为治疗是一种合作性和以行为为中心的治疗方式。来访者在治疗过程中起着积极的作用,不允许像其他治疗过程中那样对治疗师较为依赖。国外,行为治疗师由心理治疗师、临床社会工作者或精神科医生担任,他们需要接受行为治疗专业训练。与其他心理治疗相比,行为治疗疗程较短,一般平均不超过16周。

行为治疗有许多不同技术帮助来访者改变行为。这些技术包括:

1.行为家庭作业布置 治疗师通常要求来访者在两次访谈之间完成家庭作业。这些作业包括现实生活中的一些行为实验。实验过程中,来访者被鼓励对新情境做出新反应。

2.意外事件协议治疗 师对治疗目标列出一份书面或口头协议。这个协议包括对合适行为积极强化(奖励)或对不良行为消极强化(惩罚)。

3.示范 来访者通过观察治疗师的行为学习一种新行为。

4.行为演习 治疗师和来访者通过角色扮演对相应情境做出合适行为反应。

5.技能训练技术 来访者通过一个教育计划系统学习社交、扶养子女或其他相关生活技能。

6.强化 治疗师使用鼓励强化来访者某个具体行为。例如,一个多动症孩子注意力集中并完成某项布置任务后,每次都能得到一个红五星,这个红五星能强化和增强某种目标行为。强化同样可以通过阴性强化消退某种问题行为。

7.系统脱敏 来访者想象某个他们害怕的情境,治疗师使用相关技术帮助来访者放松,应对他们感到恐惧的情境,最后消除焦虑。例如,对一个治疗中的广场恐惧症患者,治疗师指导他放松,然后让他(或她)想象自己走在自家旁边一条人行道上。下一次访谈中,他(或她)再次放松,然后想象自己到超市去购物。这种想象的焦虑随诱发情境逐渐增加,最终治疗师和来访者一起到某个导致其焦虑的真实情境中,如去超市购物。暴露可以达到“彻底暴露”程度,即彻底暴露在一个真实情境中的目标,通过放松反应来应对一种让人焦虑的情境,来访者逐渐变得对原来的恐惧反应不敏感,学会用放松去作出反应。

8.暴露 暴露疗法是系统脱敏的一种快速形式。在治疗过程中,来访者被直接暴露于让来访者感到最紧张的情境中,或通过想象,或真实生活情境,以消除其恐惧反应。

9.渐进式放松 顾名思义,渐进式放松包括全身肌肉的全部放松,甚至呼吸,直到身体紧张全部释放为止。这被行为治疗师作为放松练习,用于消除焦虑和压力,或为系统脱敏做准备。渐进式放松的具体方法是先紧张,后根据全身不同肌肉群逐个放松。治疗师可以建议来访者使用事先录好的放松练习录像带在家练习。

行为治疗将传统认知重建整合进行了行为校正,形成了介于行为治疗和认知治疗之间的认知行为治疗。在认知行为治疗中,治疗师与来访者鉴别出引起问题的思维,使用相应认知行为治疗技术改变问题行为。

\subsection{治疗准备}

来访者可以独立寻找治疗,或由内科医生、心理治疗师及精神科医生转诊来。因为来访者与治疗师为达到具体治疗目标需要密切合作,所以他们的工作关系必须让人感到舒服,目标必须一致。开展治疗前,来访者与治疗师必须先见一次面。通过这次见面,治疗师可以对来访者做一个初步评估,同时也为来访者提供机会了解治疗师治疗有关情况,如治疗特点、职业资格证书及其他相关问题。

有些治疗场合,在来访者与治疗师见面前需要安排一次面谈。在国外,这种面谈通常由精神科护士、咨询员或社会工作者通过面谈或电话来完成。

\subsection{治疗研究}

行为矫正技术用于治疗各种精神卫生问题早已广泛见诸医学文献。行为治疗是否应视为某些精神疾病的一线治疗方式,在业内尚存不少争议。但毫无疑问,行为治疗技术是帮助来访者改变问题行为的一种强有力的治疗方法。

\subsection{培训及资格认证}

在国外,行为治疗师通常是由心理治疗师、临床社会工作者及精神科医生担任。其他卫生工作者可以做简单行为干预,但要求更高的行为治疗应优先让位于那些受过系统行为治疗培训的治疗师来完成。

\subsection{适应证和禁忌证}

1.适应证 行为治疗是一种对许多精神疾病或不良行为有益的治疗方法,如物质滥用、攻击性行为、愤怒处理、摄食障碍、恐惧症和焦虑障碍。另外,还被用于治疗尿失禁或失眠等。

认知行为治疗是认知治疗的一个分支,主要通过改变错误思维方式来改变不良行为。它可用作许多精神障碍的备用治疗选择,如情感障碍、人格障碍、社交恐惧症、精神分裂症、强迫症(OCD)、广场恐惧症、创伤后应激性障碍(PTSD)、阿尔茨海默病、注意力缺陷与多动症(ADHD)。另外,它还被用于治疗患风湿性关节炎及肿瘤等引起的慢性疼痛。行为治疗有时还可与药物治疗合并使用。可以根据来访者的具体情况及行为问题严重程度灵活使用。

2.禁忌证 行为治疗可能不适合有些来访者。那些没有具体行为问题但希望找人倾诉的来访者,其治疗目标只是想对过去经历过的事有所顿悟,因此最好接受心理动力治疗。在行为治疗中,来访者必须积极参与治疗。

行为治疗也不适合于某些认知功能严重受损的来访者,如有脑外伤和器质性脑病变者。


\section{人本主义治疗}

\subsection{定义}

人本主义治疗(humanistic
therapy)是一种强调人的唯一性及有能力控制自己命运的理论和治疗方法。人本主义的典型治疗方法是Rogers提出的来访者中心疗法。来访者中心疗法是一种将治疗过程的责任放在来访者身上,治疗师只起到非指导作用的治疗方法。

\subsection{发展过程}

人本主义心理学产生于20世纪60年代,当时是作为对心理动力学及行为主义的反叛而崛起的。人本主义者反对心理动力学有关人对快乐自私追求是其行为根源的悲观主义论点。他们还认为,行为主义有关人的行为取决于环境影响,将人看成机器的观念并不能充分解释人类行为。

人本主义纠正了心理动力心理学家及行为主义者把人的行为看作受个人控制以外的因素主宰的观点,强调人的内在潜能,人有决定自己命运的能力。人本主义心理学的最终目的是帮助人发挥自己的潜能,实践自己的能力。人本主义心理治疗有两个理论取向。

第一个理论是Rogers倡导的“来访者中心疗法”。其基本理念是,相信人的经验,并且认为自我是自我实现的最重要因素。在来访者中心疗法中,变态行为被看作是个体不相信经验的结果,导致对自我歪曲和错误的看法,现实的“我”和理想的“我”之间存在一种不协调。来访者中心治疗师试图通过传递共情、温暖和无条件关注帮助来访者认识到,不管来访者说什么做什么,始终是一个有价值的人,从而让他们获得自我理解和自我接纳。

Rogers认为心理治疗应放在一个来访者和治疗师密切关系建立起来的支持性环境中。Rogers所讲的“来访者”不是“患者”。传统意义上医生和患者之间的关系带有强烈的医生优越感和权威性,充满了对患者的藐视和拒绝。这样会破坏治疗师与来访者之间的平等关系。在来访者中心疗法中,来访者决定治疗的总体方向,而治疗师只是通过非正式的提问来帮助来访者提高其对问题的顿悟和对自我的了解。

人本主义第二个影响较大的理论是Abraham
Maslow的理论。Maslow认为人的天性是善良的,并且天生有一种自我实现的潜能。他还认为,人的需要发展呈金字塔状,从低级到高级逐个实现,最后达到自我实现。首先是满足生理和安全需要,然后满足归属需要,继而是人自尊的需要,最后才是自我实现的需要。Maslow认为心理问题源于自尊需要的满足困难,因此影响自我实现。治疗的目标是纠正人对自我的错误看法,提高自尊,并使他们继续朝自我实现方向发展。

\subsection{对治疗师的要求}

Rogers认为成功治疗的关键不是治疗师的技能及所受训练,而是治疗师的态度。治疗师的三种态度,即一致性、无条件积极关注及共情,对来访者中心疗法的成功至关重要。

1.一致性 指的是治疗师的开放性和坦诚,愿意抛弃治疗师的职业面具,坦诚地与来访者进行沟通。这样做的治疗师在治疗过程中能获得来访者所有的体验,并与来访者进行最大限度的分享。但是一致性并不意味着治疗师将自己的个人隐私告诉来访者,或将治疗的重点通过任何形式转移到治疗师自己身上。

2.无条件积极关注 指的是治疗师无条件、全方位接受来访者的思想、情绪、行为和个性特点等。治疗师在与来访者沟通时通常使用的方法是耐心倾听,不随便打断、下判断及提出建议。这样做的目的是为来访者创造一个安全气氛,让其畅所欲言,自由地探索和分享个人痛苦、敌意、防御性或变态思想和情绪,而不用担心会遭到治疗师的拒绝。

3.共情 即精确的、共情的理解。治疗师需要站在来访者的角度努力理解来访者的想法和情绪,治疗过程中对来访者表达的一切表现出敏感和理解。在其他心理治疗过程中,共情是开展治疗的先决条件。但在来访者中心疗法中,实际上共情就是治疗过程的一个主要组成部分。传递共情的主要方法是对来访者所说内容进行细致入微和准确地积极倾听。另外,来访者中心疗法中治疗师还使用一种叫反映的技术,主要是必要时,治疗师用自己的话对来访者所说的内容加以解释或概括。这种技术表明治疗师在细心准确倾听,并且通过另外一个人的复述给来访者机会,检查其思想和情绪。总之,在治疗师传递共情过程中,来访者会将其表达的内容逐渐细化和精确化。

\subsection{治疗目标}

来访者中心疗法的两个主要目标是提高自尊和经验开放性。在治疗过程中,治疗师应努力培养来访者作相应改变,包括理想的“我”和现实的“我”之间的一致性,更好地理解自我,减少防御性、罪恶感及不安全感,与他人建立更为积极和舒适的关系,提高现实情境下体验和表达自我情绪的能力。20世纪60年代初,来访者中心疗法与人类潜能运动结为联盟。Rogers采用了如“个人中心疗法”和“存在方式”等术语,并开始关注个人成长和自我实现,他甚至创造性地使用了邂逅小组(encounter
group),采用了Kurt Lewin等人首创的敏感性训练。

\subsection{治疗影响}

尽管来访者中心疗法与精神分析、认知治疗和行为治疗一样被看作是主要心理治疗流派之一。但Rogers的基本治疗理念对其他学派的影响甚至大于对来访者中心疗法本身的影响。他提出的许多理念和方法早已在世界范围内被不同流派的心理咨询师和治疗师以一种折中的观点整合进各自的咨询和治疗中。


\section{认知行为治疗}

\subsection{基本概念}

认知行为治疗(cognitive behavioral therapy,
CBT)是一组以行为主义的学习理论和认知心理学的有关理论为基础,通过目标定向的系统程序,以改变认知、情绪和行为功能障碍为目标的心理治疗方法。

上述定义包含三个要素:基本假设、治疗方法和治疗目标。CBT是以学习理论和认知理论为基础的,它有几个基本假设:①人的情绪和行为是受认知调节的,不合理的信念、错误的思维和非真实的评价是心理障碍的根本原因;②不合理的信念、错误的思维和非真实的评价是通过不良学习获得的,而且被病理的情绪不断强化;③不合理的信念、错误的思维和非真实的评价通过再学习、去学习和现实检验是可以改变的。CBT是行为治疗和认知治疗联姻的产物,是一组治疗方法的总称,包括认知治疗、理性情绪行为治疗和多通道治疗等,在具体技术上是综合性的,每种方法都可能涉及认知技术、行为技术和其他技术,这些技术适用于个别治疗、团体治疗和自助。CBT是限时和目标定向的,强调此时此地,目标在于用合理信念代替不合理信念、消除情绪和行为症状。许多研究证实CBT对某些心理障碍是有效的,包括焦虑障碍、心境障碍、人格障碍、进食障碍、物质滥用和某些精神病性障碍。

\subsection{发展简史}

CBT起源于20世纪早期的行为治疗和20世纪60年代的认知治疗,于20世纪80~90年代两种治疗在理论和技术上真正走向融合,形成目前的认知行为治疗。行为治疗始于1924年MC.Jones的工作,他用去学习技术成功地治愈了一名儿童恐惧症,然而它真正的发展还是20世纪50~70年代的事,在这段期间,英、美和南非的一些研究者受I.
Pavlov、JB. Watson和CL.
Hull等行为主义学习理论的影响,创建了一系列行为治疗技术。在英国,J.Wolpe把动物实验结果用于临床,创建了系统脱敏治疗,H.
Eysenck受K.
Popper著作的影响转向行为治疗,他们的主要治疗对象是神经症性障碍。在美国,许多心理学家把BF.
Skinner的操作条件反射理论应用于临床,设计了一些行为治疗技术,他们的主要治疗对象是慢性精神病性障碍,如慢性精神病的退缩行为和儿童孤独症。

虽然早期的一些行为治疗技术在神经症性障碍的治疗中获得成功,但在抑郁症的治疗中收效甚微,又因“认知革命”的影响,行为主义渐渐失去了市场,而AT.
Beck和A.Ellis的认知治疗技术受到行为治疗师的青睐。Ellis的治疗系统始于1955年,当时称为理性治疗,以后相继用过语义治疗、理性行为训练和理性情绪治疗。Beck在Ellis的鼓励下,于20世纪60年代创建了认知治疗,在70年代后,认知治疗很快成为心理治疗的主要流派,在80年代认知治疗和行为治疗联合产生了认知行为治疗,与此同时,AA.
Lazarus创建了更广谱的认知行为治疗,1995年Ellis把他的治疗方法改称为理性情绪行为治疗。

\subsection{主要体系和治疗目标}

1.治疗体系 认知行为治疗由许多具体治疗方法或治疗体系组成,因为处于不断发展中,很难确定到底有多少种,最常见的治疗体系包括Beck的认知治疗、Ellis的理性情绪行为治疗、Lazarus的多通道治疗、Meichenbaum的认知行为矫正和Cormier的认知重建治疗。这些治疗的理论体系尽管不完全相同,但都坚持一个最基本的理论假说:人的行为受认知调节,认知偏差是心理障碍的根本原因,治疗的焦点是改变认知。

2.治疗目标 CBT治疗认为病人的情绪和行为问题是由不合理信念或适应不良的认知模式造成的,因此治疗的总体目标是改变适应不良的认知模式,以合理信念代替不合理信念,消除各种情绪和行为问题,最终使病人达到无条件的自我接受。具体目标有以下几个方面:

(1)消除一些外显症状,如情绪和行为问题,降低病人的焦虑(自我责备)和敌意(责备别人和环境)。

(2)通过理论教育和实践检验,使病人了解思维与情绪行为的关系,认识到适应不良认知是情绪和行为障碍的根源。

(3)教给病人一些自我观察和自我评价的方法,使他们能识别和改变不合理信念,以适应的思维、态度和行为对抗适应不良的、自我挫败的思维和行为。

(4)矫正适应不良的概念和功能失调的认知图式,减少病人的自我挫败行为,获得更现实的、高耐受的人生观。

(5)长期目标是帮助病人积极参与各种有意义的活动,治疗者的主要任务不是告诉病人什么能使他们幸福,而是帮助他们发现他们是怎样阻碍自己获得幸福的及他们怎样才能排除那些障碍。

\subsection{特别治疗技术}

CBT是一种相对灵活的治疗体系,对其他体系的治疗技术有很大的包容性,能针对不同心理障碍个体灵活地选用各种有效的治疗技术,这里只介绍CBT体系共有的特别治疗技术。

1.认知性家庭作业 认知性家庭作业是要求病人根据REBT原理确定激发事件(A)和目前的主要症状(C),然后去找A与C之间的不合理信念(B),并对不合理信念进行质疑式的辩论(D),感受信念改变的情绪和行为效果(E)。在辩论时,病人需回答如下具体问题:①在事件发生与情绪行为反应产生前,我想到了什么(信念)?②这些信念是否正确?别人在这种情况下会怎样想?③是什么东西使我得出这样的结论?若从别的角度去想又有什么结论?④这种信念产生的结果是什么,换种方式去想又会产生什么结果?

2.不合理信念的识别与抵制 在不合理信念识别中,治疗者详细倾听病人诉说病状的发生、发展过程,了解他们生活中发生过一些什么事情,他们对事件的看法是什么,是如何评价的,产生了什么样的情绪和行为反应。通过分析找出病人可能存在的不合理信念,在与病人对质的过程中,进一步确认不合理信念的存在,向病人指出这些不合理信念可能是心理问题产生的原因。在不合理信念抵制中,治疗者与病人先就不合理信念展开辩论,帮助病人认清其信念的不合理性,以后利用病人人格中的合理部分和不合理部分就不合理信念展开自我辩论,并在现实生活体验不合理信念和合理信念对情绪行为的影响,最终以合理信念代替不合理信念。

3.认知重建 认知重建就是教病人用正性的、自我强化的思维和行动代替负性的、自我挫败的认知。这种方法是教训式的和指导式的,治疗者必须积极投入,给病人示范适应应激情境的自我强化思维和行为。Cormier(1991)建议治疗者在重建病人的认知结构时,可采取六个步骤:①言语准备,包括方法的介绍、目的、采用的理由;②确认病人在问题情境中的思想;③应对思维的介绍和练习;④从自我挫败思维转向应对思维;⑤正性或自我强化评价的介绍和练习;⑥布置家庭作业和随访。通常,治疗者和病人协作通过这六个步骤,治疗者充当顾问、帮助者、良师和教练等角色,重要的是病人必须了解整个过程、想改变自己、遵守双方制订的行动计划。在用认知重建时,常同时联用其他技术,如放松训练、想象、示范、再加工、复述、应激预防和思维暂停等。

4.认知示范 认知示范是许多治疗家在实践中总结出来,在具体细节上不完全一致。它的目的是克服病人的自我挫败性思维和行为。具体做法分为以下几个步骤:①治疗者详细了解病人问题发生的情境,在那种情境病人脑子里出现过的信念、自我评价和态度,有什么样的情绪和行为反应。根据这些资料设计具有情境针对性的有效的思维和行为,并以此作为指导。②用想象再现当时的情境,治疗者示范以有效的方式面对问题情境,完成预定的任务。③病人在治疗者有声指导下,完成治疗者示范的任务。④病人用大声的自我指导完成同样的任务。⑤病人用耳语指导完成同样的任务。⑥最后病人用无声的内部语言指导完成预定的任务。

5.现实检验 布置一些家庭作业,让病人去面对一些原来不敢从事或回避的活动,通过实践来检验他的不现实看法或歪曲的认知。如作业布置恰当,就可以使病人不需依赖治疗家,自己去修正那些适应不良性行为,自己发现哪些假设是不正确的,以便发展新的、更适当的假设,发展一些新的、更有效的行为。为了使家庭作业达到预期效果,治疗家必须将任务交代清楚,阐明布置这些作业的道理,所有作业必须用书面形式,以便以后检查。由于家庭作业在治疗中如此重要,如果在作业中遇到困难时要及时讨论,如病人对作业不完全理解,不相信自己能完成这些作业,不相信作业真正对他有帮助,讨厌做家庭作业,担心做不好等。

\subsection{适应证和禁忌证}

1.适应证 从理论上讲,认知行为治疗可以用于治疗任何病态行为。这种治疗可以治疗许多精神障碍,如情感障碍、人格障碍、社交恐惧症、强迫症、摄食障碍、物质滥用、焦虑或惊恐发作障碍、广场恐惧症、创伤后应激性反应及注意力缺失和多动症。有关研究表明,它还可以用于对肿瘤、风湿性关节炎及其他慢性疼痛的辅助治疗。另外,它还能治疗失眠症。

2.禁忌证 认知行为治疗可能不适合有些来访者。那些没有具体行为问题但希望找人倾诉的来访者,其治疗目标只是想对过去经历过的事有所顿悟,因此最好接受心理动力治疗。在认知行为治疗中,来访者必须积极参与治疗。认知行为治疗也不适合于某些认知功能严重受损的来访者,如有脑外伤和器质性脑病变者。


\section{家庭治疗}

\subsection{定义}

家庭治疗(family
therapy)是一种将家庭成员纳入治疗过程的心理治疗形式。它可以由两个或多个治疗师组织开展。多数情况下,治疗师通常由男女两位组成,用来治疗性别相关问题,或者作为家庭成员的角色楷模。尽管有些家庭治疗本质上是行为主义或心理动力学的,但其治疗基础是家庭系统理论。这种治疗将家庭成员作为一个治疗整体,强调家庭成员之间的互动关系。

\subsection{发展过程}

家庭治疗有许多不同历史渊源。长期以来精神分析和其他心理动力学研究发现,早年家庭关系在个体人格和心理障碍的形成中起着重要作用。家庭治疗产生的另外一个原因是,临床研究发现,来访者在治疗过程中取得的进步一回家就消失了。有些治疗师开始对个别治疗效果表示不满,因为个别治疗无法解释不良家庭关系与来访者问题之间的关系。

家庭治疗作为一种全新心理治疗形式出现于二次大战后。当时治疗精神分裂症的医生注意到来访者与家庭成员之间的沟通非常混乱。另外,他们还发现来访者的症状随其父母的焦虑水平而起落。这一发现使他们将家庭视为一个有其内部规则、功能及拒绝变化的有机体或系统。治疗师开始将精神分裂症来访者家庭成员作为一个整体来治疗,而不是把注意力单独集中在来访者身上。他们发现,当将来访者置于家庭系统中,精神分裂症来访者家庭成员之间的问题也随之得到改善。尽管家庭问题会使病情恶化,但不能将精神分裂症误解为由家庭问题引起。这种将整个家庭纳入治疗计划的家庭治疗方法随后被用于精神分裂症之外的问题家庭处理中。随着家庭结构的变化,家庭治疗将逐渐成为一种常见治疗方法。

\subsection{主要理论}

家庭治疗有许多不同类型。最著名的是Salvador
Minuchin创立的结构化家庭治疗。这是一种关注当前而不是过去的短程心理治疗。这一流派是将家庭行为模式看作导致个体问题的重要原因。沟通技能缺乏是导致家庭内部破坏性人际关系形成的主要原因,如家庭中一些成员结盟对抗另外一些成员。结构化家庭治疗的目标包括强化父母的领导地位,规定家庭成员之间人际关系界线,提高应对技能,将家庭成员从家庭中所处不利位置摆脱出来。Minuchin将家庭成员间相互作用方式分成两种基本类型,即困境型(enmeshed)和分离型(disengaged),并将此看作两种极端病理行为。大多数家庭都处于这个连续体的中间。Minuchin认为问题家庭系统使个体无法情绪健康。其原因是由于整个家庭系统无法摆脱来自问题成员的消极影响,而且这种消极影响具有相当惰性,一旦形成很难改变。

心理动力学家庭治疗强调的是,一个家庭成员无法接受的人格特质投射到另外一个家庭成员的潜意识过程,及父辈家庭早年未解决的冲突对当前问题的影响。这些诸如父母离异和孩子受虐待等创伤性经验在治疗过程中得到分析。这种家庭治疗更关注家庭历史而不是症状,因此持续的时间很长。治疗师还使用客体关系强调让父母与祖父母共同解决冲突的重要性。有些治疗师将祖父母纳入家庭治疗中的目的是为了帮助家庭成员更好理解代际动力学和行为模式的根源。Ivan
Boszormenyi-Nagy是这种治疗方法的积极倡导者,只对由三代人组成的家庭进行家庭治疗。

行为主义家庭治疗将家庭成员间的相互作用看成是一系列奖励或惩罚性的行为。行为治疗师教育家庭成员如何用一些积极或消极的强化,对其他成员的行为作出反应。例如,为了消除孩子某种消极行为,家庭成员可以使用取消其该行为特权或限定时间等方法。积极行为则可以通过使用标有奖励和惩罚标志的激励图加以强化,如累积得了多少红五星可以获得某种具体奖励,反之获得某种惩罚。行为主义家庭治疗中有时还在家庭成员之间签订行为协议,具体注明行为规则及奖罚措施,由此强化或消退某种行为。

另外,还有其他一些家庭治疗方式,如Virginia
Satir强调人际沟通,Satir强调教育家庭成员沟通技能,增进自尊,排除影响个人成长的障碍。

\subsection{主要技术}

家庭治疗是一种短程心理治疗,通常持续几个月,用于解决一些具体问题,如摄食障碍、学习困难、亲人丧失或搬家等。对严重功能障碍性家庭则通常需要长程治疗。

在家庭治疗中,家庭中所有成员和治疗师均参加大多数治疗。治疗师将家庭成员之间相互作用和沟通作为一个整体来分析,不将某些成员放在一边。治疗师会偶尔提醒家庭成员需要考虑的某些行为模式或结构。家庭治疗师作为一个团队在治疗过程会通过相互沟通向家庭成员展示新行为。

家庭治疗的基础是家庭系统理论。该理论将家庭看成一个有生命的有机体,而不是家庭成员的简单相加。家庭治疗使用系统理论将家庭成员对有机体中的一个部分来加以评价。解决问题的方法也是从改变这个系统着手而不针对某个具体成员。有关家庭治疗的基本概念有以下几个:

1.有问题的来访者 有问题的来访者是家庭中一员,因为其症状治疗师才将全家人整合进行治疗。家庭治疗师使用有问题的来访者这个概念是想让家庭成员不要做来访者的“替罪羊”。

2.动态平衡 动态平衡这个概念指的是家庭系统维持其习惯性功能,它倾向于拒绝改变。家庭治疗师使用动态平衡这个概念来解释为什么家庭问题存在那么长时间,为什么家庭成员会成为来访者,家庭成员开始变化时可能会出现哪些迹象等。

3.家庭范围扩展 家庭范围扩展指的是一个小家庭,加上其长辈在内的大家庭。这个概念用于解释代与代之间的态度、问题、行为及其他问题的形成。

4.差异 差异指的是每个家庭成员保持其独立意识的一种能力,而在情感上依赖于家庭成员。一个健康家庭的标志是,一方面允许其成员保持其个性,另一方面家庭成员仍然感觉能在这个家庭中找到相应位置。

5.三角关系 家庭系统理论认为家庭内的情感关系通常是三角形的。家庭系统中任何两个人彼此之间发生矛盾,就会找一个第三者来平衡他们的关系。家庭系统中的这种三角关系通常相互作用,从而维持家庭动态平衡。常见家庭三角关系包括父母和一个孩子,两个孩子和父母一方,父母一方、一个孩子和一个祖父母,三个兄弟姐妹,或丈夫、妻子和一个女婿或媳妇等。

\subsection{治疗的准备}

有些情况下,家庭成员是被儿科医生或其他医护人员转诊找家庭治疗师的。通常情况下,综合性医院儿科门诊中大概有50%的孩子都与发育问题有关。在国外,有些医生还会使用症状问卷或心理筛查量表对一个家庭是否需要接受家庭治疗做相应评估。

家庭治疗通常由精神科医生、临床心理治疗师或其他在婚姻和家庭治疗领域接受过系统训练的专业人员担任。他们通常在治疗前为家庭成员安排数次面谈,包括问题来访者、家庭中与问题来访者有密切接触或也有问题的成员。这样做的目的是让治疗师发现家庭中每个成员是如何看待问题的,并且对其家庭功能形成一个初步印象。家庭治疗师通常还要寻找家庭中情绪表达的水平和类型、主宰和服从模式、家庭成员各自起的作用、沟通方式等。他们还会观察这些模式是否刻板或相对灵活。

准备工作通常还包括画一张家谱图,图上标出家族成员及家庭主要生活事件。家谱图还注明疾病史及每个成员的个性特点。家谱图有助于揭开代与代之间行为模式、婚姻选择、家庭联盟和冲突、家族内秘密,还有有关家庭目前状况的信息。

\subsection{治疗的风险}

家庭治疗的主要风险是可能无法解决家庭成员的刻板性心理防御或改善脆弱的夫妻关系。深度家庭治疗对有精神疾病成员的家庭开展起来较为困难。

\subsection{治疗的结果}

理想情况下,正常家庭治疗结果应该是家庭成员对问题有更深的领悟,家庭成员间的差异性增加,家庭内部沟通改善,以前病态自动化行为模式消退,家庭寻求治疗的问题得到解决。

八.适应证和禁忌证

1.适应证 家庭治疗通常适用于下列情况:

(1)对患有精神分裂症或多种人格障碍的家庭人员进行治疗。家庭治疗帮助家庭其他成员理解其家人所患精神障碍,并调整状态为来访者症状改善创造条件。

(2)家族问题:这些问题通常由于祖孙三代同居一家,由祖父母们通过遗传和教养方式世代相传下来。

(3)偏离社会常模的家庭:如未婚同居、同性恋同居者抚养孩子等。这些家庭可能没有内部问题,但得忍受来自外界舆论的压力。

(4)家庭成员来自不同种族、不同文化及宗教背景的家庭。

(5)为某一家庭成员当“替罪羊”或对某一家庭成员治疗起破坏作用的家庭。

(6)家庭某一成员是来访者,而且其问题与家庭内其他成员密切相关的家庭。

(7)有适应困难的混合家庭。

2.禁忌证 有些家庭不适合做家庭治疗:

(1)家庭中父母亲一方或双方患精神疾病或有反社会或偏执型人格障碍的。

(2)家庭成员中文化或宗教信仰对心理治疗开展有抵触的。

(3)家庭成员中有因为躯体疾病或相关缺陷无法参加治疗的。

(4)家庭成员中有非常刻板的人格结构,或成员间可能引发一场情绪或心理危机的。

(5)家庭成员中有不能或不愿意定期接受治疗的。

(6)处于崩溃边缘的家庭。


\section{团 体 治 疗}

\subsection{定义}

团体治疗是一种由一小组来访者和治疗师通过定期交谈、讨论问题等方式解决来访者心理问题的心理社会治疗方式。

\subsection{发展过程}

团体治疗在第二次世界大战之后才得以广泛传播,并吸收了许多心理治疗技术,如心理动力学、行为主义、现象学等。Fritz
Perls格式塔团体治疗中,治疗师每次只对一个团体进行治疗。其他如JL
Moreno的心理剧则强调成员之间互动,心理剧要求团体成员在治疗师的指导下将相关具体情境表演出来。在Moreno心理剧影响下,20世纪60年代出现了许多新的团体治疗方式,如邂逅小组、敏感性训练、马拉松团体、互动分析等。马拉松团体持续很长时间,为的是消磨成员心理防御,鼓励更多互动。除了满足团体中个别治疗的需要,团体治疗还使用了团体治疗以外的一些方法,其中包括Kurt
Lewin 20世纪40年代的T-小组训练。

团体治疗可以在不同地方开展,包括病房和门诊。团体治疗可以用来治疗的精神障碍也有许多种,如焦虑、情感障碍、人格障碍等。从20世纪80年代开始,从团体治疗中借鉴来的治疗技术也被广泛用于自助小组。这些小组主要由一些某方面存在具体问题的人组成,如单亲、摄食障碍、毒品成瘾、儿童受虐待等。这些小组与传统意义上的团体治疗的主要区别在于没有精神卫生专业人员指导。

严格来说,自助小组如对滥用酒及控制体重等治疗,均不属于心理治疗范畴。这些自助团体因为给成员提供了社会支持、认同和归属感等,因而对团体中绝大多数人有益。自助团体成员定期聚会并讨论一些共同关心的问题,如酗酒、摄食障碍、亲人丧失或扶养孩子等。团体治疗不受治疗师管理,而是由一个非专业的小组长、团体成员或全体成员管理。

\subsection{治疗的目的}

团体治疗的目的是努力给成员提供一个安全舒适的地方。成员不但可以理解他们自己的思想和行为,而且还能给其他人提供建议和支持,最终解决相关问题。另外,如果治疗安排科学,那些有人际交往问题的成员就可以从团体治疗中获益。

\subsection{主要治疗技术}

团体治疗通常由心理治疗师、精神科医生、社会工作者及卫生专业人员来开展。有些治疗团体还安排两位治疗师负责治疗。成员是根据他们想从团体治疗中获得什么及他们能给团体成员提供什么来选拔的。

治疗团体可以是同质或异质的。在同质性团体中,成员通常有类似问题,如患有抑郁症。异质性团体中则将不同心理问题的人放在一起。参加团体治疗的成员人数不等,通常不超过12人。团体治疗在治疗次数上也有限定的和不限定两种。团体治疗开始以后对新成员也有不接纳和接纳两种。

治疗次数根据团体组成及目标而定。例如,如果团体治疗是物质滥用住院来访者治疗计划的一部分,即短程团体治疗。长程团体治疗可以长达半年、一年或更长。治疗方法根据团体治疗的目标及治疗师的经验而定。基本技术包括心理动力学、认知行为和格式塔等。

1.鼓励成员开放和诚实讨论问题 在团体治疗中,成员被鼓励开放和诚实讨论他们的问题,帮助其他成员就遇到的问题提供建议、理解和共情。团体治疗没有规则,只要成员在团体中能最大限度地发挥其潜能就行。但大多数治疗团体通常有其基本要求,并在第一次访谈中讨论清楚。来访者被告知不得将治疗过程的所见所闻告诉团体以外任何一人。这样做是为了保护其他成员的隐私权。他们另外还被要求不允许在治疗以外的地方与成员见面,因为其所带来破坏作用可能对团体治疗产生负面影响。

2.治疗师主要任务是指导团队自我发现 根据团队的目标、训练及治疗师的风格,治疗师可以指导团体互动或允许团体自己决定行动方向。通常情况下,治疗师都采用这两种方法,当团队走得太远时指导一下,或由他们自己来设置进度。治疗师可以单纯通过强化积极行为来指导团体。例如,如果一个成员向另外一个成员传达共情,或提出一个建设性建议,治疗师即时点评该行为对团体的价值。在几乎所有团体治疗情境中,治疗师都强调团体成员的共同特质,以便让成员获得一种团体认同感。通过这一技术,团体成员认识到其他人跟自己一样有着类似问题,并为以后分享、互动和改变打下基础。

3.团体互动是心理治疗的一部分 团体治疗有比个别治疗无可比拟的优点。有些来访者在团体治疗中会感觉比与治疗师单独在一起时更加真实自如。治疗师通过观察成员间的互动,向成员提供成员无法获得的信息。来访者能通过倾听其他人讨论问题获得帮助,他们同样可以通过目睹别人症状改善看到希望,或体验到帮助别人后的成就感。团体治疗还向个体提供学习的机会。成员可以通过模仿别人的积极行为而习得该行为。除了彼此学习外,团体中培养起来的信任和凝聚力可以促进成员的自信和人际交往技能。团体治疗还给个体提供了一个安全环境,那里成员新学到的技能可以得到使用和检验。另外,在纠正不良家庭互动方面,团体治疗可以向成员提供观察学习的机会。最后,团体治疗的特点是成本低效益高,因此客观上减少了治疗师的时间消耗。

当然团体治疗也有弊端。例如,有些成员会因为在团体中公开谈论自己的问题和感受而感到不舒服。有些团体反馈对成员有伤害作用。另外,团体互动过程本身会成为一个讨论焦点,会花费大量时间,甚至成员来参加治疗的主要目的有时会被忽视。

\subsection{治疗准备}

成员通常由心理治疗师或精神科医生转诊参加团体治疗。团体治疗开始前,治疗师通常会安排一次简短的面谈来决定该团体是否适合该来访者。这种面谈同样需要治疗师决定来访者的加入是否对团体有益。治疗开始前,治疗师通常会向来访者提供一些有关治疗的基本信息,主要是成功的秘诀,如开放性、倾听别人、风险等,及一些团体规则如隐私保密等。另外,还有一些有关团体治疗的教育性信息。

\subsection{治疗结果}

长程团体治疗结束时可能会引起有些成员悲伤、被遗弃、愤怒及被拒绝等体验。团体治疗师应努力鼓励成员探索其体验或使用新习得解决问题的方法来培养一种治疗结束感。治疗收尾工作是治疗过程的重要部分。

相关研究表明,无论是团体还是个别治疗,最多只有85%的来访者能从中受益。在理想情况下,通过团体治疗,成员能对他们自己的问题获得一种更好的理解及习得一些人际关系应对技能。有些成员可能会在团体治疗结束后继续参加治疗,或者个别,或者团体。

\subsection{治疗风险}

有些脆弱的来访者可能无法忍受来自团体成员的攻击性或充满敌意的评价。那些在团体情境中有沟通障碍的来访者可能有治疗脱落风险。如果对他们的沉默不作出反应或不与他们交往,他们可能会感到更加孤独而不能将自己认同为团体中一员,因此,治疗师通常需要努力鼓励沉默寡言的成员在治疗一开始就加入进来。

\subsection{禁忌证}

有自杀、他杀、精神障碍或正处于急性发病期的来访者不适合参加团体治疗,除非其行为或情绪状态比较稳定。根据心理和行为功能水平不同,认知功能受损的来访者,如患器质性脑病或脑外伤的人同样也不适合团体治疗。有些社会功能存在病理性缺陷的来访者也不适合通常意义上所说的团体治疗。

\section{心理治疗在临床治疗中的应用}

\subsection{来访者对心理治疗合理及不合理的期望}

在临床心理治疗过程中,常常会出现来访者因为不了解什么是心理治疗,对心理治疗抱有不切实际的期望而出现频繁治疗脱落现象。因此治疗一开始,一位训练有素的治疗师会了解来访者对心理治疗的了解程度,来访者期望通过心理治疗获得哪些结果,不能获得哪些结果,并对其所持的相关错误观念及时予以纠正。这对心理治疗顺利开展,避免来访者脱落,鼓励来访者更加积极参与到心理治疗中来有积极作用。

1.来访者对心理治疗合理的期望

(1)促进自己更加成熟起来(例如减少家人对自己过多关心,增加自己的独立性,减少自己只索取不付出的行为,多理解别人的需要等)。

(2)改变自己的思维方式,包括改变自己的认知方式及不合理信念。

(3)减少自己的紧张心理,不过分依赖不成熟的心理防御机制。

(4)增加不同情境中的沟通应对技巧。

2.来访者对心理治疗不合理的期望

(1)症状完全消失。

(2)听治疗师一讲症状就消除。

(3)彻底摆脱不合理信念。

(4)不努力就取得成功。

(5)彻底忘掉过去,使自己脱胎换骨。

\subsection{来访者选择}

不是所有想接受心理治疗的来访者都适合接受心理治疗。除了来访者应该对心理治疗抱有合理期望外,治疗师在选择来访者过程中应该注意把握好下面一些基本要求:

1.来访者的问题必须在治疗师的解决能力范围之内。

2.来访者必须具备一定自知力。

3.来访者必须对自己的问题有一定陈述能力。

4.来访者必须对治疗师的话具有一定理解能力。

5.来访者必须遵守赴约、付费及其他治疗规定。

6.来访者必须对心理治疗有一定了解,明确双方在治疗过程中的职责。

7.来访者必须对治疗关系有一定了解和驾驭能力。

\subsection{心理治疗的基本过程}

1.第一次访谈 评估来访者问题。必要时,第一次访谈需要对来访者做一次医学检查,以排除相关躯体疾病,并对来访者的精神状态进行系统评估。这种面谈通常是开放式的。

第一次访谈应该完成下列三件事:

(1)治疗师需要对来访者作一个临床诊断。

(2)治疗师检查来访者与治疗师之间的合作适合程度。

(3)随着来访者对自己问题的陈述,治疗师应该能够理解来访者问题性质及精神病理学原因,并且有一种“胜任感”。

2.第二次访谈 了解心理治疗过程及治疗协议。第二次访谈中,治疗师需要解决的问题如下:

(1)来访者对第一次访谈的想法和反应。

(2)就来访者问题作更细致的探索,并且排除其他新问题。

(3)初步建立治疗关系。

(4)来访者需要不需要接受药物治疗。

(5)心理测验是否必要。

(6)来访者应该选择何种形式心理治疗及治疗频率。

有关心理治疗,治疗师有必要向来访者解释清楚以下问题:①心理治疗的名称、频率、时间长度以及收费标准。②该治疗给来访者可能带来的结果。③其他治疗选择的特点及相关问题。

3.第三次访谈 第三次访谈是治疗的开始。对来访者来说,从治疗师答应对其做心理治疗开始,心理治疗就开始了。但是对治疗师来说,评估与真正意义上的心理治疗是两个概念。无论治疗性质如何,治疗开始阶段都由两个部分组成:

(1)来访者接受治疗契约,进入治疗阶段。因为不同治疗结构,“来访者进入治疗阶段”的含义也有不同。精神分析中,只有当来访者感到其在情感上对治疗师或治疗师来访者这种关系产生依恋时,才能说来访者“进入治疗”。在认知治疗或家庭治疗中,当来访者能向治疗师主动谈出其对上一次访谈的想法时,才能说来访者“进入治疗”。

(2)来访者的问题在治疗师和来访者面前得以逐渐暴露和展开。来访者用来访者的语言,治疗师用治疗师的语言,彼此陈述双方对问题的理解和看法,然后在相互沟通中努力寻找一种默契。治疗师始终会用一种带明显理论倾向(如精神分析、认知行为、家庭治疗等)或视角去审视来访者的问题,而来访者会以自己独特建构理解知识的方式看待自己的问题。一旦问题得以最终定位,开始阶段宣告结束。

4.治疗中间阶段 治疗中间阶段开始前,来访者和治疗师之间的合作关系应该相对比较稳固,并且在来访者核心问题上双方达成一致共识。治疗的中间过程就是双方协同解决问题的过程。在精神分析中,这一过程可以表现为移情与反移情。在认知行为或家庭治疗中,治疗师的角色是顺着来访者意识潜意识的态度和行为,通过共情、探索、澄清、面质和解释等技术与来访者共同寻找解决问题答案的过程。

(1)行为宣泄:行为宣泄这个术语有不同用法,但通常指的是治疗过程中情感性宣泄行为。在治疗过程中,来访者因为受意识潜意识动机驱使,通常会表现出不同行为。有些行为具有冲动性,有时会影响治疗的正常开展,如迟到、失约、不付费、突然中止治疗、自杀行为、物质滥用行为或午夜给治疗师打电话等行为。如果治疗师缺乏处理这些突发性行为的能力,他(或她)在治疗过程中会显得很被动。

(2)治疗关系维护:有时来访者做的一些事会让治疗师感到尴尬,随时有越过工作关系、破坏治疗关系的可能。例如,有些来访者会给治疗师礼品,问治疗师私生活有关的问题,穿很性感的衣服,邀请治疗师外出喝茶或吃饭等。这时就需要治疗师熟练处理这些不利于治疗的问题,不但需要治疗师具有同情心、灵活性和客观性,而且还需要掌握有关人际关系界限的知识、医疗道德及相关精神病理学知识。治疗师在缺乏经验的情况下,宁可直言拒绝其要求,也不可无原则地默许来访者的不合理要求。必要时,治疗师可以就此话题与来访者进行探讨,直至其放弃为止。

(3)对峙:对峙有许多含义,但其核心内容是来访者因为没有取得显著进步而对治疗不满,甚至想中止治疗。这种现象传统上被认为是治疗过程中阻抗的一部分。当代治疗理论则将其看作是对治疗师和来访者关系的一种评价。这时候治疗师及时对这种状态做一些积极解释,甚至适当发表一些自己的想法是必要的,因为它有利于恢复治疗合作关系。

(4)第三方干预:在治疗过程中,可能出现第三方对治疗过程进行干预的情况。例如,丈夫会因为嫉妒不停地敲治疗室的门,妈妈会因为认为“去精神病院都是脑子有毛病的”而将正在接受治疗的儿子拽出治疗室等。这时,治疗师需要用敏感性、灵活性、勇气、智慧和力量来保护来访者在治疗过程中的隐私、权利和利益。

(5)其他负面效应:治疗师的弱点在治疗过程中也会暴露无遗。首先治疗师在移情和反移情过程中不但随时有超越治疗师来访者关系界线的风险,而且还会受来访者不良情绪影响。其他负面效应还有:治疗师工作一天下来会感到精疲力竭,甚至情绪低落,在治疗过程中,治疗师应该保持好中立和客观性,同时又不对来访者的观点和情绪过分认同。

5.结束治疗 结束治疗可以出现不同的表现,有的双方皆大欢喜,有的则充满痛苦或具有伤害性。

(1)双方满意,结束治疗:这种情况是最让人满意的。治疗师和来访者通过协作达到预期治疗目标,来访者已经摆脱或基本摆脱过去问题行为,自己能独立开始生活。

结束治疗前,治疗师通常会教来访者一些常用应对症状复发的技能。目的是让来访者在未来可能的症状再次出现时作出相应的应对性反应。另外,治疗师还需要对来访者家属及其密切接触者进行心理健康教育。内容可以包括如何控制或减少导致来访者症状复发的相关风险因素及相关应对策略。

在成功心理治疗中,治疗师与来访者之间的工作关系应该不受到破坏。双方应该是相互尊重的,治疗关系不能超越工作关系允许的界限。

(2)双方基本满意,结束治疗:双方对治疗效果不完全满意的原因很多,有治疗师对治疗总体时间限制的原因,有来访者资源有限,在治疗过程中不能充分利用和发挥的原因,有治疗师心理诊断和治疗经验不足的原因等。这种治疗虽然某种程度上让来访者感到有些痛苦,但并不具有创伤性和伤害性。

(3)治疗突然中止:有些情况下,突然中止心理治疗对来访者具有创伤性和伤害性。例如治疗师另就高职或移居他乡、来访者因为缺乏资金来源而不得不中止治疗等。反过来,如果来访者突然不辞而别,对治疗师或多或少也是一次创伤性经验。在这种情况下,治疗师仍然需要保持治疗师角色,要客观冷静思考和处理所遇到的问题。虽然来访者的做法多少让人感到无法理解,但其人格必须受到尊重。必要时,治疗师有义务为来访者安排一次访谈,讨论如何结束治疗。这种情况下,治疗师仍然有责任向来访者推荐其他可能合适的心理治疗师。这样对来访者来说,寻医过程本身就是一个对自己问题及心理治疗了解的过程。

总之,处理治疗突然中止时,治疗师需要敏锐、智慧、技巧、毅力和勇气。

\subsection{心理治疗在临床主要精神障碍治疗中的应用}

1.精神分裂症 绝大多数精神分裂症来访者在急性期症状控制后都能或多或少受益于心理治疗。一般来说,对精神分裂症来访者不建议做精神分析。但行为治疗通常能帮助来访者重新恢复其日常生活和社会交往技能。行为治疗通常可以和工作治疗(工疗)合并使用,最终恢复来访者的职业功能。

家庭治疗对精神分裂症来访者十分重要。它能帮助家庭成员消除对来访者的消极态度和情绪,而家人对来访者的态度和行为恰恰是导致疾病复发的重要因素。一方面,家庭治疗能帮助家庭成员为来访者创造一个相对宽松、积极的环境;另一方面,也能帮助家庭成员学会来访者发病时的积极应对策略。国外,许多精神分裂症来访者的家庭成员常常能够受益于类似精神分裂症家庭成员组成的支持小组。

总之,精神分裂症心理治疗的主要治疗目标有以下几点:

(1)帮助来访者理解其所患精神障碍的性质,尤其是对某个具体应激源作出反应。

(2)对环境的快速整合和适应。

(3)长程治疗目标可以考虑训练来访者应对疾病复发相关技能,尤其是来访者或家庭成员应对复发早期症状采取相关措施。

2.抑郁症 心理治疗通常与药物治疗合并使用。抑郁症相关临床对照研究表明,认知行为治疗、家庭治疗合并药物治疗与控制组对比,这些治疗的效果证明是显著的。抗抑郁药物治疗在改善快感缺乏、精神运动性迟滞、睡眠障碍、食欲不振、精神病性幻觉及妄想等方面有优势,而心理治疗在改善来访者兴趣、心境、社会关系和职业问题等方面有优势。传统精神分析治疗因为研究数量不足,不能证明其有效。

认知行为治疗的理论基础是帮助来访者理解和纠正其错误信念,消除其情绪痛苦。相关临床研究证明,认知行为治疗在消除除内源性抑郁之外的轻中度抑郁症状方面有效。认知行为治疗因为具有比较持久的效果,所以在预防抑郁症复发方面有效。对比较严重的抑郁症,无论是认知行为治疗还是其他治疗,均不如药物治疗效果好。

家庭治疗主要针对家庭因素在来访者抑郁症发病过程中所起的重要作用。研究表明,改善相关家庭关系可以改善抑郁症状以及抑郁症复发。

3.焦虑症 绝大多数焦虑障碍来访者一般同时接受药物和心理治疗。许多来访者对顿悟认知相关的心理治疗效果比较明显,如精神分析,因为这些治疗能够帮助来访者理解其症状背后的潜意识心理冲突及心理防御机制。解决潜意识心理冲突从而消除躯体焦虑症状,这是传统精神分析对焦虑形成机制的经典精神病理学解释。

另外,行为治疗和认知行为治疗两种心理治疗方法对焦虑症有效。行为治疗主要通过帮助来访者进行渐进式肌肉放松来缓解焦虑。如果来访者刻意回避使其产生焦虑的情境,渐进式系统脱敏法对消除该症状有效。

还有一种心理治疗方法是认知行为治疗。该治疗主要针对纠正导致来访者焦虑的错误认知,从而消除或缓解焦虑情绪。这些来访者通常对自我、他人或将来抱有错误认知。对威胁或危险的灾难性夸大是导致其过度焦虑的原因。认知行为治疗帮助来访者认识到具体情境与病理性认知歪曲之间的关系。治疗关键是帮助来访者认识到其错误信念背后的错误基本信念系统。通过治疗,来访者能学会用更加积极健康的思维方式替代那些错误信念,最终消除自动性思维及由此产生的焦虑。

4.强迫症 强迫症行为治疗有暴露和反应阻止。根据学习理论,强迫症来访者习得了一种对焦虑不恰当的回避反应。治疗师必须鼓励来访者置身于厌恶情境(暴露),并且不允许其执行强迫性仪式行为(反应阻止)。单纯逐级暴露能减少焦虑,但不能消除强迫行为。所以反应阻止至关重要。通过逐级暴露,加上反应阻止,来访者开始可能表现得异常焦虑,但随着暴露的进一步深入,焦虑的持久时间及强度会逐渐下降。

相关研究表明,强迫症来访者具有完美主义、行为刻板等特点,而这些特点与其早年家庭中父母亲依恋和教养方式有着密切关系。通常这些来访者的主要依恋对象,父母亲或祖父母至少一方有类似行为或个性问题。因此让来访者及与其密切接触的主要家庭成员认识到来访者这种症状与其依恋及教养方式之间的关系,这样做有利于为来访者创造一个相应解决问题的家庭环境,帮助来访者纠正其完美和刻板行为。另外,家庭成员治疗过程中需要鼓励来访者多用一些积极灵活的行为,减少刻板完美行为,尤其是强迫性仪式行为。为了协调家庭成员之间的关系,治疗过程中来访者和家庭成员之间可以签订一份协议来规范相互间的治疗行为。

5.人格障碍 精神病学中,只有在人格障碍上,心理学和生物学才真正走到了一起。人格主要由两个主要成分组成的:一个是气质,即神经活动类型,主要通过遗传获得;另一个是性格,主要受后天环境及教育等因素影响。

一些临床心理治疗师认为,精神分析心理治疗对那些适合接受顿悟相关心理治疗的人格障碍来访者有用。这些人格障碍来访者包括依赖型、强迫型及回避型人格障碍。对自恋型及边缘型人格障碍通常适合做个体心理治疗,如国外许多临床研究表明,辩证行为治疗(DBT)对边缘型人格障碍很有效。但顿悟性心理治疗对偏执型和反社会型人格障碍不适合,因为这些治疗可能会增加来访者对治疗师的憎恨,认为治疗师想控制他们。支持性治疗看来最适合于分裂样人格障碍。

因为人格障碍本身具有顽固性特点,因此人格障碍的心理治疗非常富有挑战性和艰巨性。有临床研究表明,人格障碍合并精神障碍的心理治疗效果并不理想。有一项研究对抑郁症伴人格障碍的来访者进行心理治疗和药物治疗,结果表明,接受两种治疗的来访者与同样接受两种治疗、但无人格障碍的抑郁症来访者相比,无显著差异。

6.摄食障碍 摄食障碍很久以来就有心理治疗合并抗抑郁药治疗的传统。对神经性厌食和神经性贪食的心理和药物合并治疗相关临床研究主要有以下发现:

(1)在减少神经性厌食和贪食引起的呕吐及暴饮暴食方面,认知行为治疗要优于单纯支持性治疗。

(2)接受药物和心理治疗的摄食障碍来访者与安慰剂合并心理治疗的摄食障碍来访者相比,前者在暴饮暴食及抑郁症状改善方面优于后者。

(3)认知行为治疗合并抗抑郁药物治疗的效果要优于单纯药物治疗。

\textbf{【附】 生物反馈}

\subsection{定义}

生物反馈是一种由来访者来操纵的治疗方式,主要通过放松、显示设备及其他认知控制技术训练个体控制肌肉紧张、疼痛、体温、脑电波等。生物反馈这个词的意思是指将被反馈的生物信号给来访者,为的是让来访者通过学习来控制它们。

\subsection{发展过程}

1961年,一位名叫Neal
Miller的实验心理学家发现,自主神经系统反应如心率、血压、胃肠道反应、局部血流等能够受人的意志控制。作为实验结果,他认为自主神经系统过程是可以人为控制的。这项工作导致了生物反馈治疗的诞生。Wilier的工作得到了其他研究人员进一步延续和发展。20世纪70年代加州大学洛杉矶分校的研究人员Barry
Sterman博士发现老鼠和猴子可以通过训练控制其脑电模型。Sterman继而将他的研究技术用于癫痫
来访者,并通过生物反馈技术取得了60%癫痫
降低的效果。20世纪70年代后,其他研究人员陆续发表了许多有关生物反馈用于治疗心律不齐、头痛、Raynaud综合征、胃酸分泌过多,甚至深度放松的报告。从Miller和Sterman早年工作至今,生物反馈已发展为一种用于治疗众多障碍和症状的前沿性行为治疗技术。

\subsection{主要治疗技术}

在生物反馈过程中,一些特殊传感器被安置在人体上。这些传感器测量的是引起来访者症状的身体变化,如心率、血压、肌肉紧张(Emg或肌电图)、脑电波(EEG或脑电图)、呼吸、体温(体温反馈),并将这些信息输入一个可视或可听的输出设备,如图纸、光显示或声音。

尽管来访者从生物反馈监视器中看到瞬时反馈,但可以知道是什么想法、恐惧及心理想象影响了其生理反应。通过控制心理与生理这种关系,来访者可以使用相关思维和心理想象作为精细线索,如这个动作代表深放松而不是焦虑。这些暗示同样可以控制心率、脑电波、体温及其他身体功能。这些是通过放松练习、心理想象及其他认知治疗技术取得的。

这些生物反馈反应发生后,来访者可以立即通过生物反馈感官输出设备直接看到和听到他(或她)努力的结果。一旦这些技术被习得,来访者就能够分辨出用来缓解症状的放松状态或可视化信号。这时候生物反馈设备就不再需要了,来访者可以通过使用习得的生物反馈技术来解决其症状了。

通过生物反馈,个体可以在一个训练有素的专业人员指导下通过30课时的学习习得这种疗效长久的可以控制其症状的技术。治疗师建议来访者可以在家里同时做生物反馈和放松练习。

\subsection{治疗准备}

在做生物反馈治疗前,治疗师和来访者会有一次面谈,以记录来访者的病史、治疗情况,并讨论治疗目标。

生物反馈通常在一个安静和放松气氛下,来访者在舒服坐姿下进行。根据治疗的类型和目标,至少一个以上的感受器用传导凝胶安置在来访者身上。这些感受器包括:

1.肌电(Emg)感受器 测量肌肉的电生理活动,尤其是紧张。

2.流电皮肤反应(GSR)感受器 电极安置在手指上,以监测出汗或汗腺活动。

3.温度感受器 测量体温及血流变化。

4.脑电(EEG)感受器 电极安置于头皮,以测量脑电活动。

5.心率感受器 脉搏监测器安置于指尖,以测脉搏频率。

6.呼吸感受器 呼吸感受器监测氧摄入及二氧化碳呼出。

\subsection{治疗师的培训训练与资格认证}

目前国外经资格论证的生物反馈治疗师通常由执业心理治疗师、精神科医生、内科医生等受过生物反馈系统培训的专业人员来担任。

\subsection{适应证和禁忌证}

1.适应证 生物反馈已被成功用于治疗许多疾病,如颞颚关节障碍(TMJ)、慢性疼痛、肠易激综合征(IBS)、Raynaud综合征、癫、注意力缺失及多动症(ADHD)、偏头痛、焦虑症、抑郁症、脑外伤及睡眠障碍。

部分由应激引发的疾病也可用生物反馈治疗。常见头痛、高血压、磨牙、创伤后应激性反应、摄食障碍、物质滥用及某些焦虑障碍,均可以通过训练来访者使用肌肉和心理放松等方法进行成功治疗。对某些障碍来说,生物反馈通常是综合治疗方案中的一个组成部分。

2.禁忌证 装有心脏起搏器的个体应在生物反馈开始前自动告知治疗师,以免生物反馈感受器干扰起搏器正常工作。

生物反馈对有些来访者不适合。来访者必须愿意在治疗过程中起积极作用。因为生物反馈严格控制在行为变化层面上,那些希望对其症状做深刻理解的来访者最好考虑是否可以接受精神分析及其他形式心理治疗。生物反馈对认知功能受损的来访者同样不适合,如患有器质性脑病及脑外伤等。