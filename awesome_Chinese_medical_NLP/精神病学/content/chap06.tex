\chapter{躯体疾病所致精神障碍}

\section{概  述}

许多躯体疾病可以引起或诱发精神障碍。因此,医学实践中经常遇到精神障碍与躯体疾病共存或共病的情况,导致临床医生诊断和治疗上的困难。精神障碍与躯体疾病的关系越来越引起临床各科的重视,及时正确地处理好躯体疾病所致的精神障碍,对于减少医疗纠纷、减轻患者经济负担,均有重要意义。

本章所述的躯体疾病所致精神障碍(mental disorders due to physical
diseases)是指由于各种原因引起的除脑以外的躯体疾病或障碍影响脑功能变化所致的一类精神障碍,又称体因性精神病或症状性精神病(symptomatic
psychosis),包括在器质性精神病(organic
psychosis)中。临床主要表现为意识障碍、遗忘综合征、认知障碍、人格改变、精神病性症状、情感障碍、神经症样症状(脑衰弱综合征)或各种行为障碍,以及以上症状的混合状态。诊断时除标明主要精神病理综合征外,应同时作出导致精神障碍的躯体疾病诊断。

关于躯体疾病所致精神障碍的归类,在1990年出版的ICD-10中归于“F06节脑损害和功能紊乱以及躯体疾病所致的其他精神障碍”。其病因包括影响大脑的全身性疾病、内分泌障碍以及某些外源性毒性物质等。当然,此类精神障碍的起病与上述原因引起的大脑功能紊乱直接有关。此类精神障碍不包括与躯体疾病存在偶然联系的精神障碍,也不是机体对躯体疾病症状的心理反应。在该分类中对各种器质性精神病理综合征做了详细的描述,而并未过多地涉及具体病因。美国DSM-IV的有关分类与ICD-10(v)类似。2001年中国精神障碍分类与诊断标准第三版(CCMD-3)在0节03分节,器质性精神障碍中按病因学分类对躯体感染、内脏器官疾病、内分泌疾病、营养或代谢疾病、染色体异常、物理因素引起疾病所致的精神障碍以及有机化合物、一氧化碳、重金属和药物所致的精神障碍进行了描述,并给出了相应诊断标准。

\subsection{病因及发病机制}

\subsubsection{病因}

本章所描述的精神障碍,其病因应是比较明确的、可被证实的。导致精神障碍的脑功能紊乱是继发于大脑外全身性躯体疾病或障碍,脑也是众多受侵害的器官或系统之一。然而在患某种躯体疾病的患者中仅有少数患者发生精神障碍,故躯体疾病并非为此类精神障碍的唯一病因,尚有其他因素与精神障碍发病有关。一般认为患者的性别、年龄、遗传因素、人格特征、应激状态、人际关系以及既往神经精神病史等可能影响精神障碍的发生。

此外,心理社会因素如长期应激状态、长期的心理矛盾等,可能削弱个体的心理承受能力,而在躯体疾病时容易发生精神障碍;环境因素如居住拥挤、环境嘈杂、潮湿、空气污染等,都可能促发躯体疾病所致的精神障碍。可以说各种原因引起的躯体疾病(障碍)是本类精神障碍发病的主要因素,而上述其他生物学因素、心理和环境因素为精神障碍的促发因素。

\subsubsection{发病机制}

躯体疾病所致精神障碍的病因归纳如下:

1.能量供给不足 Lipowski提出躯体疾病所致精神障碍的共同病理生理改变为弥漫性大脑能量改变。这个观点现在仍受到普遍的赞同。由于躯体疾病引起机体代谢障碍,导致能量产生不足,影响大脑能量供应。而大脑又对能量供应非常敏感,而且当躯体疾病或受累时,大脑对能量的需求增长。此时机体发生能量供求矛盾,大脑正常的生理功能势必发生紊乱。因此,机体在受累时发生的能量供给不足是发生此类精神障碍的主要机制。

2.脑缺氧 由于躯体疾病特别是心脑血管疾病引起机体和脑部血液循环障碍,或贫血携氧能力不足,或机体在有害因素影响下出现微循环障碍等,均可导致脑供血、供氧不足,发生脑功能障碍,亦是发生精神障碍的重要机制。

3.毒素作用 外源性物质如细菌、病毒、寄生虫、化学物质、有害气体等侵入机体,其毒素或中间代谢产物直接作用于脑细胞,造成脑细胞受损而发生脑功能紊乱,导致精神障碍。特别是有些细菌、病毒产生的毒素对中枢神经系统有很强的亲和力,更易发生精神障碍。

4.水和电解质紊乱、酸碱平衡失调、内分泌激素与维生素不足等 这些均是躯体在疾病或受害情况下容易发生的问题,可引起脑功能紊乱,导致精神障碍。

5.应激反应 外源性有害因素作为应激源作用于机体,产生一系列生化生理反应。此类应激反应主要通过神经生理、神经生化、神经内分泌及免疫机制进行。在应激反应中,大脑或直接参与或间接受累,其正常的生理功能受到影响而发生紊乱,导致精神障碍。

6.中枢神经递质的改变 有研究表明,某些有害物质、药品或机体必需物质不足时,可直接引起脑内单胺递质代谢异常,如锰选择性地作用于苍白球及视丘抑制多巴脱羧酶使多巴胺含量减少;利血平可使中枢去甲肾上腺素、多巴胺及5-HT含量下降;烟酸缺乏时儿茶酚胺甲基化增高。以上种种中枢神经递质的代谢变化均可引发精神障碍。

对于上述种种发病机制的作用不能单独地考虑。实际上,在躯体疾病的不同病程阶段出现精神障碍,是多种发病因素错综复杂交互作用的结果。

\subsection{临床表现}

临床表现的主要特征为:认知功能障碍,表现为记忆、思维及学习功能减退;感觉中枢障碍,表现为意识和注意障碍;知觉、思维内容障碍如错觉、幻觉和妄想;心境和情绪障碍,如情感低落或高涨、焦虑;人格和行为障碍。根据躯体疾病的轻重缓急,精神症状可表现为脑衰弱综合征、急性脑病综合征和慢性脑病综合征。三者之间可有移行或转变,视躯体疾病病情变化而走。此外,精神病性症状群及情感障碍亦常见到。

\subsubsection{脑衰弱综合征}

多见于躯体疾病的初期、恢复期或慢性躯体疾病的过程中。患者感到疲倦、虚弱无力、思维迟钝、注意力不集中、情绪不稳或脆弱,常伴头部不适如头痛、头晕,感觉过敏及躯体不适如虚汗、心悸、食欲不振等。此种衰弱状态往往需要较长时间恢复。

\subsubsection{急性脑病综合征}

多继发于急性躯体疾病或机体处于急性应激状态时。精神障碍多起病急骤,症状鲜明,但持续短暂。精神症状多随机体状况的好转而恢复;或随躯体疾病的迁延而转为慢性精神障碍。主要表现为:

1.意识障碍 该类障碍中约2/3的患者表现有轻重不等的意识障碍。轻者意识模糊,重至昏迷,谵妄最常见。确定意识障碍需依据:

(1)感觉迟钝,对外界刺激的反应减弱,知觉清晰度降低,对周围环境认知模糊。

(2)注意转移、集中和持久的能力减退。

(3)定向障碍,可为时间、地点、人物或自我定向障碍。

(4)伴有下述症状:①错觉、幻觉;②言语不连贯、思维结构解体、回答不切题;③精神运动性兴奋或迟滞,或出现紧张综合征;④睡眠觉醒节律紊乱,失眠或嗜睡;⑤理解困难或错误;⑥瞬间记忆障碍,回忆困难。病情缓解后患者对病中经历部分遗忘或全部遗忘。

2.急性精神病性状态 主要表现为不协调的言语运动性兴奋、紧张综合征或类躁狂样兴奋;或短暂性幻觉妄想状态,一般无明显意识障碍,可伴有轻度意识模糊或意识范围狭窄。病愈后患者对病中细节可能有部分遗忘。

\subsubsection{慢性脑病综合征}

慢性脑病综合征主要由慢性躯体疾病引起或发生于严重躯体疾病之后;可由急性脑病综合征迁延而来,亦可缓慢发病;不伴有意识障碍。主要表现为:

1.类精神分裂症状态 患者多表现为持续性幻觉妄想状态,虽幻觉妄想内容比较接近现实,但亦可出现某些精神分裂症特征性症状,如评论性幻听、假性幻觉和内心被揭露等。

2.抑郁状态 是多种慢性躯体疾病如慢性肝炎、肺结核、系统性红斑狼疮、高血压等常见的精神障碍。患者表现为情感低落,失去生活乐趣和信心,对前途悲观绝望,躯体疲倦,饮食、睡眠障碍,轻者有自杀观念和企图,重者有自杀行为。

3.类躁狂状态 多见于某些药物如皮质激素、阿的平急慢性中毒和甲状腺功能亢进。患者可表现为协调性言语运动性兴奋,如话多、活动多、情绪易激动、好打抱不平,但情感高涨、思想奔逸及对外界的感染力多不明显或不典型。

4.人格改变 在慢性长期躯体疾病的影响下,患者原来的人格特点逐渐发生变化。主要表现为行为模式和人际关系出现显著而持久的改变,如情绪不稳;心境可由正常突然变得忧郁、焦虑或易激惹;或反复暴怒发作或攻击行为,与所遇的诱发因素明显不相称;或明显情感淡漠,对周围事物不关心;或社会性判断明显受损,如性行为轻率,行为不顾后果;或偏执、多疑。

5.认知障碍 严重躯体疾病引起脑部弥漫性病理改变,导致认知功能减退,或患者经较长时间昏迷,醒后可表现为轻重不等的认知障碍。此认知障碍通常具有不可逆性、慢性或进行性的性质。患者意识多是清楚的,也可合并谵妄。表现为抽象概括能力明显减退,如不能解释成语、谚语,掌握词汇量减少,不能理解抽象意义的词汇,难以概括同类事物的共同特征。判断能力明显减退,对于同类事物之间的差别不能作出正确判断。轻度认知障碍,包括记忆损害,学习和集中注意力困难,短程记忆缺损,对新近发生的事件常有遗忘,远程记忆相对保存,这是认知障碍的另一重要表现。认知障碍常常影响患者的社会适应能力和日常生活。轻者仅表现工作、学习和社交活动能力下降,尚能保持独立的生活能力。中度障碍者进食、穿衣、大小便尚可自理,其余活动则需他人帮助。重者生活完全不能自理,痴呆,变成“植物状态”。

6.遗忘综合征 是一种以短程记忆和远程记忆缺损为突出表现的综合征,其瞬间记忆保持。虚构是本综合征另一特点,但并不一定均存在。无意识障碍,其病程和预后取决于原发病的病程,但上述症状至少持续一个月方可诊断。

\subsection{临床特点}

各种躯体疾病引起的精神障碍正如1909年K
Bonhoeffer指出“脑组织对各种外源性有害因素都反映出类似的精神病理现象”,多无特征性症状,即不同的躯体疾病所致的精神症状大致相同,一般具有以下特点:

1.精神症状一般多发生于躯体疾病高峰期,大多以精神症状为首发,如肝性脑病、系统性红斑狼疮,精神症状的出现往往先于其他躯体症状。

2.精神症状多与躯体疾病的严重度平行,即躯体疾病严重时精神症状亦明显,待躯体疾病好转后精神症状亦减轻。

3.精神症状多具有昼轻夜重的波动性及随着躯体症状的轻重而多变。

4.病程和预后主要取决于原发躯体疾病的状况及处理是否得当。一般精神障碍持续的时间较短,预后较好,但如患者曾经长期陷入昏迷,可遗留人格改变或智能减退。

5.躯体疾病所致精神障碍的患者除表现为明显的精神症状外,多伴有躯体和(或)神经系统的病理体征及实验室的阳性发现。

\subsection{诊断与鉴别诊断}

\subsubsection{诊断}

按照我国制定的精神障碍分类及诊断标准,诊断躯体疾病所致精神障碍须符合以下诸条标准:

1.从病史、体检(包括神经系统检查)、实验室和其他辅助检查可以找到躯体疾病的证据;并能确定精神障碍的发生和病程与躯体疾病相关。

2.精神障碍表现为下列综合征之一

(1)认知障碍;

(2)遗忘综合征;

(3)意识障碍;

(4)人格改变;

(5)精神病性症状(包括幻觉、妄想、紧张综合征、思维障碍和行为紊乱);

(6)情感障碍(包括抑郁和躁狂状态);

(7)脑衰弱综合征;

(8)以上症状的混合状态或不典型表现。

3.严重程度符合下述标准之一

(1)现实检验能力减退;

(2)社会功能减退。

\subsubsection{鉴别诊断}

1.脑器质性疾病所致精神障碍 原发病在脑部,可查知明显的脑部病理改变,如颅脑CT、MRI、脑脊液检查等阳性所见及突出的定位性神经体征。

2.伴发的功能性精神病 躯体疾病所致的慢性精神障碍的类精神分裂症状态、抑郁状态、类躁狂状态有时难以从临床症状与功能性精神病鉴别,但掌握疾病过程及阳性躯体征及实验室所见可以鉴别。

\subsection{治疗原则}

1.针对病因治疗,积极治疗原发躯体疾病。

2.精神症状的控制根据具体临床表现可给予小剂量不良反应轻的抗精神病药、抗抑郁药及抗焦虑药。如为意识障碍则以支持疗法为主,如表现出明显的躁动不安,可适当给予异丙嗪注射或口服治疗。

3.支持疗法 ①保证营养、水分,维持电解质及酸碱平衡;②改善脑循环;③促进脑细胞功能的恢复,如给予能量合剂等。

4.护理 躯体疾病的护理至关重要。环境和心理护理有助于消除患者的恐惧、焦虑情绪,对有意识障碍的患者要特别注意安全护理,以防其自伤、摔倒、冲动的意外发生。对有抑郁心境的患者应警惕其自杀企图,给予预防。

\subsection{病程和预后}

躯体疾病所致精神障碍的病程和预后取决于原发躯体疾病的性质、病程和治疗以及对精神症状的处理是否恰当。一般来说,急性精神障碍如意识障碍、急性精神病性症状持续时间较短,预后良好。但若躯体疾病恶化,上述障碍则转入昏迷。慢性精神障碍如智能障碍、人格改变则迁延时间至少达2~4个月。

\section{躯体感染所致精神障碍}

躯体感染所致精神障碍是指由病毒、细菌、螺旋体、真菌、原虫,或其他微生物、寄生虫等所致的脑外全身性感染,如败血症、梅毒、伤寒、斑疹伤寒、恶性疟疾、血吸虫病、人类免疫缺陷性病毒(HIV)感染等所致的精神障碍。颅内未发现直接感染的证据。

\subsection{病因及发病机制}

\subsubsection{病因}

多为外界病毒、细菌、螺旋体、真菌、原生虫及寄生虫等侵入机体引发疾病。

\subsubsection{发病机制}

精神障碍的发生与上述病原体进入机体发生作用有关,但尚有其他因素参与。以下诸点具有重要意义:

1.病毒、细菌的毒素对脑组织造成直接损害。

2.由于疾病而使代谢亢进,造成中间代谢产物在颅内蓄积。

3.急性感染时造成暂时性脑缺氧和脑水肿。

4.由于感染引起机体高热、大量出汗,患者不能正常进食而致体力消耗、营养缺乏、衰竭,能量供应不足,以及酸碱失衡、电解质紊乱,影响脑功能活动。

5.个体差异,如高龄者、儿童、既往体质不强者,在躯体感染时易发生精神障碍。

在上述诸因素综合作用中,感染的性质(如病原体对大脑细胞的亲和力)、程度、速度、病原体的数量、作用时间以及抗感染措施是否得力,对精神障碍的发生有着关键性的作用。

\subsection{临床表现}

感染所致精神障碍的临床表现是根据急性感染还是慢性感染而定。急性感染多导致急性精神障碍,以意识障碍为主,慢性感染则多见于慢性精神障碍,如类精神分裂症状态、抑郁状态、类躁狂状态、人格改变以及智能障碍。

几种常见的感染疾病所见的精神障碍:

\subsubsection{流行性感冒所致精神障碍}

流行性感冒为流感病毒引起的急性传染性呼吸道疾病。由于流感病毒对中枢神经系统具有很强的亲和力,大多导致精神障碍。有报道,发病率为25%~35%。主要临床表现为:前驱症状为头痛、衰弱无力、疲乏、睡眠,醒觉节律紊乱,继之表现有嗜睡、感知障碍、非真实感。高热时或重症病例可出现意识障碍,如意识蒙眬甚至谵妄。随着病情好转而进入恢复期,此时主要表现可见抑郁状态和脑衰弱综合征。少数病例可出现脑炎症状,病程较短,一般预后良好。

\subsubsection{肺炎所致精神障碍}

急性肺部感染,在疾病高峰期可以出现意识障碍,多见意识模糊,有时发生谵妄。慢性肺部感染如肺结核,则主要表现为抑郁状态,伴记忆力减退、注意力集中困难及思维迟钝。

\subsubsection{疟疾所致精神障碍}

普通型疟疾在其高热阶段可出现意识恍惚、定向力障碍及感知障碍。恶性疟疾,或称脑型疟疾,其疟原虫具有毒力强、亲神经的特点,可形成脑部病理变化,如灶性坏死、出血和脑水肿等。见于疟疾流行区或免疫力差的患者。精神症状主要表现剧烈头痛伴恶心、呕吐,烦躁不安,继之表现意识障碍如蒙眬或谵妄状态,甚至昏迷。此时神经系统可查出明显的病理征或表现为抽搐或瘫痪,患者表情淡漠。恢复期患者表现为抑郁状态或脑衰弱综合征。重症病例在后期可发生智能障碍。

\subsubsection{流行性出血热所致精神障碍}

流行性出血热为一种流行于秋、冬季节的急性传染病。病原可能是病毒,其发病机制尚不清楚。主要表现为发热、出血。临床分为发热期、低血压期、少尿期、多尿期和恢复期。有研究报道,在173例出血热患者中有53例(占30.6%)出现中枢神经系统症状,全部表现有精神障碍,病理解剖可见脑表面和脑实质内充血、血管扩张和坏死灶。精神症状多发生于低血压期和少尿期,主要表现为昏睡、谵妄、昏迷,或兴奋、躁动不安,持续1~2周。同时可出现明显的神经系统症状和病理征。

\subsubsection{狂犬病所致精神障碍病}

因为狂犬病毒通过带病毒的狗或其他动物咬伤或抓伤人体,而侵入机体。在大脑皮质和基底神经节可发现Negri小体。临床表现分猛烈型及瘫痪型两种。初期患者感头痛、不安、低热、愈合的伤口出现痛痒或麻木等异常感觉。2~3天后猛烈型者表现为恐水、恐风、恐光,水、风、光均可激惹反射性咽喉痉挛发作。患者紧张不安、恐惧、烦躁。病情逐渐加重,并有全身痉挛、颈强直、唾液分泌增多,高热,出现心力衰竭、呼吸麻痹。治疗无效可突然死亡。瘫痪型主要表现为肢体瘫痪、昏迷而死亡。

\subsubsection{艾滋病所致精神障碍}

艾滋病亦称获得性免疫缺陷综合征(acquired immunodeficiency syndrome,
AIDS),主要通过性接触传染,也可由血液和母婴传播。若病毒侵及中枢神经系统可出现神经精神症状。有30%~40%的艾滋病患者出现中枢神经病理学改变:神经元减少、脑萎缩、神经液质结节和小灶性脱髓鞘。疾病初期患者多受社会心理因素影响而表现为焦虑、抑郁状态,有的患者表现突出的抑郁、自杀倾向。随着病情的恶化,患者出现痴呆状态,如健忘、迟缓、注意力不集中,解决问题的能力下降和阅读困难,表情淡漠、主动性差、社会退缩。躯体症状表现为昏睡、畏食和腹泻,并导致体重明显下降。有的患者出现疼痛发作、缄默和昏迷。艾滋病目前已成为世界各国关注的公共卫生问题,尚无较好的治疗办法,可试用抗病毒药和免疫增强剂。关键是普及有关科学知识,严格管理血液制品和严肃性生活,以预防为主。

\subsection{诊断与鉴别诊断}

诊断要点为确定感染依据。鉴别诊断着重于与非感染性器质性精神病及伴发的功能性精神病鉴别。详见本章第一节有关部分。

\subsection{治疗}

查出病原,针对病原进行系统的、积极的抗感染治疗和中西医结合治疗。其余治疗原则参考本意第一节有关部分。

\subsection{预后}

本病预后取决于感染性质、躯体疾病的严重度和有效治疗。

\section{内脏器官疾病所致精神障碍}

内脏器官疾病所致精神障碍是指由重要内脏器官如心、肺、肝、肾等发生严重疾病,继发脑功能紊乱所致的精神障碍。精神障碍的严重程度随原发疾病的严重程度而波动。

\subsection{病因与发病机制}

病因主要是由于重要内脏器官受到各种有害因素的作用而发生病变。内脏器官的病变直接影响了其重要的生理功能,造成机体循环、代谢障碍及水与电解质紊乱和酸碱不平衡,导致脑供血、供氧不足及代谢产物蓄积,而发生精神障碍。

\subsection{临床表现}

\subsubsection{肺性脑病}

肺性脑病又称肺脑综合征,是一种重度肺功能不全所致的精神神经障碍。凡能引起严重肺功能不全的因素如慢性支气管炎、肺纤维化、肺结核以及神经肌肉疾病造成的呼吸肌麻痹症,都可引发此类障碍。由于肺功能不全导致动脉低氧血症而引起脑缺氧是肺性脑病主要的发病机制。临床表现:前驱症状可有头昏、耳鸣、不安、淡漠等;数小时至数天后出现间歇性意识障碍,由嗜睡状态渐加重至朦胧状态、谵妄状态,重症者可进入昏迷。神经系统表现有头痛、癫痫
发作、震颤、扑翼样震颤、不自主运动、锥体束征、颅压增高。动脉血二氧化碳分压增高,氧分压偏低,pH值降低。脑电图为弥漫性Q波及δ波。治疗原则为积极控制肺部感染,保持呼吸道通畅,改善缺氧状态,纠正酸中毒,减轻或消除脑水肿。控制精神症状时用药宜慎重,禁用麻醉药和催眠药,慎用酚噻嗪类制剂。兴奋患者必要时可给予适量地西泮口服或注射。注意呼吸中枢被抑制。

\subsubsection{肝性脑病}

肝性脑病又称肝脑综合征,是由严重肝病如暴发性肝炎、亚急性肝炎、慢性肝炎、肝硬化及肝癌后期所致的精神障碍。肝为机体的重要解毒器官,一旦肝受损,其功能严重失调,致使氨基酸代谢紊乱,血氨及脑脊液中的氨增多,其他各种中间代谢产物积聚,是导致精神障碍的重要机制。此外,酪氨酸、蛋氨酸和色氨酸代谢障碍、中枢单胺递质代谢紊乱等都将影响大脑功能而发生精神障碍。临床表现早期可见患者迟钝、少动、寡言或兴奋躁动不安,继而出现嗜睡、蒙眬、谵妄状态,最终进入肝昏迷。慢性肝炎、肝硬化引发的精神障碍多缓慢进展,少有意识障碍,多出现人格障碍和智能障碍。神经系统主要表现为扑翼样震颤、肌阵挛、肌张力增高、病理反射和痉挛发作。躯体检查可发现肝大、腹水、黄疸,实验室检查血氨增高以及脑电图的θ波基本节律等,有助于诊断。治疗原则为对原发病积极治疗。静脉滴注谷氨酸钠或精氨酸有助于患者恢复意识和改善精神症状。应禁用麻醉药、催眠药和酚噻嗪类药物。如患者兴奋,必要时可给予水合氯醛、副醛、地西泮或氯硝西泮;氯硝西泮可以肌内注射或静脉滴注,对兴奋、躁动疗效好。

肝豆状核变性,是一种铜代谢障碍的隐性遗传病。主要的病理生理变化是血浆铜蓝蛋白减少,导致铜沉积在豆状核、肝脏、角膜和肾脏上。可以出现锥体外系症状,如肌痉挛和肌张直。此外,在早期就可以出现精神症状。早期起病进展较快,青少年期以后起病疗程多迁延。精神症状可以多种形式出现。临床诊断可根据角膜K-F环(Kayser-Fleischer
ring),尿和粪铜排泄量增加以及血浆铜蓝蛋白减少确诊。

\subsubsection{心源性脑病}

心源性脑病又称心脑综合征,是由各种原因的心脏病如冠状动脉硬化性心脏病、风湿性心脏病、先天性心脏病等所致的精神障碍。其发病机制与各种心脏疾病引起的心排血量减少、血压下降致使脑血流量减少,脑供血不足有关。各种心脏病所致的精神障碍均可出现脑衰弱综合征。当有心力衰竭、心绞痛发作、心肌梗死以及发作性心动过速时,患者常常表现为焦虑、抑郁、恐惧或易激惹。重症病例或风湿活动期会发生程度不等的意识障碍。治疗原则为积极治疗心脏病,可用一些具有活血化瘀作用的中成药改善脑循环。如出现兴奋或幻觉妄想状态可采用小剂量氟哌啶醇每日4~8mg,分次口服,或5mg肌注。

\subsubsection{肾性脑病}

肾性脑病又称尿毒症性脑病,是由于各种原因引起慢性肾衰竭或急性肾衰竭导致的精神障碍。发病机制尚不清楚。肾为机体主要排泄器官,其功能受损,势必导致内毒素的积蓄,对中枢神经系统产生有害作用,导致脑内葡萄糖和氧代谢障碍。尿毒症时,体内电解质紊乱、酸中毒,血液和脑脊液通透性增强,以致颅内压增高,发生脑水肿。以上机体和脑内的种种病理生理改变与精神障碍的发生有关。精神症状早期主要表现为脑衰弱综合征,部分患者可有被害性幻觉妄想,或类躁狂样表现,或为抑郁状态。在慢性进行性肾衰竭时,患者可出现记忆减退、智能障碍。当肾衰竭严重时,可有由轻而重的意识障碍,最后出现昏迷。神经系统可见全身性痉挛发作、扑翼性震颤、瘫痪等。脑电图检查具有诊断参考和判断预后的意义:基本节律变慢,广泛慢波,额部阵发性慢波,或癫痫
发作波;当肾衰竭好转时,脑电波亦有改善。积极预防和治疗原发病,预防肾衰竭至关重要。用药宜慎重,以防药物在体内蓄积。患者兴奋不安时,可给予地西泮和水合氯醛。

透析疗法是治疗急慢性肾功能不全的有效办法。但在透析治疗过程中,血内尿素氮浓度急剧下降而脑脊液和脑组织内尿素氮等下降缓慢,脑脊液渗透压高于血液渗透压,最终引起颅内压增高和脑水肿而出现精神障碍,称为透析性脑病,或称平衡失调综合征(disequilibrium
syndrome)。主要表现为兴奋、精神错乱、昏迷和痉挛发作,可伴有头痛、恶心、呕吐。因此在透析治疗中应重视躯体并发症的防治。一旦发生,应积极处理,纠正水、电解质紊乱和代谢性酸中毒。可采用小剂量精神药物和抗痉挛药物治疗精神症状和痉挛发作,并注意药物的不良反应。

\subsection{预后}

内脏器官所致精神障碍的预后取决于原发疾病。一般经恰当治疗可有短期缓解,最终预后不佳。

\section{内分泌疾病所致精神障碍}

内分泌疾病引起内分泌功能亢进或低下可导致精神障碍。

\subsection{病因和发病机制}

各种原因所致的内分泌疾病,其发病机制尚不完全清楚。一旦某一内分泌器官发生病变引起内分泌紊乱,将影响中枢神经系统的功能,而发生脑代谢障碍和弥漫性脑损害,出现精神症状。

\subsection{临床表现}

\subsubsection{甲状腺功能异常时的精神障碍}

1.甲状腺功能亢进 早期表现为情绪不稳、紧张、过敏、急躁、易激动、失眠、注意力不集中。随着病情发展多见类躁狂状态,而老年患者则多表现焦虑抑郁状态,也可见幻觉妄想状态。甲状腺危象时,则以意识障碍为主,表现为嗜睡、昏睡、谵妄甚至昏迷,同时伴有高热、多汗、震颤,显示明显的甲状腺中毒症状。此外,由于躯体疾病恶化或心理因素的作用可急剧出现谵妄或精神错乱状态。甲状腺功能亢进时,除上述精神症状外还可见重症肌无力、周期性瘫痪、舞蹈样动作、帕金森综合征及癫痫
样发作等。治疗原则为积极治疗原发病,控制甲状腺功能亢进,防止感染。对症治疗精神症状,配合心理治疗。

2.甲状腺功能低下 由于甲状腺素分泌不足或缺乏引起躯体、发育和精神障碍。其功能减退始于胎儿期或新生儿期,以智力发育低下和躯体发育矮小为特征,称呆小病。功能低下始于儿童或成年称为黏液性水肿,可表现为智能障碍、抑郁或躁狂状态、幻觉妄想状态。病程较长的老年患者常在冬季出现意识障碍,主要表现昏迷状态,死亡率可达50%。甲状腺功能低下可有听力减退、构音障碍、视神经萎缩、步态不稳、共济失调、眼球震颤、面瘫、神经炎及痉挛发作等多种神经系统症状。甲状腺素治疗具有良好效果。应避免受寒、感染。慎用抗精神病药物及麻醉药和催眠药,因患者对药敏感易诱发昏迷。

\subsubsection{脑垂体功能异常时的精神障碍}

1.脑垂体前叶功能亢进所致精神障碍 是由于脑垂体前叶各种生长激素分泌过多引起躯体、精神、神经等障碍。患者表现肢端肥大、巨人症。精神障碍主要为人格改变、情绪不稳、易怒、焦虑不安和(或)迟钝、少动、寡语,可见妄想状态和抑郁、躁狂状态以及以领悟困难、反应迟钝和思维贫乏为特点的痴呆状态。患者有时表现嗜睡状态。神经系统可查知视野缩小、视力模糊、视乳头水肿。

2.脑垂体前叶功能减退所致精神障碍 由于垂体前叶的炎症、肿瘤、坏死或手术引起垂体功能减退、激素分泌减少,导致性腺、甲状腺及肾上腺皮质等继发性功能减退,发生内分泌系统与神经系统的相互调节紊乱,出现躯体、精神、神经症状。由分娩大出血引起的原发性垂体前叶功能减退,称为席汉病(Sheehan's
disease)。临床表现可见闭经、性欲减退、乳房和生殖器萎缩、阴毛和腋毛脱落等躯体症状。精神障碍主要表现为早期的脑衰弱综合征。急性发病者出现意识障碍,或幻觉妄想状态和抑郁状态。部分患者可渐发展为慢性器质性脑病,出现淡漠、懒散、迟钝等人格改变。神经系统症状有痉挛发作、肌阵挛、手足徐动等。应用肾上腺皮质激素、甲状腺素、雌激素等替代疗法可改善精神症状。慎用地西泮、奋乃静,禁用氯丙嗪,以防止休克或昏迷的发生。

\subsubsection{肾上腺应激功能异常所致精神障碍}

1.肾上腺皮质功能亢进所致精神障碍 肾上腺皮质功能亢进,皮质激素分泌过多,影响儿茶酚胺代谢,导致精神障碍和神经系统症状。肾上腺皮质醇增多症又名库欣(Cushing)综合征。精神症状主要表现为抑郁状态、幻觉状态,人格改变主要为情绪不稳、易激惹、好伤感哭泣以及类似老年痴呆的痴呆表现。神经系统症状和体征可见四肢肌肉无力或萎缩、震颤及痉挛发作。可选用适量精神药物治疗精神症状。

2.肾上腺皮质功能减退所致精神障碍 肾上腺皮质功能减退,皮质激素分泌不足引起的躯体、精神障碍,又名阿狄森病(Addison
disease)。该病的病因考虑为自身免疫性疾病引起原发性肾上腺皮质萎缩所致。精神障碍表现有:①易激惹,情绪不稳,时而欣快时而抑郁的情感障碍;②周期性的幻觉妄想状态(有人称为阿狄森精神病);③全面性痴呆状态,同时伴有性欲、食欲减退,睡眠障碍,烦渴,月经不调等。神经系统可见视力减退、复视、眩晕、痉挛等。在肾上腺危象发作时,多突然发生意识障碍,主要表现为谵妄、错乱甚至昏迷。主要以肾上腺皮质激素替代治疗。慎用酚噻嗪类药物。

\subsubsection{性腺功能异常所致的精神障碍}

由于生理和病理性原因引起性腺激素平衡失调,以致性腺功能异常引发精神障碍。神经系统与性腺功能关系密切,其发病机制复杂。临床常见生理性原因如月经、妊娠等引起的精神障碍,一般症状较轻,预后良好。

1.月经前期综合征 主要因雌激素和孕酮失衡导致精神障碍。临床表现为情绪不稳、抑郁、烦闷、焦虑、易激惹、疲倦感,偶见失神发作,伴乳房胀痛,腹胀,恶心、呕吐及性欲、食欲增强。治疗可采用促性腺激素以调整雌激素的平衡;用抗焦虑药物如安他乐、苯二氮䓬
类对症治疗,如抑郁症状明显宜用抗抑郁药。

2.妊娠期精神障碍 妊娠早期3~4个月丘脑、垂体-性腺内分泌系统处于动荡变化过程中,患者表现为情绪不稳、不安、易怒、敏感多疑和脑衰弱综合征。重症者可出现类躁狂状态伴意识障碍,躯体症状有血压降低、口渴、少尿等。妊娠中期内分泌系统稳定。到妊娠后期3个月内分泌系统重又发生变化,肾上腺皮质功能亢进,雌激素分泌增高,甲状旁腺功能减退。临床表现为脑衰弱综合征、抑郁状态,并发妊娠中毒症者可发生谵妄或精神错乱状态等意识障碍。分娩后或在产后2~3周症状消失。为避免药物对胎儿影响宜慎用药。重症者可酌情应用肾上腺皮质激素、甲状腺素、雌激素等激素治疗。精神症状可采用抗焦虑药物治疗。

\section{营养缺乏和代谢疾病所致精神障碍}

营养不良、某种(或多种)维生素缺乏、血卟啉病、糖尿病以及水、电解质平衡失调等营养缺乏和代谢疾病等可引起精神障碍。根据原发疾病(或障碍)的不同,其发病机制和临床表现有差异。

\subsection{营养不良和维生素缺乏所致精神障碍}

包括各种原因如饮食不当、疾病、手术、酒中毒等引起机体营养不良和维生素缺乏,导致酶活性障碍、新陈代谢紊乱、能量供应不足而出现躯体、精神神经障碍。

1.烟酸缺乏症 又称糙皮病,是由于烟酸(维生素B2)缺乏导致脑垂体细胞、基底神经节以及脊髓前角细胞发生广泛性变性引起精神障碍。轻者表现为精神萎靡、谵妄等脑衰弱综合征。急性起病者可出现发热、腹泻、意识模糊、昏睡、昏迷,死亡率高,称为烟酸缺乏性脑病。慢性患者反应迟钝,有智能障碍,重者呈痴呆状态。疾病过程中可有幻觉、妄想、木僵、焦虑、抑郁等症状。躯体症状则皮炎、腹泻明显。神经系统可查出瞳孔改变、眼球震颤、锥体束征和癫痫
发作。皮炎、腹泻、痴呆被称为烟酸缺乏症的三主征。治疗原则为大量补充烟酸或烟酰胺,每次200mg,每日3次,同时补充其他B族维生素和维生素C;给予糖、蛋白质等营养支持疗法很重要。精神症状不需特别治疗。

2.低血糖所致精神障碍 血糖低于2.78~3.05mmol/L为低血糖。由于自主神经系统功能障碍、胃部手术可引起功能性低血糖;胰岛B细胞瘤、肝脏疾病引起器质性低血糖。由于血糖低导致脑内葡萄糖量供应减少,脑内氧化过程障碍,出现脑缺氧。低血糖发作时,临床上可表现为烦躁不安、头昏、恐惧、焦虑、易激惹,随之表现注意力不集中、记忆减退、计算困难,伴兴奋、躁动、意识障碍如谵妄,甚至昏迷,补糖后上述症状缓解。如发作次数频繁或长时间发作可见患者情感淡漠、人格改变以及智能障碍,重者痴呆。治疗原则:查找引起低血糖的原因予以积极治疗。低血糖发作时应立即静脉注射高渗葡萄糖。发作时要避免应用抗精神病药物,以免引发昏迷。如疑为低血糖发作,应即进行快速检测血糖,以明确诊断和进行有效治疗。

\subsection{代谢疾病所致精神障碍}

1.糖尿病所致精神障碍 糖尿病因胰岛分泌胰岛素不足引起糖、蛋白质、脂肪代谢紊乱,导致酮中毒,并发动脉硬化和微血管病变导致脑供血不足,出现精神障碍、躯体和神经系统症状。临床表现为疲倦、无力、失眠等脑衰弱综合征、抑郁状态、焦虑状态和幻觉状态。当糖尿病病情恶化或血糖水平明显升高时,可出现嗜睡、精神错乱、昏迷以及多发性神经炎、眼底改变、肌萎缩及腱反射减低等神经系统常见症状。治疗原则为控制糖尿病的发展。

2.血紫质病(血卟啉病)所致精神障碍 血紫质病可能与先天性酶缺陷有关,多发生于青年。患者d-氨基酮戊酸合成酶活性增强,尿卟啉原Ⅱ合成酶或尿卟啉原Ⅲ辅合成酶缺乏,产生多量卟啉和卟啉前体。后者可导致中枢神经、周围神经、自主神经病变和腹痛。扁桃体炎、感冒、饮酒、应用某些药物,或在妊娠与产后容易诱发精神障碍。急性期精神障碍主要表现为抑郁状态、幻觉妄想状态、精神分裂样精神病,或意识混浊伴高度兴奋。急性精神障碍迁延常表现为木僵、嗜睡等状态。间歇期则多表现为癔病样痉挛发作,急躁、易怒、孤独、少动等。神经系统症状可见动眼神经、面神经麻痹及延髓性麻痹、瘫痪或癫痫
样发作。上述精神障碍常被误诊为功能性精神病或神经症。查尿紫胆原增加可确诊。治疗原则:①应用肾上腺皮质激素、雄激素治疗,如发作与月经周期有关,可用口服避孕药治疗;②急性期静脉滴注葡萄糖;③应用氯丙嗪、奋乃静等药治疗精神症状。患者死亡率较高,多死于延髓性麻痹。

\subsection{水及电解质紊乱所致精神障碍}

临床上经常遇到各种原因所致的水及电解质紊乱,出现精神症状和意识障碍。其发病机制复杂。

1.脱水症(高渗综合征) 高热、躯体疾病、意识障碍、进水困难;或机体排水量过多,或给患者高渗糖、盐、蛋白质治疗,引起脱水、血钠升高(\textgreater{}150mmol/L),导致精神、躯体障碍。临床表现为兴奋状态、幻觉或意识障碍,甚至昏迷,伴口渴、口腔干燥、尿少。

2.水中毒(低渗综合征) 由于手术后、脑垂体前叶功能减退、肾上腺皮质功能减退等引起抗利尿激素分泌过多,或长期大量使用利尿药、急性肾衰竭和糖尿病酸中毒引起水潴留、血钠降低(\textless{}120mmol/L),导致脑水肿、颅内压升高,出现精神、神经和躯体障碍。临床表现为抑制状态,情感淡漠、言语减少、运动缓慢及不同程度的意识障碍,如意识混浊、谵妄和昏迷,伴头痛、肌阵挛、抽搐及恶心、呕吐、食欲减退、乏力等症状。

3.高钾血症 急性肾衰竭、糖尿病、烧伤、代谢性酸中毒等原因可致血钾升高,超过5.5mmol/L,可出现精神、神经和躯体障碍,临床表现为兴奋状态、躁动不安、情绪不稳以及嗜睡和意识障碍,可见弛缓性瘫痪,重者有呼吸肌麻痹。

4.低钾血症 肠梗阻,肝、肾手术后,肾衰竭,呼吸性或代谢性碱中毒以及激素和利尿药过度使用,抗精神病药物治疗中可出现血钾低于3.5mmol/L,导致精神障碍。临床主要表现为抑制状态、木僵和抑郁、嗜睡和谵妄。神经系统症状为周期性瘫痪、四肢肌无力和弛缓性瘫痪。

水及电解质紊乱所致精神障碍的治疗原则为:①查找病因,积极治疗;②调整和恢复水及电解质代谢的平衡;③慎用药物以防意识障碍的发展。

\section{系统性红斑狼疮所致精神障碍}

系统性红斑狼疮为结缔组织疾病中最常见的一种。此病损害常累及中枢神经系统,表现有精神症状者占20%~30%。

\subsection{病因和发病机制}

系统性红斑狼疮被认为是自体免疫性疾病,机体多系统受损。精神障碍的发病机制复杂,与下列脑部病理生理变化有关:①脑部血管壁和脉络丛可查出免疫复合物和γ-球蛋白的沉淀;脑脊液中补体水平下降;血脑屏障存在DNA和抗DNA抗体免疫复合物,使淋巴细胞毒得以通过血脑屏障,引起脑部广泛性损害。②自身抗体中有抗脑细胞原生质抗体,直接损害中枢神经系统。③由于系统性红斑狼疮损害心、肝、肾等重要脏器,继发严重的并发症可引起脑功能损害,出现精神症状。

\subsection{临床表现}

女性多见,青壮年发病。本病早期或恢复期表现为脑衰弱综合征。严重病例可见各类意识障碍的表现,甚至昏迷。病情进展迅速,经治疗有病情恢复者。慢性迁延性病例多见类功能性精神病样表现,如类分裂症状态类、类抑郁状态和类躁狂状态。后者需与皮质激素治疗中的中毒反应鉴别。神经系统症状:因颅内压增高引起全身痉挛发作、偏瘫、失语、舞蹈样不自主动作。躯体症状为受累脏器的生理功能改变的表现。多见脑电图异常,异常率为60%~80%,以慢波为主。脑电图的变化与中枢神经系统症状的消长呈平行关系。

\subsection{诊断与鉴别诊断}

需确诊系统性红斑狼疮,精神症状多在疾病高峰期出现,且随躯体症状的改善而缓解。实验室检查:荧光抗核抗体阳性对确诊有意义。精神症状和躯体症状常与其滴度呈正比。本病需与躯体疾患伴发的心因性反应鉴别。

\subsection{治疗和预后}

肾上腺皮质激素是本病的主要治疗措施。病情危重时进行冲击治疗和椎管给药治疗。免疫抑制药如环磷酰胺、硫唑嘌呤与激素合并使用可使病情缓解。出现精神症状或抑郁症状可慎用抗精神病药物和抗抑郁焦虑治疗。每次发作的预后随着治疗的进展而趋于乐观。应尽力防止系统性红斑狼疮的反复发作和病情迁延,避免和预防诱发因素如暴晒、感染,避免不恰当用药,如抗结核药、磺胺类药、抗痉挛药、抗精神病药以及抗生素等,多易诱发红斑狼疮样反应。
