\chapter{自杀与危机干预}

\section{概  述}

自杀(suicide)的历史几乎与人类历史同步,远古时期就有自杀现象存在。但真正把自杀作为社会现象和病理现象科学地加以研究,是从19世纪开始的。基于以下几方面的重要原因,医学生应对自杀有所了解和进行研究。一是自杀已成为当今社会严重的公共卫生问题,在死亡谱中位于第八位,位于年轻人死亡原因的第二位,据统计,2000年全球约有100万人采取自杀行为,中国的年自杀率22/10万;二是自杀不仅使自杀者本人死亡或造成残疾,而且给社会和自杀者家庭带来严重的负担和影响,自杀的负担占全球总疾病负担的1.8%,每一个自杀死亡或自杀未遂者至少有6位亲近的人受到影响;三是自杀与精神疾病的关系密切,通过对自杀的研究不仅可以更好地干预自杀,而且会对精神疾病的认识和防治起到积极的促进作用。

\section{自杀的定义与分类}

\subsection{定义}

不同的学科对自杀有不同的定义,这主要是因为各学科研究的出发点和目的不同。生物学派强调的是生物学因素在自杀中所起的作用,认为自杀是“一种先天的遗传性行为模式”。心理学从心理动力学或应激角度出发,将自杀定义为“是对某一客体的一种不自觉的敌对”或“自杀是一种急性的实在的情景应激反应”。社会学着眼于社会文化因素,认为自杀是社会整合力异常,作用于个体的结果。从社会学角度讲,人类许多导致躯体伤害的不良嗜好,如酗酒等也属于慢性自杀范畴。一般说来,多数学者都认同这样的观点,即自杀是对个体行为的动机及行为的控制能力的反映,是个体主动行为的结果。如《大不列颠百科全书》将自杀定义为:自杀是一个人自愿并故意地伤害自己生命的行动。Kaplan等在《精神病学概要》中认为:自杀是有意自我伤害导致的死亡。中国精神障碍分类与诊断标准第三版(CCMD-3)将自杀定义为:故意采取的自我致死的行为。因此,从这样的角度看,自杀是在意识清晰的前提下有目的的主动行为,意识障碍者的自伤或自杀显然不属于真正的自杀。

\subsection{分类}

在不同的自杀定义下,有许多不同的自杀分类方法。19世纪法国社会学家涂尔干根据社会对个人关系的影响以及控制力的强弱,把自杀分为三类:①利他性自杀(altruistic
suicide):是指在社会习俗或群体压力下,或为追求某种目标而自杀;②利己性自杀(egotistic
suicide):是指个人失去社会的约束与联系,对身处的社会及群体毫不关心,孤独而自杀;③失范性自杀(anomic
suicide):是指个人与社会固有的关系被破坏,令人不知所措而自杀。CCMD-3将自杀分为:①自杀死亡(committed
suicide):指死亡的结局系故意采取自我致死的行为所致;②自杀未遂(attempted
suicide):有自杀的动机和可能导致死亡的行为,但未造成死亡的结局;③自杀观念(suicide
idea):有自杀意念,而未采取自杀的行动;④准自杀(pa-rasuicide):又称类自杀(也有学者称为蓄意自伤),有自伤的意愿,但并不真正想死,采取的行为导致死亡的可能性很小,可以是一种呼救行为或威胁行为,通常不会造成死亡。尽管按照自杀的严格定义准自杀不能称之为自杀,这些人的真正意图不是死亡,但在以后的一段时间内其自杀率远高于普通人群,因此不应忽视。

\section{自杀流行病学调查}

近年来统计资料表明,全球自杀人数有逐年增加的趋势。自杀水平一般是以自杀率作为指标,以10万人为基数。据世界卫生组织统计,1950~1960年,全球30多亿人口中,自杀率为10/10万。1990年世界卫生组织公布的30个国家自杀率表明,自杀率呈上升趋势。自杀率最高的是匈牙利,自杀率高达44.9/10万,其次为丹麦,为31.57/10万。自杀率较低的国家是冰岛、西班牙和希腊等国,每10万人口低于5人。美国的自杀率较低,为11.5/10万,位于死亡原因的第八位。我国2000年统计表明,自杀率为22/10万。对于自杀未遂的发生率,由于资料来源的困难和可靠性等原因,报道不尽相同,但估计不会低于自杀死亡者,的10多倍。据有关研究结果,一般人群自杀终身发生率为1%~12%。

调查资料显示,男女之间自杀率有区别。但国内与国外的情况截然相反。西方国家的研究资料表明,男性自杀率高于女性,自杀死亡中,男女性别之比为3:1左右。而我国的调查结果是女性自杀率略高于男性,男女性别比为1:1.25。究其原因可能与不同的文化背景有关。

自杀有明显的年龄差异,突出的问题是青少年和老年自杀。一般来讲,自杀的年龄有两个高峰期,15~25岁是第一高峰期,随后下降;55岁后是第二个自杀高峰期。自杀是青少年最主要的死亡原因之一;老年人自杀率高于青壮年,并随着年龄的增长而增高。有研究发现,14岁以下儿童自杀死亡者罕见,但自杀未遂和自杀意念者并不少见。

不同职业的人群之间自杀率不同。一般认为,社会阶层较低者自杀率低于高社会阶层者。西方研究资料表明,从事专门职业的人员如医生、律师、作家及行政管理人员自杀率较高,特别是医生的自杀率远高于一般人群。其原因可能是高社会阶层者承受的社会责任和精神压力较大,在面临重大精神压力时,极力维护自己“社会强者”的形象而封闭自己的痛苦和孤独,很少寻找帮助,这种背景下很容易出现自杀倾向。医生的自杀率较高的另一原因可能是与容易接近药物有关。

人们结束自己生命的方式有多种,不同国家、不同职业、不同性别自杀者的自杀方法不尽相同。在英国伦敦服药自杀者占女性自杀者一半以上,占男性自杀者1/3。我国农村常见的自杀方法之一是服用农药。通常自杀者采用的自杀方式有:自缢、溺水、枪击、跳楼、自焚、制造交通事故等。男性与女性相比更多采用暴力的方法自杀。

\section{自杀的相关因素及自杀风险评估}

\subsection{相关因素}

尽管对于自杀是否是一种疾病,目前还有争论,但大多数人都认同,自杀是在风险因素和保护因素消长的影响下,个人素质和应激相互作用的产物,是心理、社会和生物等多种因素作用下形成的行为。

\subsubsection{自杀者的心理特征}

研究发现,自杀者通常在认知、情感、人际关系和社会交往等方面存在一些缺陷。自杀者常采用两分思维的方式,即“非此即彼”、“非好即坏”的思维分析处理问题,以偏概全,且常常将问题绝对化,更多的是看到事物的负面影响,心存敌意和偏见;对困难不能正确地估计,常用简单或粗暴的方式摆脱困境,行为具有冲动性和盲目性。在情感方面自杀者一般都伴有各种慢性的痛苦、焦虑、抑郁、内疚和愤懑等负性情绪。多数自杀者在不稳定、不够成熟的神经类型背景下,对负性刺激常以冲动的行为方式如自杀、自伤来发泄自己的情绪。与此同时,自杀者在社会交往、人际关系方面存在缺陷,通常因回避社交缺乏强有力的社会支持,对新环境适应也较困难。

\subsubsection{精神应激}

除精神疾病导致的自杀外,几乎所有的自杀者都可以追溯出自杀前存在相当大的负性生活事件,重大的负性应激事件可能是自杀的直接原因或诱因。与自杀有关的负性生活事件通常是:亲人去世,离婚,财产、社会地位及名誉受损,失业等。这些生活事件大多带有“丧失”的性质,自杀者对这些带“丧失”性质的应激无法应对,在得不到有效的社会支持的情况下,往往导致或触发自杀行为的产生。

\subsubsection{社会文化因素}

虽然自杀是个体行为,但社会文化因素在其中起着不可忽视的作用。这些因素包括:社会整合力、社会角色冲突、社会生活节奏、婚姻家庭制度、宗教信仰等。一般来讲,一个稳定、整合力强的社会,自杀率较低;家庭关系融洽、夫妻和睦者,自杀率较独身、离婚、丧偶或无子女者低。宗教信仰对自杀的影响是显而易见的,如伊斯兰教把自杀列为禁忌,信奉此教者自杀率低。值得注意的是,自杀还具有模仿性,德国作家歌德的名著《少年维特之烦恼》中主人公因失恋而自杀,该书1774年出版后,青少年模仿自杀的很多。日本著名的女歌星冈田有希子跳楼自杀后头10天内,有20多个人跳楼,两个月内有114名男女先后模仿自杀,日本学者称之冈田有希子综合征。

\subsubsection{躯体疾病}

躯体疾病,特别是慢性或难治性疾病如癌症、癫痫
、糖尿病等是自杀的重要触发因素。有资料表明,自杀死亡者中患有各种躯体疾病者占20%以上。一般认为躯体疾病对自杀的影响是疾病造成躯体极大痛苦、限制了正常的职业和社会交往、增加绝望感,从而导致或触发自杀的产生。

\subsubsection{精神疾病}

自杀与精神疾病的关系极为密切。一方面相当多的自杀死亡者可以诊断为各种精神疾病,另一方面相当多的精神疾病患者有自杀企图或自杀行为,有时自杀就是某一精神疾病的症状之一。

有资料表明,50%~90%的自杀者可以追溯诊断为精神疾病,其中以心境障碍最多见,其次是滥用精神活性物质、精神分裂症、人格障碍等。

精神疾病患者的自杀率远高于一般人群。抑郁症是自杀者中最常见的精神疾病,约2/3的抑郁症患者有自杀的意念,15%的抑郁症患者最终死于自杀。值得注意的是,最严重的抑郁症会因缺乏始动力和体力无法实施自杀计划,常常是在起病初期和抑郁发作末期出现自杀行为。约10%的精神分裂症患者最终死于自杀。此外,滥用精神活性物质者如酒依赖、海洛因依赖及人格障碍者自杀危险性是一般人群的5~10倍。

\subsubsection{遗传及神经生物学因素}

近年来,许多证据表明自杀与遗传及神经生物学因素有密切关系。Mann于1999年提出应激-脆弱性模式,认为自杀是应激和自身素质相互作用的结果。事实上,任何单一的因素可能都不足以引起自杀,只有在脆弱的素质背景下,自杀者遭遇自身无法应对的应激才有可能出现自杀行为。这也解释了为什么严重程度相似的抑郁症患者,或者遭遇同样的应激的不同个体,有些出现自杀行为,有些不出现自杀行为。目前自杀的神经生物学方面研究较多集中于5-羟色胺(5-HT)系统。

1.中枢5-HT的研究 中枢5-HT与自杀的关联性是通过对自杀者脑脊液中5-HT的代谢物5-羟吲哚醋酸(5-HIAA)的研究发现的。研究发现,自杀者脑脊液中的5-HIAA水平显著下降。自杀方法的致死性越高,脑脊液中5-HIAA水平减低越显著。自杀者中不论其精神疾病的诊断如何,这种相关性均存在,因而推测5-HIAA是自杀者素质的标志,并可作为今后预测自杀可能性的指标之一。D-芬氟拉明(D-FF)是选择性5-HT释放和再摄取抑制剂,其综合净效应是使5-HT释放增多,5-HT可促进催乳素(PRL)的分泌。抑郁症患者特别是自杀企图强烈的患者,D-FF激发的PRL分泌迟钝,说明5-HT功能障碍与自杀有关。

2.自杀死亡者脑组织的研究 研究发现,自杀死亡者脑部前额叶皮层5-HT受体密度增加,5-HT载体位点减少,受体结合力下降,揭示脑部前额叶皮层5-HT功能降低。脑部前额叶皮质与行为控制有关,此区的损伤可导致脱抑制行为,增加自杀的风险。

3.自杀相关基因多态性研究 大量研究显示,自杀具有家族倾向性。家系调查及双生子、寄养子等遗传学研究揭示了自杀行为具有遗传易感性。5-HT系统的自杀候选基因的研究主要从5-HT合成、失活、作用、转运等过程相关的酶或受体基因入手的,如色氨酸羟化酶(TPH)基因、5-HT转运体(5-HTT)基因、5-HT\textsubscript{1B}
、5-HT\textsubscript{2A}
受体基因。研究发现自杀与这些基因的多态性存在某种关联性。尽管这方面的研究目前还处于初级阶段,有些研究结果尚存在矛盾,但足以提示遗传因素在自杀中起一定的作用。

\subsection{风险评估}

一般认为,自杀行为是由其风险因素与保护因素之间的消长确定的。每一位医生都应对自杀的风险进行评估,通过对自杀风险因素的分析和评估,有助于识别自杀的高危人群,从而进行有效的自杀干预。

\subsubsection{自杀的风险因素}

自杀的风险因素主要指能激发自杀行为的急性应激(如分离、丧失、负性和创伤性事件、躯体和精神疾病复发或恶化等),人际关系不良,自杀倾向。具有下列特征的个体可视为自杀的高危人群:年龄大于45岁,独身、离婚或丧偶,无固定职业或失业者;患有精神疾病或慢性、难治性躯体疾病者;既往有自杀意念和自杀行为者;个体情绪不稳、行为冲动,事业无成就,人际关系不良,缺乏家庭温暖,社会系统较差者。

\subsubsection{自杀的保护因素}

保护因素是与愉快的源泉相联系的。就业或事业有成就感,家庭成员之间的情结,皈依宗教,参加群体活动等对个体都有保护作用;专业机构提供咨询和帮助以及危机干预都有支持作用。

\section{自杀的治疗和预防}

影响自杀的因素多种多样,因此,自杀的治疗和预防也涉及许多方面。其中动员社会各阶层共同识别、评定那些自杀的高危人群,纠正对自杀者的偏见和错误认识,提高全民心理素质,对自杀者提供及时有效的治疗,加强农药、精神药品、武器等管理是预防自杀的重要环节。

\subsection{治疗}

\subsubsection{心理治疗}

心理治疗主要包括对自杀高危人群的心理社会干预、自杀未遂者在综合医院救治时心理干预以及自杀死亡亲人的心理指导。

\subsubsection{抗抑郁药治疗}

自杀观念、自杀行为是抑郁的组成部分。抑郁症患者的自杀率极高,因此,对自杀者进行抗抑郁药物治疗是有效的措施之一。这些药物目前使用较多的是选择性5-HT再摄取抑制剂如氟西汀、西酞普兰、舍曲林等。其他新型抗抑郁药物如米氮平、文拉法辛等。

\subsubsection{心理稳定药物}

锂盐除保护抑郁症患者不再发作外,可通过加强中枢5-HT功能,对自杀者发挥有益的作用。

\subsubsection{电休克治疗}

抗自杀效应明显,3~5次电休克治疗后即可改善精神病患者的自杀行为。

\subsubsection{抗精神病药物}

精神病患者是自杀的高危人群,使用抗精神病药物不仅有助于减低自杀的风险,而且对精神病患者,特别是精神分裂症患者自杀有效。

\subsection{预防}

自杀的一般预防措施包括及时发现自杀迹象,纠正对自杀的错误认识,普及心理健康知识,减少自杀工具的可获得性,建立预防自杀的专门机构等。

\subsubsection{发现自杀可能的迹象}

实施自杀行为前常常会有迹象,通过对这些迹象的了解可以给自杀的干预提供有益的帮助。

1.向亲友、同事或医务人员询问,或在个人日记中发现出消极、悲观情绪或自杀意愿。

2.与朋友讨论自杀方法,或购买可用于自杀的毒物、药物、刀具,或常在江河、悬崖、高楼徘徊等,提示可能已有自杀计划。

3.慢性难治性躯体疾病患者突然不愿接受医学干预,或突然表现情绪好转,与亲人交代家庭今后的安排和打算。

4.突然无缘无故地与人诀别,或将平时珍视的物品送人。

5.精神病患者中有自责自罪、虚无妄想或存在带有生与死内容的幻听,有抑郁情绪的患者,情绪出现突然的“好转”等。

\subsubsection{改变目前对自杀的模糊观念}

目前社会上对自杀存在不同程度的误解,甚至有相当多的医务人员也同样存在误解。如不加以纠正,对自杀预防不利。

1.自杀无规律 可循自杀事件常带有突然性,一旦发生,周围人会感诧异。其实大部分自杀者都曾有过明显的直接或间接的求助信号,或流露出一些自杀的迹象。

2.宣称自杀的人不会自杀 研究表明,50%的有自杀企图者在自杀前曾向他人谈论过自杀,这种人很可能会有自杀行为,必须高度重视。

3.所有自杀的人都是精神异常者 事实上相当多的自杀者不是精神病患者,给自杀未遂者贴上“精神病”的标签,会使他们觉得受到了侮辱和歧视,往往成为他们再次自杀的原因。

4.自杀危机改善后不会再有问题 研究发现,自杀危机改善后,至少在3个月内还有再度自杀的可能,尤其是抑郁症患者在症状好转时最有危险性。

5.对有自杀危险的人不能提及自杀 很多人担心,对有自杀意念的人谈及自杀会加重他们的动机,事实上与可能自杀的人讨论自杀问题,可以及时发现他们的自杀企图,对其危险性进行正确评估,使他们体会到关爱、同情、支持和理解。当然,谈话时最好不要涉及有关自杀的方法。

6.自杀未遂者并非真正想死 事实上,部分自杀未遂者死亡的愿望很强烈,只是自杀的方法不足以致死或抢救及时,这些人再次自杀的可能性很大。

\subsubsection{普及心理健康知识}

普及心理卫生常识,利用各种形式宣传心理卫生知识。对中小学生讲授各种生活技能,如分析和解决问题、应付挫折、表达思维和情感的能力。建立社区心理咨询和心理保健网络,使有心理障碍或处于心理危机的个体能得到及时、有效的专业化帮助与诊疗。

\subsubsection{控制和管理可能成为自杀的工具}

控制和管理好那些可能被自杀者就近利用的自杀工具,如有机磷农药、煤气、精神药物、毒鼠药、汽车排放废气、枪支等。在常被自杀者选择用以结束生命的一些危险地带,设立“珍视生命”一类的规劝标记,并建立经常性救援组织。

\subsubsection{建立预防自杀的专门机构}

建立危机干预中心或自杀预防中心,一方面对处于心理危机者提供支持,另一方面可以推动当地的自杀预防工作。西方国家在这方面已有一些经验,我国有一些大城市也正在做类似工作。与此同时,对相关医务人员进行专门的培训,以点带面,对自杀的预防可能起到事半功倍的效果。

\subsubsection{精神病患者的自杀预防}

精神疾病患者的自杀风险远高于一般人群。由于精神疾病原因自杀者占自杀总数的50%~90%,因此,对精神病患者的预防自杀应列为重点。

1.对生活在社区的精神疾病患者,其家人应陪同患者定期看医生,由医生动态评估患者自杀的可能性。医生应告知患者的看护者必要的自杀高危特征的识别和预防自杀的有关知识,患者的药品由家属保管,限制每次的处方量,为患者和家属提供24小时支持体系。

2.对住院精神病患者做到:掌握病情尤其是有无消极言行和自杀意念;提供有效的治疗,必要时行电休克治疗;加强巡视,特别是晚间对闭塞角落和隔离房间的巡视;与家属配合,互通情况,及时发现患者想法;注意精神病房的设计,如安装护栏,以防患者跳楼。

\section{危 机 干 预}

\subsection{危机干预的定义}

危机干预(crisis
intervention)最先由Lindeman于1944年提出,美国学者Caplan在美国哈佛地区进行了大量心理卫生工作后加以发展。目前公认的危机干预的含义是:个体处于有威胁的应激状态,当对个体生理或心理的刺激强烈到超过自身承受能力时,正常个体维持与其环境相平衡的状态会打乱,个体的心理反应将变得越来越无组织性和目的性,若不能使用惯用应对机制解决应激,就会导致紧张、焦虑、无助等情绪紊乱。这种危机不仅使个体体验到巨大的痛苦,而且还可能导致自杀、攻击等行为的出现。危机干预就是给处于危机中的个体及时的帮助,使症状得到立即缓解和持久的消失,使个体很快恢复到危机前的生理、心理和社会功能水平,并获得新的应付技能,以预防将来心理危机的发生。

\subsection{影响危机发生的因素}

1.应激事件的知觉和评价 即个体对应激事件发生的意义、对自己将来的影响的理解和估价。

2.状态支持 即个体的周围环境中的人能否提供有效的帮助。

3.有效的应对机制 即个体对生活事件常用的应对办法,如哭泣、愤怒、外跑或向别人倾诉等。

\subsection{心理危机的一般过程}

1.易感状态 应激事件发生后,如果运用通常的应付办法不能解决问题时,可能导致个体心理危机出现前的易感状态。个体一方面动员心理应付策略,增加紧张感,另一方面有心理不平衡的出现。处于易感状态的个体可以体验到受威胁、失落、恐惧和挑战感,同时也希望问题能够被解决。如果问题不能解决,紧张将进一步发展,而进入危机活动状态。

2.危机活动状态 这个时期一般在6~8周,此期,多数个体将寻求医疗帮助。一些人由于不能耐受如此高的紧张和焦虑而出现精神崩溃;另有一些人会采取适应不良行为,如物质滥用、攻击和自杀行为等。这些适应不良行为一方面能带来继发性获益,如过瘾后产生快感、受到更多照顾,另一方面可能导致社会功能的进一步损害。

3.重整期 重整期是危机期的延伸部分。此期患者的焦虑和紧张水平已下降,并有一定的认识能力。应激后的个体适应水平较应激前可能减低或相等,少数甚至会提高一些。职业化的危机干预有可能使个体的适应水平增加或至少保持原有水平。

\subsection{危机干预的过程和重点}

\subsubsection{危机干预的过程}

对处于心理危机中的个体,干预一般包括评估、制定干预目标、进行具体实施和干预终止等过程。

1.对处于心理危机中个体的评估 医生接触患者后应对其进行评估,包括应激事件是什么?患者是否有能力应对?应用的心理机制是什么?这些机制作用如何?患者的社会支持系统如何?是住院治疗还是门诊治疗?在这一过程中治疗者应与患者建立一种良好的关系,取得患者的信任。交谈中尽可能使患者有充分表达情感的机会,维护患者的自尊心。

2.制定干预目标 对患者进行全面评估后,要与患者及家属制定明确同时又是切实可行的干预措施。措施的基本要求是根据患者的需要,适合患者的原有心理素质状况,直接解决患者的痛苦。

3.实施和终止 治疗者从应激事件相关的所有资料和信息中找出认知和行为方面的共同问题,然后寻求解决这一问题的办法,同时注意防止和克服治疗过程中出现的不良移情和认知缺陷。一旦某一目标达到,再确定下一步目标,逐步解决患者的心理危机。如果患者已基本达到了情绪平衡和危机前的功能水平,并经双方同意,就可终止治疗。

\subsubsection{危机干预的重点和目的}

Caplan提出危机干预的重点和目的有4个方面。

1.解决问题的行为 在遭遇应激事件后,个体由于情绪改变往往难以摆脱或改变紧张状态,此时,治疗者和社会组织的帮助十分重要。措施包括提供信息、帮助估价事件及可能的后果。

2.特殊反应的能力 即使用已学会的解决意外问题的技能,学会应付挫折和失败,增加接受紧张处境的技能。

3.内在应付机制的动员 即采用一些防御机制,如否认、分离幽默等。

4.解决事件后果的行为 主要来自于社会支持,处理好事件发生的善后工作,如灾难后的慰问、安抚等。

总之,危机干预无论是从干预的客体还是目的性等方面都不同于普通的心理治疗。治疗者应把握干预的重点和目的,力求被干预的个体尽快摆脱危机,迅速恢复应激前的生理、心理和社会功能。


\protect\hypertarget{text00020.html}{}{}

