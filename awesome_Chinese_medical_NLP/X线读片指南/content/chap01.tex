\chapter{X线诊断基础}


X线自伦琴(Wilhelm Conrad
Röntgen)于1895年发现后不久,就被医学用于疾病的诊断,并形成了放射诊断学。

20世纪50年代到60年代开始应用超声与核素扫描进行人体检查,出现了超声成像和γ闪烁成像。20世纪70年代到80年代又相继出现了X线计算机断层成像(CT)、磁共振成像(MRI)、发射计算机断层成像(ECT)[如,单光子发射计算机断层成像(SPECT)与正电子发射计算机断层成像(PECT)]等新的成像技术。虽然各种成像技术的成像原理与方法不同,诊断价值与限度亦各异,但都是通过人体内部结构和器官的影像来了解人体解剖、生理功能及病理变化,以达到诊断的目的,这样就形成了影像诊断学。

放射诊断学是影像诊断学中重要的组成部分,从某种意义上讲亦是医学影像学的基础。了解其成像原理、方法和图像特点,掌握图像的观察、分析与诊断方法及其在疾病诊断中的价值与限度,从而加以合理应用,这对影像科医师来说是极其重要的。

\section{X线成像的基本原理及图像特点}

\textbf{一、X线成像的基本原理}

一般来说,高速行进的电子流被物质阻挡即可产生X线。具体地讲,X线是在真空管内高速行进成束的电子流撞击钨(或钼)靶时而产生的。

X线是一种波长很短的电磁波。目前,X线诊断常用的X线波长范围为0.008~0.031nm。X线具有以下几种与成像相关的特性:

1.穿透性 X线具有很强的穿透力,能穿透一般可见光不能穿透的各种不同密度的物质,并在穿透过程中受到一定程度的吸收即衰减。X线的穿透力除与X线波长有关外,还与被照体的密度和厚度相关。X线穿透性是X线成像的基础。

2.荧光效应 X线能激发荧光物质,产生肉眼可见的荧光,荧光效应是进行X线透视的基础。

3.摄影效应 X线能使涂有溴化银的胶片感光,经显影和定影处理,产生黑和白的影像。感光效应是进行X线摄片的基础。

4.电离效应 X线通过任何物质都可产生电离效应。它是放射防护学和放射治疗的基础。

基于以上X线特征,加之当X线透过人体各种不同组织结构时,由于其密度和厚度的差别,它被吸收的程度不同,所以到达荧光屏或胶片上的X线量即有差异。这样,在荧光屏或X线片上就形成黑白对比不同的影像。这就是X线成像的基本原理。

传统X线检查可区分四种密度:①高密度的有骨组织和钙化灶等,在X线片上呈白色;②中等密度的有软骨、肌肉、神经、实质器官、结缔组织以及体液等,在X线片上呈灰白色;③较低密度的有脂肪组织,在X线片上呈灰黑色;④低密度的为气体,在X线片上呈黑色。

人体组织器官的形态不同,厚度也不一致。厚的部分,吸收X线多,透过的X线量少;薄的部分相反,从而在X线片上或荧光屏上显示出黑白或明暗差别。

由此可见,密度和厚度的差别是产生影像对比的基础,是X线成像的基本条件。而密度与厚度在成像中所起的作用要看哪个占优势。例如,肋骨密度高但厚度小,而心脏大血管系软组织,为中等密度,但厚度大,因而心脏大血管在X线胸片上的影像反而比肋骨影像白。

\textbf{二、X线图像特点}

X线图像是X线束穿透某一部位的不同密度和厚度组织结构后的投影总和,是一种叠加影像,使原本三维的立体结构变成了一个二维平面图像。

由于X线束是从X线管向人体做锥形投射,因此X线影像有一定程度放大并产生伴影。这其中处于中心射线部位的X线影像,虽有放大,但仍保持被照体原来的形状;而边缘射线部位的X线影像,由于倾斜投射,使被照射体既有放大,又有歪曲失真。

\section{X线检查技术}

X线图像是由从黑到白不同灰度的影像所组成的,这些不同灰度的影像反映了人体组织结构的解剖及病理状态。传统的X线检查可区分骨骼、软组织、脂肪和气体,这就是自然对比。对于缺乏自然对比的组织或器官,可人工地引入一定量的在密度上高于或低于它的物质,便产生了人工对比。自然对比和人工对比是X线检查的基础。

\textbf{一、普通检查}

X线普通检查包括X线透视和X线摄影。

1.X线透视 主要优点是可以转动患者体位进行多方位观察,了解人体组织器官的全貌;了解器官的动态变化,如心脏、大血管的搏动、膈的运动及胃肠道蠕动等;操作方便,费用较低。主要缺点是透视图像欠清晰;密度与厚度较大部位难以观察,如头颅、脊柱等;透视无法留下永久性记录;透视照射时间长,X线量大。

2.X线摄影 优点是图像清晰,可留有永久性记录,便于复查时对照和会诊。缺点是仅能获得一个方位一个区域的影像;无法进行动态观察;费用比透视稍高。

这两种方法,根据检查的需要可配合使用,以提高诊断的正确性。

\textbf{二、特殊检查}

1.体层摄影 普通X线片是一个重叠的影像,故有部分组织结构或病变不能充分显示。体层摄影则可通过特殊装置和操作获得某一选定层面上组织结构的影像,而不属于选定层面的结构则在投射过程中被模糊掉。多用于了解病变内部结构,有无空洞、钙化、病灶边缘情况;还可显示气管、支气管通畅情况等。

2.软线摄影 如常用钼靶摄影,主要用于检查软组织,特别是乳腺组织的检查。

3.其他特殊检查 放大摄影,以显示较细微的病变;荧光摄影,多用于集体体检;记波摄影,以了解心脏大血管搏动、纵隔肿瘤的鉴别、心脏瓣膜钙化、膈肌运动、胃肠蠕动等。

\textbf{三、造影检查}

人体内有很多器官和系统缺乏密度的差异,例如胃肠道、胆系和泌尿系统等。即使在天然对比较明显的胸部和四肢,也不能完全满足诊断要求。为了扩大诊断范围,必须在密度相近的管腔内或器官的周围,注入密度高于或低于它们的物质,进行人工对比。这种方法通常称为造影检查,引入的物质称为造影剂。造影检查及其应用,大大地扩大了X线检查的范围。

(一){造影剂}  按密度高低分为高密度造影剂和低密度造影剂两类。

1.高密度造影剂 为原子序数高、密度(比重)大的物质。常用的有钡剂和碘剂。

钡剂为医用硫酸钡粉末,按粉末微粒大小、均匀性和一定量胶分为不同类型。市场上有不同类型和规格的成品销售,使用时只需加入适量水,达到一定浓度,可适应不同部位检查需要。硫酸钡混悬液主要用于食管及胃肠道造影,目前多采用钡气双重对比检查,以提高检查质量。

碘剂种类繁多,应用很广,分为有机碘和无机碘制剂两类。

有机碘水剂类造影剂注入血管内以显示器官和大血管,已有数十年历史。广泛应用于胆管及胆囊、肾盂及尿路、动静脉及心脏造影、CT增强检查等。20世纪70年代以前均采用离子型造影剂,系高渗,故可引起血管内液体增多和血管扩张、肺静脉压升高、血管内皮损伤及神经毒性较大等缺点,使用中可出现毒副反应。近30多年来开发出数种非离子型造影剂,这类造影剂具有相对低渗性、低粘度、低毒性等优点,大大降低了毒副反应,更适用于血管、神经系统及造影增强CT扫描,但费用较贵。

有机碘水剂类造影有以下三种类型:①离子型:以泛影葡胺(Meglumine
Diatrizoate)为代表。②非离子型:以碘苯六醇(Iohexol)、碘普罗胺(Iopromide)、碘必乐(Iopamidol)为代表。③非离子型二聚体:以碘曲仑(Iotrolan)为代表。

无机碘制剂中,碘化油(Iodinate
Oil)含碘40%,常用于支气管、子宫输卵管造影等。碘化油造影后吸收极慢,故造影完毕应尽可能将碘化油吸出。

脂肪酸碘化物的碘苯酯(Iophendylate),可注入椎管内做脊髓造影,但近年来已用非离子型二聚体碘水剂取代。

2.低密度造影剂 为原子序数低、密度(比重)小的物质。目前应用于临床的有二氧化碳、氧气和空气等。体内二氧化碳吸收最快,空气吸收最慢。空气与氧气均不能注入正在出血的血管,以免发生气栓,可用于蛛网膜下腔、关节囊、腹腔、胸腔及软组织间隙的造影,近年来已较少使用。

(二){造影方法及其应用}
 现就各系统目前常用造影方法及其应用做一简要介绍:

1.骨与关节系统 为了了解关节囊内软组织损伤和病理改变,可行关节造影。

2.呼吸系统 支气管碘油造影是直接观察支气管病变的检查方法,诊断效果好,有一定的痛苦,自CT广泛应用于临床后,这种造影已较少应用。支气管动脉造影用于肺癌的诊断,进而可行介入放射学治疗。肺动脉造影可用于肺动静脉瘘畸形的诊断及栓塞治疗,亦有助于肺隔离症的诊断。

3.循环系统 心血管造影是将造影剂快速注入心脏和大血管内,以显示心脏和大血管腔内解剖结构及血流动力学改变,从而进行疾病的诊断。

对造影剂的要求:高浓度、低粘度、毒性小。目前常用70%泛影葡胺或非离子型水溶性碘造影剂,用量按千克体重计算,每千克体重为1ml,总量一般不超过50ml,注射速度要求每秒15~25ml,快速连续摄影或行数字减影血管造影。

造影方法:根据造影目的、造影剂注入的方式和部位的不同,现介绍几种造影方法:

(1)右心造影:先行右心插管,根据需要将导管前端置于右心房或右心室心尖部,注入造影剂,显示右侧心腔和肺血管。主要适用于右心、肺血管的异常及伴有发绀的先天性心脏病。

(2)左心造影:导管自周围动脉插入,其前端一般置于左心室心尖部。适用于二尖瓣关闭不全、主动脉瓣口狭窄、心室间隔缺损、永存房室共道及左心室病变。

(3)主动脉造影:导管自周围动脉插入,其前端置于主动脉瓣上3~5cm处。适用于显示主动脉本身的病变如动脉瘤、主动脉夹层、主动脉缩窄、大动脉炎等以及主动脉瓣关闭不全、主动脉与肺动脉或主动脉与右心之间的异常沟通如动脉导管未闭、主肺动脉隔缺损、主动脉窦瘤破裂等。

(4)冠状动脉造影:用特定的导管从周围动脉插入主动脉,进而入冠状动脉内进行选择性冠状动脉造影。适用于冠状动脉粥样硬化性心脏病的检查,是冠状动脉搭桥术或血管成形术及内支架置放术术前必须的检查步骤。

心血管造影是一种比较复杂而有一定痛苦和危险的检查方法,必须慎行。

4.胃肠道 胃肠道疾病的检查主要用钡剂造影。血管造影用于胃肠道血管性疾病、胃肠道出血的检查和介入治疗。

(1)钡剂造影:按检查范围可分为:①上消化道造影:包括食管、胃、十二指肠及上段空肠。②上中消化道造影:在做完上消化道造影检查后每隔1小时检查一次,观察空肠、回肠及回盲部情况。③结肠造影:分为钡剂灌肠造影及口服法钡剂造影,前者为检查结肠的基本方法。

按造影方法可分为传统的钡剂造影法和气钡双重造影法。后者已较广泛使用,在检查过程中,除需注意气钡双重相外,尚应结合充盈相、粘膜相及压迫相,才不至于遗漏病变。

为了检查小肠还可用小肠灌钡造影。

胃肠道钡剂造影应注意以下几点:①X线透视与摄片结合,前者可动态观察,后者可观察细微病变。②形态与功能并重。③触诊和加压交替使用。

必要时可用抗胆碱药,以降低胃肠道张力,有利于观察细微结构,亦可帮助鉴别狭窄是痉挛性还是器质性。

(2)血管造影:动脉造影主要用于钡剂检查无阳性发现的胃肠道出血和肿瘤。造影方法是经股动脉穿刺,在透视监视下,将导管插入腹腔干、肠系膜上动脉或肠系膜下动脉,注入造影剂,快速连续摄影。

对于门静脉高压症、食管或胃静脉曲张的患者,可做肝门静脉造影以显示侧支循环的走向、程度,为治疗方案提供资料。

5.肝、胆、胰 肝动脉造影对肝占位病变的血管病变有较大价值,常在超声成像和CT不能确诊的情况下或在介入治疗前施行。胆系造影检查种类较多,分述如下:

(1)口服法胆囊造影:主要用于观察胆囊的形态和功能,从而进行疾病诊断。口服胆囊造影剂(常用碘番酸)后,造影剂被小肠吸收进入血液,然后经胆汁排入胆管到胆囊,经胆囊浓缩后,使胆囊显影。一般于检查前1日晚服造影剂后14小时摄取充盈相,若充盈良好,即吃脂肪餐,之后1小时再摄一片,观察胆囊收缩排空功能。

(2)静脉法胆系造影:静脉注射胆影葡胺,使胆管和胆囊显影。由于USG(超声图)的广泛应用,以上两种造影已退居次要地位。

(3)术后经引流管(T管)造影:手术后经T管做胆管造影,主要观察胆管与十二指肠通畅情况,并了解胆管内有无残留结石或其他疾患。

(4)内镜逆行胰胆管造影(ERCP):是将十二指肠纤维镜送至十二指肠降部,经过十二指肠大乳头插入导管注入造影剂,以显示胆管或胰管。对诊断胆管病变有很大价值。

(5)经皮经肝胆管造影(PTC):是用细针穿刺皮肤、肝脏进入胆管后,注入造影剂使胆管显影。主要用于鉴别阻塞性黄疸的原因并确定阻塞的部位,通常于CT或USG确定有胆管阻塞后,才施行该项检查。在该检查基础上发展了胆管引流术。

胰腺小,位置深,USG、CT可以在无损伤情况下显示胰腺,并对其疾病做出诊断,是目前首选的方法。有些造影检查亦对其有所帮助。如低张十二指肠造影,能较好显示胰腺肿瘤或胰腺炎对十二指肠和胃窦部大弯侧造成的压迫或浸润,当然这是在病变大到一定程度时才能显示的。ERCP对诊断慢性胰腺炎、胰头癌和壶腹部癌有一定帮助。胰头癌是造成胆管阻塞的原因之一,故PTC检查对其诊断亦有一定帮助。选择性腹腔干造影主要用于胰岛细胞瘤的诊断,但多在USG或CT难于确诊后应用。

6.泌尿系统 造影检查可显示泌尿系统器官的解剖结构及其功能情况,进而对疾病做出诊断。

(1)排泄性尿路造影:是泌尿系统常用的造影检查方法。常用的造影剂为60%泛影葡胺,经静脉注射后,几乎全部经肾小球滤过排入肾盂、肾盏而使之显影,不但可以显示肾盂、肾盏、输尿管及膀胱内腔的解剖形态,而且可以了解两肾的排泄功能。严重的肝、肾和血管疾病是本法的禁忌证。

常规法尿路造影:成人用60%泛影葡胺20~40ml,约4分钟内静脉注射完毕,于注射后7分钟、15分钟、30分钟摄取两肾区腹部加压片,如显影良好,除压后摄全腹部片。如有肾盂积水而显影不清,可延长摄影时间2~4小时,甚至更长时间。

(2)逆行肾盂造影:膀胱镜检查时,以导管插入输尿管,注入造影剂而使肾盂、肾盏显影。多用于排泄性尿路造影显影不良或不适于做排泄性尿路造影患者。

(3)膀胱及尿道造影:是将导管插入膀胱,注入造影剂,使膀胱显影。用于诊断膀胱肿瘤、膀胱憩室、前列腺增生等。将导尿管插入前尿道或将注射器直接抵住尿道口,注入造影剂,可显示男性尿道的病变。亦可进行排尿期尿道摄影。

(4)腹主动脉造影和选择性肾动脉造影:主要用于诊断大动脉炎和肾血管性疾病,也可观察肾肿瘤和肾上腺肿瘤。在选择性肾动脉造影的基础上,可对肾癌进行化疗、栓塞等介入治疗。

7.女性生殖系统造影 生殖系统的X线造影可了解子宫和输卵管情况,为女性生殖系统某些炎症和肿瘤的诊断提供依据。

(1)子宫输卵管造影:是经子宫颈口注入40%碘化油或有机碘水剂以显示子宫和输卵管内腔。主要用于观察输卵管是否通畅、子宫有无畸形等。

(2)盆腔动脉造影:经皮穿刺股动脉插管,将导管先置于腹主动脉分叉处或髂总或髂内动脉进行造影,可显示髂内动脉及子宫动脉。该法主要用于诊断生殖器官的血管性病变,确定盆腔肿块的血供和性质,并通过导管进行介入治疗。

8.中枢神经系统 早先该系统造影检查方法较多,自从CT及MRI使用以来,已大大减少,目前常用的有脑血管造影,必要时可进行脊髓造影。

(1)脑血管造影:目前多采用经皮穿刺股动脉插管,根据需要导管前端可分别进入右颈总动脉、左颈总脉和椎动脉,将有机碘水剂引入脑血管中,使其显影。需摄动脉期、静脉期和静脉窦期照片。主要用于诊断脑动脉瘤、血管发育异常和血管闭塞等病变,并可了解脑瘤的供血动脉,在造影的基础上,尚可进行介入治疗如动静脉瘘的栓塞治疗。

(2)脊髓造影:是将造影剂引入蛛网膜下腔中,通过改变患者体位,在透视下观察其在椎管内流动情况和形态,以诊断椎管内病变。

除上述各系统常用的造影检查方法外,尚有一些较少应用的造影检查,此处不再赘述。

总之,以上各种造影检查可概括为两种方式:直接引入和间接引入。直接引入又可分为:①口服法:食管及胃肠钡餐检查。②灌注法:钡剂灌肠、支气管造影、逆行胰胆管造影、逆行泌尿道造影及子宫输卵管造影等。③穿刺注入法:可直接或经导管注入器官或组织内,如心脏造影、血管造影、关节腔造影和脊髓造影等。间接引入为造影剂先被引入某一特定组织或器官内后,经吸收并聚集于欲造影的某一器官内,使之显影,分为:①吸收性:淋巴管造影。②排泄性:口服胆囊造影、静脉胆管造影和静脉泌尿道造影等。

(三){检查前准备及造影反应的处理}
 各种造影检查都有相应的检查前准备和注意事项,必须严格执行,认真准备,以保证检查效果和患者的安全。在造影反应中,以碘过敏较常见并较严重。造影前,必须了解患者有无造影的禁忌证,进行碘过敏试验。阳性者,不宜造影检查;但阴性者,造影中也可发生过敏反应,因此造影过程中自始至终应密切观察患者。一旦出现严重反应,立即终止造影并进行抗休克、抗过敏和对症治疗。

\textbf{四、X线检查方法的选择原则}

X线检查方法的选择,应该在了解各种X线检查方法的适应证、禁忌证和优缺点的基础上,根据临床初步诊断,提出一个X线检查方案。一般应该选择安全、准确、简便且又经济的方法。X线透视和X线摄片是比较简单的检查方法,通常被首先考虑,如应用这些方法可达到诊断目的要求,就无需再进行其他复杂检查,以免增加患者的痛苦和负担。对活动性器官进行动态观察,需了解其功能,以透视为宜。而有些部位检查如颅骨、脊柱和骨盆等只能摄片,X线透视无助于事。有时两三种检查方法都是必需的,如胃肠道检查,既要透视又要摄片;再如对于某些先天性心脏病准备手术治疗的患者,不仅需要心脏透视和摄片,还必须做心血管造影。对于可能产生一定反应和一定危险的检查方法或价格昂贵的检查,必须慎用,不可视作常规检查、不加选择的使用,以免给患者带来不必要的痛苦和损失。

\section{X线诊断的原则和方法}

\textbf{一、X线诊断的原则}

X线诊断是重要的临床诊断方法之一。诊断以X线影像为基础,综合X线各种病理表现,联系临床资料,进行分析推理,才可能提出比较正确的X线诊断结论。在诊断过程中,应根据下列原则来进行:①根据正常解剖、生理的基础知识,认识人体器官和组织的X线影像表现。②根据病理解剖学和病理生理学的基础知识,认识人体病理改变所产生的阴影。③结合临床资料,包括病史、症状、体征以及其他临床检查资料进行分析推理,做出结论。

为了正确做出X线诊断,除应参照上述原则外,还必须具有全面的X线检查程序,包括检查部位和检查方法的选择、摄片位置和曝光因素的确定等,均应根据临床需要及患者具体情况决定。临床所提出X线检查的要求,一般是为了进一步明确疾病的诊断、病变的范围和程度,或者是为排除临床上可疑之疾患。基于临床上的不同要求,在提出X线诊断意见时,则应力求做出具体回答。在分析病例时,应对所有的X线表现按照检查的程序,客观地、全面地进行观察与研究。首先从常见病入手,通常情况下,常见病毕竟常见,少见病毕竟少见,罕见病毕竟罕见,作为一名称职的医师,头脑中应具有较广泛的罕见疾病、少见疾病的知识,这一点很重要。在分析过程中,可多考虑些疾病,然后根据X线影像及相关临床资料,将其中部分疾病排除掉,提出几种可能性,原则上诊断意见不宜超过3个,并指出何种可能性最大,以便临床上参考和处理。对那种根据X线表现和临床资料能做出十分肯定X线诊断的,必须做出肯定诊断。对无法诊断或无法明确诊断的,尚可提出进一步检查的建议。

\textbf{二、X线诊断的方法}

(一){系统周密的观察}
 要想做出正确的诊断,必须从系统周密的观察开始。观察的要求和内容包括以下几方面。

1.X线片的技术条件 阅读分析X线片时,首先应注意照片的质量是否符合诊断的要求。一张良好的照片必须达到位置正确、黑白对比鲜明、细微结构清晰可见、照片清洁不带污迹及其他伪影、标记(左、右及片号)鲜明无误。

2.按一定程序观察X线片 为了不至于遗漏重要X线征象,应按一定顺序,全面而系统地进行观察。如以胸片为例,应按胸廓、肺、纵隔、横膈及胸膜等逐步观察。在分析肺部的X线表现时,可从肺尖到肺底,从肺门到肺周依次进行观察。在分析骨关节片时,应依次观察骨骼、关节及软组织。在分析骨骼时,则应注意骨皮质、骨松质及骨髓腔等。否则很易被引人注目的部分所吸引,忽略观察其他部分,而往往这部分正是更重要而必须阅读的部分。

3.对病变观察的要点

(1)病变的部位和分布:某些病变好发于人体的一定部位,它的分布可表现出一定规律。如骨肉瘤多发生于长骨干骺端,而尤文肉瘤多在骨干;后纵隔肿瘤多为神经源性肿瘤,而中纵隔的肿瘤则多为淋巴类肿瘤等。

(2)病变的数目:病变的数目常与其性质相关。类风湿性关节炎常为多发性关节病变,而结核性关节炎则多为单个关节发病;肺内转移瘤常多发,而原发性周围性肺癌多为单发。

(3)病变的形状:肺内斑片状阴影多为炎症,而球形阴影多为肿瘤,有时亦可为结核球或炎性假瘤;骨囊肿往往呈囊状透光区,而恶性骨肿瘤表现为不规则状骨破坏。

(4)病变的大小:在肺内弥漫性病变中,急性粟粒型肺结核其斑点状阴影直径为1~2mm,矽肺结节一般为2~4mm,而转移瘤的斑点因转移时间不一,往往大小不一。

(5)病变的边缘:肺内良性肿瘤边缘光滑锐利,恶性肿瘤边缘呈分叶状,并见细小毛刺;肺内炎症往往边缘模糊不清。又如胃的良性溃疡龛影口部边缘光整,而恶性溃疡边缘不光整,可见指压迹、裂隙征。

(6)病变的密度:病变的密度可以较周围组织增高或减低。急性骨髓炎时以骨破坏为主,表现为密度减低;当骨质破坏转向修复时,可见骨质增生硬化,密度增高。浸润性肺结核初期病灶的密度较为浅淡,当硬结钙化时密度增高。又如2cm以下周围性肺癌,病灶往往密度不均匀,可见颗粒征、空泡征,而3cm以上病灶通常密度增加且均匀一致。

(7)器官本身的功能变化:如胃窦炎的患者,行上消化道钡餐检查时,可见胃窦部处于半收缩状态,但形态可有所变化;而胃窦癌,其壁变得僵硬,蠕动消失。左心功能不全时可表现为肺瘀血,甚至肺水肿;慢性支气管炎伴肺气肿患者,透视下呼吸改变时,膈移动甚小,肺野透亮度改变甚微,反映患者换气功能下降。

(8)病变周围的组织结构:观察病变时,对其周围情况也应有所了解,才能使诊断正确、全面。如中央型肺癌,早期可致外围出现阻塞性肺气肿,而后可引起阻塞性炎症,甚至出现肺不张,局部肋间隙变窄,同侧膈肌升高,更有甚者出现局部肋骨破坏。又如结核球,其周围往往可见斑点索条状阴影,即所谓卫星灶。恶性溃疡,龛影周围可见癌灶浸润所致的僵硬的环堤征,并见周围粘膜不规则、破坏等。

4.掌握临床情况 对患者的临床情况,除病历中或检查申请上记载外,根据诊断需要,有时可亲自做进一步询问和必需的查体。也可与有关临床医师共同研究,以便掌握更可靠和全面的临床资料,这对完成正确X线诊断是非常重要的。

(二){客观的逻辑分析判断}
 当我们通过对X线影像的观察取得了大量丰富的材料后,产生了许多印象,必须经过科学地分析和研究,才能得出正确的结论。

在进行综合分析时,如何使X线表现与临床资料紧密结合起来,这是非常重要的。以下问题值得注意:

1.性别 有些疾病的发生,与性别有一定关系。如下腹部肠道外的肿块,对女性患者应考虑为卵巢或子宫的疾病,而在男性则来自睾丸或精囊肿瘤转移的可能。再如前列腺之疾病则仅为男性所有。类风湿性关节炎多发生于30岁以上女性,而强直性脊柱炎多见于青少年男性,大动脉炎多见于女性。

2.年龄 根据受检者的年龄对疾病进行分析。如儿童肺内肿块,多为良性;而老年则多为恶性。冠状动脉粥样硬化性心脏病多于40岁以上发病,而原发性心肌病发病年龄偏轻。骨肉瘤一般发病年龄在15~25岁之间,而骨巨细胞瘤多于20~40岁发病,年长者骨肿瘤以转移瘤多见。

3.体形 人的体形对心脏方面影响较明显。如瘦长者心脏多呈垂位心,而矮胖者心脏多呈横位。因此,对瘦长者来说,出现横位心时往往提示心脏增大。

4.职业史和接触史 受检者的职业史与接触史,为诊断职业病和寄生虫病的重要依据。如在诊断尘肺或放射损伤等,则均应具有特殊的职业史和接触史;在诊断肺部血吸虫病时,必须询问是否有疫水接触史。

5.生活史 在诊断地方病或区域性疾病时,应详细了解受检者的生长和居住过的地区,这对确定某些疾病的性质是有决定作用的。如包虫病多发生于西北牧区,而华东地区则有血吸虫病,大骨节病又以东北多见。

6.过去史 病史对决定病变的急性与慢性有很大帮助。如位于肺底部斑片状模糊阴影,如系发病突然且伴有发热、咳嗽、胸痛者,可诊断为急性肺炎;而对病程较久,又有长期咳嗽、咳痰与咯血者,则应考虑为支气管扩张合并感染的可能;如在一个局部反复出现炎症,年龄偏大,应想到阻塞性肺炎可能。又如,当发现关节有狭窄或破坏征象时,病史短者多为化脓性病变,而结核性关节炎则往往病史较长。

7.体征 心脏病的早期或心脏的X线征象改变不典型时,心脏听诊更为必要。对肺多血、心影呈梨形患者,胸骨左缘第2至第3肋间闻及较柔和的收缩期杂音,多为房间隔缺损;胸骨左缘第3至第4肋间闻及粗糙响亮的收缩期杂音,可扪及震颤,多为室间隔缺损;胸骨左缘第2肋间闻及滚筒样连续性杂音,则多为动脉导管未闭;听到心包摩擦音或心音遥远者,应想到心包炎。小肠、大肠广泛积气,明显扩张,腹部听不到任何肠鸣音,可诊断为麻痹性肠梗阻。

8.重要的检查结果 临床检验结果,对分析病变性质和确定X线诊断也是非常重要的。如在痰中发现结核杆菌时,上肺野出现斑片状阴影,应首先考虑为肺结核;颅骨和扁骨出现斑点样破坏而尿中本周蛋白阳性,可诊断为骨髓瘤。

9.治疗经过 对某些X线影像一时难以确定其性质时,可通过诊断性治疗,观察病灶的变化,最终给予判断。如在肺部发现片状阴影,而临床症状轻微,经过抗感染治疗后,阴影消失,则可诊断为肺炎而不是结核。又如骨骼的广泛性囊样骨吸收,经过甲状旁腺摘除后,病情好转,则可诊断为甲状旁腺功能亢进。

应该指出,X线诊断是有价值的,但也有一定限制。传统X线检查只能分辨4种密度,如只能诊断胸腔积液,而无法区别是脓胸、血胸或胸水;有些疾病临床表现已经出现,而X线可暂时表现为阴性,有滞后现象,如大叶性肺炎、骨髓炎、右心衰竭等。

总之,X线诊断结果基本上有三种情况:①肯定性诊断,即确诊。②否定性诊断,即经过X线检查,排除了某些疾病,此种判断应慎重,因有时X线表现比临床症状与体征有滞后现象,必须跟踪观察。③可能性诊断,即经过X线检查,发现了某些X线征象,但一时难以明确性质,可列出几种可能性,当然可能性最大者应放在首位。

X线诊断用于临床已有近百年历史,积累了丰富的经验,尽管其他一些先进的影像检查技术,例如CT、MRI、PET等对一部分疾病的诊断,显示出很大的优越性,但它们不能取代X线检查,而且X线诊断学是这些先进的影像检查技术的基础。一些部位的检查,例如胃肠道、骨与关节及心血管,仍主要使用X线检查。X线片的空间分辨率高,图像清晰,且经济、简便,因此X线诊断仍然是影像诊断中使用最广泛和最基本的方法。当然,在检查中应该高度重视防护。

\section{医学影像诊断报告书写原则}

医学影像诊断报告是医学影像学科质量管理体系的重要组成部分。为了提高医疗服务质量,能够客观、准确地反映医学影像检查结果,避免发生漏诊和误诊,在书写医学影像诊断报告时也要遵循一定的原则。具体步骤包括以下几个方面:

\textbf{一、充分做好书写前的准备工作}

1.认真审核影像检查 申请单申请单上记载患者的姓名、性别、年龄等一般资料,以及临床病史、症状、体征、实验室检查和其他辅助检查(包括其他影像检查)结果,此外还包括临床拟诊情况、本次影像检查的要求和目的等。在正式书写影像诊断报告之前,要认真审核这些内容。当这些项目,尤其是病史、症状、体征等临床资料填写不够详细和不够充分时,应及时予以补充,可与临床医师、患者和家属进行沟通,因为这些内容是做出正确影像诊断的重要参考资料。

2.认真审核影像检查影像 它包括如下内容:

(1)影像的患者信息是否与申请单相符:影像上标有患者的信息,包括姓名、性别、年龄、检查号等,要认真核对这些信息是否与申请单内容一致,避免发生错误,否则将会导致重大医疗事故。

(2)影像的检查技术和检查方法是否合乎要求:临床对不同系统的不同疾病进行影像检查有着不同的目的和要求,而不同影像检查技术和检查方法对于这些目的和要求有着不同的价值和限度。因此,针对临床的目的和要求,首先要认真审核所进行的影像检查能否满足这些需要,如不符合需要,则应及时进行补充检查。其次,要仔细核对影像与申请单所要求的检查技术、方法和部位是否相符,是否完全。对于不相符或不完全者,要及时安排重新检查。

(3)影像的质量是否合乎标准:对于各种成像技术和检查方法所获得的图像,良好的对比度和黑化度是发现异常表现的关键。在数字化成像包括CR、DR影像上,正确运用窗技术对于异常表现的显示也非常重要。此外,影像上的各种伪影也可干扰正常和异常表现的识别,从而影响了诊断的准确性。因此,在书写诊断报告之前要认真审核影像质量,对于不符合质量标准的影像,不能勉强书写,以免发生漏诊和误诊。

3.准备齐全相关资料 相关资料包括与疾病诊断有密切关系的各项实验室检查结果、各项功能检查和各项其他辅助检查结果,还包括其他影像检查结果。申请单上,这些检查结果未必填写详细或填写有误,然而这些检查结果可以支持、但也可以否定影像诊断时的最初考虑,因此对最终影像诊断和鉴别诊断有着非常重要的影响。再有,对于疾病治疗后随诊复查的影像检查,要准备好既往影像检查影像和诊断报告,以资进行比较。

\textbf{二、集中精力认真书写影像诊断报告}

影像诊断报告要求计算机打印。对于不具备打印条件的单位,手写时要求字迹清楚,字体规范,禁用不规范的简化字和自造字。书写影像诊断报告要使用医学专用术语,要语句通畅,具有逻辑性,且要正确运用标点、符号。特别要提出的是,无论计算机打印或手写的诊断报告,均不得用笔涂改,否则日后一旦对诊断报告发生异议时,难以确定责任。

影像诊断报告的格式要规范化。一般包括以下五项基本内容:一般资料,成像技术和检查方法,影像检查表现,印象或诊断,书写医师和复核医师签名。

1.一般资料 要认真填写诊断报告上一般资料,其中包括患者的姓名、性别、年龄、门诊号或住院号、影像检查号、临床拟诊情况、检查部位、检查日期和报告日期。不应有空项,并特别注意填写时要与申请单和影像上相应项目的内容保持一致。

2.成像技术和检查方法 是对所要分析、诊断的影像,叙述清楚采用何种影像设备、使用何种成像技术和检查方法获取的。其中,对于与影像分析有关的检查步骤(如胃肠道造影前的胃肠道准备、静脉尿路造影中的摄片时间)、使用的材料和方法(如所用对比剂的名称、剂量、引入途径等)以及检查时患者的状态(如屏气检查、神志欠清等)均要予以说明。这些内容的叙述对于正确评估影像,确认异常表现非常重要。

3.影像检查表现 影像检查表现的叙述是影像诊断报告的核心内容,是最终印象或诊断的依据。书写这部分内容时,应在全面、系统观察影像的基础上进行。在书写时,还应注意以下几点。

(1)关于异常表现:要重点叙述异常表现即病灶所在的部位、数目、大小、边缘、形态、密度、回声和信号强度,邻近组织结构的改变及其与病灶的关系。这些征象是进行疾病诊断的主要依据。需指出的是,在异常表现的描述中,不应出现有关疾病名称的术语,也就是说不能与最后的印象或诊断相混淆。

(2)关于正常表现:应简单、扼要叙述影像上所显示但未发现异常表现的组织结构和器官。如此可以表明书写者已观察了这些部位,并且排除了病变的可能性,从而避免了这些部位病变的遗漏;但需注意,对正常表现的叙述要防止走过场,即并非仔细观察,而只是进行习惯性书写。

(3)其他方面:要注意叙述对病变诊断和鉴别诊断有重要意义的阳性和阴性征象。例如孤立性肺结节,其内有无钙化、轮廓有无分叶、边缘有无细短或粗长毛刺、周围有无卫星病灶、邻近胸膜有无改变等,这些征象对于结节良、恶性的鉴别非常有帮助,均应一一叙述。

4.印象或诊断 印象或诊断是影像诊断报告的结论部分,应特别注意它的准确性,把握好度,既不要诊断不足,也不应过诊。印象或诊断基本有以下三种情况:①肯定性诊断,即经影像检查不但发现了病变,而且可以做出准确的定位、定量和定性诊断。②否定性诊断,即经影像检查排除了临床所怀疑的病变。但应注意,在此方面影像学检查有一定限度,因为疾病自发生至影像检查发现异常表现常需一定的时间,而且某些疾病可能影像检查难以发现异常。因此,对于否定性诊断,要正确评估它的意义,必要时应建议临床进行影像复查。③可能性诊断,即经影像检查,发现了一些异常表现,甚至能够清楚显示出病变的位置、范围和数目,但难以明确病变的性质,此时印象或诊断宜提出几种可能的病变,并依它们可能性的大小进行排序。在这种情况下,还应根据拟诊的可能性,建议临床行其他影像检查、相关的实验室检查和(或)其他辅助检查、随诊影像检查,乃至诊断性治疗或影像导向下穿刺活检。

在书写印象或诊断时,还应注意以下几点:

(1)关于印象或诊断与影像检查表现叙述的一致性:即印象或诊断应与影像检查表现的叙述内容相匹配,绝不能相互矛盾,也不应有遗漏,即表现中已叙述异常,而印象或诊断却无相应异常表现的结论,反之亦然。

(2)关于印象或诊断中的疾病描述:①应指明疾病的部位、范围和性质。②当同一次影像检查发现几种疾病并存时,印象或诊断中要依这些疾病的临床意义进行排序。

(3)关于用词的准确性:在书写印象或诊断时,更要注意用词的准确性,疾病的名称要符合相关规定。特别是不要有错字、别字、漏字及左、右侧之误,否则可导致严重的医疗事故。

5.书写医师和复核医师签名这是影像诊断报告的最后一项内容,不要用计算机打印,而要用笔手签(已得到主管部门同意使用电子签名的除外),以表示对报告内容负有责任。

最后,书写医师在完成报告的书写后,要认真核对报告内各项内容,确认无误后,提交给复核医师。复核医师一般年资和职称要高于书写医师,其应逐一复核医学影像报告中的每一项内容是否准确,并要再次核对申请单、影像和报告所示患者姓名、性别、年龄和检查项目的一致性。各项内容无误后,由复核医师签字,并准发报告。

\section{X线鉴别诊断思路}

X线鉴别诊断几乎贯穿了所有的疾病诊断,在遵循上述诊断原则的同时,首先要全面、仔细观察X线片上的各种征象,提取其特征性的改变,联系它们与病理之间的关系,合理地解释X线征象的意义。一个征象可能是一个或几个甚至多个原因的结果,对其加以鉴别诊断以尽量减少诊断中的“可能”,达到最后、最可能的诊断目的。因此,X线鉴别诊断应当先从认识患者(X线片及检查单)开始,仔细发现和解读各种X线征象,再从有哪些疾病可发生该征象来考虑。

首先应当快速观察患者的姓名、性别、检查技术以总揽全局的概要;任何时候都要关注有无快速致命的征象(如膈下游离气体、气胸等征象)以减少重大事故的发生;熟知检查部位及检查方法可能存在的盲区并重点观察以减少漏诊;仔细观察并发现各种X线征象;对征象的解读应当综合考虑可能出现的疾病大类,记住CINTV:先天性(C),感染性、特发性(I),肿瘤性(N),创伤性(T)和血管性(V)。

其次对征象的分析要先考虑是否为常见病中的某种典型或不典型征象,再考虑是否为少见或罕见病中的典型或不典型征象。

如有以往检查,尽可能地与以往X线片比较,进行动态观察。

经过上述考虑后,取其中最符合的鉴别作为最可能的诊断;如均不符合,应当考虑可能是征象太不典型或尚未被大家所认识的新征象或新疾病。

得出初步印象后,需结合临床、其他实验室检查进行分析,做出尽可能准确的诊断。

需要强调的是,结合临床资料应该在根据X线征象的特征得出初步印象后进行,否则很可能受临床资料先入为主的影响,而妨碍医师的独立思考。

资料不足或特殊疾病的鉴别诊断(如骨肿瘤等)强调影像与病理、临床的三结合诊断,必要时可以提出进一步的检查步骤和方法。

\protect\hypertarget{text00007.html}{}{}

