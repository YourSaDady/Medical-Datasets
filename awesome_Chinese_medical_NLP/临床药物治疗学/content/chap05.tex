\chapter{疾病对临床用药的影响}

疾病是影响临床用药的重要因素,其通过改变药物在体内的吸收、分布、生物转化及排泄过程,导致药物代谢动力学的改变;同时也通过改变某些组织器官受体数目和功能,导致药效动力学的改变。因此,只有充分认识在治疗过程中病理状态对临床用药的影响,及时调整药物剂量、给药途径及给药间隔,才能达到对患者实施合理性、个体化药物治疗的方案,获得最佳治疗效果和最低治疗风险的目的。

\section{疾病对药动学的影响}

\subsection{疾病对药物吸收的影响}

\subsubsection{改变胃排空时间}

延长胃排空时间的疾病如胃溃疡、抑郁症、帕金森病、创伤或手术恢复期等,能推迟药物的吸收时间,使得药物达峰时间延长,药物起效时间变慢;而缩短胃排空时间的疾病如甲状腺功能亢进、胃酸过多及处于焦虑兴奋状态等则相反。

\subsubsection{改变小肠的药物吸收}

小肠是药物吸收的主要部位,能改变小肠吸收功能的疾病,如节段性回肠炎,可减慢克林霉素、甲氧苄啶及磺胺类药物的吸收;慢性胰腺炎或胆囊纤维化的患者,可明显减少头孢氨苄、头孢噻肟的吸收。

\subsubsection{胆汁分泌减少}

胆汁缺乏的患者可发生脂肪泻及并发吸收障碍综合征,而对一些脂溶性高的药物如脂溶性维生素、地高辛等,一般难以吸收。

\subsubsection{慢性疾病}

慢性肝功能不全、肾功能不全、肾病综合征、心力衰竭、营养不良伴有低蛋白血症的患者,血浆中游离型药物浓度升高,降低了药物透过肠黏膜的浓度梯度,使口服药物吸收减少。肾功能减退者维生素D羟化不足,导致肠道\ce{Ca^2+}
吸收减少;慢性尿毒症患者常伴有胃肠功能紊乱,如腹泻、肠黏膜水肿等,能减少药物吸收,同时由于胃内氨的含量增高,使pH值升高,可减少弱酸性药物在胃内的吸收。

\subsubsection{心力衰竭}

心力衰竭患者由于胃肠道瘀血,影响药物吸收,药物生物利用度可减少达50%。

\subsubsection{营养不良}

营养不良、恶性贫血、糜烂性胃炎的患者,由于内因子分泌减少,可造成B{12}
缺乏。

\subsubsection{药物吸收量与注射部位血流量有关}

当患者处于休克状态时,由于周围循环衰竭,皮下或肌内注射药物吸收受阻,应采取静脉给药的方式才能达到抢救目的。

\subsection{疾病对药物分布的影响}

药物的体内分布主要受血浆蛋白含量、体液pH值、药物脂溶性等多种因素影响。其中血浆蛋白含量及其与药物结合能力是影响药物体内分布的最重要因素之一,药物与血浆蛋白结合率稍有改变,就可能明显改变药物的药理作用。

\subsubsection{疾病对药物血浆蛋白结合率的影响}

肝脏疾病时,蛋白合成减少,从而使血浆蛋白结合率降低,游离型药物增加,可使药物的组织分布范围扩大。血浆蛋白含量低的患者,按常规剂量用药时,有可能发生不良反应。低白蛋白血症患者使用地西泮、泼尼松等药物,可出现明显毒性反应,使用苯妥英钠、华法林及洋地黄等蛋白结合率高的药物也可出现此种现象。故此类患者用药应注意减少用量,从最小有效剂量开始,必要时做血药浓度监测。

\subsubsection{疾病对血液pH的影响}

肾病可引起血液pH值的变化,影响药物解离度及药物向组织的分布,如肾病伴酸中毒可使水杨酸和苯巴比妥等弱酸性药物分布到中枢组织,可能增加中枢毒性。

\subsection{疾病对药物代谢的影响}

\subsubsection{影响肝脏功能的疾病}

肝脏在药物的代谢中起着重要的作用,大多数药物在肝脏内经过生物转化后转变为无活性的代谢产物而排出体外。肝脏功能减退时,肝药酶数量减少、活性下降,药物在肝脏的代谢灭活减少,可使药物效应增强,甚至毒性反应增加。如肝硬化患者的地西泮半衰期可显著延长,药效也随之延长,这时常规剂量的药物也可导致昏迷。此外,能影响肝血流量的疾病对药物代谢也有一定的影响,如甲状腺功能亢进的患者交感神经兴奋,心率加快,肝血流量随心输出量增加而增加,利多卡因、维拉帕米、普萘洛尔、吗啡、哌替啶等药物在肝脏代谢加快,半衰期缩短;而充血性心力衰竭的患者,上述药物在肝脏代谢则减慢。有些药物须经肝脏活化才具有药理作用,如泼尼松等,故肝功能不全的患者,血液中活化的泼尼松龙浓度下降,因而药理作用降低。

\subsubsection{影响肾脏功能的疾病}

肾脏是仅次于肝脏的药物代谢器官。肾脏能代谢很多药物。近曲小管含有高浓度的葡萄糖醛酸转移酶,使药物大量与葡萄糖醛酸结合。例如,静脉注射呋塞米,20%在肾脏葡萄糖醛酸化,50%胰岛素的消除是通过肾脏代谢。肾脏疾病时,药物在体内的转化速度和途径均可以发生改变,如尿毒症患者对苯妥英钠的氧化代谢加快,表现为常规剂量下难以控制癫痫
发作。

\subsubsection{呼吸系统疾病}

呼吸系统疾病也可以影响药物的代谢,如慢性呼吸功能不全患者对普鲁卡因的代谢减慢;慢性哮喘对甲苯磺丁脲的代谢加快;急性肺水肿患者,因肺血气交换减少,影响肝内血供,使氨茶碱代谢减慢,半衰期延长。

\subsection{疾病对药物排泄的影响}

药物有多种排泄途径,如尿液、胆汁、肠液、唾液、汗腺等,其中最主要的排泄器官是肾脏。肾功能不全的患者,主要是经肾脏排泄的药物容易在体内蓄积,药物半衰期延长,药理效应增强,甚至发生毒性反应。许多药物的不良反应发生率明显高于肾功能正常者,而且与肾功能损害程度密切相关。

\subsubsection{肾小球滤过率(GFR)改变}

急性肾小球肾炎及严重肾功能减退患者的GFR下降,主要经肾小球滤过而排出体外的药物如地高辛、氨基糖苷类等排泄减慢,半衰期延长,药效增强。因此,肾功能减退患者使用上述药物时,应根据肾功能调节剂量。肾病综合征时,肾小球毛细血管通透性增加,致使药物排除增多,药效降低。

\subsubsection{肾小管分泌的改变}

肾小管分泌是主动转运过程,需要有载体参加,一般不受血浆蛋白结合的影响。弱酸性或弱碱性药物从肾小管主动分泌,各自的分泌通道并不相同,但同类分泌通道却缺乏特异性。如弱酸类利尿药呋塞米及氢氯噻嗪一般通过有机酸转运机制分泌进入肾小管管腔达到作用部位,但在尿毒症时,体内积蓄的内源性有机酸阻止其达到作用部位,以致要增大药物剂量才能在管腔内达到有效浓度,发挥利尿作用。

\subsubsection{肾小管和集合小管的重吸收改变}

尿液pH值能影响非解离型药物的比例,从而影响药物的被动重吸收。弱酸性药物在碱性环境中易解离,当患者pH值升高时,排泄增多。弱碱性药物(吗啡、可待因、氨茶碱)在碱性环境中难解离,当pH值升高时排泄减少。故临床上可通过调节尿液pH值的方法来治疗药物中毒,如碳酸氢钠碱化尿液治疗苯巴比妥中毒。

\subsubsection{肾血流量减少}

休克、心力衰竭、肾动脉病变均可使肾血流量减少,肾小球滤过、肾小管分泌和肾小管重吸收等功能均可发生障碍,从而影响药物的经肾排泄。

\subsubsection{肝脏疾病影响药物经胆汁排泄}

某些药物以原型或其代谢产物的形式按主动转运经胆汁排泄,如红霉素。当肝功能减退时,由于肝血流量减少,进入肝细胞的药物减少,同时药物从肝细胞到胆汁的主动转运过程发生障碍,可使药物经胆汁排出减少,药物的肝肠循环减弱。如肝功能正常者服用地高辛后7d内从胆汁中的排出量为给药量的30%;而肝功能减退者服用同等剂量后,7d内的排出量仅为8%。任何影响肝血流量、肝细胞对药物的摄取、药物在肝内的代谢、药物向胆汁的转运、胆汁形成的速度等因素,均可影响药物自胆汁的排泄。

\section{疾病对药效学的影响}

\subsection{疾病引起受体数目改变}

大多数药物与靶细胞上的受体结合,激动或阻断受体,产生药理效应。而组织细胞内受体的数目、亲和力及内在活性可因疾病的影响而产生改变。研究发现,某些疾病产生针对自身受体的抗体,可阻断受体与药物的正常结合,某些疾病还可以引起体内cAMP、IP3/DG和G蛋白等细胞内信使的活性产生改变,此种状态下用药,药物效应必然发生改变。例如,甲状腺功能亢进患者的β受体比正常人多1倍,用了β受体激动剂很容易引起心律失常。因此,疾病对药物靶受体的影响是改变药物效应的一个重要因素。

\subsubsection{高血压病}

高血压病患者交感神经活性增高,使β受体暴露于高浓度的肾上腺素和去甲肾上腺素中,致使β受体下调。普萘洛尔的降压作用是通过阻断β受体数目实现的,有利于β受体数目的向上调节。对于内源性儿茶酚胺高的患者,其减慢心率、降低血压的作用相当显著,而对体内儿茶酚胺浓度不高的患者,其治疗效果较差。

\subsubsection{支气管哮喘}

哮喘患者支气管平滑肌上的β受体数目减少,且与腺苷酸环化酶的偶联有缺陷,体内cAMP含量降低,使α受体的功能相对占优势,引起支气管收缩,诱发哮喘。治疗时应用β受体激动剂如沙丁胺醇等舒张支气管平滑肌的同时,加用α受体阻断药或糖皮质激素后可出现良好的治疗效果,因为糖皮质激素能使cAMP含量升高,哮喘得以缓解。而大剂量β受体激动剂可拮抗机体内源性糖皮质激素的功能,对哮喘产生不良效果,故目前临床不主张大剂量使用β受体激动剂。

\subsection{疾病引起机体对药物的敏感性改变}

\subsubsection{肝脏疾病}

肝病患者体内氨及短链脂肪酸等代谢异常,使脑功能处于非正常状态,对较常用的镇静催眠药、镇痛药和麻醉药的敏感性几乎都增强,甚至诱发肝性脑病。如慢性肝炎患者,尤其是发生肝性脑病的患者,在使用氯丙嗪和地西泮镇静时,使用常规剂量就会使患者产生木僵和脑电波减慢,宜选用奥沙西泮,但仍需慎重给药,宜从小剂量开始。肝硬化水肿和腹腔积液的患者使用过强的利尿剂治疗,由于过度失钾,能加重肝性脑病症状,诱发肝昏迷,应用保钾利尿剂治疗。

\subsubsection{肾脏疾病}

肾功能衰竭引起尿毒症时,引起电解质和酸碱平衡紊乱,导致机体内各种生物膜的电位及平衡机制改变,以致改变机体对药物的敏感性。由于血脑屏障有效性降低,对镇静催眠药的中枢神经系统抑制效应更敏感。由于凝血机制的改变,使机体对抗凝血药更敏感,使用阿司匹林和其他非甾体类抗炎药(NSAIDs)更易引起胃肠出血。

\subsubsection{心脏疾病}

心脏自律性紊乱与心肌损害相伴,并会被药物增强。地高辛的心脏毒性会被低钾血症和高钙血症所增强,低钾血症还能明显减弱许多抗心律失常药的效应,故在治疗心律失常时要注意电解质的平衡,同时药物的剂量需要适当调整。有严重缺氧疾患者,地高辛更易引发心律失常。对药物敏感性的显著改变也可由治疗的终止而诱发。如冠心病患者长期使用β受体阻断药治疗停止后,会持续数日对肾上腺素有高敏性。此类患者必须缓慢地减少β受体阻断药的治疗剂量,以免引起反跳。

\subsection{疾病引起受体后效应机制改变}

疾病引起受体后效应机制改变可以地高辛对不同类型心力衰竭的效应为例。不同原因所致的心力衰竭,其\ce{Na^+}
-\ce{K^+}
-ATP酶后效应机制受到抑制或损害的程度也不一致,使用强心苷的临床疗效也不一样。对低心输出量型心力衰竭,如高血压、心瓣膜病、先天性心脏病等心脏长期负荷过重引起的心力衰竭,应用强心苷治疗效果较好,药物强心苷受体后效应机制没有受损,它能增加心肌收缩力,降低前后负荷,增加心输出量;而高心输出量型心力衰竭,如甲状腺功能亢进、贫血继发的心力衰竭、肺源性心脏病所致心力衰竭,由于存在心肌缺氧或能量代谢障碍,使强心苷受体后效应机制受到严重影响,因而应用强心苷治疗效果较差,易引发毒性反应,应治疗原发病。

\section{疾病状态下的临床用药}

肝脏是药物代谢的主要场所,肾脏是药物排泄的主要器官,肝肾疾病或其他脏器的病变引起肝肾功能减退时,药物代谢排泄必然受到影响,从而影响药物的药理效应,甚至造成药物在体内的蓄积,引起严重毒性反应。

\subsection{肝脏疾病的临床用药}

\subsubsection{肝脏疾病对临床用药的影响}

肝脏疾病可引起肝血流量减少或肝药酶活性降低,使药物的肝清除率减少,药物在体内蓄积。如钙通道阻滞药非洛地平、硝苯地平、尼莫地平等在肝硬化患者的血浆清除率和首过消除明显降低,半衰期显著延长。肝硬化患者口服这些药物时,剂量仅为正常剂量的25%~50%。

急性病毒性肝炎或肝硬化时,许多药物的血浆蛋白结合率降低,血浆中游离型药物浓度增高,这与肝病时血浆蛋白合成减少、血浆蛋白结合部位减少或内源性抑制物蓄积有关。为确保肝病时用药安全,肝硬化患者应从小剂量开始用药,并随时观察临床反应以便及时调整剂量及给药间隔,必要时可进行血药浓度监测。

口服给药存在首过消除,肝病患者首过消除减少,药物的生物利用度增加,药物的血药浓度升高,故对肝病患者使用普萘洛尔、美托洛尔、拉贝洛尔、阿司匹林、利多卡因、氯丙嗪、吗啡、哌替啶等具有明显首过消除效应的药物时,应减少给药剂量、并延长给药间隔时间。

\subsubsection{肝功能不全时用药注意事项}

肝脏疾病时,药物的消除速率减慢,血药浓度升高,药物的半衰期延长,但只要血药浓度的变化不超出2~3倍,且机体没有受体敏感性的改变,则该血药浓度的变化并没有太大的临床意义。但据统计,药物引起肝功能损害占药物不良反应的10%~15%,而多数药物都能引起不同程度的肝功能损害。肝脏疾病用药应注意以下几点:①禁用或慎用具有肝功能损害作用的药物,如必须应用,应进行生化监护;②慎用经肝脏代谢且不良反应多的药物;③禁用或慎用可诱发肝性脑病的药物。

\subsection{肾脏疾病时的临床用药}

\subsubsection{肾脏疾病对临床用药的影响}

肾脏是药物排泄的主要器官,肾功能减退时,药物的吸收、分布、生物代谢、排泄以及机体对药物的敏感性均可能受到影响。肾功能不全患者,药物易在体内蓄积,药物半衰期延长,药效提高,甚至发生毒性反应。例如,肌酐清除率近似正常值的患者({Qc}
=83mL/min)肌内注射卡那霉素7mg/kg,$t_{1/2}$
为1.5h,而肾功能衰竭患者({Qc} =8mL/min)$t_{1/2}$ 可达25h。

\subsubsection{肾功能不全时选药原则}

肾功能不全的患者在选择治疗药物及制定用药方案时,应遵循以下几点原则:①尽可能选用肾毒性较低或无肾毒性的药物;②选择那些在较低浓度即可生效且不良反应容易辨认的药物;③评估患者的肾功能,确定适当的给药剂量及给药间隔时间。

\subsubsection{肾功能减退时给药方案调整}

肾功能减退时,如仍按照常规方案给药,可因药物在体内蓄积而引起毒性反应。故对肾功能不全的患者使用主要经肾排泄且毒性较大的药物,应先评估患者的肾功能,然后根据患者的肾功能减退程度调整给药方案,确定适当的给药剂量及间隔时间。