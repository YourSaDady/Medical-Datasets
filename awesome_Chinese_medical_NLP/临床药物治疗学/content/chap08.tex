\chapter{抗菌药物的合理应用}

\section{抗菌药物的相互作用}

药物相互作用(drug-drug
interaction,DDI)是指同时或相继使用两种或两种以上药物时,由于药物之间的相互影响而导致其中一种或几种药物作用的强弱、持续时间甚至性质发生不同程度改变的现象。可分为体外相互作用与体内相互作用两大类。前者指药物在体外配伍时直接发生的物理性变化或化学性相互作用,又称配伍禁忌。后者多由于配伍用药物在体内过程生成效应环节发生的相互影响,包括药动学相互作用和药效学相互作用。

\subsection{抗菌药物的配伍禁忌}

应用一种药物疗效不佳时,常采用几种药物合理配伍,有些配伍使作用降低,导致治疗失败;有些配伍使毒副作用增强,引起严重的不良反应;还有些使作用过度增强,超出机体所能耐受的能力,引起不良反应,乃至危害患者生命,均属配伍禁忌。

\subsubsection{分类}
\paragraph{物理性配伍禁忌}

药物配伍使用时所产生的分散状态或其他物理性质改变,导致药物外观、性质、药效发生变化,即属于物理性配伍禁忌。物理性配伍禁忌一般不改变药物成分,但会造成制剂的外观或均匀性发生变化,亦可能影响疗效或对患者造成损伤,并引起患者疑虑。
\paragraph{化学性配伍变化}

药物配伍使用时所引起的药物成分的化学变化,导致疗效降低、毒副作用增大等不良反应后果,这些就属于化学性配伍禁忌。

\subsubsection{常见药物配伍举例}
\paragraph{理化配伍禁忌}

理化配伍禁忌多是由液体pH值、离子电荷等条件改变引起的混浊、沉淀、漂浮物、变色等(见表\ref{tab8-1})。

\begin{longtable}[]{@{}lll@{}}
\caption{理化配伍禁忌}
\label{tab8-1}\\
\toprule
\endhead
配伍药物 & 配伍药物 & 结果\tabularnewline
\midrule
阿莫西林克拉维酸钾 & 庆大霉素 & 沉淀\tabularnewline
头孢曲松 & 万古霉素 & 沉淀\tabularnewline
& 喹诺酮类 & 沉淀\tabularnewline
美洛西林 & 氨基糖苷类 & 絮状沉淀\tabularnewline
头孢哌酮 & 维生素B{6} & 白色沉淀\tabularnewline
& 氨溴索、氧氟沙星 & 白色混浊\tabularnewline
环丙沙星 & 维生素C、5%碳酸氢钠 & 沉淀\tabularnewline
& 氨苄西林 & 环丙沙星沉淀\tabularnewline
亚胺培南西司他丁 & 乳酸钠注射液 & 沉淀\tabularnewline
万古霉素 & 氨茶碱 & 沉淀\tabularnewline
\bottomrule
\end{longtable}
\paragraph{中药注射液的配伍禁忌}

中药因其历史悠久、标本兼治、不良反应相对较小而被广泛应用。随着制剂水平的提高,中药注射剂已成为临床工作中常用的品种,新的配伍禁忌也不断出现(见表\ref{tab8-2})。

\begin{longtable}[]{@{}lll@{}}
    \caption{中药注射液的配伍禁忌}
\label{tab8-2}\\
\toprule
\endhead
配伍药物 & 配伍药物 & 结果\tabularnewline
\midrule
复方丹参 & 喹诺酮类注射液 & 沉淀\tabularnewline
灯盏花素 & 甲硝唑注射液 & 立即沉淀\tabularnewline
双黄连 & 氨基糖苷类药物 & 黄色沉淀\tabularnewline
& 氨苄西林 & 混浊\tabularnewline
& 林可霉素 & 混浊\tabularnewline
& 喹诺酮类药物 & 沉淀\tabularnewline
盐酸川穹嗪 & 头孢哌酮 & 沉淀\tabularnewline
穿琥宁 & 氨基糖苷类 & 沉淀\tabularnewline
& 喹诺酮类注射液 & 沉淀\tabularnewline
莪术油 & 头孢哌酮 & 变棕色\tabularnewline
\bottomrule
\end{longtable}

\subsection{抗菌药物的体内相互作用}

DDI通常由药动学和(或)药效学方面的改变引起的,药效学方面的相互作用包括无关、协同、相加和拮抗4种。药动学方面主要由于药物在吸收、分布、代谢和排泄方面的相互影响引起。

\subsubsection{药动学方面的影响}
\paragraph{影响吸收的相互作用}

口服给药为常见的给药途径,影响因素也较多。介导药物肠道吸收的转运方式主要有简单扩散和主动转运两种。简单扩散主要受胃肠道功能及理化环境的影响,主动转运则主要受转运体功能状态的影响。

(1)胃肠道pH值改变。以简单扩散方式吸收的药物,吸收速度受胃肠道pH值影响。一般情况下,弱酸性药物在酸性环境下,解离度低,吸收快。碱性药物在碱性环境下,解离度低,易吸收。因此,弱酸性药物与抑酸药或碱性药物联用,吸收速度与程度明显下降;若与酸性药物联用,吸收速度与程度明显增加。弱碱性药物则相反。

(2)肠道转运体竞争现象。肠道内存在多种转运体参与药物吸收,介导药物肠道主动转运的转运体及相关药物,不过有临床意义的实例尚鲜有报道。

(3)吸收与配合作用。多价钙、镁、铝、铁、铝等阳离子可与喹诺酮类药物形成不溶性配合物,减少后者的吸收,使其生物利用度降低,应避免联用。同时钙、镁、铝等离子也可与四环素结合形成溶解度低的沉淀物或复合物,可干扰后者的吸收。

(4)胃肠道功能。胃排空速度影响药物进入小肠的时间,因而影响药物的吸收。
\paragraph{影响分布的相互作用}

药物进入血液后,部分溶于血浆转运至作用部位或进入其他无效部位而贮存。但大部分与血浆蛋白,特别是白蛋白形成可逆性结合,使血浆中游离的药物以及蛋白质结合的药物之间形成动态平衡。当两种药物同时存在时,与蛋白质亲和力较大的药物,可将另一亲和力较小的药物自结合状态置换出来,这样就可使后一种药物的游离浓度相对增高,到达作用部位的药物浓度也就相应增多。如保泰松及水杨酸盐类,可自血浆蛋白中置换磺胺类药物,从而增强后者的抗菌作用。
\paragraph{影响药物代谢的相互作用}

药物的生物转化主要在肝脏进行。已知肝脏有多种药物代谢酶,参与药物代谢的CYP酶主要有CYP3A4、CYP1A2、CYP2C9、CYP2C19和CYP2D6五种。其中,约55%药物由CYP3A4代谢,20%药物由CYP2D6代谢,15%药物由CYP2C9和CYP2C19代谢。倘若所用的药物中,有的能促使或抑制肝微粒体酶的活性,就必然会影响另一药物的代谢或其本身的代谢,转而缩短或延长药物的作用时间。如利福平对氯霉素代谢酶的诱导,而降低氯霉素在血和脑脊液中的浓度。
\paragraph{影响排泄的相互作用}

肾脏是药物排泄的主要器官,药物的排泄与尿液的pH值有关。如酸性尿使保泰松、磺胺类及水杨酸类的排泄减少,而碱性尿则使之增加。如两药竞争同一主动转运系统,则一种药物可抑制另一种药物的主动转运,减少其排泄,延长其作用时间。临床常见案例为丙磺舒与β内酰胺类药物的联用。两药在临床治疗剂量使用时竞争主动转运系统,使两药的血药浓度均明显增加,故丙磺舒可增强和延长β内酰胺类药物的疗效。

\subsubsection{药效学方面的影响}
\paragraph{相加作用}

指两种药物联用的效应等于单用各药双倍效量的效应。相加作用的两药联用时均应减量,否则可引起药物中毒。如阿托品与氯丙嗪合用可引起胆碱能神经功能低下的中毒症状。
\paragraph{协同作用}

指两种药物联用后的药物效应大于各药单用双倍剂量的效应。如头孢菌素与氨基糖苷类抗生素联用可增强杀菌作用;降压药与利尿药联用,可增强降压作用。
\paragraph{拮抗作用}

指两种药物联用时的药效小于各药单用的效用。如甲苯磺丁脲的降糖作用是通过促进胰岛β细胞释放胰岛素,但这一作用可被氢氯噻嗪类利尿药所拮抗,可抑制胰岛β细胞释放胰岛素。

总之,为减少不良反应的发生,增强某种药物的疗效,经常通过合并用药以达到最佳临床效果,但配伍不当易产生不良后果。流行病学研究提示,配伍用药品的种类越多,不良反应发生率越高,发生相互作用种类越复杂(见表\ref{tab8-3})。

\begin{longtable}[]{@{}lll@{}}
    \caption{生理配伍禁忌}
    \label{tab8-3}\\
\toprule
\endhead
配伍药物 & 配伍药物 & 结果\tabularnewline
\midrule
亚胺培南西司他丁 & 更昔洛韦 &
引起癫痫 发作\tabularnewline
氨苄西林 & 10%葡萄糖溶液 & 过敏反应增加\tabularnewline
第一代头孢菌素 & 氨基糖苷类 & 肾毒性增加\tabularnewline
阿米卡星 & 林可霉素 & 毒性增加\tabularnewline
克林霉素 & 氨基糖苷类 & 对神经肌肉阻滞作用增加\tabularnewline
喹诺酮类 & 氨茶碱 & 氨茶碱毒性增加\tabularnewline
氨基糖苷类 & 异丙嗪 & 掩盖耳毒性\tabularnewline
氨基糖苷类 & 葡萄糖苷 & 肾毒性增加\tabularnewline
红霉素 & 氨茶碱 & 毒性增加\tabularnewline
林可霉素 & 糖皮质激素 & 抑制免疫作用增加\tabularnewline
\bottomrule
\end{longtable}

\subsection{抗菌药物体内相互作用的发生机制}

由于药效学方面的DDI比较明显,临床药师和医师很容易掌握和趋利避害。临床常见的DDI主要出现在药动学方面,即药物在吸收、分布、代谢和排泄方面的相互影响,其中代谢方面的DDI约占全部DDI的40%,最具有临床意义。绝大多数药物在肠道和(或)肝脏中要经过细胞色素P450(CYP酶系)代谢,CYP酶系被抑制或被诱导是导致代谢性DDI的主要原因,而其中酶抑制作用所致DDI的临床意义远远大于酶诱导作用,约占代谢性相互作用的70%,酶诱导引起的DDI约占23%。

药物转运蛋白也是产生DDI的一个重要的因素。多药耐药现象(multidrug
resistance,MDR)就是其中的一个重要方面。MDR的机制是由位于细胞膜上的一系列ATP结合膜转运蛋白将药物从细胞内外排,其中主要是MDR和多药耐药相关蛋白(multidrug
resistance
associated-protein,MRP)家族,MDR主要转运疏水性阳离子化合物,MRP既转运疏水性非带电化合物,也转运水溶性的阴离子化合物。现在研究最多的MDR是P-糖蛋白(P-glycoprotein,P-gp)。另外,肝脏通过药物转运蛋白对某些药物具有主动摄取和浓集的作用,药物对这些转运蛋白的抑制也是产生DDI的重要原因之一。

\subsubsection{细胞色素P450酶系}

药物在体内的代谢包括两相反应:Ⅰ相反应是氧化还原反应,主要涉及CYP酶家族;Ⅱ相反应是结合反应,涉及谷胱甘肽、葡萄糖醛酸、硫酸盐和甘氨酸等。通常情况下,一种药物要经过多个CYP酶代谢,仅少数药物经单一的药酶代谢。估算认为,大约有60%的处方药需经过CYP酶代谢。现在已经确定的细胞色素P450家族为18个家族42个亚族。参与代谢的CYP酶主要是CYP3A4、CYP1A2、CYP2C9、CYP2C19、CYP2D6五种,占CYP酶的95%。其中大约55%的药物经CYP3A4代谢,20%经CYP2D6代谢,15%经CYP2C9和CYP2C19代谢。

药物对CYP酶的抑制包括①相互竞争性抑制:通常发生在两种药物都是同一酶的底物时。②作用机制基础上的抑制:也叫自杀性抑制,如红霉素等大环内酯类抗生素在肝脏经CYP3A4代谢,脱去氨基糖分子中的叔胺基的N-甲基,代谢物与酶分子中血红蛋白的亚铁形成亚硝基复合物,从而使酶失活。14元环的红霉素和克拉霉素作用最强,罗红霉素和16元环的交沙霉素、乙酰螺旋霉素次之,15元环的阿奇霉素最弱,基本没有抑制作用。③非选择性抑制:指药物对多个CYP亚族酶都有抑制作用。如西咪替丁咪唑环上的N与CYP亚铁血红素部分的配体非选择性结合,引起酶活性障碍。由于每个CYP都有这个配位,因此西咪替丁对所有的CYP都有影响,但主要是影响CYP3A4,其次是CYP2D6;另外酮康唑、咪康唑、伊曲康唑、克霉唑中的杂环氮原子与CYP血红蛋白中的铁原子结合,从而非选择性抑制酶活性。特比萘芬不含杂环氮原子,没有酶抑制作用。

\subsubsection{P-gp}

P-gp是一种细胞膜上的磷糖蛋白(phosphor-glycoprotein)相对,分子质量为170000,由1280个氨基酸组成,它是ATP结合盒转运蛋白超家族成员之一。推测P-gp以类似药泵模型的方式转运药物,但其转运药物的确切分子机制迄今尚未完全阐明。
\paragraph{吸收方面}

P-gp的底物广泛,它所介导的药物外排是口服药物吸收差异和生物利用度变异的一个主要原因。P-gp的底物如地高辛、环孢素、他克莫司等吸收很容易受到P-gp的抑制剂(如维拉帕米)或诱导剂(如利福平)的影响,导致生物利用度增加或减少。对于一些治疗指数窄的药物,生物利用度的改变可以引起血药浓度的相应改变而导致中毒或治疗无效。

P-gp对药物的转运存在饱和性,给予高剂量的口服药物,肠道中药物的浓度远远大于其代谢速率常数($K_m$
)时,P-gp的转运即可到达饱和,但是对于一些水溶性差、分解慢、分子量大的药物如紫杉醇和环孢素,其中游离药物浓度始终低于$K_m$ ,即使给予高剂量P-gp仍影响药物的吸收。
\paragraph{分布方面}

P-gp广泛分布在血脑屏障、胎盘屏障、血睾屏障等毛细血管内皮细胞,这些部位的P-gp可以减少药物在脑内、胎盘、睾丸等部位的分布,导致这些部位血药浓度降低。如P-gp降低长春新碱、鬼臼毒素等脂溶性高的药物在脑组织中的含量;而维拉帕米等P-gp抑制剂可减少P-gp的外排作用,提高药物的疗效或产生毒性。
\paragraph{代谢方面}

P-gp对代谢的影响主要体现在肠道的首过效应,P-gp虽然不直接参与药物的代谢,但是P-gp和肠道中的CYP3A4酶分布相近,有共同的底物和调控机制,大部分药物在进入肠腔上皮细胞后,被P-gp泵出肠腔,部分药物会被重新吸收,在反复的泵出和重吸收过程中,延长了药物和CYP3A4酶的作用时间,增加了药物的肠道首过效应。
\paragraph{排泄方面}

分布在肾脏近端小管上皮的P-gp主要参与药物的排泄。实验发现,阿霉素在MDR(-/-)小鼠中的分泌要比MDR(+/+)低。实验还发现,阿奇霉素、长春新碱、柔红霉素和地高辛的胆汁排泄也受P-gp影响,清除率在MDR(+/+)的小鼠要大于MDR(-/-)小鼠。静脉给予地高辛后,野生型小鼠在90min内有16%的地高辛排泄到肠腔,而基因敲除小鼠的肠腔中没有发现地高辛。因此,可以认为P-gp介导的药物从肠道细胞的泵出功能不仅是吸收功能,也是药物的一种清除模式。
\paragraph{P-gp的诱导、抑制及其应用}

P-gp具有广泛的底物专属性,临床上的抗癌药物、免疫抑制剂、抗高血压药物、抗过敏药物、抗感染药物大部分是P-gp的底物或抑制剂。P-gp的转运具有饱和性,临床给药剂量通常超过该药的$K_m$
值,因此底物之间也存在竞争性的抑制作用。P-gp的过度表达也是肿瘤化疗失败的原因之一,临床可以利用P-gp抑制剂来提高靶组织的药物浓度。如脂溶性高的长春新碱容易通过血脑屏障,利用P-gp抑制剂可以减少长春新碱的外排,提高对脑内肿瘤的治疗浓度;P-gp抑制剂也可以提高口服紫杉醇、环孢素、他克莫司和地高辛等药物的生物利用度。P-gp抑制剂主要包括维拉帕米、缬沙坦、奎尼丁等,但是亲和力低,达到所需要的抑制效果时会产生很大的毒性。水果蔬菜、草药和其他食物中的一些成分如黄酮醇、香豆素类都能调节P-gp的活性,进而影响药物的体内处置。常用抗菌药物中,红霉素、利福平、左氧氟沙星是P-gp的底物,红霉素、伊曲康唑、酮康唑是P-gp的抑制剂。

\subsection{药物相互作用的预防和解决方法}

(1)尽可能减少不必要的联合用药,或者延长合用药物的给药间隔,并密切观察患者的病情和药物不良反应。

(2)应用常见药物相互作用数据库即时监测,对临床需要进行联合用药的处方进行即时审查,以减少潜在DDI的发生。

(3)根据理论基础上的药物相互作用模型,利用体外试验数据预测体内潜在的DDI。考虑到潜在代谢产物的影响,体外实验仅仅可以用于证实无某种药物相互作用;如果体外实验显示有可能存在某种相互作用,则需进行充分的体内实验。

\section{抗菌药物的不良反应及防治}

抗菌药物的应用的确为临床感染性疾病的防治做出了很大的贡献,使很多严重感染(如鼠疫、炭疽等),以及以往认为无法根治或挽救的疾病(如结核性脑膜炎、感染性心内膜炎等)的预后得到了很大的改观。但抗感染药物应用的同时也带来很多不良反应或后果,严重时致残或致死。抗菌药物除可以产生毒性反应和变态反应外,还可引起人体内菌群失调而导致的二重感染和病原微生物对药物产生或增加耐药性等。因此,在应用抗菌药物时除了要考虑其治疗作用,还应对其不良反应给予足够的重视,以减少药物不良反应给患者带来的伤害。

\subsection{毒性反应}

药物(包括抗感染药物)的毒性反应是指药物引起的生理、生化等功能异常和(或)组织、器官等的病理改变,其严重程度随剂量增大和疗程延长而增加。其机制可为药物的化学刺激、人体细胞蛋白质合成或酶系功能受阻等,也可因宿主原有的遗传缺陷或病理状态而诱发。毒性反应和变态反应常互相掺杂,有时不易截然区分,如大环内酯类的胆汁淤积、磺胺类药物的肾功能损害、氯霉素的贫血等。

\subsubsection{毒性反应的分类}

毒性反应是抗菌药物所引起的各种不良反应中最常见的一种,主要表现在肝、肾、神经系统、血液、胃肠道、给药局部等方面。
\paragraph{肝脏}

肝为体内的主要代谢器官,对口服药尤甚。很多药物包括抗菌药物及其代谢物可引起肝脏损害,或影响其代谢酶的功能。其机制一般可分为:①中毒,主要由药物代谢引起。②过敏,与剂量大小虽无关系,但剂量大者的发生率有时较高。③药物对代谢酶的影响,分“酶促”和“酶抑”两类。能引起肝脏损害的药物主要有四环素类、红霉素酯化物、磺胺类药物、抗结核药(异烟肼、利福平等)、呋喃唑酮等,其他尚有β-内酰胺类(青霉素类、头孢菌素等)、两性霉素B等。临床表现主要有黄疸、上腹痛、肝大、转氨酶升高,重者可有全身出血倾向。伴变态反应者可有发热、皮疹、关节痛及嗜酸性粒细胞增多等。

静脉注射量较大或长期口服四环素有可能引起急性或亚急性肝脂肪变性,孕妇、长期口服避孕药者、肾功能或肝功能减退者及血浆蛋白低下者尤易发生。临床表现与急性病毒性肝炎相似,病情进展迅速,有中等程度黄疸、上消化道出血及全身出血倾向等,妊娠者多早产死婴,病死率高。病理检查可见肝细胞内广泛分布小脂滴,主要为三酰甘油,细胞核仍位于中心。免疫荧光检查显示四环素定位于线粒体,干扰细胞内蛋白质合成,进而使脂蛋白合成减少,加以三酰甘油排泄受损,因而导致脂类在肝脏中沉积。

红霉素的酯化物可引起胆汁郁积性黄疸,鉴于其发生率较高,因此有人认为属毒性反应。但由于①初次发病一般在服药10~20d以后,再次用药则发病时间提前;②发病与剂量大小无关;③肝组织学检查偶见肝细胞坏死,主要为胆汁淤积及嗜酸性粒细胞浸润,故不少人认为是变态反应所致。临床表现主要有黄疸、瘙痒、上腹痛,可伴发热,上腹痛较显著者可误诊为胆管疾病;恢复迅速,无后遗症,周围血象示嗜酸性粒细胞增多。大环内酯类中最易引起本病者为依托红霉素(俗称“无味红霉素”),其他如红霉素乳糖酸盐、交沙霉素、罗红霉素等偶也可引起。

磺胺类药物也有引起肝脏损害的可能,临床表现类似肝炎,肝活检呈肝细胞坏死,可伴诱发热、关节痛、皮疹、嗜酸性粒细胞增多等,严重者可发展为急性或亚急性重型肝炎。大环内酯类与磺胺类引起肝功能损害的发病机制可能与毒性反应和变态反应均有关系。

肝脏损害是抗结核药物应用过程中最常见的并发症,典型的药物有异烟肼、利福平、对氨基水杨酸、吡嗪酰胺、乙胺丁醇等。异烟肼在肝内分解为异烟酸和乙酰肼,乙酰肼可与大分子物质以共价键结合而引起肝功能损害,快乙酰化者由于乙酰肼积聚较多,故发生率较高。异烟肼的肝毒性可分为轻型和肝炎型两种,前者占应用者中10%,后者约1%~2%。轻型可完全无症状,仅有转氨酶升高;而肝炎型则可出现病毒性肝炎的各种表现。利福平对肝脏毒性的发生率更高,可达20%,其临床表现为转氨酶升高、肝大、黄疸等,以一过性转氨酶升高最为多见;与异烟肼合用尤易发生。此外,利福平尚可使胆红素增多而导致高胆红素血症。

呋喃唑酮、呋喃妥因等的导致肝功能损害可能是一种免疫反应,主要表现为胆汁淤积,偶伴有散在性肝细胞坏死,多发生在用药后数周。临床上可出现发热、皮疹、黄疸、嗜酸性粒细胞增多等,预后良好。

两性霉素B的疗程一般较长,在过程中常出现肝毒性,如转氨酶升高、黄疸、肝大等,剂量较大时尤为常见。

其他抗菌药物如β-内酰胺类(青霉素类、头孢菌素等)、氟喹诺酮类(依诺沙星、氧氟沙星等)、林可霉素类、大环内酯类(麦迪霉素、乙酰螺旋霉素、红霉素及其新衍生物罗红霉素、克拉霉素、阿奇霉素等)、灰黄霉素等均偶可引起肝功能损害,表现为一过性或短暂的血清转氨酶升高。
\paragraph{肾脏}

肾脏血管丰富,是大多数抗菌药物的主要排泄途径,药物在肾皮质内常有较高浓度积聚,因此肾毒性相当常见,表现轻重不一,轻者呈单纯尿常规和(或)血生化异常,重者可有不同程度肾功能减退至尿毒症等。肾毒性所造成的病变时,以肾小管病变最为常见,还有间质性肾炎、肾间质水肿等,其他造成肾功能损害的因素有肾血流灌注减少、药物结晶阻塞肾小管或尿路等。

发生肾毒性的抗菌药物主要有氨基糖苷类、多黏菌素类、两性霉素B、万古霉素、头孢菌素类、青霉素类、四环素类、磺胺类药物和利福平等。大多为可逆性,停药后可逐渐恢复。

(1)氨基糖苷类与肾小管的刷边膜易于结合,局部浓度高,从而直接损伤肾小管上皮细胞,严重时引起肾小管坏死和急性肾功能衰竭,其毒性大小与药物浓度成正比。老年人、脱水者、两种以上肾毒性药物联用者尤易发生。庆大霉素较阿米卡星和奈替米星更易引致肾毒性。

(2)多黏菌素类常量即可引起肾毒性,应用后肌酐清除率多有下降;约有20%的患者在用药后4d内发生蛋白尿、血尿和少尿等,约2%出现肾小管坏死。

(3)两性霉素B可改变肾小管上皮细胞通透性,导致排清障碍而增加尿钾排泄,从而引起多种肾功能损害,发病率高,剂量较大时可能导致不可逆的急性肾功能衰竭。还可引起肾血管收缩,导致肾皮质缺血和肾小球滤过率减少,影响浓缩功能而出现肾性尿崩症。

(4)头孢菌素类中的头孢噻啶由于肾毒性强现已基本不用,其他第一代头孢菌素如头孢噻吩和头孢唑林在用量较大时也具有一定肾毒性,与其他肾毒性药物如氨基糖苷类、强利尿剂等合用时尤宜注意。

(5)青霉素类中甲氧西林主要引起急性间质性肾炎,与应用剂量大小无关,使用者约有10%~15%发病,氨苄西林、阿莫西林等偶也可引起。一般于用药7~10d后发生皮疹、发热、嗜酸性粒细胞增高、血尿等,甚至导致进行性肾功能损害。

(6)磺胺类药物的肾毒性主要由药物在肾小管内析出结晶,引起血尿或梗阻性肾病,甚至发生少尿或急性肾功能衰竭。本类药物还可通过免疫反应引起急性间质性肾炎、肾小球炎、坏死性血管炎等。

(7)其他如四环素和土霉素,在肾功能中重度减退时应用,可因其抗合成代谢作用而加剧氮质血症、酸中毒等。利福平引起的间质性肾炎,常伴有流感样综合征,因在肾小球基膜处曾找到沉积的相应抗体,故可认为可能有免疫反应参与。万古霉素与其他多肽类抗生素(多黏菌素类、杆菌肽等)一样,主要损及肾小管,其肾毒性发生率一般为5%,与庆大霉素合用可增至30%以上。

肾毒性(氨基糖苷类、两性霉素B、万古霉素等)的最早表现为蛋白尿和管型尿,此时尿量可无明显改变,继而尿中出现红细胞,并发生尿量改变(增多或减少)、pH值改变(大多自酸性转为碱性)、氮质血症、肾功能减退、尿钾排泄增多等,其损害程度与剂量及疗程成正比(间质性肾炎除外)。一般于给药后3~6d发生,停药后5d内消失或逐渐恢复。少数患者可出现急性肾功能衰竭、尿毒症等。
\paragraph{神经精神系统}

(1)中枢神经系统。青霉素全身用药剂量过大和(或)静脉注射速度过快时,可对大脑皮质产生直接刺激作用,出现肌痉挛、惊厥、癫痫
、昏迷等严重反应,称为“青霉素脑病”,一般用药后早则8h,迟至9d发生。尿毒症时肾排泄青霉素类的功能降低,血浆蛋白对该类药物的结合力下降,致使游离药物浓度增高,因而有较多药物通过血脑屏障,脑膜有炎症时更甚。当脑脊液中的青霉素浓度超过8IU/mL时,可因大脑皮质兴奋性增高而诱发癫痫
发作。异烟肼、环丝氨酸等的剂量过大可使脑内谷氨酸脱羧酶的活性减低、维生素B{6}
缺乏和γ-氨基丁酸的含量减少而导致癫痫
。

亚胺培南西司他丁和氟喹诺酮类易透过血脑屏障,在中枢神经系统的浓度过高,容易出现头痛、头晕、焦虑、烦躁、失眠等症状,当采用相应剂量(按肾功能计算)较大时,可发生惊厥、癫痫
。

鞘内或脑室内注入青霉素类、氨基糖苷类、多黏菌素B、两性霉素B等,常用量时即可引起一些脑膜刺激征如头痛、颈项轻度强直、呕吐、感觉过敏、背和下肢疼痛、尿频、发热等,脑脊液中的蛋白和细胞也有增加,注射后即刻或数小时内发生,多次注射后蛛网膜下组织可发生粘连。用量较大时可发生高热、惊厥、昏迷、尿潴留、呼吸和循环衰竭,甚至导致死亡,

(2)脑神经。第8对脑神经损害或耳毒性为氨基糖苷类的重要毒性反应之一,与其他耳毒性药物如强利尿剂(呋塞米、依他尼酸等)、水杨酸类、抗癌药(长春碱、长春新碱)、砷、汞、奎宁、万古霉素、多黏菌素类等合用时毒性反应将加剧,噪音、失水、缺氧、肾功能减退等均系诱发因素,老年人和婴儿尤易发生。对高敏易感者及有家族史者更应特别注意。对肾功能不全者应缜密观察氨基糖苷类的耳毒性,但必须指出耳、肾毒性可同时出现于耳、肾功能原来正常的患者。耳毒性的发生机制与内耳淋巴液中药物浓度较高有关,但内耳组织并无浓缩药物的功能,只是因为药物在内耳淋巴液中的半衰期远较血浆半衰期长(10~15倍)而已。

由于药物在内耳中的滞留,从而引起一系列的生化和组织学的反应,以柯蒂器受累最显著。早期变化为可逆性,但当柯蒂器毛细胞消失后则不能再生,使耳聋成为进行性和永久性。耳前庭损害的主要病变在周围迷路感觉上皮。

氨基糖苷类均具有一定耳毒性,分为耳蜗损害和前庭损害。耳蜗损害较为严重者有新霉素、卡那霉素、阿米卡星等。其临床表现的先兆为耳饱满感、头晕、耳鸣等,也可并无预兆,高频听力先有减退,继以耳聋。孕妇应用氨基糖苷类药时药物可通过胎盘而影响胎儿耳蜗。对耳前庭损害较显著者为链霉素和庆大霉素。氨基糖苷类中以奈替米星的耳毒性为弱。其他抗生素如万古霉素、多黏菌素类、米诺环素、卷曲霉素等也具有一定耳毒性,红霉素,氯霉素等偶也引起。耳前庭损害的表现为眩晕、头痛、急剧动作时可发生恶心、呕吐,伴眼球震颤;严重者可致平衡失调,步态不稳,每一动作停止后似仍在继续进行,向左右转侧有持续滚动感,前俯有倾跌感。大多为暂时性,少数可持续较长时间。

在较大量和较长期应用时,抗菌药物对视神经偶也可产生一定毒性。氯霉素长期口服或滴眼,有引起视神经炎、视神经萎缩甚至失明的可能。口服乙胺丁醇2~6月内可能发生球后视神经炎、视网膜出血及色素变化,应用较大量者更甚。链霉素、异烟肼等也可引起视神经炎及视神经萎缩。磺胺类药物、卡那霉素、新霉素、四环素等对视神经也有影响。

(3)神经肌肉接头。大剂量氨基糖苷类静脉快速注射,因神经肌肉接头阻滞,可表现为四肢软弱、周围血管性血压下降,以及心肌抑制症状等,严重者可因呼吸肌麻痹而危及生命。除氨基糖苷类外,多黏菌素、林可霉素类、四环素类等也偶可引起。神经肌肉接头的阻滞现象现已少见,但重症肌无力和肌营养不良者应用氨基糖苷类等时仍需注意这一现象的可能发生。

(4)周围神经。链霉素、多黏菌素类、庆大霉素等注射后可引起口唇及手足麻木,严重者伴头昏、面部和头皮麻木、舌颤等。异烟肼和乙胺丁醇可因维生素B{6}
缺乏而导致周围神经炎。

(5)神经症状。首先必须辨别精神症状是否为原发性疾病或其他药物所引起。抗菌药物如氯霉素、青霉素、异烟肼等有时可引起精神症状,如幻视幻听、定向力丧失、狂躁吵闹、失眠、猜疑等;或表现为抑郁症,可有自杀企图。链霉素和四环素类偶可引起精神失常或欣快症。
\paragraph{血液系统}

(1)贫血。很多抗菌药物可以引起贫血,发病机制也有多种。氯霉素是其中突出的一种,可引起红细胞生成抑制所致的贫血、再生障碍性贫血(再障)、G-6-PD缺乏所致的贫血。当氯霉素血浓度较高,特别在较长期使用时,氯霉素分子中的“硝基苯基团”或“苯环对位基团”可损害红细胞的线粒体抑制其生成。血红素合成酶(铁螯合酶)紧密结合在线粒体内膜上,线粒体受损与剂量大小和疗程长短有关,一般在用药期间发生,停药后大多恢复。除周围血象呈明显贫血外,骨髓象可示红细胞成熟受阻,早期红细胞内出现空泡;实验室检查可出现血清铁和血浆饱和铁升高。

氯霉素是最易引起再障的抗菌药物,与剂量大小无关,发生率虽低,但病死率高于50%。多见于12岁以下的女性儿童,患者大多有慢性荨麻疹、湿疹等过敏性疾病。G-6-PD参与红细胞无氧酵解途径,通过还原型谷胱甘肽而保持红细胞的稳定性。G-6-PD缺乏时红细胞已处于不稳定状态,氯霉素可使还原型谷胱甘肽氧化,因而易于诱发溶血性贫血。在G-6-PD缺乏时可诱发溶血性贫血的抗菌药物尚有磺胺类药物、呋喃类等。

两性霉素B可与红细胞膜上的固醇结合,使细胞膜的通透性发生改变而发生溶血。

β-内酰胺类如青霉素类、头孢菌素类等偶可引起溶血性贫血。其机制为附着于红细胞膜上的抗原和相应抗体,或免疫复合物在补体的作用下非特异地吸附在红细胞膜上,并与其发生作用而引起。

(2)白细胞计数和血小板计数减少。很多抗菌药物如氯霉素、磺胺类药物、β-内酰胺类、大环内酯类、氟胞嘧啶、氨基糖苷类、四环素类、两性霉素B、灰黄霉素等均可引起白细胞和(或)血小板计数减少,但发生率较低,停药后恢复快,临床上可全无症状。白细胞计数减少以氯霉素所致者较多见,该药品可抑制幼粒细胞的蛋白质合成,在后者的胞质中出现空泡,并发生退行性改变。头孢菌素所致的血小板计数减少与免疫机制有关。药物与血浆蛋白的结合成为全抗原,抗原与相应抗体结合为免疫复合物,在补体的参与下覆盖在血小板膜上而导致血小板破坏。此外,氯霉素、灰黄霉素等偶可引起粒细胞缺乏症而出现高热、咽痛、口腔糜烂等。

(3)凝血机制异常。由于β-内酰胺类(主要为青霉素类和头孢菌素类)的抗菌活性强、毒性低、用量较大,应用后有因凝血酶原减少、血小板凝集功能异常等发生出血,如鼻出血、消化道出血(包括大便隐血阳性)等的可能,虽大多属轻、中度,但仍值得重视。文献报道较多的是头孢菌素类中的拉氧头孢,其他尚有头孢哌酮、头孢孟多、头孢唑林等,以及青霉素类中的青霉素、羧苄西林、替卡西林、甲氧西林、阿洛西林等,大多与用量较大有关。

β-内酰胺类可抑制肠道内产生维生素K的菌群,而维生素K是肝细胞微粒体羧化酶必需的辅助因子,参与凝血酶原前体中谷氨酸的γ羧化反应,因而其缺乏将使凝血酶原的合成减少和依赖维生素K的凝血因子Ⅱ、Ⅶ、Ⅸ、Ⅹ等的水平降低。拉氧头孢、头孢哌酮、头孢孟多等的结构中尚含有N-甲基硫化四氮唑,后者与谷氨酸的结构相似,因而可干扰维生素K所参与的羧化反应。

二磷酸腺苷(adenosine
diphosphate,ADP)是诱导血小板凝集的重要刺激因子,现已发现多种β-内酰胺类可阻断这一作用,剂量增大时更甚。现已证实,拉氧头孢、青霉素、羧苄西林等能非特异性地与血小板膜结合,从而阻断了ADP与特异性受体的结合,使血小板的凝聚功能发生障碍。
\paragraph{胃肠道反应}

大多数抗菌药物口服或注射后胆汁中浓度较高者均可引起一些胃肠道不良反应,如恶心、上腹不适、胀气、腹泻等,偶伴呕吐。化学性刺激是胃肠道反应的主要原因,但也可是肠道菌群失调的后果,或二者兼而有之。四环素类(多西环素、金霉素)引起的胃肠道反应最为常见。大环内酯类中以红霉素口服后的不良反应最常见,乙酰麦迪霉素、罗红霉素、阿奇霉素、克拉霉素等的胃肠道不良反应较少且轻微。氯霉素、氨基糖苷类(链霉素、新霉素、卡那霉素、庆大霉素等)、磺胺类药物等口服后也易发生胃肠道反应,但反应程度较四环素类为轻。

除菌群交替性腹泻外,很多抗菌药物(不仅是林可霉素类)偶可引起假膜性肠炎。
\paragraph{局部刺激}

很多抗菌药物肌肉注射、静脉注射或气溶吸入后可引起一些局部反应。肌肉注射后发生局部疼痛者相当多见,可伴硬结形成,青霉素钾盐尤为突出,应用后诉剧痛者达20%。静脉滴注红霉素乳糖酸盐时,浓度过高或速度过快可导致血栓性静脉炎,伴不同程度的疼痛和静脉变硬。气溶吸入氨基糖苷类、两性霉素B等的浓度过高,易出现咽痛、呛咳等上呼吸道刺激症状。
\paragraph{其他}

其他毒性反应尚有乳齿黄染及牙釉质发育不全、灰婴综合征、颅内压升高、心脏损害、不纯制剂的发热反应、内毒素导致的治疗休克等。另外,抗菌药物的相互作用也有可能出现一些毒性反应。

四环素类可沉积在牙齿及骨质内,影响牙釉质及骨骼的发育。新生儿短期应用四环素类即可引起乳齿的色素沉着,染成黄或棕黄色。儿童服药次数较多者则除乳齿黄染外,并可导致牙釉质发育不全,从而易促成龋齿和使恒齿失去光泽而成暗灰色。妊娠25周以上的妇女服用四环素后,药物也可沉积于胎儿乳齿中。幼儿服用四环素类后,部分儿童可出现骨骼生长抑制现象。用于婴幼儿还可导致前囟隆起伴呕吐;用于成人偶有头痛、呕吐、视神经盘水肿等,除颅内压升高外无异常,因停药后症状迅速消失,故也称“良性颅内压升高症”。

早产儿和新生儿应用氯霉素每日剂量大于100mg/kg时,用药3~4d后出现呕吐、进行性苍白、发绀、循环衰竭、呼吸不规则,一般于症状出现后数小时内死亡,及时停药则有迅速恢复的可能。早产儿和新生儿由于肝内酶系发育不全,影响了葡萄糖醛酸与氯霉素的结合,加以早产儿和新生儿的肾脏排泄药物的功能较差,故血中氯霉素浓度常显著升高而导致此严重反应。

某些抗菌药物如两性霉素B、万古霉素等不易提纯,或其本身即是一种致热源,应用后(一般为静滴)数小时内即可出现寒战、高热等反应;应注意与变态反应中的药物热区别。

某些抗菌药物能够对心脏引起直接和间接损害,但发生率均较低。直接损害的如两性霉素B,可引起心肌损害,静脉滴注过快时有导致心室颤动、心脏骤停的可能。万古霉素静脉滴注也有引起心脏骤停的报道。青霉素大量静脉滴注时偶可引起暂时性心电图变化,可能是冠状动脉水肿导致的心肌缺血。动物实验显示氨基糖苷类可使心肌收缩力减弱。间接的心脏损害乃电解质紊乱所致,如两性霉素B、氨基糖苷类所引起的低血钾症等。

\subsubsection{毒性反应的防治原则}

(1)抗菌药物应用后均可发生一些毒性反应,某些是比较严重的。因此,应用任何抗菌药物前应充分了解其可能发生的各种反应及其防治对策,这对新上市的药物尤为重要。剂量宜按生理和病理状况(特别是肝、肾功能)而确定。因药动学的个体差异较大,故有条件时应定时监测血药浓度,毒性较大的抗菌药物如氨基糖苷类、万古霉素、氯霉素用于新生儿等有此指征。疗程必须适当,并及时停药。在疗程中严密观察可能发生的一切反应及其预兆,并作必要的血、尿常规,以及血小板计数和肝、肾功能的检查。

(2)毒性较强的抗菌药物如氨基糖苷类、两性霉素B、万古霉素、多黏菌素类等,对老年人、婴幼儿、孕妇等尤应特别注意。联用抗菌药物时应警惕协同毒性及相互作用的可能发生。早产儿和新生儿以不用氯霉素为宜,必须应用时要监测血药浓度;孕妇和乳妇应避免用四环素类。

(3)发生轻度毒性反应时一般可采用对症处理,中至重度毒性反应及时减量、停药或改用毒性反应较低的抗菌药物,必要时可加用肾上腺皮质激素。大多数毒性反应于停药后可迅速减轻或消失。

(4)除少数例外,避免在鞘内、脑室内应用抗菌药物,在胸腹腔、关节腔内注入抗菌药物一般无必要。

(5)氨基糖苷类、氟喹诺酮类、亚胺培南西司他丁等的静脉滴注速度宜慢,一次给药时间不宜少于1h。氨基糖苷类所致的神经肌肉接头阻滞可采用新斯的明静脉注射或肌肉注射处理,成人每次0.125~1mg。后二者引起的惊厥或癫痫
一般在减量、停药和应用地西泮静注(成人每次5~10mg)后可望控制。

链霉素注射后的手足麻木可应用葡萄糖酸钙、氯化钙等以减轻症状;异烟肼等引起的周围神经炎可用较大量的维生素B{6}
治疗;药物引起的血小板计数减少及出血可考虑输血或血小板;口服抗菌药物宜采用最小适宜量以减少反应,但一般不宜与氢氧化铝制剂或牛奶同服;肌肉注射疼痛显著者可加用局麻药(如利多卡因等);氯霉素引起的精神症状,患者可能有自杀企图,宜严加防卫。

\subsection{变态反应}

变态反应是应用抗菌药物后除毒性反应外最常见的不良反应,其发病机制为外来的抗原性物质与体内抗体间所发生的一种非正常的免疫反应。抗菌药物的分子结构比较简单,虽非蛋白质,但大多可作为半抗原,与体内(偶或体外)的蛋白质结合而成为全抗原,从而促使人体产生特异性抗体(或致敏淋巴细胞);当人体再次接触同种抗菌药物后即可产生各种类型的变态反应,所以几乎每一种抗菌药物均可引起一些变态反应。此类反应药理学上难以预测,与剂量无关,减少剂量后症状不会改善,必须停药。药物变态反应可波及全身各器官、组织,最多见者为皮疹,其他尚有过敏性休克、血清病型反应、药物热、血管神经性水肿、嗜酸性粒细胞增多症、溶血性贫血、再生障碍性贫血、接触性皮炎等。按其发病机制分属于四种类型。
\paragraph{Ⅰ型变态反应}

Ⅰ型变态反应又称速发型(immediate type
hypersensitivity),包括过敏性休克、支气管哮喘、喉头水肿、即刻型荨麻疹等。

过敏原(抗菌药物)可刺激人体B细胞产生IgE,再次接触后过敏原可与吸附在肥大细胞和嗜碱性粒细胞表面的IgE结合,致使细胞内环磷腺苷(cAMP)的生成锐减。cAMP具有有效控制肥大细胞和嗜碱性粒细胞颗粒内组胺释放的作用,组胺可作用于呼吸道和消化道的黏膜及皮肤,引起支气管哮喘、黏膜水肿、呕吐、腹泻、皮肤红肿、荨麻疹等。肥大细胞内的嗜酸性粒细胞趋化因子也同时释放,从而加重了局部组织的水肿及嗜酸性粒细胞浸润。血清素、慢反应物质的相继释出和缓激肽激活等使支气管痉挛及喉头水肿等进一步加剧,并导致小血管扩张、血管通透性增加。上述生物活性物质的共同作用导致有效血循环量减少、微循环障碍、组织缺血、血压下降,造成过敏性休克。

过敏性休克一般呈闪电样发作,甚至在注射针头尚未拔出即可发生,也可在皮试时出现。约半数患者的症状发生在注射后5min内,注射后30min内发生者占90%,但也有个别病例于数小时内或连续用药的过程中(甚至3周后)发病。多见于20~40岁的成年人,女性多于男性。各种用药途径均可引起,注射最多见。

为防止过敏性休克的发生,应用抗菌药物,特别是青霉素、链霉素等前必须详细询问既往史,其内容包括①以往用药史,是否用过青霉素类、氨基糖苷类等药物;②应用后有无荨麻疹、瘙痒、胸闷、发热等反应;③对其他抗菌药物如磺胺类药物、解热镇痛药等有无过敏;④个人有无变态反应性疾病如支气管哮喘、过敏性鼻炎、湿疹等;⑤家属中有无上述类似病史。有对照的青霉素皮试对预测包括过敏性休克在内的变态反应有一定价值,皮试阴性仍宜提高警惕。使用青霉素类各种制剂前必须先做皮试,已停用7d以上(小儿3d以上)而需要再次使用时应重做皮试;换用另一种批号也以再做皮试为妥。鉴于过敏性休克90%于给药后30min内发生,故应观察30min才可放行。

过敏性休克必须分秒必争就地抢救,切忌远道运输。肾上腺素为首选药物,成人患者可立即肌内注射0.1%的肾上腺素0.5~1.0mL,病情严重者于静脉内给药。本药可重复应用,剂量同上。其他选用药物有血管活性药物、扩容剂、肾上腺皮质激素、抗组胺药物、葡萄糖酸钙等。喉头水肿严重引起窒息时,应及早做气管切开术。

青霉素(G)以往在临床上的应用最广,发生过敏性休克也最为多见,发生率为0.004%~0.015%,病死率为5%~10%。链霉素、庆大霉素等氨基糖苷类和头孢菌素类次之,磺胺类、四环素类、林可霉素类、大环内酯类、链霉素、利福平等也偶可发生过敏性休克。青霉素类与头孢菌素类之间可发生交叉变态反应,发生率虽然不高,仍应密切注意。

皮疹,常见荨麻疹、斑丘疹、麻疹样皮疹,也有红斑、猩红热样皮疹、天疱疮样皮疹、湿疹样皮疹、结节样红斑、多形性红斑、紫癜、剥脱性皮炎、大疱性表皮松解萎缩性皮炎、渗出性红斑等,以后三者的预后较严重。每一种抗菌药物均可引起皮疹,青霉素所致者以荨麻疹及麻疹样皮疹最为常见,发生率为2%;链霉素所致者则多表现为广泛的斑丘疹,发生率约5%;氨苄西林所致者多为斑丘疹或荨麻疹,口服后发生率约7%,注射给药时可达20%;磺胺类药物所致者以麻疹样皮疹较多见,发生率为1.5%~2.0%。

皮疹多于治疗开始后10d左右出现,在以往曾接受同一抗菌药物的患者中,则可于数小时到1或2d内迅速出现;一般持续5~10d后消退。在抗菌药物应用过程中所发生的稀疏皮疹虽多数可自行消退,但因少数患者可发展为剥脱性皮炎而危及生命,故以及时停药为妥。对有轻型皮疹而必须继续用药者,则宜有相应挽救措施(肾上腺皮质激素、抗组胺药物等),并严密观察;如皮疹继续发展,并伴有其他变态反应和发热者应立即停药,同时加强抗过敏治疗。

血管神经性水肿是较常见的一种变态反应,绝大多数为青霉素所引起,其后果一般并不严重,但波及呼吸系统及脑部时有危及生命的可能。过敏性休克的呼吸道阻塞显然也是血管型水肿所致。四环素、氯霉素、红霉素、链霉素等也偶可引起本病。
\paragraph{Ⅱ型变态反应}

Ⅱ型变态反应又称细胞毒性型(cytotoxic type
hypersensitivity)。临床表现有溶血性贫血以及白细胞和血小板计数减少等,系由于吸附于细胞表面的过敏原(抗菌药物)与相应的抗体IgG、IgM或IgA结合后,在补体的参与下引起细胞表面的破坏和溶解所致。溶血性贫血临床少见,且很少伴有其他过敏反应,持续时间可达数周,停药后溶血可停止。青霉素类与某些头孢菌素类可引起此类变态反应。
\paragraph{Ⅲ型变态反应}

Ⅲ型变态反应又称免疫复合物,包括血清病样反应和药物热。

血清病样反应的发病机制为:大抗原决定簇刺激人体产生特异性IgG,两者结合成可溶性复合物,沉积在毛细血管壁上,并激活补体系统,生成血管活性物质,导致局部充血与水肿。嗜中性趋化因子等的产生和释放可造成局部中性粒细胞浸润。粒细胞溶酶体酶的释放可引起组织的炎症和破坏。血清病样反应90%的病例见于应用青霉素的患者,另有一些见于青霉素、链霉素联用。其症状与血清病疾病相同,有发热、关节疼痛、荨麻疹、淋巴结肿大、腹痛、蛋白尿、嗜酸性粒细胞增多等。除并发喉头水肿或脑部的血管神经性水肿者外,血清病样反应是一种较轻的变态反应,脱离与药品接触外无须特殊处理。其他如四环素、氯霉素、红霉素和新生霉素等偶可引起。

几乎每一种抗菌药物都可以引起药物热,以青霉素最多,链霉素、新生霉素、多黏菌素B、氨苄西林次之,其他青霉素类、头孢菌素类、庆大霉素、卡那霉素、四环素类也可引起。药物热的潜伏期一般为7~12d,短则1d,长者达数周。热型大多为弛张热或稽留热。多同时伴有皮疹,皮疹的出现且可先于发热。停药后2或3d内大多可以退热,周围血象中嗜酸性粒细胞没有增多。药物热的主要诊断依据如下:①应用抗菌药物后感染得到控制,体温下降后又再上升。②原来感染所致的发热未控制,应用抗菌药物后体温反而较未用前为高。③发热或热度增高不能用原有感染解释,而且也无继发感染的证据。患者虽有高热,但其一般情况良好。④某些患者尚伴有其他变态反应,如皮疹、嗜酸性粒细胞增多等,停用抗菌药物后热度迅速下降或消退。
\paragraph{Ⅳ型变态反应}

Ⅳ型变态反应又称迟发型、细胞介导型,主要为经常接触抗菌药物如青霉素、链霉素等患者发生的接触性皮炎。

药物可与皮肤组织结合成复合抗原,并引起针对这一抗原的细胞免疫。当皮肤再次接触同一药物时,致淋巴细胞在局部被激活,产生一系列淋巴细胞,导致单核细胞浸润性皮炎。由于组织分化、增殖需要一定时间,故反应出现较迟,是一种迟发型变态反应。一般于接触后3~12个月内发生。皮炎一般出现于双手、手臂、眼睑、颈部等处,表现为皮肤瘙痒、发红、丘疹、眼睑水肿、湿疹等,停止接触后皮炎逐渐消退。

\section{抗菌药物临床应用管理}

\subsection{概述}

\subsubsection{抗菌药物不合理使用的表现}

(1)适应证。掌握不严格,如一些非细菌感染性疾病大量应用抗菌药物。

(2)药物选择。对于治疗来说没有根据药物的药代、药效特性、病原学依据来选药,特别是经验治疗,存在着过于广覆盖的问题。对于预防用药,存在着缺乏针对性、选药档次偏高、用药时机不当、用药时间过长等问题。

(3)联合用药。所选的联合应用的药物,不能达到增强疗效、减少不良反应的目的,或者病程没有必要联合用药。如相同药理作用的药物重复使用,加重了药物的不良反应,而没有扩大抗菌谱或增强疗效。

(4)给药方案。存在着药液配制溶媒选择不当、配制方法不妥、剂量应用不当以及药液浓度过高或滴速过快、给药间隔不当等问题。

(5)换药。存在着换药过于频繁的情况。抗感染药物的使用,要在用药3d后根据临床表现、化验指标等因素确定疗效,再根据实际情况判断是否换药。

(6)过敏史。存在着未询问患者过敏史的情况。抗菌药物在应用中,有些要求做皮肤过敏试验,有些没要求。但过敏反应在此类药物中很常见,详细询问过敏史非常重要。有些不良反应就是因为不知患者过敏史所造成的,特别要注意药物的交叉过敏问题。

\subsubsection{抗菌药物滥用的结果}
\paragraph{耐药性增加}

尽管细菌耐药是一种自然的生物现象,它的发生是不可避免的,但抗菌药物的不规范使用,可以加速细菌耐药性的产生。过量使用或长时间使用抗菌药物,会使一些细菌发生变异,导致常用的抗菌药物耐药性产生或增加;频繁更换药物而未完成疗程,或者使用亚有效量的抗菌药物,均会使与之接触而未被杀灭的细菌对其产生耐药。随着广谱抗菌药物的广泛应用,细菌耐药现象日趋严重。目前,细菌耐药产生的速度远远高于新药开发的速度。近年来陆续出现的耐药革兰阳性菌MRSA、VRE,产ESBLs和头孢菌素酶的革兰阴性菌,超广谱抗生素广泛应用选择出来的嗜麦芽窄食单胞菌以及鲍曼不动杆菌、多药耐药铜绿假单胞菌等,已成为临床抗感染治疗的难点,严重影响着抗感染治疗的有效率与感染导致的病死率的升高。由于耐药现象的出现,使得一些本来得以控制的感染性疾病又变得困难起来,如结核杆菌引起的传染性疾病肺结核,多年前已基本得以控制,但是近年来不仅结核的发病率明显增加,而且随着耐药结核杆菌的出现,治疗非常困难。这不仅会增加患者医疗上的花费(治疗耐药性结核花费的社会资源是治疗非耐药结核的10倍以上),而且增加了患者遭受不良反应的痛苦(抗结核药肝脏毒性明显),甚至可引起病死率增加。长此以往,可能会退回到20世纪80年代以前的状态,没有抗生素使用,人类将再一次面临很多感染性疾病的威胁。
\paragraph{不良反应增加}

抗菌药物同其他药物一样,是一把双刃剑。抗菌药物进入人体以后,在抑制、杀灭细菌发挥治疗作用的同时,也会引起很多的不良反应,甚至造成药源性疾病,导致患者死亡。用的药物越多,引起不良反应的机会越高。我国药品不良反应监测中心的记录显示,我国的药品不良反应三分之一是由抗菌药物引起的,这个比例和抗菌药物的使用比例是一致的。
\paragraph{增加卫生资源的浪费与医疗花费}

由于抗菌药物使用的起点高以及超时、超量等不合理应用现象,导致药物不良反应发生率增加,不仅延长了患者的住院时间,增加了患者的痛苦,而且增加了我国有限卫生资源的耗费,增加了患者的经济负担。据1998年一项统计表明,仅不合理使用第三代头孢菌素一项就使我国每年浪费卫生资源7亿元人民币。
\paragraph{医疗纠纷增加}

近年来,因用药不合理引起的医疗纠纷在国内时有发生。我国2002年统计全国各级人民法院受理的医疗诉讼案件多达170万件,其中37%涉及临床用药。通过具体案例分析,不少医学事件确与临床选药不当、剂量错误、药物配伍不当或用药对象错误等有关,按现有医学知识判断这些临床用药失误,均属可防范事件,而在发生的这些案例中有相当的比例是由抗菌药物的选择与应用不当引起。

\subsection{抗菌药物的应用管理}

\subsubsection{严格掌握抗菌药物治疗的适应证}

抗菌药物治疗的适应证主要为细菌感染,其次为支原体、衣原体、立克次体、螺旋体、真菌等病原微生物感染,非上述感染原则上不用抗菌药物。临床应用时,必须重视病原学检查并进行细菌药敏试验,根据致病菌的变化以及细菌耐药性的变迁选用适当的抗菌药物。
\paragraph{临床诊断明确}

应用抗菌药物,必须根据临床诊断,严格掌握适应证,除病情危重且高度怀疑上述致病原感染外,发热原因不明者不宜采用,否则会使感染的临床表现不典型,干扰诊断,延误正确治疗。抗菌药物对各种病毒性感染并无疗效,除合并继发性细菌感染外,病毒性疾病不宜选用,临床医师切忌盲目迎合患者心理而轻率用药,从而真正走出“应用抗菌药物追求品种多、产品新、价格贵、疗效长”的治疗误区。
\paragraph{预测病原菌进行经验治疗}

细菌等感染性疾病经临床初步诊断明确后,在尚未有细菌鉴定和药敏报告的情况下,应根据临床表现和临床标本涂片,预测病原菌种类,进行经验治疗,参考本医疗机构近期药敏统计资料,选择可能敏感的广谱抗菌药物,如大环内酯类、β-内酰胺类、半合成青霉素类等。待药敏结果报告出来以后再调整用药,选择敏感抗菌药物,或经过充分使用后,如未见效再换用其他敏感抗菌药物。获得性感染或初治患者,可选用抗菌药物;对医疗机构内感染、严重感染、难治性感染应根据临床表现及感染部位,推断可能的病原菌及其耐药情况,选用抗菌活性强、安全性好的杀菌剂,必要时可以联合用药;轻中度感染尽量选择生物利用度高的口服制剂,病情较重可用注射剂。由多种药物可供选择时,应优先选用抗菌作用强、窄谱、不良反应少的抗菌药物。制定抗菌药物治疗方案时,应考虑药物的成本-效果比。
\paragraph{尽快分离病原菌}

为获得准确的病原学诊断,力争在应用抗菌药物之前尽早尽快采集相应的临床标本(如血、浓汁、痰、尿、脑脊液等),立即送至微生物实验室进行涂片染色检查及细菌鉴定,必要时可连续多次采样送检,进行细菌技术、细胞学检查等,迅速明确致病原。在一般细菌培养连续呈阴性而又不能排除细菌感染时,除作涂片镜检外,还应进行微需氧菌、厌氧菌及真菌培养,同时鉴别致病菌及污染菌;对于特殊种类致病源(如军团菌属、支原体、衣原体等)还可配合血清学检查进行诊断。未获得结果前或危急的情况下,可根据临床诊断推测最可能的病原菌,先进行经验治疗;一旦明确病原菌,应根据临床用药效果并参考药敏试验结果,调整用药方案,再进行目的治疗。临床无感染表现而病原检查获阳性结果者,应排除污染菌、正常菌和定植菌的可能。
\paragraph{常规进行药敏试验}

病原菌查明后应进行常规药敏试验。体外药敏试验是临床选用抗菌药物的重要依据,选用敏感抗菌药物治疗,临床治愈率可以达到80%以上。药敏试验方法必须标准化,以期各实验室获得的结果具有可比性。医师根据药敏报告,尽量选用敏感、窄谱的抗菌药物,因广谱抗菌药物易导致微生态平衡失调及二重感染。对一些严重感染以及混合感染,常用抗菌药物联合治疗,两药联合时可出现协同、拮抗、相加或无关4种效应,故联合用药最好用参考药敏试验,以供临床选择用药时参考。

\subsubsection{根据患者的生理、病理状况合理用药}
\paragraph{抗菌药物在不同生理状况患者中的应用}

(1)新生儿患者抗菌药物的应用。①新生儿期肝、肾均未发育成熟,肝药酶分泌不足或缺乏,肾清除功能较差,因此新生儿感染时应避免应用毒性大的抗菌药物,包括主要经肾排泄的氨基糖苷类、万古霉素、去甲万古霉素等,以及主要经肝代谢的氯霉素。确有应用指征时,必须进行血药浓度监测,据此调整给药方案,个体化给药,以确保治疗安全、有效。不能进行血药浓度监测者,不宜选用上述药物。②新生儿期禁用可影响生长发育的四环素类、氟喹诺酮类,避免应用可导致胆红素脑病及溶血性贫血的磺胺类药和呋喃类药。③新生儿期由于肾功能尚不完善,需减量应用主要经肾排出的青霉素类、头孢菌素类等β-内酰胺类药物,以防止药物在体内蓄积而导致严重中枢神经系统毒性反应的发生。④新生儿的体重和组织器官生长发育很快,抗菌药物在新生儿的药动学亦随日龄增长而变化,因此使用抗菌药物时应按日龄调整给药方案。

(2)小儿患者抗菌药物的应用。①氨基糖苷类抗生素有明显耳、肾毒性,万古霉素和去甲万古霉素也有一定耳、肾毒性,小儿患者应尽量避免应用。临床有明确指征且又无其他毒性低的抗菌药物可供选用时,方可选用该类药物,并在治疗过程中严密观察不良反应。有条件者应进行血药浓度监测,根据监测结果个体化给药。②四环素类抗生素可导致牙齿黄染及牙釉质发育不良,不可用于8岁以下小儿。③氟喹诺酮类抗菌药物对骨骼生长发育可能产生不良影响,避免用于18岁以下未成年人。

(3)妊娠期患者抗菌药物的应用。①妊娠期应避免应用对胎儿有致畸或明显毒性作用的药物,如四环素类、氟喹诺酮类等。②妊娠期应避免应用对母体和胎儿均有毒性作用的药物,如氨基糖苷类、万古霉素、去甲万古霉素等,确有应用指征时,须在血药浓度监测下使用,以保证用药安全、有效。③妊娠期感染时可选用毒性低、对胎儿及母体均无明显有害影响的药物,如青霉素类、头孢菌素类等β-内酰胺类和磷霉素等。

(4)哺乳期患者抗菌药物的应用。哺乳期患者接受抗菌药物后,药物可自乳汁分泌,存在对乳儿潜在的影响,并可能出现不良反应,因此治疗哺乳期患者时应避免选用氨基糖苷类、氟喹诺酮类、四环素类、氯霉素、磺胺药等。哺乳期患者应用任何抗菌药物时,均宜暂停哺乳。

(5)老年患者抗菌药物的应用。①老年人肾功能呈生理性减退,按一般常用量接受主要经肾排出的抗菌药物治疗时,由于药物自肾排出减少,导致在体内蓄积,血药浓度增高,容易发生不良反应。因此,老年患者尤其是高龄患者接受主要自肾排出的抗菌药物(如青霉素类、头孢菌素类等β-内酰胺类药物)治疗时,应按轻度肾功能减退情况减量给药,可用正常治疗量的1/2~2/3。②老年患者宜选用毒性低并具有杀菌作用的抗菌药物,青霉素类、头孢菌素类等β-内酰胺类为常用药物,而毒性大的氨基糖苷类、万古霉素、去甲万古霉素等药物应尽可能避免应用,有明确应用指征时须在严密观察下慎用,同时应进行血药浓度监测,据此调整剂量,使给药方案个体化,以达到安全、有效的用药目的。
\paragraph{抗菌药物在不同病理状况患者中的应用}

(1)肾功能减退患者抗菌药物的应用。①根据感染的严重程度、病原菌种类及药敏试验结果等选用无肾毒性或肾毒性低的抗菌药物。②主要由肝胆系统排泄或由肝脏代谢,或经肾脏和肝胆系统同时排出的抗菌药物用于肾功能减退者时,维持原治疗量或剂量略减,如大环内酯类、利福平、β-内酰胺类等。③主要经肾排泄,本身并无肾毒性或仅有轻度肾毒性的抗菌药物在肾功能减退者可应用,但剂量需适当调整,主要包括青霉素类和头孢菌素类的大多品种。④肾功能减退时不宜或尽量避免应用的药物有四环素类、磺胺类、头孢噻啶等;肾功能减退时必须酌情减量的药物有氨基糖苷类、羧苄西林、多黏菌素类、万古霉素等,此类药物均有明显肾毒性,并主要经肾排泄。在确有应用指征时须进行血药浓度监测,据此调整给药方案,达到个体化给药;也可以内生肌酐清除率为准,在轻、中和重度肾功能减退者依次调整剂量为正常剂量的1/2~2/3、1/5~1/2和1/10~1/5,疗程中需密切监测患者肾功能。

(2)肝功能减退患者抗菌药物的应用。①主要由肝脏清除,肝功能减退时清除明显减少,无明显毒性反应发生的药物在肝病时仍可正常应用,但需谨慎,必要时减量给药,治疗过程中需严密监测肝功能,如红霉素等大环内酯类(不包括酯化物)、林可霉素、克林霉素等。②主要经肝脏或有相当量肝脏清除或代谢,肝功能减退时清除减少并可导致毒性反应发生的药物在肝功能减退患者应避免使用,如氯霉素、利福平、红霉素酯化物等。③经肾、肝两途径排出的青霉素类、头孢菌素类等药物在严重肝病患者,尤其肝、肾功能同时减退的患者使用时需减量应用。④主要由肾排泄的药物(如氨基糖苷类)在肝功能减退者不需调整剂量。

\subsection{制订抗菌药物治疗方案的原则}

根据病原菌、感染部位、感染严重程度和患者的生理、病理状况及抗菌药物的作用特点制订抗菌药物治疗方案,包括抗菌药物的选用品种、剂量、给药次数、给药途径、疗程及联合用药等。在制订治疗方案时应遵循下列原则。

\subsubsection{品种选择}

根据病原菌种类及药敏试验结果选用抗菌药物品种。

\subsubsection{给药剂量}

按各种抗菌药物的治疗剂量范围给药。治疗重症感染(如败血症、感染性心内膜炎等)和抗菌药物不易达到的部位的感染(如中枢神经系统感染等),抗菌药物剂量宜较大(治疗剂量范围的高限);而治疗单纯性下尿路感染时,由于多数药物尿药浓度远高于血药浓度,则可应用较小剂量(治疗剂量范围的低限)。

\subsubsection{给药途径}

(1)给药方式。轻症感染可接受口服给药者,应选用口服吸收完全的抗菌药物,不必采用静脉或肌内注射给药。重症感染、全身性感染患者初始治疗应予静脉给药,以确保药效;病情好转能口服时应及早转为口服给药。

(2)局部用药。抗菌药物的局部应用只限于少数情况,宜尽量避免,例如全身给药后在感染部位难以达到治疗浓度时可加用局部给药作为辅助治疗。局部用药宜采用刺激性小,不易吸收、不易导致耐药性和不易致过敏反应的杀菌剂,而青霉素类、头孢菌素类等易产生过敏反应的药物不可局部应用。氨基糖苷类等耳毒性药不可局部滴耳。

(3)给药次数。为保证药物在体内能最大限度地发挥药效,杀灭感染灶病原菌,应根据药动学和药效学相结合的原则给药。

(4)疗程。抗菌药物疗程因感染不同而异,一般宜用至体温正常、症状消退后72~96h,特殊情况需酌情处理。败血症、感染性心内膜炎、化脓性脑膜炎、伤寒、布鲁菌病、骨髓炎、溶血性链球菌咽炎和扁桃体炎、深部真菌病、结核病等需较长的疗程才能彻底治愈,并防止复发。

\subsection{严格控制抗菌药物的预防用药}

\subsubsection{明确预防性应用抗菌药物的适应证}

预防性应用抗菌药物应有适应证,滥用抗菌药物预防并不能减少感染的发生,有时反有促进耐药菌株生长和导致二重感染的危险,甚至掩盖症状导致延误诊断及治疗的时机。预防性应用抗菌药物的主要适应证有以下几点。

(1)严重创伤、开放性骨折、火器伤、腹内空腔脏器破裂、有严重污染和软组织破坏的创伤等。

(2)大面积烧伤。

(3)结肠手术前肠道准备。

(4)急症手术患者的身体和其他部位有化脓性感染。

(5)营养不良、全身情况差或接受激素、抗癌药物等的患者需作手术治疗时。

(6)进行人造物留置手术。

(7)有心脏瓣膜病或已植入人工心脏瓣膜者因病需作手术时。

\subsubsection{加强Ⅰ类切口手术预防性使用抗菌药物的管理}

Ⅰ类切口手术是指手术未进入炎症区、呼吸道、消化道及泌尿生殖道,以及闭合性创伤手术符合上述条件者。一般不预防性使用抗菌药物,确需使用时,要严格掌握适应证、药物选择、用药起始与持续时间。

Ⅰ类切口手术常用预防抗菌药物的单次使用剂量为:头孢唑林1~2g;头孢拉定1~2g;头孢呋辛1.5g;头孢曲松1~2g;甲硝唑0.5g。对β-内酰胺类抗菌药物过敏者可选用克林霉素预防葡萄球菌、链球菌感染;可选用氨曲南预防革兰阴性杆菌感染,必要时可联合使用。如若在耐甲氧西林葡萄球菌检出率高的医疗机构进行人工材料植入手术(如人工心脏瓣膜置换、永久性心脏起搏器置入、人工关节置换等)时,也可选用万古霉素或去甲万古霉素预防感染。

给药方法:术前0.5~2h内,或麻醉开始时首次给药;手术时间超过3h或失血量大于1500mL,术中可给予第2剂;总预防用药时间一般不超过24h,个别情况可延长至48h。因为术前和术中应用抗菌药物后,在手术过程中患者血内便能始终保持一定的抗菌药物浓度,可防止术后感染的发生;还可避免术后长期应用一些抗感染药物可能发生的副作用和不良结果,如二重感染等。仅术后才应用抗感染药物时,对预防感染的效果较差。

\subsection{防治联合用药的滥用}

抗菌药物的联合应用存在严重的滥用问题。实际上,联合用抗菌药物有时不如单独应用安全、有效。联合应用的药物种类越多,产生不良反应的可能性越大。抗菌药物的联合应用要有明确指征,单一药物可有效治疗的感染不需联合用药,仅在下列情况时有指征联合用药。

(1)病原菌尚未查明的严重感染。可采用扩大抗菌谱的经验方法。例如对病原菌尚不清楚的脓毒血症,首先联合应用抗葡萄球菌和革兰阴性菌的药物控制重症感染,一旦有了细菌培养结果,则停用不必要的抗菌药物。

(2)单一抗菌药物不能控制的需氧菌及厌氧菌混合感染。典型的混合感染常涉及需氧菌和厌氧菌,可采用氨基糖苷类或第三代头孢菌素等抗革兰阴性菌和甲硝唑、克林霉素等抗厌氧菌药物联合治疗,如腹腔脓肿。选择抗菌药物联合治疗旨在覆盖绝大多数已知或可疑的致病菌,由于亚胺培南等广谱抗菌药物的出现,已减少了联合用药在混合感染的应用。

(3)单一抗菌药物不能有效控制的感染性心内膜炎或败血症等重症感染。

(4)需长程治疗,但病原菌易对某些抗菌药物产生耐药性的感染。如结核病的治疗,使用单一药物常易产生耐药菌株,通常临床采用二联、三联用药,以防止长期用药时耐药菌的出现。

(5)患者免疫功能低下。如吞噬细胞功能缺损者,用β-内酰胺类抗菌药物加利福平,使药物分别在细胞内、外同时发挥抗菌作用,增强抗菌效果。

(6)降低不良反应。由于药物协同抗菌作用,联合用药时可将毒性大的抗菌药物剂量减少。如两性霉素B与氟胞嘧啶联合治疗隐球菌脑膜炎时,前者的剂量可适当减少,从而减少其毒性反应。联合用药时宜选用具有协同或相加抗菌作用的药物联合,如青霉素类、头孢菌素类等β-内酰胺类药物与氨基糖苷类联合,两性霉素B与氟胞嘧啶联合。联合用药通常采用两种药物联合,3种及3种以上药物的联合仅适用于个别情况,如结核病的治疗。此外,必须注意联合用药后药物不良反应将增多。

\subsection{严格执行抗菌药物分级管理制度}

各级医疗机构应结合本单位实际,根据抗菌药物的特点、临床疗效、细菌耐药、不良反应以及当地社会经济情况、药品价格等因素,将抗菌药物分为非限制使用、限制使用与特殊使用三类进行分级管理。

\subsubsection{非限制使用}

经临床长期应用证明安全、有效,细菌不易产生耐药性,价格相对较低的抗菌药物。

\subsubsection{限制使用}

与非限制使用抗菌药物相比较,这类药物在疗效、安全性、对细菌耐药性影响、药品价格等方面存在局限性,不宜作为非限制性药物使用。

\subsubsection{特殊使用}

不良反应明显,不宜随意使用或临床需要倍加保护以免细菌过快产生耐药而导致严重后果的抗菌药物;新上市的抗菌药物,其疗效或安全性任何一方面的临床资料尚较少,或并不优于现用药物者;价格昂贵的药品。

临床选用抗菌药物应遵循前述几项基本原则,根据感染部位、严重程度、致病菌种类以及细菌耐药情况、患者病理生理特点、药物作用特点和价格等因素加以综合分析考虑,一般对轻度与局部感染患者应首先选用非限制使用抗菌药物进行治疗;严重感染、免疫功能低下者合并感染或病原菌只对限制使用抗菌药物敏感时,可选用限制使用抗菌药物治疗;特殊使用抗菌药物的选用应从严控制。

临床医师可根据诊断和患者病情开具非限制使用抗菌药物处方;患者需要应用限制使用抗菌药物治疗时,应经具有主治医师以上专业技术职务任职资格的医师同意并签名;患者病情需要应用特殊使用抗菌药物时,应具有严格的临床用药指征或确凿依据,须经由医疗机构药事管理与药物治疗学委员会认定的具有抗感染临床经验或相关专业专家会诊同意,处方需经具有高级专业技术职务任职资格医师签名,药师要严格审核处方。紧急情况下,临床医师可以越级使用高于权限的抗菌药物,但仅限于1d用量,并做好相关病历记录。

根据抗菌药物临床应用监测情况,以下药物作为“特殊使用”类别管理:①第四代头孢菌素:头孢吡肟、头孢匹罗、头孢噻利等;②碳青霉烯类抗菌药物:亚胺培南西司他丁、美罗培南、帕尼培南倍他米隆、比阿培南等;③多肽类与其他抗菌药物:万古霉素、去甲万古霉素、替考拉宁、利奈唑胺等;④抗真菌药物:卡泊芬净、米卡芬净、伊曲康唑、伏立康唑、两性霉素B含脂质体等。医疗机构可根据本机构具体情况调整“特殊使用”类别抗菌药物品种。