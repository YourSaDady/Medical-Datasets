\chapter{免疫性疾病的药物治疗}

\section{类风湿关节炎}

类风湿关节炎(rheumatoid
arthritis,RA)是一种病因未明的、以关节滑膜炎为特征的慢性全身性自身免疫性疾病,主要特征为对称性、慢性、进行性多关节炎。关节滑膜的慢性炎症、增生,形成血管翳,侵犯关节软骨、软骨下骨、韧带和肌腱等,造成关节软骨、骨和关节囊破坏,最终导致关节畸形和功能丧失,可伴有发热、皮下结节及淋巴结肿大等关节外表现,血清中可出现多种自身抗体,病变呈持续、反复发作过程。

RA呈全球性分布,发病高峰在30~50岁,女性多发,我国的患病率约为0.32%~0.36%。

\subsection{病因和发病机制}

\subsubsection{病因}

可能与环境、细菌、病毒、遗传和性激素等相关。
\paragraph{感染因素}

已经证明某些病毒和细菌微生物可通过其体内的抗原性蛋白或多肽片段介导RA患者的自身免疫反应。

(1)细菌因素。A组链球菌及菌壁的肽聚糖长期存在于体内成为持续的抗原,刺激机体产生抗体,发生免疫病理损伤而致病。结核分枝杆菌的热休克蛋白也与机体有共同的序列,提示可能与RA有关。

(2)病毒因素。EB病毒的gp110糖蛋白可能通过分子模拟机制在RA的发生和演变中发挥作用。77%的RA患者滑膜中有细小病毒B19基因,活动性滑膜炎患者的滑膜组织均表达B19抗原VP-1,而骨关节炎及健康对照组无VP-1表达。与RA有关联的病毒还包括巨细胞病毒、肝炎病毒及多种反转录病毒,它们对于本病致病中的机制尚待研究。
\paragraph{遗传因素}

流行病学调查显示,同卵孪生子同患RA的概率为27%,而异卵孪生子同患RA的概率为13%,说明有一定的遗传倾向。
\paragraph{性激素}

RA患者雄激素及其代谢产物的水平明显降低。妊娠期间病情减轻,服避孕药的女性发病减少,更年期女性RA的发病明显高于同龄男性及老年女性。
\paragraph{其他因素}

寒冷、潮湿、疲劳、营养不良、吸烟、创伤和精神因素等,常为本病的诱发因素。

\subsubsection{发病机制}

一般认为,RA是一种自身免疫性疾病,RA发病是多种因素共同作用的结果,由易感基因参与、感染因子及自身免疫反应介导的免疫损伤和修复是RA发病及病情演变的基础。
\paragraph{人类白细胞抗原}

人类白细胞抗原(human leukocyte
antigen,HLA)是由单一链组成的HLA-DR、HLA-DP、HLA-DQ跨膜蛋白,主要功能是结合抗原、与T细胞受体形成HLA-抗原-TCR三分子复合物,并激活T细胞。
\paragraph{T细胞的免疫反应}

T细胞是RA滑膜组织中的主要炎性细胞。CD$_4^+$
细胞占大多数,而CD$_8^+$细胞相对较少。
\paragraph{滑膜细胞的免疫反应}

正常滑膜关节的滑膜只有一、二层细胞厚度,但在RA时由于滑膜细胞大量增加,每个细胞的体积增大,滑膜明显增厚。大多数病程长的患者增厚的滑膜层下有淋巴细胞浸润。滑膜细胞可分为两大类:类似巨噬细胞的甲型滑膜细胞,类似成纤维母细胞的乙型滑膜细胞。甲型滑膜细胞在RA中起主要作用。近年来研究发现,RA滑膜出现异常凋亡与增殖,其中包括RA成纤维样滑膜细胞凋亡不足,导致炎症反应持续。
\paragraph{自身抗体}

RA患者B细胞产生自身抗体,其中类风湿因子(rheumatoid
factor,RF)是最常见的一种,RF可在RA患者出现临床症状之前出现。RF包括IgM、IgG、IgA和IgE四型。RF与RA的滑膜炎症及关节外病变有关。RF可在关节腔内形成的抗原抗体复合物,可陷于透明软骨和纤维软骨,激活补体,刺激滑膜细胞增生,激活成纤维细胞,引起关节局部或其他部位病损。目前,在RA患者血清中还发现了抗核周因子及抗环状胍氨酸抗体等多种抗体。

\subsection{临床表现}

\subsubsection{好发人群}

女性好发,发病率为男性的2或3倍,可发生于任何年龄,高发年龄为40~60岁。

\subsubsection{症状体征}

可伴有体重减轻、低热及疲乏感等全身症状。
\paragraph{关节表现}

(1)晨僵:早晨起床时关节活动不灵活的主观感觉,它是关节炎症的一种非特异表现,其持续时间与炎症的严重程度成正比。

(2)关节受累的表现:①多关节受累,呈对称性多关节炎(常≥5个关节)。易受累的关节有手、足、腕、踝及颞颌关节等,其他还可有肘、肩、颈椎、髋、膝关节等。②关节畸形,手的畸形有梭形肿胀、尺侧偏斜、天鹅颈样畸形、钮孔花样畸形等。足的畸形有跖骨头向下半脱位引起的仰趾畸形、外翻畸形、跖趾关节半脱位、弯曲呈锤状趾及足外翻畸形。③其他,可有正中神经/胫后神经受压引起的腕管∕跗管综合征等表现。
\paragraph{关节外表现}

(1)一般表现:可有发热、类风湿结节(属于机化的肉芽肿,与高滴度RF、严重的关节破坏及RA活动有关,好发于肘部、关节鹰嘴突、骶部等关节隆突部及经常受压处)、类风湿血管炎(主要累及小动脉的坏死性小动脉炎,可表现为指、趾端坏死、皮肤溃疡、外周神经病变等)及淋巴结肿大。

(2)心脏受累:可有心包炎、心包积液、心外膜、心肌及瓣膜的结节、心肌炎、冠状动脉炎、主动脉炎、传导障碍、慢性心内膜炎及心瓣膜纤维化等表现。

(3)呼吸系统受累:可有胸膜炎、胸腔积液、肺动脉炎、间质性肺疾病和结节性肺病等。

(4)肾脏表现:主要有原发性肾小球及肾小管间质性肾炎、肾脏淀粉样变和继发于药物治疗(金制剂、青霉胺及NSAIDs)的肾功能损害。

(5)神经系统:除周围神经受压的症状外,还可诱发神经疾病、脊髓病、外周神经病、继发于血管炎的缺血性神经病、肌肥大及药物引起的神经系统病变。

(6)贫血:是RA最常见的关节外表现,属于慢性疾病性贫血,常为轻至中度。

(7)消化系统:可因RA血管炎、并发症或药物治疗所致。

(8)眼:幼年患者可有葡萄膜炎,成人可有巩膜炎,可能由血管炎所致。还可有干燥性结膜角膜炎、巩膜软化、巩膜软化穿孔和角膜溶解。

\subsection{诊断}

\subsubsection{实验室检查}
\paragraph{一般检查}

血、尿常规、肝功能损害、C反应蛋白(C reactive
protein,CRP)、生化(肝肾功能,A/G)、免疫球蛋白、蛋白电泳、补体等。
\paragraph{自身抗体}

RF-IgM、抗环状胍氨酸抗体、RF-IgG及RF-IgA、抗核周因子、抗角蛋白抗体,以及抗核抗体、抗ENA抗体等。
\paragraph{遗传标记}

HLA-DR4和HLA-DR1亚型。
\paragraph{影像学检查}

(1)X线片。关节X线片可见软组织肿胀、骨质疏松及病情进展后的关节面囊性变、侵袭性骨破坏、关节面模糊、关节间隙狭窄、关节融合及脱位。X线可分4期;Ⅰ期,正常或骨质疏松;Ⅱ期,骨质疏松,有轻度关节面下骨质侵袭或破坏,关节间隙轻度狭窄;Ⅲ期,关节面下明显的骨质侵袭和破坏,关节间隙明显狭窄,关节半脱位畸形;Ⅳ期,上述改变合并有关节纤维性或骨性强直。胸部X线片可见肺间质病变和胸腔积液等。

(2)CT检查。胸部CT可进一步提示肺部病变,尤其高分辨CT对肺间质病变更敏感。

(3)MRI检查。手关节及腕关节的MRI检查可提示早期的滑膜炎病变,对发现RA患者的早期关节破坏很有帮助。

(4)超声。关节超声是简易的无创性检查,对于滑膜炎、关节积液以及关节破坏有鉴别意义。研究认为其与MRI有较好的一致性。
\paragraph{特殊检查}

(1)关节穿刺术。对于有关节腔积液的关节,关节液的检查包括关节液培养、RF检测、抗CCP抗体检测、抗核抗体等,并做偏振光检测鉴别痛风的尿酸盐结晶。

(2)关节镜及关节滑膜活检:对RA的诊断及鉴别诊断很有价值,对于单关节难治性的RA有辅助的治疗作用。

\subsubsection{RA的诊断标准、分期、功能及活动性判断}
\paragraph{美国风湿病学会1987年修订的RA分类标准}

标准具体见表\ref{tab16-1}:其中≥4条可以确诊RA。①晨僵至少1h(≥6周);②3个或3个以上关节受累(≥6周);③手关节(腕、MCP或PIP关节)受累(≥6周);④对称性关节炎(≥6周);⑤有类风湿皮下结节;⑥X线片改变;⑦血清RF阳性(滴度>1∶32)。
\paragraph{病情分期}

(1)早期:有滑膜炎,无软骨破坏。

(2)中期:介于上、下间(有炎症、关节破坏、关节外表现)。

(3)晚期:已有关节结构破坏,无进行性滑膜炎。
\paragraph{关节功能分级}

(1)Ⅰ级:功能状态完好,能完成平常任务无碍(能自由活动)。

(2)Ⅱ级:能从事正常活动,但有1个或多个关节活动受限或不适(中度受限)。

(3)Ⅲ级:只能胜任一般职业性任务或自理生活中的一部分(显著受限)。

(4)Ⅳ级:大部分或完全丧失活动能力,需要长期卧床或依赖轮椅,很少或不能生活自理(卧床或轮椅)。

\begin{longtable}[]{ll}
    \caption{2010年ACR/EULAR关于RA新的分类标准}
    \label{tab16-1}\\
    \toprule
    关节受累                           & 血清学(至少需要1条)\tabularnewline
    \midrule
    \endhead
    1个大关节                          & RF和ACPA均阴性\tabularnewline
    2~10个大关节                      & RF和(或)ACPA低滴度阳性\tabularnewline
    1~3个小关节(伴或不伴大关节受累) &
    RF和(或)ACPA高滴度阳性\tabularnewline
    \vtop{\hbox{\strut 4~10个小关节(伴或不伴大关节受累)}\hbox{\strut >10个关节(至少1个小关节受累)}\hbox{\strut 急性时相反应物(至少需要1条)}}
                                       & \tabularnewline
    CRP和ESR正常                       & 症状持续时间<6周\tabularnewline
    CRP或ESR异常                       & 症状持续时间≥6周\tabularnewline
    \bottomrule
\end{longtable}
\paragraph{活动性指标}

(1)关节疼痛≥4个。

(2)晨僵>30min。

(3)ESR≥30mm/h。

(4)CRP增高。

(5)血小板计数增高。

(6)贫血。

(7)RF(+)1∶20以上。

(8)有关节外表现(发热、贫血、血管炎等)。

\subsection{治疗}

\subsubsection{药物治疗}

本病尚缺乏根治方案以及预防措施,当前应用的药物均不能完全控制关节破坏。治疗目的主要:①控制关节及其他组织炎症,延缓病情的进展;②保持关节功能和防止畸形;③修复受损关节以减轻疼痛和恢复功能。为达到上述目的,早期诊断和早期治疗极为重要。

治疗原则:①早期诊断,尽早治疗。早期发现进行性或侵袭性疾病患者,尽早应用改变病情药物以控制RA病变的进展。②联合用药:对重症患者应联合应用两种以上改变病情的药物,以使病情完全缓解。联合用药可减少单独用药的剂量,减少不良反应的发生。③治疗方案个体化:根据患者的病情、对药物的作用及不良反应等多方面情况选择个体化治疗方案。④功能锻炼:除急性期、发热以及内脏受累的患者需卧床休息和关节制动外,只要患者可以耐受,应及早地进行关节功能锻炼,以免过久的卧床导致关节废用,甚至促进关节强直。

\subsubsection{药物分类和作用机制}

根据药物作用机制,WHO将抗RA药物分为:①改善症状抗风湿药(symptom
modifying anti-rheumatic
drugs,SM-ARDs):这类药物改善炎症性滑膜炎的症状和临床表现,包括NSAIDs、糖皮质激素(glucocorticoid)、改变病情抗风湿药(disease-modifying
anti-rheumatic drugs,DMARDs);②控制疾病的抗风湿治疗药(disease
controlling anti-rheumatic
therapy,DC-ART):这类药物能改变RA的病程。其作用为:①改善并维持关节功能,降低滑膜炎症;②防止或明显降低关节病变的进展,且至少维持1年。迄今尚无堪称DC-ART的药物。习惯上把治疗RA的药物分为一线、二线及三线药物(具体见表\ref{tab16-2})。一线药物主要是NSAIDs;二线药物是指DMARDs;三线药物是糖皮质激素类药物。

\begin{longtable}[]{lp{8cm}}
    \caption{RA的主要治疗药物}
    \label{tab16-2}\\
    \toprule
    分类       & 药物\tabularnewline
    \midrule
    \endhead
    NSAIDs     &
    \vtop{\hbox{\strut 布洛芬、双氯芬酸、塞来昔布}\hbox{\strut 萘丁美酮、美洛昔康、依托度酸}}\tabularnewline
    DMARDs     &
    柳氮磺吡啶、羟氯喹、金制剂、青霉胺、MTX、来氟米特、环孢素A、AZA、雷公藤\tabularnewline
    糖皮质激素 & 泼尼松、泼尼松龙、醋酸曲安西龙、得宝松\tabularnewline
    \bottomrule
\end{longtable}

\paragraph{NSAIDs}

此类药物作用机理是抑制环氧化酶(cyclooxygenase,COX),导致炎症介质前列腺素合成减少,从而使炎症减轻,达到消炎止痛的作用。特点是起效快,可缓解关节疼痛及晨僵等症状,但不能控制病情进展。现已发现COX有两种亚型,即COX-1和COX-2,COX-1为结构酶,其底物PGs主要参与调节机体的生理功能,如调节外周血管张力,维持肾血流量,保护胃黏膜及调节血小板聚集等功能。COX-2是诱导酶,存在于白细胞中,炎症和疼痛时合成PGs,临床宜选用不良反应小的COX-2选择性抑制剂。常用的NSAIDs用法及不良反应如表\ref{tab16-3}所示。 

\begin{longtable}[]{p{5cm}p{5cm}p{5cm}}
    \caption{常用的NSAIDs用法及不良反应}
    \label{tab16-3}\\
    \toprule
    
    药名                                                             & 用法用量                                                  & 常见不良反应\tabularnewline
    \midrule
    \endhead
    \vtop{\hbox{\strut 布洛芬}\hbox{\strut (Ibuprofen)}}           &
    每次0.4~0.6g,每日3或4次,RA比骨性关节炎用量稍大                &
    20%~30%的患者有胃肠道不良反应,严重者有上消化道出血\tabularnewline
    \vtop{\hbox{\strut 吲哚美辛}\hbox{\strut (Indometacin)}}       &
    初量每次25~50mg,每日2或3次,每日最大量不超过150mg              &
    不良反应较多,包括胃肠道反应、神经症状、肾脏受损等\tabularnewline
    \vtop{\hbox{\strut 萘普生}\hbox{\strut (Naproxen)}}            &
    每次0.25g,每日早晚各1次,如无医师意见疗程不超过10d              &
    与吲哚美辛比较,胃肠道和神经系统不良反应发生率和严重程度均较低\tabularnewline
    \vtop{\hbox{\strut 双氯芬酸}\hbox{\strut (Diclofenac sodium)}} &
    每日75~150mg,分3次服用,疗效满意后可逐渐减量                   &
    胃肠道不良反应常见,服药量大于每日20mg时胃溃疡发生率明显增高,中性粒细胞减少、头痛等\tabularnewline
    \vtop{\hbox{\strut 美洛昔康}\hbox{\strut (Meloxicam)}}         &
    每次15mg,每日1次                                                &
    胃肠道反应、肝酶升高、肾功能损害、头痛、皮疹等\tabularnewline
    \vtop{\hbox{\strut 依托度酸}\hbox{\strut (Etodolac)}}          &
    400~1000mg,每日1次,应根据患者病情和临床疗效增减剂量           &
    胃肠道不良反应较少\tabularnewline
    \vtop{\hbox{\strut 塞来昔布}\hbox{\strut (Celecoxib)}}         &
    每日0.2g一次服用,疗效不明显者可增加至每日0.4g,分2次服用。每日最大剂量0.4g
                                                                     & 胃肠道不良反应、头痛、磺胺过敏反应、浮肿等\tabularnewline
    \vtop{\hbox{\strut 萘丁美酮}\hbox{\strut (Nabumetone)}}        &
    每次1g,每日1次,每日最大量为2g,分两次服。体重不足50kg的成人以每天0.5g起始,逐渐上调至有效剂量
                                                                     & 胃肠道不良反应、神经症状、皮疹、瘙痒等\tabularnewline
    \bottomrule
\end{longtable}
\paragraph{DMARDs}

该类药物可以减轻RA的关节症状和体征并延缓关节病变的进展。本组药物包括传统的青霉胺、金制剂、柳氮磺吡啶、抗疟药(氯喹、羟氯喹)等。由于这组药物起效时间平均在6周以上,且其中多数药物并不具备阻止关节破坏的性能,故曾称为慢作用抗风湿药(slow
acting anti-rheumatic
drugs,SAARDs)。近年来本类药物包括了起效迅速的治疗关节炎的生物制剂(TNF阻滞剂,IL-1阻滞剂),故倾向用DMARDs名词。在广义上,DMARDs类药物中也应包括某些免疫抑制剂,通过抑制细胞和体液免疫反应使组织损伤得以减轻,如MTX、来氟米特、环磷酰胺、AZA等。此外,还包括多种用于RA的植物药制剂,如白芍总苷、雷公藤等,对缓解关节肿痛、晨僵均有较好的作用,但长期缓解病变的作用还需进一步研究。本类药物中常用的药物及其作用机制如表\ref{tab16-4}所示。

\begin{longtable}[]{p{3cm}p{4cm}p{4cm}p{4cm}}
    \caption{常用治疗RA的DMARDs}
    \label{tab16-4}\\
    \toprule
    药名                                                                                                                                                                                       & 作用机制                                                                                                                                & 用法                                                                                                                                                                               & 主要不良反应                                                                                                                                 \\
    \midrule
    \endhead
   
    青霉胺(penicilamine)                                                                                                                                                                       & 其值基可破坏血浆中的巨球蛋白,使RF滴度下降,抑制淋巴细胞转化,使抗体生成减少,稳定溶廓体酶                                                & 初用时,每日口服125~ 250 mg,以后每1~2个月增加125~250mg.平均日剂量为500~750 mg,最大量一般每日不超过1.0g,常用维持量为250mg                                                           & 不良反应较多,剂量大时更为明显.有蛋白尿、白细胞和血小板计数减少.胃扬道反应、骨髓抑制、皮疹、口异味、肝肾功能损害等。青霉素过敏者对本品亦过敏
    \\
    柳氮磺此啶(sulfaselazine)                                                                                                                                                                  & 抑制前列腺素合成、抑制脂氧化酶代谢物的形成、抑制白细胞功能                                                                              & 每日1.5~3.0g.分2次服用,与饭同时服用。用法:初始每日用量宜小,每周或2周递增剂量,达到日所需剂量                                                                                      & 恶心、呕吐、白细胞计数减少、皮每I用量宜小每疹.男性不育等,对磺胺过敏者禁用
    \\
    环磷酰胺(cyclophosphamide, CTX)                                                                                                                                                            & 其活性产物与细胞成分的功能基因发生烷化作用。影响DNA的结构和功能。主要作用于B淋巴细胞而发挥免疫抑制作用                                  & 活动性SLE\footnote{SLE表示系统性红斑狼疮(systemic lupus erythematosus)。}、狼疮肾炎、静脉给药,按体表面职每次500~1000 mg/m$^2$.每34周1次;每次200mg.每日或隔日1次。口服给药每日100mg 1次服用,维持量减半                                            & 骨髓抑制,白细胞及血小板计数下降、肝脏毒性及消化道反应脱发、闭经和出血性膀胱炎等
    \\
    甲氨蝶呤(methotrexate, MTX)                                                                                                                                                                & 可引起细胞内叶酸缺乏抑制细胞增生 和复制,抑制白细胞的趋向性,有直接的抗炎症作用                                                           & 治疗各种关节炎的起效期为6~8周,故评价疗效必须在8周后,口服初始剂量每次7.5mg.每周1次,可耐情增加,20mg每周1次.分1次或2次服用。对口服吸收不良者可改用肌内注射或静脉注射,10~15mg,1周1次 & 恶心、呕吐、口轻溃疡腹泻、肝功能损害、肺泡及肺间质浸润、骨髓抑制等。同时服用甲酰四氢叶酸钙可减少不良反应的发生,不影响疗效
    \\
    环孢素(cyclosporin)                                                                                                                                                                        & 对诱导期的细胞免疫和胸腺依赖性抗原的体液免疫有较高的选择性抑制作用。可明显缓解RA的病情并可降低ESR.CRP及血清RF滴度                       & 每日按体重3~3.5mg/kg.每日1次(也可分为2次)口服。4-8周后疗效不佳者,可增量至每日5 mg/ kg,病情稳定后减量                                                                               & 主要不良反应为肾毒性.要注意监测肾功能尽量避免与NSAIDs合用,防止加重肾毒性反应
    \\
    金制剂(gold compounds)                                                                                                                                                                     & 抑制淋巴细胞和DNA合成,抑制单核和中性粒细胞的趋化反应,降低免疫球蛋白的产生及抑制溶酶体酶释放                                            & 口服金(瑞得)服用方便,每日3mg,每日1次,2周后增到每日6mg,每日1次.维持治疗直到病情控制起效于3~9个月后,适用于早期或轻型患者                                                          & 皮疹和腹泻、口腔溃疡、白细胞及血小板计数减少、蛋白尿                                                                                         \\
    氯喹(chloroquine)                                                                                                                                                                          & 抑制巨噬细胞释放氧离子和APC的递呈功能,可减少炎症渗出,减轻关节症状,防止关节挛缩                                                          & 氯喹:每日0. 25~0.5g,每日2次;羟氯喹:每日0.15~0.2g,每日3次。1~3月起效,6个月无效停药                                                                                                  & 胃肠道反应如恶心、呕吐、食欲减退等,长期使用须注意视网膜的退行性变和视神经萎缩
    \\
    硫唑嘌呤(azathioprine, AZA)                                                                                                                                                                & 嘌呤类似物,可抑制腺嘌呤和鸟嘌呤的合成.最终影响DNA的合成                                                                                 & 起始剂量每日100 mg,1次服用,每日最大剂量为150mg。疗效明显,将剂减至每日50mg                                                                                                          & 服药期间需监测血象及肝肾功能量
    \\
    来氟米特(leflunomide)                                                                                                                                                                      & 抗代谢性免疫抑制剂.抑制二氢乳酸脱氢酶和酪氨酸激酶的活性,可明显减轻RA的关节肿痛和晨僵                                                    & 每次20mg,每日1次。病情控制后可每日10~20mg                                                                                                                                          & 胃肠道反应、皮疹及白细胞计数减低\\
    雷公藤(tripterygiumwilfordii hook)
                                                                                                                                                                                               & 抑制CD$_4^+$T细胞和B细胞的功能,并能活化CD$_8^+$T细胞、促使其增生,降低CD$_4$/CD$_8$比值,抑制外周血单核细胞和滑膜细胞产生TNF、IL-1和IL-6 &
    雷公藤多苷片,口服.每日1~1. 5mg/kg,分次服。一般为每次10mg,每日4次,或1次20mg,每日3次。必要时在医师密切观察下可短期应用1. 8~2.0mg/kg。病情控制后可减量或采用间歇疗法,疗程根据病种及病情而定 & 性腺抑制,导致男性不育和女性闭经,引起纳差、恶心呕吐、腹泻等,有骨髓抑制作用,并有可逆性肝酶升高和血肌酐清除率下降等                                                                                                                                                                                                                                                                                                                                                           \\
    \bottomrule
\end{longtable}


\paragraph{糖皮质激素(Glucocorticoid)}

糖皮质激素可有效减轻炎症肿痛、迅速缓解病情,但效果不持久,对病因和发病机理无影响,其主要机制是与靶细胞胞浆内的受体结合,抑制多种炎症因子介导的炎症。

\subsubsection{治疗药物的选用}
\paragraph{NSAIDs}

NSAIDs的功效并无太大差异,影响其选用的主要因素包括药物不良反应、服药是否方便、作用持续时间和费用等。无论选择何种NSAIDs,剂量都应个体化;只有在一种NSAIDs足量使用1~2周无效后才更改为另一种;避免同时服用2种或2种以上NSAIDs;老年人宜选用半衰期短的NSAIDs药物,对有溃疡病史的老年人,宜服用选择性COX-2抑制剂以减少胃肠道不良反应。
\paragraph{常用的DMARDs}

DMARDs不能迅速地抗炎止痛,但有改善和延缓病情进展的作用。从疗效和费用等考虑,一般首选MTX,并将它作为联合治疗的基本药物。一般对单用一种DMARDs疗效不好或进展性、预后不良和难治性RA患者可采用机制不同的DMARDs联合治疗。目前常用的联合方案有:①MTX+SSZ;②MTX+羟氯喹(或氯喹);③MTX+青霉胺;④MTX+金诺芬;⑤MTX+AZA;⑥SSZ+羟氯喹。国内还可采用MTX和植物药(雷公藤多甙和白芍总甙等)联合治疗。如患者对MTX不能耐受,可改用来氟米特或其他DMARDs,难治性RA可用MTX+来氟米特或多种DMARDs联合治疗。治疗RA应迅速给予NSAIDs缓解疼痛和炎症,尽早使用DMARDs,以减少或延缓骨破坏。早期积极、合理使用DMARDs治疗是减少致残的关键。症状缓解后,为防止病情复发原则上不停药,但也可依据病情逐渐减量维持治疗,直至最终停用。
\paragraph{糖皮质激素}

一般说来激素不能作为治疗RA的首选药,一旦停药短期内复发,长期应用可导致严重不良反应。

\subsubsection{治疗药物的相互作用}

(1)糖皮质激素与MTX合用可加重后者的毒性作用,两者联用应减少MTX的用量。两药长期联用有可能引起膀胱移行细胞癌,应定期尿检。糖皮质激素与环磷酰胺联用可提高免疫抑制作用,并减少用量。

(2)几乎所有的NSAIDs都可抑制MTX肾排泄,增加MTX毒性;老人、肾衰者及叶酸耗竭者易受影响,老人和肾功能不全者慎用。

\subsubsection{常用治疗方案的选用}
\paragraph{金字塔模式}

金字塔模式的原则是对RA初发患者从一线药物开始,即以NSAIDs为首选药物。如果不能控制病情或患者不能耐受时,再改用二线药物,即DMARDs。如仍不能控制病情则改用三线药物,即糖皮质激素。在使用二线、三线乃至更高层次的药物时,底层的基础治疗仍可继续。
\paragraph{下台阶模式}

对病情较重的RA患者,一般在2年内发生骨侵蚀性改变。如果遵循金字塔模式易错过治疗时机。Wilske于1990年提出了下台阶模式,其原则是早期使用起效快的糖皮质激素(或NSAIDs)和MTX控制炎症,一旦炎症得到控制,即逐渐停用第一台阶的药物而改用DMARDs,最大限度地发挥各种药物的不同作用。下台阶模式的目的是为了在关节出现损害之前控制病情,并缩短DMARDs的使用时间减轻其不良反应。目前下台阶模式已被广泛采用。
\paragraph{锯齿形模式}

1990年Fries提出了锯齿形模式,其原则是RA一旦确诊,早期使用DMARDs;一旦发现病情加重或复发,即改变DMARDs,重新控制病情,使RA的病程呈锯齿形。

\subsubsection{免疫净化}

RA患者血中常有高滴度自身抗体、大量循环免疫复合物、高免疫球蛋白等。因此,除药物治疗外,可选用免疫净化疗法,快速去除血浆中的免疫复合物和过高的免疫球蛋白、自身抗体等。如免疫活性淋巴细胞过多,还可采用单个核细胞清除疗法,从而改善T、B细胞及巨噬细胞和自然杀伤细胞功能,降低血液黏滞度,以达到改善症状的目的,同时提高药物治疗的效果。目前常用的免疫净化疗法包括血浆置换、免疫吸附和淋巴细胞/单核细胞去除术,被置换的病理性成分可以是淋巴细胞、粒细胞、免疫球蛋白或血浆等,应用此方法时需配合药物治疗。

\subsubsection{功能锻炼}

必须强调,功能锻炼是RA患者关节功能得以恢复及维持的重要方法。一般说来,在关节肿痛明显的急性期,应适当限制关节活动。但是,一旦肿痛改善,应在不增加患者痛苦的前提下进行功能活动。对无明显关节肿痛,但伴有可逆性关节活动受限者,应鼓励其进行正规的功能锻炼。在有条件的医疗机构,应在风湿病专科及康复专科医师的指导下进行。

\subsubsection{外科治疗}

经内科治疗不能控制及严重关节功能障碍的RA患者,外科手术是有效的治疗手段。外科治疗的范围从腕管综合征的松解术、肌腱撕裂后修补术至滑膜切除及关节置换术。

\section{系统性红斑狼疮}

系统性红斑狼疮(systemic lupus
erythematosus,SLE)是严重危害人体健康的常见病,由自身免疫介导,以免疫性炎症为突出表现的弥漫性结缔组织病。患者体内出现大量自身抗体(以抗核抗体为代表)和多系统累及是SLE的两个主要临床特征。本病女性约占90%,常为育龄妇女。我国患病率约为70/10万。

\subsection{病因与发病机制}

\subsubsection{病因}

SLE病因尚不清楚,可能与环境、性激素、药物和遗传等多种因素有关。
\paragraph{遗传素质}

SLE有遗传倾向性及家族发病聚集性,同卵孪生者共患SLE的频率(25%~50%)明显高于异卵孪生者(5%),SLE患者近亲中本病的发生率高于一般人群SLE的发病率,说明SLE存在遗传的易感性。
\paragraph{环境因素}

(1)日光。日光过敏见于40%的SLE患者,表现为光照部位出现红斑,皮疹加重或全身情况恶化等,称为光敏感现象。使SLE患者出现光敏感现象的紫外线主要是波长为290~320nm的紫外线B,具有较强的穿透能力,可穿过云雾层和玻璃。

(2)感染。许多间接依据提示SLE的发病与某些感染因素有关,特别是病毒感染,可通过分子模拟或超抗原作用引起自身免疫反应。

(3)药物。含有芳香族胺基团或联胺基团的药物可诱发药物性狼疮,可分为两类:第一类是诱发SLE症状的药物,如青霉胺、磺胺类、保泰松和金制剂等;第二类是引起狼疮样综合征的药物,如肼屈嗪、普鲁卡因胺和苯妥英钠,这类药物在应用较长时间和较大剂量后患者可出现SLE的临床症状及实验室改变,SLE患者应慎用。
\paragraph{性激素}

SLE在女性多发,在育龄妇女多发,妊娠期和产后哺乳期可诱发本病或使病情活动,提示雌激素对SLE的发病及加重有促进作用。目前研究发现,SLE患者体内雌激素水平增高,雄激素降低,同时泌乳素增高可能对SLE病情有影响。

\subsubsection{发病机制}

本病发病机制尚未完全清楚。一般认为是在遗传、环境和性激素等多种因素作用下,引起机体免疫调节功能紊乱,出现T、B淋巴细胞和巨噬细胞功能异常,造成免疫耐受异常,致使淋巴细胞不能正确识别自身组织,导致自身免疫反应的发生和持续。
\paragraph{B淋巴细胞}

SLE的突出特点是B淋巴细胞功能亢进,95%以上的患者抗核抗体呈阳性。可能是环境因素(感染)或超抗原等刺激B淋巴细胞引起B细胞丧失自身耐受,或T辅助淋巴细胞功能亢进促使B细胞保持持续的高活跃状态而产生多种自身抗体,特别是抗核抗体(antinuclear
antibodies),这是本病的免疫学特点,也是本病发生和延续的主要因素之一。
\paragraph{T淋巴细胞}

T细胞是产生和保证自身免疫耐受的主要原因,SLE存在T细胞的多种异常。表现为T抑制细胞(CD$_8^+$
)减少,T辅助细胞(CD$_4^+$
)功能过强及“双阴性”(CD$_4^-$CD$_8^-$
)T细胞增加。SLE中大多数致病性自身抗体的产生都是依赖于T细胞。
\paragraph{细胞因子异常}

研究发现SLE患者细胞因子分泌异常。狼疮中单核细胞自发产生IL-1和IL-6增加,在活动期更明显,这些细胞因子引起B细胞多克隆活化产生自身抗体。IL-1可诱导IL-8、IL-6、TNF等炎症因子产生,与狼疮肾炎有关,IL-1活性与光过敏亦有关。几乎所有SLE患者血清中IL-2水平均升高,且活动期比缓解期高,IL-2为T细胞的生长因子,由CD$_4^+$
细胞产生。在SLE活动期,IL-10水平升高,IgG生成增加。

\subsection{临床表现}

\subsubsection{全身症状}

起病可急可缓,多数早期表现为非特异的全身症状,如发热,尤以低热常见,以及全身不适、乏力、体重减轻等。病情常缓重交替出现。感染、日晒、药物、精神创伤、手术等均可诱发或加重。

\subsubsection{皮肤和黏膜}

皮疹常见,约40%患者有面部典型红斑称为蝶形红斑。急性期有水肿、色鲜红,略有毛细血管扩张及鳞片状脱屑,严重者出现水疱、溃疡、皮肤萎缩和色素沉着。手掌大小鱼际、指端及指(趾)甲周红斑,身体皮肤暴露部位有斑丘疹、紫斑等。出现各种皮肤损害者约占总患病数的80%,毛发易断裂,可有斑秃。15%~20%的患者有雷诺现象。口腔黏膜出现水泡、溃疡,约占12%。少数患者病程中发生带状疱疹。

\subsubsection{关节、肌肉}

约90%以上患者有关节肿痛,且往往是就诊的首发症状,最易受累的是手近端指间关节,膝、足、髁、腕关节均可累及。关节肿痛多呈对称性。约半数患者有晨僵。X线检查常无明显改变,仅少数患者有关节畸形。肌肉酸痛、无力是常见症状。

\subsubsection{肾脏}

约50%患者有肾脏疾病临床表现,如蛋白尿、血尿、管型尿、白细胞尿、低比重尿、浮肿、血压增高、血尿素氮和肌酐增高等,电子显微镜和免疫荧光检查几乎均有肾脏病理学异常。

\subsubsection{胃肠道}

部分患者可表现为胃肠道症状,如上消化道出血、便血、腹水、麻痹性肠梗阻等,这是由于胃肠道的血管炎所致,如肠系膜血管炎。肠系膜血管的动、静脉伴行,支配胃肠营养和功能,如发生病变,则所支配的部位产生相应症状,严重时累及生命。肠系膜血管炎可以导致胃肠道黏膜溃疡、小肠和结肠水肿、梗阻、出血、腹水等,出现腹痛、腹胀、腹泻、便血和黑粪、麻痹性肠梗阻等临床表现。如不及时诊断、治疗,可致肠坏死、穿孔,造成严重后果。

\subsubsection{神经系统}

SLE的神经累及又称为神经精神性狼疮,往往与SLE活动性相关。轻者仅有偏头痛、性格改变、轻度认知障碍;重者可表现为脑血管意外、昏迷、癫痫
持续状态等。癫痫
是中枢神经系统受累的最常见表现,以大发作多见。神经精神狼疮以弥漫性的高级皮层功能障碍为表现的多与抗神经元抗体、抗核糖体P蛋白抗体相关,以局灶性神经定位体征为表现的多伴有抗磷脂抗体阳性或有全身血管炎或明显病情活动。

\subsubsection{肝}

SLE引起的肝功能损害主要表现为肝肿大、黄疸、肝功能异常以及血清中可存在多种自身抗体等。其中,肝肿大约占10%~32%,多在肋下2~3cm,少数可明显肿大。红斑狼疮引起黄疸的原因很多,主要有溶血性贫血、合并病毒性肝炎,胆道梗阻及急性胰腺炎等。约30%~60%的红斑狼疮患者可有肝功能试验异常,主要表现为转氨酶水平升高、血清白蛋白水平降低、球蛋白水平及血脂水平升高等。红斑狼疮合并肝功能损害常常为轻、中度肝功能异常,严重肝功能损害者较少见。SLE可并发I型自身免疫性肝炎(狼疮性肝炎),多发生于年轻的女性,临床上可表现为乏力、关节痛、发热、肝脾肿大、黄疸等。

\subsubsection{心脏}

约10%~50%的患者出现心脏病变,可能由于疾病本身,也可能由于长期服用糖皮质激素治疗引起。心脏病变包括心包炎、心肌炎、心内膜及瓣膜病变等,依个体病变不同,表现有胸闷、胸痛、心悸、心脏扩大、充血性心力衰竭和心律失常等。

\subsubsection{肺}

肺和胸膜受累约占50%,其中约10%患狼疮性肺炎、胸膜炎和胸腔积液较常见,肺实质损害多数为间质性肺炎和肺间质纤维化,引起肺不张和肺功能障碍。狼疮性肺炎的特征是肺部有斑状浸润,可由一侧转到另一侧,激素治疗能使影消除。在狼疮性肺损害基础上,常继发细菌感染。

\subsubsection{血液系统}

SLE患者常有贫血、白细胞和血小板计数减少。活动性SLE约60%的患者有慢性贫血,10%属溶血性贫血,约40%患者白细胞计数减少或淋巴细胞绝对数减少,本病所致的白细胞计数减少一般发生在治疗前或疾病复发时,大多对激素敏感,与细胞毒药物所致细胞减少不同。约20%患者有血小板计数减少,发生各系统出血如鼻出血、牙龈出血、皮肤紫癜、血尿、便血、颅内出血等。

\subsubsection{其他}

部分患者在病变活动时出现淋巴结和腮腺肿大。眼部受累较普遍,如结合膜炎和视网膜病变,少数视力障碍。患者可有月经紊乱和闭经。

\subsection{诊断}

\subsubsection{实验室改变}

活动期ESR增快,IgG、IgA和IgM均增高,75%~90%的患者血补体C{3} 、C{4}
降低,50%~60%的患者抗心磷脂抗体IgG型或IgM型阳性,80%~95%的患者抗核抗体(ANA)阳性,血清ANA效价≥1∶80,对结缔组织病有诊断意义,狼疮带实验SLE阳性率为70%,IgG沉着诊断意义更大。

\subsubsection{SLE的分期}

(1)轻型SLE:约占25%,实验室检查符合,症状较轻,仅有皮疹,或虽有轻度活动性,但症状轻微,如疲倦、关节痛、肌肉痛、皮疹等,而累及的靶器官功能正常或稳定。

(2)重型SLE:活动程度较高,病情较严重,患者有发热、乏力等全身症状,实验室检查有明显异常。

(3)急性暴发性危重SLE:指SLE患者病情突发恶化,狼疮高热持久不退、急性肾衰竭、狼疮脑病癫痫
发作、狼疮心肌损害、狼疮血液危象等。

\subsubsection{诊断标准}

根据1972年及1982年的美国风湿医学会标准,凡具有下列11项准则中的任意4项以上,可诊断为红斑性狼疮:①面颊有红斑------脸上之蝶形皮疹;②圆盘红斑;③阳光过敏;④口腔溃疡;⑤非糜烂性关节炎;⑥浆液膜炎包括有肋膜炎或心包膜炎;⑦肾功能障碍,出现蛋白尿或尿中有圆柱体;⑧神经障碍如抽搐或精神病;⑨血液障碍主要是溶血性贫血,网状血球增多,加上下列之一:白细胞计数过低(<4000/mm{3}
)、淋巴细胞数量过少(<1500/mm{3} )、血小板计数过少(<100000/mm{3}
);⑩免疫系统障碍有下列之一:狼疮细胞阳性反应、DNA抗体、核酸核蛋白抗体、梅毒血清假阳性反应半年以上;⑪抗核抗体(ANA)阳性反应。

\subsection{治疗}

\subsubsection{药物治疗}

SLE目前虽不能根治,但合理治疗可以缓解,尤其是早期患者。强调早期诊断、早期治疗,以避免或延缓不可逆的组织脏器的病理损害,维持重要脏器功能。在疾病的活动期应积极控制病情,使其逐步稳定,达到缓解或长期平稳。SLE的治疗主要包括控制炎症反应、免疫抑制、免疫调节和对症治疗。

\subsubsection{药物分类和作用机制}

治疗SLE药物主要包括抗炎药物如NSAIDs、抗疟药和糖皮质激素;免疫抑制剂有环磷酰胺、AZA和CsA。

抗疟药有抗光敏和稳定溶酶体膜作用,并具有在皮肤积蓄较多的特点,对控制皮损、光敏和轻度关节炎有效。该类药物的主要不良反应是引起视网膜退行性病变,服药期间应定期检查眼底,一般在治疗开始用药6个月后检查1次,以后每隔3个月复查。另有心脏病患者,特别是心动过缓或传导阻滞者禁用。

糖皮质激素直接作用于G{0}
期细胞及免疫效应淋巴细胞,抑制炎症反应,对炎症细胞及炎症介质也有强力的抑制作用,是目前治疗SLE的主要和基本药物。适用于有内脏损害或血液病变和病情活动者。

环磷酰胺(CTX)作用于免疫系统的定向干细胞S期,抑制细胞分化、增殖,从而抑制免疫效应细胞,作用较缓慢但持久。对体液免疫的抑制作用较强,能抑制B细胞增殖和抗体生成,对T细胞介导的免疫非特异性炎症反应也有效。AZA主要抑制非特异性炎症反应,抑制细胞免疫。其不良反应较CTX少,尤其对肿瘤免疫监视系统和性腺影响小。CsA可特异性抑制T淋巴细胞IL-2的产生,发挥选择性的细胞免疫抑制作用,主要影响G{0}
和G{1} 早期细胞,对已合成DNA的细胞无明显作用。

\subsubsection{治疗药物的选用}
\paragraph{轻型SLE}

症状轻微,无明显内脏损害。平时要避免日光刺激,穿保护性衣服,外出时用光防护系数在15以上的遮光剂,避免日光浴;如只有皮疹者,可短期局部应用激素,但脸部应尽量避免使用强效激素类外用;皮疹多,外用无效者可用抗疟药,氯喹每日200mg,每日1次,或羟氯喹每日200~400mg,每日1或2次,治疗2~3周。氯喹对光过敏和关节症状也有一定疗效。也可用NSAIDs如双氯芬酸25mg,每日3次。如上述治疗无效,应及早服用小剂量糖皮质激素治疗,必要时加用免疫抑制剂。应注意轻型SLE可因过敏、感染、妊娠生育、环境变化等因素而加重。
\paragraph{重型SLE}

治疗主要分诱导缓解和维持治疗两个阶段。诱导缓解目的在于迅速控制病情,阻止或逆转内脏损害,力求疾病完全缓解,多数患者的诱导缓解期需要超过半年至1年,但应注意过分免疫抑制诱发的并发症,尤其是感染、性腺抑制等。

(1)糖皮质激素:通常选用中效制剂泼尼松,常规剂量为每日60~100mg,病情较轻可采用中小剂量如每日30~40mg。足量激素持续6~8周,病情明显好转后缓慢减量,最后以最小有效剂量长期维持(≤每日10mg)。在治疗过程中应同时或适时加用免疫抑制剂,如环磷酰胺、AZA等,以便更快地诱导病情缓解和巩固疗效,并避免长期使用较大剂量激素导致的严重不良反应。SLE的激素疗程较漫长,应注意保护下丘脑---垂体---肾上腺轴,避免使用对该轴影响较大的长效激素。

(2)细胞毒类药物与激素合用能有效地诱导疾病缓解,提高疗效,阻止和逆转病变的发展,改善远期预后,并有助于激素顺利撤减。该类药物中环磷酰胺为首选,其次为AZA。CTX口服每日1.0~2.5mg/kg,也可静脉用CTX
200mg/kg,每周3次,或400mg/kg,每周2次;或采用大剂量静脉冲击疗法。用药期间应密切注意血象监测,避免导致白细胞计数过低。AZA在控制肾脏和神经系统病变效果较差,而对浆膜炎、血液系统、皮疹等较好,可引起造血危象,严重粒细胞和血小板缺乏症。MTX通过抑制核酸的合成发挥细胞毒作用,疗效不及环磷酰胺冲击疗法,但长期用药耐受性较佳。主要用于关节炎、肌炎、浆膜炎和皮肤损害为主的SLE。如大剂量激素联合环磷酰胺或AZA使用4~12周,病情仍不改善,应加用CSA,每日5mg/kg,分2次服用,服用3个月,以后每月减1mg/kg,至每日3mg/kg作维持治疗。其主要不良反应为肝肾功能损害,使用期间应予以监测。在需用CTX的病例,由于血白细胞计数减少而暂不能使用者,亦可用本药暂时替代。

SLE达到诱导缓解后,应继续巩固治疗。目的在于用最少的药物防止疾病复发,尽可能使患者维持在“无病状态”。口服泼尼松7.5~20mg和每日口服硫唑嘌呤50~100mg维持,部分患者需终身服用激素治疗。必须强调对患者的长期随访。
\paragraph{急性暴发性危重SLE}

在SLE有重要脏器累及,乃至出现狼疮危象的情况下,如急性爆发性狼疮、重症狼疮性肾炎、中枢神经狼疮和急性自身免疫性贫血、血小板减少性紫癜等需应用以下冲击疗法。

(1)甲基泼尼松龙(Methylprednisolone,MP)冲击疗法:即甲基泼尼松龙用至500~1000mg,每日1次,加入5%葡萄糖200mL,静脉滴注1h左右,连续3d为1个疗程,疗程间隔期5~30d,间隔期和冲击后需每日口服泼尼松0.5~1mg/kg,疗程和间隔期长短视具体病情而定。MP冲击疗法只能解决急性期的症状,随后的治疗必须有一定量的激素与环磷酰胺冲击疗法配合使用,否则病情容易反复。需强调的是,在大剂量冲击治疗前或治疗中应密切观察有无感染发生。如有感染应及时给予相应的抗感染治疗。对活动程度严重的SLE,加用细胞毒药物有利于更好地控制SLE活动,减少SLE暴发,同时减少激素的需用量。

(2)目前推荐采用美国国立卫生院的CTX冲击疗法:即0.5~1.0g/m{2}
体表面积,静脉注射,每月1次,连续3~6个月后每3个月1次,共2年。多数患者6~12月可缓解病情而进入巩固治疗阶段。由于对环磷酰胺的敏感性存在个体差异,年龄、病情、病程和体质等使其对药物的耐受性有所区别,所以治疗时应根据患者的具体情况,掌握好剂量、冲击间隔期和疗程。除白细胞计数减少和诱发感染外,环磷酰胺冲击治疗的不良反应包括性腺抑制、胃肠道反应、脱发和肝功能损害等。狼疮性脑病、肾炎或严重血小板计数减少可静脉注射大剂量丙种球蛋白,本疗法是一种强有力的辅助治疗措施,对危重的难治性SLE也颇有效。常用剂量200~400mg/kg静脉注射,每日1次,连续3~5d,必要时每3~4周重复1次。对狼疮脑病癫痫
发作者、急性肾衰竭者、狼疮心肌损害严重者,除使用甲泼尼龙冲击疗法和CTX冲击疗法外,还需进行对症治疗。狼疮癫痫
发作者,宜地西泮肌注或用卡马西平等抗癫痫
药;急性肾衰竭者,宜在血液透析或腹膜透析基础上加强免疫干预治疗;心力衰竭者,宜减轻心脏前后负荷和适当使用洋地黄制剂。

\subsubsection{常用治疗方案的选用}

SLE病程迁延、病变活动与缓解可反复出现,治疗后只能得到不同程度的缓解,而治疗药物常带来严重不良反应。因此,在制定治疗方案时应认真考虑如何达到最大治疗效果与最小药物不良反应之间的平衡。要考虑个体化原则,又要掌握治疗的风险与效应之比,制定具体的治疗方案(见表\ref{tab16-5})。

\begin{longtable}[]{p{5cm}p{5cm}}
    \caption{SLE治疗方案举例}
    \label{tab16-5}\\
    \toprule
    症状                 & 药物治疗\tabularnewline
    \midrule
    \endhead
    关节炎症             & ①NSAIDs;②抗疟药;③激素和/或MTX\tabularnewline
    发热                 & ①NSAIDs;②抗疟药;③激素\tabularnewline
    皮疹                 & ①紫外线防护;②局部使用激素;③抗疟药;④激素\tabularnewline
    雷诺现象             & ①戒烟、保暖;②钙拮抗剂;③哌唑嗪\tabularnewline
    浆膜炎               & ①NSAIDs;②激素\tabularnewline
    肺部疾病             & ①激素\tabularnewline
    血小板计数减少、贫血 &
    ①激素;②静脉用丙种球蛋白;③免疫抑制剂;④脾切除\tabularnewline
    肾小球肾炎           &
    ①激素;②甲基泼尼龙+环磷酰胺冲击;③免疫抑制剂\tabularnewline
    中枢狼疮             &
    ①焦虑、抑郁治疗;②激素;③抗癫痫
    药物;④免疫抑制剂\tabularnewline
    \bottomrule
\end{longtable}

\subsubsection{其他治疗}

所有SLE患者均应进行包括心理及精神支持、避免日晒或紫外线照射、预防和治疗感染或其他并发症及依据病情选用适当的锻炼方式。对于常规治疗不能控制或重症患者,可选用大剂量免疫球蛋白冲击,血浆置换等方法进行治疗。