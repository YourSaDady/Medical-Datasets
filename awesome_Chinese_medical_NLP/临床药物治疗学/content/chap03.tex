\chapter{药品不良反应}

\section{药品不良反应的基本概念}

药品不良反应最早可追溯到19世纪中叶的氯仿事件。曾广泛用于麻醉的氯仿被认为与水一样安全,但1877年英国医学协会经过多年调查与研究,认为氯仿是一种危险药物,小剂量会产生心脏毒性,大剂量时可抑制呼吸。随着20世纪初制药工业的兴起,特别是磺胺类药物及抗生素的先后问世,标志着疾病的药物治疗进入了一个新纪元。药物挽救了无数患者的健康和生命,但同时也不可避免的带来了药品不良反应。20世纪60年代初,导致数千例海豹肢畸形的沙利度胺(thalidomide)事件发生后,震惊了世界各国。自此,药品不良反应开始受到各国政府管理部门和医药界的重视。此后,世界各国纷纷成立了药品不良反应监测中心或委员会,加强了对药品上市前的安全性试验和上市后的不良反应监测。

\subsection{药品不良反应(adverse drug reaction,ADR)的定义}

世界卫生组织(WHO)对药品不良反应的定义是:药品不良反应是指在预防、诊断、治疗疾病或者调节生理功能的过程中,人接受正常剂量的药物时出现的任何有伤害的和与用药目的无关的反应。这个定义排除了药物过量、药物滥用和治疗错误。2011年7月1日起施行的《药品不良反应报告和监测管理办法》对药品不良反应的定义是:合格药品在正常用法用量下出现的与用药目的无关的有害反应。例如,采用抗菌药物治疗时,抗菌药物引起的患者皮疹、腹泻等反应以及抗肿瘤药物治疗时引起的患者脱发、骨髓抑制等皆属于药品不良反应。

\subsection{药品不良反应的危害性}

药品不良反应的危害很大。从20世纪初至今,在全世界范围内引起很多患者死亡或重大伤残事件的药品不良反应达数十起之多。随着制药工业的飞速发展,临床应用的治疗药物品种越来越多,药品不良反应越来越常见和严重。据报道,20世纪70年代美国住院患者中28%发生药品不良反应;据20世纪90年代末美国150家医院39项研究报告估计,美国每年有200多万患者由于药品不良反应导致病情恶化,其中10.6万人死亡。WHO有报告指出,临床用药实践中药品不良反应发生率高达5%~20%,在住院患者中为10%~15%。我国每年约250万住院患者与药品不良反应有关,每年因药品不良反应消耗的费用超过15亿元。近年来,我国陆续报道的严重药品不良反应事件有酮康唑导致肝坏死、左旋咪唑导致间质性脑炎、龙胆泻肝丸导致肾功能损害等。

\subsection{药品不良反应的分类}

\subsubsection{按病因学分类}

根据病因学,WHO将药品不良反应分为A、B、C三种类型。
\paragraph{A型不良反应}

A型不良反应又称剂量相关性不良反应,是由药物本身或其代谢物所致,是药物固有药理作用的增强和持续所导致。具有明显的剂量依赖性和可预见性,且与药物常规的药理作用密切相关,发生率高而致死率相对较低。例如,镇静催眠药引起的中枢抑制不良反应随剂量增加而加重。本类型不良反应发生的频率和强度与用药者的年龄、性别、机体的生理和病理状态都有很大关系。临床表现为副作用、毒性作用、首剂效应、后遗效应、继发效应等。
\paragraph{B型不良反应}

B型不良反应又称剂量不相关不良反应,是由于药物性质的变化或者用药者的特异体质引起。反应的性质通常与药物的常规药理作用无关,反应的强度与用药剂量无关(对不同的个体来说,本类不良反应的发生以及严重程度与剂量无关;对于同一个敏感个体来说,药物的量与反应的强度相关),难以预见,发生率较低而致死率相对较高。这类不良反应由患者的敏感性增高所引起,表现为药物反应发生质的改变,可能是遗传药理学变异引起的,大多数具有遗传药理学基础的反应一般在患者接触药物后才能发现,因而难以在首次用药时预防这类不良反应发生。例如,先天性缺乏血浆假性胆碱酯酶的患者,在应用琥珀胆碱时易出现严重骨骼肌松弛、呼吸抑制。临床表现为变态反应和特异质反应等。
\paragraph{C型不良反应}

C型不良反应发生机制尚不十分明确,大多是发生在长期用药之后,潜伏期长,且没有明确的时间联系,难以预测。例如,长期服用避孕药导致的乳腺癌、血管栓塞;孕期服用乙烯雌酚会导致子代女婴甚至是第三代女婴发生阴道腺癌。本类型的不良反应主要包括致畸、致癌、致突变。三型药品不良反应的区别如表\ref{tab3-1}所示。

\begin{table}[ht]
    \caption{三型药品不良反应的区别}
    \label{tab3-1}
    \centering
    \begin{tabular}{cccc}
    \toprule
    项目 & A型 & B型 & C型 \\
    \midrule
    计量 & 有关 & 无关 & 正常\\
    潜伏期 & 短 & 不定 & 长 \\
    重现性 & 能 & 能 & 不能\\
    遗传性 & 无关 & 显著 & 可能\\
    体质 & 无关 & 有关 & 可能有关\\
    家族性 & 无关 & 显著 & 可能有关\\
    种族性(民族性) & 无关 & 有关 & 无关\\
    毒理筛选 & 易 & 难 & 不定\\
    预后 & 一般良好 & 不定 & 不定\\
    \bottomrule
    \end{tabular}
\end{table}


\subsubsection{按严重程度分类}

根据发生的严重程度,药品不良反应分为轻度、中度、重度三级。
\paragraph{轻度药品不良反应}

轻度药品不良反应是指患者可忍受,不影响治疗进程,对患者康复无影响。对于轻度药品不良反应患者,不需特别处理,但应注意观察。
\paragraph{中度药品不良反应}

中度药品不良反应是指患者难以忍受,需要撤药或作特殊处理,对患者康复有直接影响。对于中度药品不良反应患者,应立即撤销引起不良反应的药品,并针对其临床表现和类型进行特殊治疗。
\paragraph{重度药品不良反应}

重度药品不良反应是指危及患者生命,致残或致死。对于重度药品不良反应患者,需立即停药并紧急处理。

\subsection{药品不良反应的临床表现形式}

\subsubsection{副作用(side effect)}

副作用是指在治疗量时出现的与治疗目的无关且机体感觉不适的药理反应。如局部麻醉药物引起的头昏、低血压等。随着治疗目的的不同,副作用也可转化为治疗作用。如阿托品具有抑制腺体分泌、解除平滑肌痉挛、加快心率等作用,在全身麻醉时利用它抑制腺体分泌的作用,其松弛平滑肌引起腹胀或尿潴留是副作用;在利用其解痉作用时,口干和心悸成为副作用。副作用是在治疗剂量下发生的,是药物本身固有的作用,一般较轻微并可以预料,所以可采取预防措施来避免或减轻。

\subsubsection{毒性反应(toxic reaction)}

毒性反应是指在剂量过大或药物在体内蓄积过多时发生的危害性反应,一般比较严重,但是可以预知并应该避免。急性毒性多损害循环、呼吸及神经系统功能,慢性毒性多损害肝、肾、骨髓、内分泌等功能;致癌(carcinogenesis)、致畸(teratogenesis)、致突变(mutagenesis)反应,即通常所指的“三致作用”,也属于慢性毒性范畴。企图增加剂量或延长疗程以达到治疗目的是有限度的,过量用药是十分危险的。

\subsubsection{后遗效应(residual effect)}

后遗效应是指停药后血药浓度已降至阈浓度以下时残存的药理效应。例如,服用巴比妥类催眠药后,次晨出现乏力、困倦现象;长期应用肾上腺皮质激素,停药后肾上腺皮质功能低下,数月内难以恢复。

\subsubsection{停药反应(withdrawal reaction)}

停药反应也称撤药反应,是指突然停药后原有疾病加剧,主要表现是症状反跳。例如,长期服用可乐定降血压,停药次日血压将激烈回升;癫痫
患者长期服用苯妥英钠,突然停用时,诱发更严重的癫痫
发作。

\subsubsection{继发反应(secondary reaction)}

继发反应是由于药物的治疗作用所引起的不良后果,又称为治疗矛盾。如广谱抗生素可引起菌群失调而致某些维生素缺乏,进而引起出血和二重感染;免疫抑制药降低机体的抵抗力也可致二重感染;阿司匹林诱发雷耶氏综合征等。

\subsubsection{首剂效应(first-dose response)}

某些药物在开始作用时,由于机体对药物的作用尚未适应而引起较强的反应。若一开始即按常规剂量易导致过度作用。例如,哌唑嗪按常用治疗量开始用药,易致血压骤降,故对于这些药物应从小剂量开始,逐渐加量至常用量。

\subsubsection{变态反应(allergic reaction)}

变态反应是一类免疫反应。非肽类药物作为半抗原与机体蛋白结合为抗原后,经过接触10d左右的敏感化过程而发生的反应,也称为过敏反应(hypersensitive
reaction)。药物变态反应可波及全身各组织和器官,可分为全身性反应和皮肤反应两大类。

药物变态反应分为四型:Ⅰ型反应为速发反应,常见药物有青霉素、胰岛素等;Ⅱ型反应为细胞毒性反应,常见药物有青霉素、甲芬那酸等;Ⅲ型反应为免疫复合物型反应,常见药物如磺胺、巴比妥;Ⅳ型反应为迟发型反应,常见药物如磺胺、四环素等。

\subsubsection{特异质反应(idiosyncrasy)}

特异质反应是指少数患者由于遗传因素对某些药物的反应性发生了改变。反应性质也可能与常人不同,但与药物固有药理作用基本一致,反应严重程度与剂量成比例,药理性拮抗药救治可能有效。例如,乙酰化酶缺乏患者服用肼苯达嗪时容易引起红斑狼疮样反应;红细胞内缺乏葡萄糖-6-磷酸脱氢酶的患者,体内还原型谷胱甘肽不足,服用某些药物如伯氨喹后,容易发生急性溶血性贫血和高铁血红蛋白血症。

\subsubsection{药物依赖性(drug dependence)}

连续使用某些作用于中枢神经系统的药物后,用药者为追求快感而要求定期连续地使用该药,为精神依赖性。一旦停药会产生严重的戒断症状,这种反应又称为生理依赖性。

\subsection{药品不良反应因果关系评定}

药品不良反应的发生与否与所用药物有关,怎样评价两者之间的相关性,这是确定某些药品不良反应的重要一环。

\subsubsection{药品不良反应报告因果关系等级评价准则}

(1)开始用药的时间和不良反应出现的时间有无合理的先后关系。

(2)所怀疑的不良反应是否符合该药品已知不良反应的类型。

(3)停药或减量后,反应是否减轻或消失。

(4)再次接触可疑药品是否再次出现同样的反应。

(5)所怀疑的不良反应是否可用并用药的作用、患者的临床状态或其他疗法的影响来解释。

\subsubsection{药品不良反应报告因果关系的判定}

将因果关系的确定程度分为肯定、很可能、可能、可能无关和肯定无关5级标准,可参照表\ref{tab3-2}进行判断。

\begin{table}[ht]
    \caption{药品不良反应判断标准表}
    \label{tab3-2}
    \centering
    \begin{tabular}{cccccc}
    \toprule
    标准 & 肯定 & 很可能 & 可能 & 可能无关 & 肯定无关\\
    \midrule
    合理的时间顺序 & 是 & 是 & 是 & 是 & 否\\
    属已知药物的反应类型 & 是 & 是 & 是 & 否 & 否\\
    停药可以改善 & 是 & 是 & 是或否 & 是或否 & 否\\
    再次给药可重复出现 & 是 & \multicolumn{3}{c}{为患者健康不宜重复给药} & 否 \\
    以已知疾病可以解释 & 否 & 否 & 是或否 & 是或否 & 是\\
    \bottomrule
    \end{tabular}
\end{table}

\subsection{中药不良反应}

“是药三分毒”,即使是服用中药,同样有可能产生药品不良反应。在汉代以前曾有记载,在400种中药中有毒者占60多种。《神农本草经》将中药分成上、中、下三品,“下品多毒,不可久服”,如大戟、莞花、甘遂、乌头、狼毒等,后来的实践证明,当时认为“无毒”、多服和久服不伤人的“上品”也发生了中毒死亡病例,如人参等。而“中品”中的百合、麻黄等也被实践证明是有一定毒性的药物,不可滥用。

\section{药品不良反应监测方法与报告系统}

\subsection{药品不良反应监测方法}

鉴于药品不良反应的严重性,许多发达国家从20世纪60年代开始,先后开展了药品不良反应的监测工作。我国卫生部于1988年在北京、上海两地进行了药品不良反应监测工作的试点,并在全国范围内逐步扩大。我国于1989年组建了国家药品不良反应监测中心。1998年加入WHO国际药品监测合作中心。2001年将药品不良反应报告制度纳入修订的《中华人民共和国药品管理法》中,且已在31个省级及1个军队系统建立药品不良反应监测中心,并逐渐完善了分支机构,还于2003年9月开始发布药品不良反应信息通报,2004年实施《药品不良反应报告和监测管理办法》。目前,常用的药品不良反应监测方法有自愿呈报系统、集中监测系统、病例对照研究、队列研究、记录联结和记录应用等。

\subsubsection{自愿呈报系统(spontaneous reporting system)}

自愿呈报系统是一种自愿而有组织的报告系统,国家或地区设有专门的药品不良反应登记处,成立有关药品不良反应的专门委员会或监测中心,委员会或监测中心通过监测报告单位把大量分散的不良反应病例收集起来,经加工、整理、因果关系评定后储存,并将不良反应信息及时反馈给监测报告单位以保障用药安全。目前,世界卫生组织国际药物监测合作中心的成员国大多采用这种方法。

自愿呈报系统的优点是监测覆盖面大、监测范围广、时间长、简单易行。药品上市后自然地加入被监测行列,且没有时间限制。药品不良反应能够得到早期警告。由于报告者及时得到反馈信息,可以调整治疗计划,合理用药。缺点是存在资料偏差和漏报现象。

\subsubsection{集中监测系统}

集中监测系统即在一定时间、一定范围内详细记录药品不良反应的发生情况。根据研究目的分为病源性和药源性监测。病源性监测是以患者为线索,了解患者用药及药品不良反应情况。药源性监测是以药品为线索,对某一种或几种药品的不良反应进行监测。

集中监测系统通过对资料的收集和整理,对药品不良反应的全貌有所了解,如药品不良反应出现的缓急和轻重程度,不良反应出现的部位和持续时间,是否因不良反应而停药,是否延长住院期限,各种药品引起的不良反应发生率及转归等。

\subsubsection{病例-对照研究(case control studies)}

病例-对照研究是对比有某病的患者组与未患病的对照组的研究,其目的是为了找出两组对先前药物暴露的差异。即在人群中患有拟研究的疾病------患者组(病例组)与未有患那种疾病的人群(对照组)相比较,研究前者拥有假说因素是否更高。在药品不良反应监测中,拟研究的疾病为怀疑药品引起的不良反应,假说因素则是可疑药品。可疑药品在病例组的暴露率与对照组比较,如果两者在统计学上有意义,说明它们的相关性成立。Herbst等发现母亲孕期服用己烯雌酚与女儿阴道腺癌的关系,就是采用病例-对照法研究的。

\subsubsection{队列研究(cohort studies)}

队列研究是将样本分为两个组进行观察,一组为暴露于某一药品的患者,另一组为不暴露于该药品的患者,验证两组间结果的差异,即不良事件的发生率或疗效。一般分为前瞻性队列研究、回顾性队列研究和双向性队列研究。前瞻性队列研究在药品不良反应监察中较常用。前瞻性调查是从现在时点起,对固定人群的观察。优点主要有①可收集到所有的资料;②患者的随访可持续进行;③可以估计相对和绝对危险度;④假设可产生,亦可得到检验。缺点主要有①资料可能偏性;②容易漏查;③假若药品不良反应发生率低,为了得到经得起统计学检验的病例数,就得扩大对象人群或延长观察时间,但有时不易做到;④费用较高。英国西咪替丁的上市后监测就是采用队列研究进行的。

\subsubsection{记录联结(record linkage)}

记录联结是指通过独特方式把各种信息联结起来,可能会发现与药品有关的事件。通过分析提示药品与疾病间和其他异常行为之间的关系,从而发现某些药品的不良反应。如通过研究发现安定类药与交通事故之间存在相关性,证实安定类药有嗜睡、精力不集中的不良反应,因此建议驾驶员、机器操作者慎用。阿司匹林与脑出血间也存在统计学关系。记录联结是一种较好地监测药品不良反应的方法,计算机的广泛应用将大大有利于记录联结的实施。

\subsubsection{记录应用}

是指在一定范围内通过记录使用研究药品的每例患者的所有相关资料,以提供没有偏性的抽样人群,从而了解药品不良反应在不同人群的发生情况,计算药品不良反应发生率,寻找药品不良反应的易发因素。根据研究的内容不同,记录应用规模可大可小。

\subsection{药品不良反应的报告系统}

各国情况不同,监测系统各不相同。我国药品不良反应监测报告工作由国家食品药品监督管理局主管,监测报告系统由国家药品不良反应监测中心和专家咨询委员会、省市级中心监测报告单位组成。

\subsubsection{药品不良反应报告要求}

2011年7月1日起施行的《药品不良反应报告和监测管理办法》明确指出:药品生产、经营企业和医疗机构应当主动收集药品不良反应,获知或者发现药品不良反应后应当详细记录、分析和处理,填写《药品不良反应/事件报告表》并报告。

\subsubsection{药品不良反应报告范围}

(1)新药监测期内的国产药品应当报告该药品的所有不良反应;其他国产药品,报告新的和严重的不良反应。

(2)进口药品自首次获准进口之日起5年内,报告该进口药品的所有不良反应;满5年的,报告新的和严重的不良反应。

(3)药品生产、经营企业和医疗机构发现或者获知新的、严重的药品不良反应应当在15d内报告,其中死亡病例须立即报告;其他药品不良反应应当在30d内报告。有随访信息的,应当及时报告。

\subsubsection{严重药品不良反应的定义}

严重药品不良反应是指因使用药品引起以下损害情形之一的反应:①导致死亡;②危及生命;③致癌、致畸、致出生缺陷;④导致显著的或者永久的人体伤残或者器官功能的损伤;⑤导致住院或者住院时间延长;⑥导致其他重要医学事件,如不进行治疗可能出现上述所列情况的。

\subsubsection{新的药品不良反应定义}

新的药品不良反应是指药品说明书中未载明的不良反应。说明书中已有描述,但不良反应发生的性质、程度、后果或者频率与说明书描述不一致或者更严重的,按照新的药品不良反应处理。对上市5年以上的药品,对已知的比较轻微的不良反应不要求报告,如三环类抗抑郁药引起的口干、阿片类药物所致便秘,地高辛引起的恶心等。

\section{药品不良反应的防治}

对医师和药师来说,药物治疗所带来的不良反应是个几乎天天面临的课题。药物治疗在取得疗效的同时也伴随着药品不良反应的风险。因此,只有充分认识药品不良反应产生的各种影响因素,才能合理有效地防止药品不良反应。理想的药物治疗是以最小的药品不良反应风险取得最佳的药物治疗效果。

\subsection{引起药品不良反应的常见因素}

临床应用的药品种类繁多,用药途径不同,体质又因人而异,因此药品不良反应发生的原因也是复杂的。引起药品不良反应的常见因素主要有三大类,即药物因素、机体因素和其他因素。

\subsubsection{药物因素}
\paragraph{药物作用的性质}

药物在体内的作用具有选择性,当一种药物对机体的组织和器官有多种作用时,若其中一项为治疗作用,其他作用就成为不良反应。如麻黄碱兼有平喘和兴奋中枢作用,当用于防治支气管哮喘时,兴奋中枢神经引起的失眠便为不良反应。这类不良反应常常是难以避免的。
\paragraph{药物剂量与剂型}

用药的剂量过大,或者连续用药时间过长发生不良反应的可能性大。同一药物剂型不同,由于制造工艺和用药方法的不同,可以改变药物的生物利用度,影响药物的吸收与血中药物的浓度,如不注意掌握,即会引起不良反应。
\paragraph{药物杂质}

由于技术原因,药物在生产过程中常残留微量中间产物或杂质,这些物质虽有限量,但也可引起不良反应。青霉素引起的过敏性休克就是由于发酵生产过程中,由极少量青霉素降解产生的青霉烯酸和酸性环境中部分青霉素分解产生的青霉噻唑酸所引起。另外,由于药物在储存过程中有效成分分解生成的某些物质也会对机体产生不良反应。例如,四环素在温暖条件下保存可以发生降解,生成棕色黏性物质4-差向脱水四环素,该降解产物可以引起肾脏近曲小管弥漫性损害,称为范科尼综合征。
\paragraph{药物添加剂}

药物生产过程中加入的溶剂、赋形剂、稳定剂、增溶剂、着色剂等也可引起各种不良反应。例如20世纪60年代,澳大利亚某制药公司将苯妥英钠的赋形剂碳酸钙改为乳糖,结果导致癫痫
患者用药后出现共济失调、精神障碍和复视等神经系统症状。其原因是碳酸钙能与苯妥英钠形成可溶性复盐减少苯妥英钠的吸收,乳糖则不与苯妥英钠发生相互作用,因而使苯妥英钠吸收率增加20%~30%,服药后产生不良反应。苯妥英钠注射液静脉注射后,出现的低血压与其溶剂丙二醇有一定关联。防腐剂对羟基苯甲酸酯则可以引起荨麻疹。

\subsubsection{机体因素}
\paragraph{生理因素}

(1)特殊人群。少年、儿童对药物反应与成年人不同,药品不良反应发生率较成年人高。小儿特别是新生儿和婴幼儿各系统器官功能不健全,肝脏对药物的解毒作用与肾脏对药物的排泄能力低下,肝酶系统发育尚未完善,因而易发生药品不良反应。例如,新生儿应用氯霉素后易出现灰婴综合征,这是由于新生儿肝酶发育不完善,葡萄糖醛酸的结合力差,以及肾脏排泄能力降低致使氯霉素在体内蓄积而引起循环衰竭。又如四环素和新形成的骨螯合,产生四环素-钙正磷酸盐络合物,在新生儿可引起骨生长抑制及幼儿牙齿变色和畸形,但对成人则无影响。老年人不良反应发生率随年龄增加而升高,50~60岁药品不良反应发生率为14.4%,61~70岁为15.7%,71~80岁为18.3%,81岁以上为24%。老年人不良反应发生率高与多种因素有关。老年人肝肾功能减退,药物的代谢和排泄能力降低;此外,老年人组织器官功能改变,靶器官对某些药物敏感性增高。这些因素均能促进药品不良反应的发生。例如,地西泮在青年人体内的平均半衰期为40h,在老年人体内则可延长至80h。

(2)性别。一般而言,药品不良反应的发生率女性高于男性。例如,保泰松引起的粒细胞减少及氯霉素引起的再生障碍性贫血,女性的发生率分别比男性高3倍和2倍。女性也较男性容易发生药物性红斑狼疮。由于男女生理功能不同,妇女在月经期和妊娠期对泻药及其他刺激性强烈的药物敏感,有引起月经过多、流产及早产的危害。女性在妊娠期间应用某些药物还有导致胎儿发育异常的不良反应。
\paragraph{遗传因素}

(1)个体差异。不同个体对同一剂量的相同药物有不同反应,这种生物学差异普遍存在。例如,水杨酸钠在男性患者中引起不良反应的剂量可相差10倍,75%的患者在服用水杨酸总量为6.5~13.0g时出现不良反应,少数患者在总量为3.25g时出现不良反应,个别患者在总量达30.0g左右时才出现反应。遗传基因的多态性是导致不同个体之间药品不良反应发生差异的重要原因。例如,编码药物代谢酶、药物转运体、药物受体或离子通道的基因发生突变,导致由这些基因编码的蛋白功能改变,进而影响药物的代谢或药物的效应。乙酰化是许多药物,如磺胺类、异烟肼等在体内灭活的重要代谢途径,乙酰化的速度也受遗传基因的控制而表现为快型和慢型两种。慢型乙酰化者可能因体内缺乏乙酰化酶,因此,消除药物的速度比其他人慢。例如,慢型乙酰化者长期服用异烟肼,约有23%的患者患多发性外周神经炎;而对快型乙酰化者,其发生率只有3%左右。

(2)种族。如果人群的种族不同,发生的药品不良反应也有所不同。日本人和因纽特人群中有不少人是快型乙酰化者,使用异烟肼易产生肝功能损害;而英国人和犹太人群中慢型乙酰化者达60%~70%,这些人使用异烟肼易产生周围神经炎。在葡萄糖-6-磷酸脱氢酶(G-6-PD)缺乏者中,非洲黑人主要缺乏G-6-PD-A,在服用伯氨喹、磺胺等药物出现溶血性贫血时,红细胞的损害不太严重;而高加索人主要缺乏G-6-PD-B,使用上述药物时,红细胞的损害就比较严重。

(3)特异质反应和变态反应。少数患者的特异性遗传素质使机体产生特异质反应,这种反应是有害的,甚至是致命的,但只在极少数患者中出现。例如,某些患者体内缺乏G-6-PD,患者的红细胞易受氧化性药物(如伯氨喹、氨苯砜、阿霉素等)的损害,最终导致溶血性贫血。卡马西平引起的特异质反应主要表现为肝脏损害、恶血质、多器官超敏反应,主要由卡马西平代谢产物作为半抗原或全抗原刺激机体而发生的非正常免疫反应,有时也称过敏反应。药物引起的变态反应占全部药品不良反应的6%~10%,其发生与剂量无关,而与患者的特异体质和免疫机制有关。

(4)病理因素。疾病可以造成机体器官功能改变,继而影响药物在体内的药效学和药动学改变,诱发药品不良反应。例如,对于便秘患者来说,口服药物在消化道内停留时间长,吸收量多,易发生不良反应。慢性肝病患者由于蛋白合成作用减弱,血浆蛋白含量减少,使血中游离药物浓度升高,易引起不良反应。肝硬化患者服用地西泮,其{t}
{1/2} 可达105h(一般患者{t} {1/2}
为46h),从而易致不良反应。肾病患者因肾功能减退,使许多药物的排泄受到影响,导致药物蓄积而诱发不良反应。如对于多黏菌素,患者的肾功能正常时,其神经系统的不良反应发生率约为7%,而肾功能不良时可达80%。因此,肝肾病患者不宜使用与一般患者相同的剂量和用药间隔时间,否则就易发生不良反应。支气管哮喘患者因气道的高反应性,使用普萘洛尔可导致哮喘发作,但普萘洛尔并不明显增加正常人的气道阻力。

(5)营养状态。饮食的不平衡亦可影响药物的作用。如异烟肼引起的神经损伤,当处于维生素B{6}
缺乏状态时则较正常情况更严重;对缺乏烟酸饲养的动物,当用硫喷妥钠麻醉时,作用增强。

\subsubsection{其他因素}
\paragraph{给药方法}

注射药物配伍使用是临床上最常用的给药方法之一。在实际应用、操作过程中,由于药物配伍不当、溶媒选择不合理等原因,使药物发生沉淀、浑浊、结晶、变色等理化反应,不仅可使药效降低,还可对人体造成损害。万古霉素与美洛西林配伍连续静脉滴注,两药可以相互反应,在输液管中生成白色浑浊乳状液,两者连续使用时存在配伍禁忌。氯化钾用于低血钾者,只宜口服或缓慢静脉滴注给药,若静脉推注可导致心搏骤停,应绝对避免。

给药途径不同关系到药物的吸收、分布,也影响药物发挥作用的快慢、强弱及持续时间。例如静脉给药直接进入血液循环,立即发生效应,较易发生不良反应;口服刺激性药物可引起恶心、呕吐等。
\paragraph{联合用药}

当多种药物不适当联合应用后,不良反应的发生率亦随之增高。据报告:5种药物合用,其不良反应的发生率为4.2%,6~10种为7.4%,11~15种为24.2%,16~20种为40%,21种以上达45%。联合用药增加不良反应发生概率的原因是多方面的,其中最常见的原因是药物在体内的相互作用影响了药物在体内的代谢过程,造成血药浓度显著升高,导致不良反应的发生。例如,苯妥英、华法林在体内主要经细胞色素P450酶(CYP450)2C9代谢,氟尿嘧啶可以抑制CYP450
2C9,当氟尿嘧啶与苯妥英或华法林合用时,可以使后两种药物代谢减少,血药浓度增高,易诱发不良反应。有些药物长期使用后能加速肝药酶的合成并增强其活性,使机体对另一些药物的代谢加速。在临床上酶诱导作用常可使药物稳态血药浓度降低,为了达到和维持疗效,必须加大剂量。一旦停用诱导剂,原来的血药浓度即升高,从而产生不良反应。

\subsection{药品不良反应的预防}

\subsubsection{新药上市前严格审查}

为了确保药物的安全有效,新药上市前必须进行严格、全面的审查。新药的临床试验必须遵循临床前试验和临床试验的指导原则,完成试验后应提供完整的试验研究报告和相关的临床试验观察资料。新药评审专家本着实事求是的原则对每个申报的资料进行全面、翔实、严格的审查,以确保新药的安全性。

\subsubsection{新药上市后的监测}

由于临床前研究和临床试验存在一定的局限性,故为了确保药物的安全性和有效性,必须继续进行新药上市后的监测。

\subsubsection{询问用药史}

用药前应仔细询问患者是否有药物过敏史或家族药品不良反应史。

\subsubsection{合理用药}

医师不合理使用药物也可能造成药品不良反应的发生。因此,提高医师合理用药的水平,同样能够避免不必要的药品不良反应的发生。

\subsection{药品不良反应的处理}

一旦发生不良反应,医护人员必须迅速采取有效措施,积极进行治疗。

(1)及时停药,及时救治。在药物治疗过程中,若怀疑出现的病症是由于药物所引起而又不能确定为哪种药物时,如果治疗允许,最可靠的方法是首先停用可疑药物甚至全部药物,这样处理不仅可及时终止致病药物对机体的继续损害,而且有助于药品不良反应的识别。若治疗不允许中断,对于A型药品不良反应往往可通过减量,或者换用一种选择性更高的同类药物;对于B型药品不良反应则必须更换药物。

(2)加强药物排泄,延缓药物吸收。如口服给药,可经洗胃(1∶1000~1∶5000高锰酸钾溶液、稀过氧化氢溶液、浓茶或淡盐水反复洗胃)、催吐等方法加强药物的排泄;有些药物在胃内尚未被吸收而又不能洗胃排空时,可给予解毒剂,常用的解毒剂为活性炭,在必要及有条件时进行人工透析,以排除体内滞留的药物。

(3)利用药物的相互作用来降低药理活性,减轻不良反应的发生。如阿托品对抗毛果芸香碱的毒性反应,纳洛酮解救吗啡中毒等。

(4)药物过敏的救治。当发生药物过敏时,可应用抗组胺药、皮质激素、皮肤局部治疗药或抗感染药物进行及时治疗。

