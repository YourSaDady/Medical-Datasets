\chapter{常见妇产科疾病的药物治疗}

\section{先兆流产}

\subsection{定义}

指妊娠28周前,先出现少量阴道流血,继之常出现阵发性下腹痛或腰背痛,妇科检查宫颈口未开。胎膜未破,妊娠产物未排出,子宫大小与停经周数相符,妊娠有希望继续者。经休息及治疗后,若流血停止及下腹痛消失,妊娠可以继续;若阴道流血量增多或下腹痛加剧,可发展为难免流产。

难免流产是指流产已不可避免,由先兆流产发展而来。此时阴道流血量增多,阵发性下腹痛加重或出现阴道流液(胎膜破裂)。妇科检查宫颈颈口已扩张,有时可见胚胎组织或胎囊堵塞于宫颈口内,子宫大小与停经周数相符或略小。

\subsection{治疗}

先兆流产应卧床休息,禁忌性生活,阴道检查操作应轻柔,必要时给予对胎儿危害小的镇静剂。黄体酮每日肌注20mg,对黄体功能不足的患者具有保胎效果。或者口服地屈孕酮片首剂40mg,随后每8h服地屈孕酮片10mg至症状消失。其次,维生素E及小剂量甲状腺粉(适用于甲状腺功能低下患者)也可应用。此外,对先兆流产患者的心理治疗也很重要,要使其情绪安定,增强信心。经治疗两周,症状不见缓解或反而加重者,提示可能胚胎发育异常,进行B型超声检查及HCG测定,决定胚胎状况,给以相应处理,包括终止妊娠。

\section{滴虫性阴道炎}

\subsection{定义}

滴虫性阴道炎是由阴道毛滴虫感染引起的阴道炎症。可由性交直接传染,也可经浴池、盆具、游泳池、衣物及污染的器械等间接传播。常于月经前后发作。滴虫性阴道炎患者的阴道pH值升高,一般在5.0~6.5。

\subsection{临床表现和诊断}

(1)白带增多,呈黄白稀薄脓性液体,常呈泡沫状。

(2)外阴瘙痒、灼热感、疼痛、性交痛。

(3)感染尿道时,可有尿频、尿痛甚至血尿。

(4)妇科检查:阴道及宫颈阴道部黏膜红肿,常有散在红色斑点或呈草莓状,后穹窿有多量黄白色、黄绿色脓性泡沫状分泌物。

\subsection{治疗}

(1)甲硝唑2g,每日1次口服;或甲硝唑500mg,每日2次,连服7d。

(2)甲硝唑2g,每日1次,连服3~5d。

(3)甲硝唑阴道泡腾片200mg,每晚1次,连用7~10d。

(4)0.75%甲硝唑凝胶,每次5g,每日2次,连用7d。

口服给药的治愈率为90%~95%,局部用药前,应用具有消毒作用的液体或降低阴道pH值的液体冲洗阴道1次,可减少阴道恶臭分泌物,利于药物吸收并减轻瘙痒症状。治疗结束后,于下次月经干净后复查分泌物,经3次月经后复查滴虫均为阴性者方称为治愈。

\section{外阴阴道假丝酵母菌病}

\subsection{定义}

外阴阴道假丝酵母菌病是常见的外阴阴道炎症,80%~90%的病原体为白假丝酵母菌,10%~20%为光滑假丝酵母菌、近平滑假丝酵母菌、热带假丝酵母菌等其他假丝酵母菌。

\subsection{流行病学资料}

白假丝酵母菌为条件致病菌,10%~20%的非孕妇女及30%的孕妇阴道中有此菌寄生,但并不引起症状。有假丝酵母菌感染的阴道pH值多在4~4.7,通常小于4.5。

\subsection{病因}

常见的发病诱因有妊娠、糖尿病、大量应用免疫制剂及广谱抗生素、胃肠道假丝酵母菌、穿紧身化纤内裤及肥胖,部分患者无并发诱因。外阴阴道假丝酵母菌病主要为内源性感染,部分患者可通过性交直接传染或通过接触感染的衣物间接传染。

\subsection{临床表现}

(1)外阴瘙痒,外阴、阴道灼痛、还可伴有尿频、尿痛及性交痛。

(2)阴道分泌物增多,呈白色豆渣样或凝乳样。

(3)妇科检查:外阴局部充血,肿胀,小阴唇内侧及阴道黏膜表面附有白色块状物或被凝乳状物覆盖,擦除后露出红肿的阴道黏膜面。

\subsection{诊断}

(1)涂片法:找芽孢和假菌丝。有假菌丝时才可报告为“阳性”。

(2)若有症状而多次悬滴法检查为阴性可能是顽固病例,为确诊是否为非白假丝酵母菌感染,有条件的可采用培养法及药敏试验,以了解菌株类型,并根据药敏结果选择药物。

(3)对反复发作的顽固病例,可检测血糖浓度,糖尿病除外。

\subsection{治疗}
\paragraph{治疗原则}

消除诱因,根据患者情况选择局部或者全身应用抗真菌药物。

外阴阴道假丝酵母菌病可通过性交传染,治疗期间应避免性生活或采用阴茎套。对反复复发者,可检查性伴侣有无假丝酵母菌龟头炎,必要时对性伴侣同时进行治疗。避免厕所、盆具、毛巾、浴室交叉感染。
\paragraph{消除诱因}

若患有糖尿病应给予积极治疗,及时停用广谱抗生素、雌激素及皮质类固醇激素。
\paragraph{一般处理}

重者可用3%硼酸水溶液冲洗阴道1次,减少阴道分泌物或外阴瘙痒;也可选择硝酸咪康唑软膏、3%克霉唑软膏等涂抹外阴。
\paragraph{ 抗真菌药物}

(1)氟康唑150mg,顿服。

(2)伊曲康唑口服200mg,每日1次,3~5d或口服200mg,每日2次,1d。

(3)咪康唑栓剂阴道给药,200mg,每晚1次,7d或400mg,每晚1次,3d。

(4)克霉唑栓剂阴道给药,100mg,每日1次,7d或100mg,每日2次,3d或500mg,每日1次。

\section{细菌性阴道病}

\subsection{定义}

细菌性阴道病为阴道内正常菌群失调所致的一种混合感染。

\subsection{病因}

正常阴道内以产生过氧化氢的乳杆菌占优势。细菌性阴道病时,阴道内产生过氧化氢的乳杆菌减少而其他细菌大量繁殖,主要有加德纳菌、动弯杆菌、普雷沃菌、紫单胞菌、类杆菌、消化链球菌等厌氧菌及人型支原体,其中以厌氧菌居多,厌氧菌数量可增加100~1000倍。厌氧菌繁殖的同时可产生胺类物质,使阴道分泌物增多并有腥臭味。

\subsection{临床表现}

(1)阴道分泌物增多,有鱼腥臭味,性交后加重。

(2)轻度外阴瘙痒或烧灼感。

(3)妇科检查:阴道内见均质分泌物,稀薄,常黏附于阴道壁,但黏度很低,用拭子易从阴道壁擦去,而阴道黏膜无充血或水肿的炎症表现。

(4)辅助检查:阴道分泌物pH值>4.5;胺试验阳性;阴道分泌物镜检查找线索细胞。

\subsection{诊断}

下列4项中有3项阳性,即可临床诊断为细菌性阴道病。

(1)均质、稀薄、白色的阴道分泌物。

(2)阴道pH值>4.5。

(3)胺臭味试验阳性。

(4)线索细胞阳性(必备项目):有条件的可参考革兰染色Nugent诊断标准。

\subsection{治疗}

(1)甲硝唑口服500mg,每日2次,7d;或甲硝唑口服2g,每日1次。

(2)克林霉素口服300mg,每日2次,7d。

(3)2%克林霉素软膏阴道涂布,每次5g,每晚1次,连用7d。

(4)0.75%甲硝唑软膏(胶),每次5g,每日2次,共7d。

\section{妊娠高血压疾病}

\subsection{定义}

妊娠高血压为收缩压≥140mmHg和(或)舒张压≥90mmHg(2次测量),并根据血压分为轻度高血压(140~150mmHg/90~109mmHg)和重度高血压(≥160mmHg/110mmHg)。妊娠高血压包括以下几种情况:妊娠前已患高血压、单纯妊娠期高血压、妊娠前高血压合并妊娠期高血压并伴有蛋白尿、未分类的妊娠高血压。

妊娠期高血压为妊娠导致的高血压合并或不合并蛋白尿,约占妊娠高血压的6%~7%。妊娠期高血压多见于妊娠20周以后,多数于产后42d缓解,主要以器官灌注不良为特点。

如果临床合并明显蛋白尿(≥0.3g/d或尿蛋白肌酐比值≥30mg/mmol)则称为先兆子痫(轻度子痫前期),发生比例为5%~7%,但在妊娠前就有高血压的孕妇中可以高达25%。血压和尿蛋白持续升高,发生母体脏器功能不全或胎儿并发症,称为重度子痫前期。如果在此基础上发生不能用其他原因解释的抽搐,称为子痫。硫酸镁是治疗子痫的一线药物。

\subsection{诊断}

测量血压前被测者至少安静休息5min。测量取坐位或卧位,注意肢体放松,袖带大小合适。通常测量右上肢血压,袖带应与心脏处于同一水平。妊娠期高血压定义为同一手臂至少2次测量的收缩压≥140mmHg和(或)舒张压≥90mmHg。若血压较基础血压升高30mmHg/15mmHg,但低于140mmHg/90mmHg时,不作为诊断依据,但须严密观察。对首次发现血压升高者,应间隔4h或以上复测血压,如2次测量均为收缩压≥140mmHg和(或)舒张压≥90mmHg,诊断为高血压。对于收缩压≥160mmHg和(或)舒张压≥110mmHg的严重高血压患者,应密切监测血压。

\subsection{治疗}

何时开始药物治疗争议较多,目前建议血压收缩压为140~150mmHg和(或)舒张压为90~95mmHg,同时不伴靶器官损害的妊娠高血压,可以先行非药物治疗(见表\ref{tab22-1}),同时密切监测血压水平。非药物治疗包括左侧卧位、低钠饮食及补充钙质。

\begin{longtable}[]{p{4cm}llp{6cm}}
    \caption{妊娠高血压疾病应用药物}
    \label{tab22-1}\\
\toprule
药物 & FDA药物级别 & 是否通过胎盘 & 不良反应\tabularnewline
\midrule
\endhead
阿司匹林(75~100mg/d) & B & 是 & 未见到致畸作用\tabularnewline
甲基多巴 & B & 是 & 新生儿轻微低血压\tabularnewline
硝苯地平 & C & 是 &
抑制分娩,与硫酸镁合用易导致孕妇低血压,胎儿缺氧\tabularnewline
拉贝洛尔 & C & 是 &
妊娠中晚期使用致胎儿宫内发育迟缓,分娩前使用致新生儿心动过缓和低血压\tabularnewline
\bottomrule
\end{longtable}

\subsubsection{拉贝洛尔}

为α、β肾上腺素能受体阻滞剂。用法:50~150mg口服,每日3~4次。静脉注射:初始剂量20mg,10min后如未有效降压则剂量加倍,最大单次剂量80mg,直至血压被控制,每日最大总剂量220mg。静脉滴注:50~100mg中加入5%葡萄糖溶液250~500mL,根据血压调整滴速;血压稳定后改口服。

\subsubsection{硝苯地平}

为二氢吡啶类钙离子通道阻滞剂。用法:5~10mg口服,每日3或4次,24h总量不超过60mg。紧急时舌下含服10mg,起效快,但不推荐常规使用。

\subsubsection{甲基多巴}

为中枢性肾上腺素能神经阻滞剂。用法:250mg口服,每日3次,以后根据病情酌情增减,最高不超过每日2g。

\subsubsection{硝普钠}

强效血管扩张剂。用法:50mg加入5%葡萄糖溶液500mL,按每分钟0.5~0.8μg/kg缓慢静脉滴注。孕期仅适用于其他降压药物应用无效的高血压危象孕妇。产前应用不超过4h。

\section{不孕症}

\subsection{定义}

凡婚后未避孕、有正常性生活、同居2年而未曾受孕者,称不孕症。据1989年资料,婚后1年初孕率为87.7%,婚后2年的初孕率为94.6%,婚后未避孕而从未妊娠者称原发性不孕;曾有过妊娠而后未避孕连续2年不孕者称继发不孕。夫妇一方有先天或后天解剖生理方面的缺陷,无法纠正而不能妊娠者称绝对不孕。夫妇一方因某种因素阻碍受孕,导致暂时不孕,一旦得到纠正仍能受孕者称相对不孕。

\subsection{病因及发病机制}

阻碍受孕的因素可能在女方、男方或男女双方。据调查不孕属女性因素约占60%,属男性因素约占30%,属男女双方因素约占10%。

\subsubsection{女性不孕因素}
\paragraph{输卵管因素}

输卵管因素是不孕症最常见因素。输卵管有运送精子、捡拾卵子及将受精卵运进到宫腔的功能。任何影响输卵管功能的因素,如输卵管发育不全(过度细长扭曲、纤毛运动及管壁蠕动功能丧失等),输卵管炎症(淋菌、结核菌等)引起伞端闭锁或输卵管黏膜破坏时输卵管闭塞,均可导致不孕。此外,阑尾炎或产后、术后所引起的继发感染,也可导致输卵管阻塞造成不孕。
\paragraph{卵巢因素}

引起卵巢功能紊乱导致持续不排卵的因素有:卵巢病变,如先天性卵巢发育不全、多囊卵巢综合征、卵巢功能早衰、功能性卵巢肿瘤、卵巢子宫内膜异位囊肿等;下丘脑-垂体-卵巢轴功能紊乱,引起无排卵性月经、闭经等;全身性疾病(重度营养不良、甲状腺功能亢进等)影响卵巢功能导致不排卵。
\paragraph{子宫因素}

子宫先天畸形、子宫黏膜下肌瘤可造成不孕或孕后流产;子宫内膜炎、内膜结核、内膜息肉、宫腔粘连或子宫内膜分泌反应不良等影响受精卵着床。
\paragraph{宫颈因素}

宫颈黏液量和性状与精子能否进入宫腔关系密切。雌激素不足或宫颈管感染时,均会改变黏液性质和量,影响精子活力和进入数量。宫颈息肉、宫颈肌瘤能堵塞宫颈管影响精子穿过,宫颈口狭窄也可造成不孕。
\paragraph{阴道因素}

阴道损伤后形成的粘连瘢痕性狭窄,或先天无阴道、阴道横隔、无孔处女膜,均能影响性交并阻碍精子进入。严重阴道炎症时,大量白细胞消耗精液中存在的能量物质,降低精子活力,缩短其存活时间而影响受孕。

\subsubsection{男性不育因素}

主要是生精障碍与输精障碍。应行外生殖器和精液的检查,明确有无异常。
\paragraph{精液异常}

如无精子或精子数过少,活力减弱,形态异常。影响精子产生的因素有:

(1)先天发育异常:先天性睾丸发育不全不能产生精子;双侧隐睾导致曲细精管萎缩等妨碍精子产生。

(2)全身原因:慢性消耗性疾病,如长期营养不良、慢性中毒(吸烟、酗酒)、精神过度紧张,可能影响精子产生。

(3)局部原因:腮腺炎并发睾丸炎导致睾丸萎缩;睾丸结核破坏睾丸组织;精索静脉曲张有时影响精子质量。
\paragraph{精子运送受阻}

附睾及输精管结核可使输精管阻塞,阻碍精子通过;阳痿、早泄不能使精子进入女性阴道。
\paragraph{免疫因素}

精子、精浆在体内产生对抗自身精子的抗体可造成男性不育,射出的精子发生自身凝集而不能穿过宫颈黏液。
\paragraph{内分泌功能障碍}

男性内分泌受下丘脑-垂体-睾丸轴调节,垂体、甲状腺及肾上腺功能障碍可能影响精子的产生而引起不孕。
\paragraph{性功能异常}

外生殖器发育不良或阳痿致性交困难等。

\subsubsection{男女双方因素}

(1)缺乏性生活的基本知识。

(2)男女双方盼孕心切造成的精神过度紧张。

(3)免疫因素:近年来对免疫因素的研究,认为有两种免疫情况影响受孕。①
同种免疫:精子、精浆或受精卵是抗原物质,被阴道及子宫内膜吸收后通过免疫反应产生抗体物质,使精子与卵子不能结合或受精卵不能着床。②
自身免疫:认为不孕妇女血清中存在透明带自身抗体,与透明带起反应后可防止精子穿透卵子,因而阻止受精。

\subsection{诊断}

\subsubsection{男方检查}

询问既往有无慢性疾病,如结核、腮腺炎等;了解性生活情况,有无性交困难。除全身检查外,重点应检查外生殖器有无畸形或病变,尤其是精液常规检查。正常精液量为2~6mL,平均为3~4mL,异常为<1.5mL;pH值为7.2~7.5,在室温中放置5~30min完全液化,正常精子总数>8000万/mL,异常为<2000万/mL;正常精子活动数>50%,异常为<35%。

\subsubsection{女方检查}
\paragraph{询问病史}

结婚年龄、男方健康状况、是否两地分居、性生活情况、是否采用避孕措施、月经史、既往史(有无结核病、内分泌疾病)、家族史(有无精神病、遗传病)。对继发不孕,应了解以往流产或分娩经过、有无感染史等。
\paragraph{体格检查}

注意第二性征发育情况、内外生殖器的发育情况,以及有无畸形、炎症、乳房泌乳等。胸片排除结核,必要时作甲状腺功能检查等。
\paragraph{女性不孕特殊检查}

如卵巢功能检查、输卵管通畅试验、性交后精子穿透力试验、宫颈黏液、精液相合试验、子宫镜检查、腹腔镜检查等。

\subsection{女性不孕症的治疗}

引起不孕的原因虽很多,但首先要增强体质和增进健康,纠正营养不良和贫血;戒烟、不酗酒;积极治疗内科疾病;掌握性知识、学会预测排卵日期性交(排卵前2~3d或排卵后24h内),性交次数适度,以增加受孕机会。
\paragraph{氯米芬}

氯米芬为首选促排卵药,适用于体内有一定雌激素水平者。月经周期第5日起,每日口服50mg(最大剂量达200mg),连用5d,3个周期为一疗程。排卵率高达80%,但受孕率仅为30%~40%,可能与其抗雌激素作用有关。若用药后有排卵但黄体功能不全,可加用绒促性素(HCG),于月经周期第15~17d开始连用5d。
\paragraph{HCG}

绒促性素具有类似LH作用,常与氯米芬合用。于氯米芬停药7d加用HCG
2000~5000IU一次肌注。
\paragraph{尿促性素(HMG)}

HMG含有FSH和LH各75IU,促使卵泡生长发育成熟。于月经来潮第6d起,每日肌注HMG
1支,共7d。用药期间需检查宫颈黏液,检测血雌激素水平,B超监测卵泡发育情况,一旦卵泡发育成熟停用HMG。停药后24~36h,加用HCG
5000~10000IU一次肌注,促进排卵及黄体形成。
\paragraph{溴隐亭}

溴隐亭属多巴胺受体激动剂,能抑制垂体分泌催乳激素。适用于无排卵伴有高催乳激素血症者。从小剂量开始,如无反应,1周后改为每日剂量2.5mg,分2次口服。一般连续用药3~4周,直至血催乳激素降至正常范围。