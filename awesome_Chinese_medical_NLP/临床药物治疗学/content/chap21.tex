\chapter{常见皮肤病的药物治疗}

\section{手足癣}

手足癣是手癣和足癣的总称。手足癣是指发生在手足皮肤或其背面部位的皮肤癣菌感染。手足癣尤其是足癣是皮肤癣菌病中最常见的疾病,人群患病率高达30%~70%,多见于成人,全世界范围流行。气候湿热和足部多汗少脂以及局部欠透气(穿鞋)是足癣重要的诱发因素,并常因足癣感染引起手癣。手足癣的致病菌主要包括红色毛癣菌、石膏样毛癣菌、絮状表皮癣菌和断发毛癣菌。

\subsection{分型}

根据致病菌和发病部位的不同,手足癣可分为四型:①角化过度型:临床表现为无水疱,皮肤角化过度,粗糙无汗,冬季常见皮肤皲裂;②水疱型:在足跖及足缘成群或散在分布小水疱为主,疱壁厚内容物澄清,干燥吸收后出现脱屑,自觉瘙痒严重,热天多见;③丘疹鳞屑型:足跖有明显的小片状脱屑,呈弧形或环状附于皮损边缘,界限清楚炎症不明显;④趾间糜烂型:跖间皮肤湿润浸渍而发白,边界清楚,去除浸渍的表皮即可暴露红斑糜烂的基层,并发症包括合并细菌感染、癣菌疹反应、丹毒、蜂窝织炎等,夏季多发或加重。

\subsection{治疗}

\subsubsection{角化过度型}

选用复方苯甲酸搽剂或10%冰醋酸浸泡,再外用抗真菌药霜剂克霉唑霜、咪康唑霜、酮康唑霜、特比萘芬霜等制剂。

\subsubsection{水疱型}

如疱小、未破,选用比较温和的水剂和霜剂,如复方苯甲酸搽剂或复方间苯二酚涂剂;如疱较大,先将疱液汲去,再外用抗真菌药霜剂加足粉;如疱已破,可用复方间苯二酚涂剂加足粉。

\subsubsection{丘疹鳞屑型}

选用外用抗真菌药霜剂。

\subsubsection{趾间糜烂型}

先用足粉或咪康唑散剂使局部干燥,再外用咪康唑霜剂或复方卡氏搽剂。常用药物如下。

\textbf{冰醋酸溶液:}

【适应证】 抑制霉菌生长。用于治疗手、足癣。

【用法用量】 浸泡手、足,每日1次。

【注意事项】 避免接触眼睛和其他黏膜(如口、鼻等)。

\textbf{复方苯甲酸涂剂:}

【主要成分】 苯甲酸、水杨酸。

【适应证】 软化和溶解角质,抑制真菌。用于治疗手足癣和体股癣等症。

【用法用量】 涂搽患处,每日2或3次。

【注意事项】 湿性、起泡或糜烂性的急性炎症患者忌用本品。

\textbf{复方间苯二酚涂剂:}

【主要成分】 间苯二酚、苯酚。

【适应证】 具有杀霉菌作用。治疗手足癣、体癣、股癣等。

【用法用量】 局部涂抹,每日2或3次。

\textbf{足粉:}

【主要成分】 水杨酸、硼酸、氧化锌、滑石粉。

【适应证】 止痒、吸湿、收敛和制霉菌。用于趾间糜烂型足癣。

【用法用量】 撒布患处。

【注意事项】 避免接触眼睛和其他黏膜(如口、鼻等)。不适用于儿童。

\textbf{酮康唑乳膏:}

【适应证】 酮康唑为抗真菌药,对皮肤癣菌如毛发癣菌属、表皮癣菌属、小孢子菌属及酵母菌如假丝酵母菌具有抑制作用。局部外用几乎不经皮肤吸收。适用于由皮真菌、酵母菌及其他真菌引起的皮肤、指(趾)甲感染,也可用于由酵母菌(如假丝酵母菌等)和革兰阳性细菌引起的阴道感染和继发感染。

【用法用量】 皮肤感染:外用,涂搽于洗净的患处,早晚各1次,症状消失后(通常需2~5周)应继续用药10d,以防复发。指(趾)甲感染:尽量剪尽患甲,将本品涂擦于患处,每日1次,患甲松动后(约需2~3周)应继续用药至新甲开始生长。疗效明确一般需7个月左右。假丝酵母菌阴道炎:每日就寝前用涂药器将药膏挤入阴道深处,必须连续用2周。月经期内也可用药。

【不良反应】 可见刺痛或其他局部刺激症状,偶见瘙痒等过敏反应。罕见病例中可出现过敏反应,如灼烧感、皮肤刺激、局部湿疹。

【注意事项】 对酮康唑、咪唑类药物或亚硫酸盐过敏者禁用。过敏体质者慎用。用药部位如有烧灼感、红肿等情况应停药,并将局部药物洗净。不得用于皮肤破溃处。避免接触眼睛和其他黏膜(如口、鼻等)。不宜大面积使用。股癣患者,勿穿紧贴内裤或化纤内裤,在外用乳膏剂时可散布撒布剂。为减少复发,对体癣、股癣、花斑癣,疗程至少需要2~4周。

\textbf{特比萘芬乳膏:}

【适应证】 本品为广谱抗真菌药,能高度选择性地抑制真菌麦角鲨烯环氧化酶,阻断真菌细胞膜形成过程中的麦鲨烯环氧化反应而干扰真菌固醇的早期生物合成,从而发挥抑制和杀灭真菌的作用。用于治疗手癣、足癣、体癣、股癣、花斑癣及皮肤假丝酵母菌病等。

【用法用量】 外用,每日2次,涂患处,并轻揉片刻。疗程1~2周。

【不良反应】 偶见皮肤刺激如烧灼感,或过敏反应如皮疹、瘙痒等。

【注意事项】 过敏体质者、孕妇及哺乳期妇女慎用。避免接触眼睛和其他黏膜(如口、鼻等)。用药部位如有烧灼感、红肿等情况应停药,并将局部药物洗净。

\section{带状疱疹}

带状疱疹是由水痘-带状疱疹病毒感染引起的,为嗜神经病毒。可长期潜伏于脊髓神经或颅神经的感觉神经节的神经元中,当宿主免疫功能减退时,病毒活跃而发病。带状疱疹可见于任何年龄,但多见于成人。带状疱疹潜伏期可长达数年至数十年,水痘病史可有可无。带状疱疹常呈散发性,与机体免疫功能有关。局部创伤后、系统性红斑狼疮、淋巴瘤、白血病以及较长期接受皮质激素、免疫抑制剂和放射治疗的患者,较正常人明显好发,且病程迁延,病情较重,后遗神经痛也较突出。

发病初期,患者常先有局部皮肤感觉异常或痛感。部分患者同时伴有轻微发热,乏力、头痛等全身症状。经1~3d开始发疹,发疹的部位往往在局部皮肤感觉异常处。最初表现为红斑,继而出现成簇成群的丘疹,继而变成水疱。水疱约绿豆大小、表面光滑,疱壁紧张而透明,周围绕以红晕。间有出现脓疱、大疱或血疱,各群之间皮肤正常。5~8d后水疱内容物稍显浑浊,或部分破溃,局部糜烂渗液,最后干燥结痂。第2周痂皮脱落,一般不留瘢痕,或暂时留存淡红的色斑或色素沉着,日久亦可消退。带状疱疹以胸腰部位为多见。皮疹多数局限于身体一侧,不超越身体正中线;偶尔有双侧分布者。病程为2~3周,能自愈,愈后极少复发,但神经痛则可持续1~2个月或更久。

治疗的基本原则为止痛、抗病毒、缩短病程和防止继发感染。

\subsection{止痛治疗}

可选用NSAIDs、卡马西平等治疗。后遗神经痛可采用物理疗法,如音频电疗法或采用氦-氖激光照射与皮肤损害相关的脊髓后根、神经支配区或疼痛区,缓解症状效果较好,可缩短病程。

\subsection{抗病毒药物}

阿昔洛韦口服,每次0.2g,每日5次。3~5d有效,一般用5~10d。

\subsection{免疫调节剂}

转移因子、α-干扰素、胸腺肽等可酌情选用,以减轻症状,缩短疗程。

\subsection{局部治疗}

酌情选用3%阿昔洛韦软膏,继发感染选用0.5%新霉素软膏等外涂。常用药物如下。

\textbf{阿昔洛韦:}

【药理作用】 本品进入疱疹病毒感染的细胞后,与脱氧核苷竞争病毒胸苷激酶或细胞激酶,药物被磷酸化成活化型阿昔洛韦三磷酸酯,然后通过两种方式抑制病毒复制:①干扰病毒DNA多聚酶,抑制病毒的复制;②在DNA多聚酶作用下,与增长的DNA链结合,引起DNA链的延伸中断。本品对病毒有特殊的亲和力,但对哺乳动物宿主细胞毒性低。

本品口服吸收差,约15%~30%由胃肠道吸收。能广泛分布至各组织与体液中,脑脊液中浓度约为血中浓度的一半,可通过胎盘。本品蛋白结合率低(9%~33%)。血消除半衰期约为2.5h。肌酐清除率15~50mL/min时,血消除半衰期延长为3.5h。无尿者的血消除半衰期长达19.5h,血液透析时降为5.7h。本品主要经肾排泄,约14%的药物以原形由尿排泄,经粪便排泄率低于2%,呼出气中含微量药物。血液透析6h约清除血中60%的药物。腹膜透析清除量很少。

【适应证】 单纯疱疹病毒感染、带状疱疹和免疫缺陷者水痘的治疗。

【用法用量】 口服。

(1)生殖器疱疹初治和免疫缺陷者皮肤黏膜单纯疱疹:成人常用量每次0.2g,每日5次,共10d;或每次0.4g,每日3次,共5d;复发性感染1次0.2g,每日5次,共5d;复发性感染的慢性抑制疗法,每次0.2g,每日3次,共6个月,必要时剂量可加至每日5次,每次0.2g,共6~12个月。

(2)带状疱疹:成人常用量每次0.8g,每日5次,共7~10d。

(3)肾功能不全的成人患者应调整剂量。

(4)水痘:2岁以上儿童按体重每次20mg/kg,每日4次,共5d,出现症状立即开始治疗。40kg以上儿童和成人常用量为每次0.8g,每日4次,共5d。

【不良反应】 偶有头晕、头痛、关节痛、恶心、呕吐、腹泻、胃部不适、食欲减退、口渴、白细胞计数下降、蛋白尿及尿素氮轻度升高、皮肤瘙痒等,长程给药偶见痤疮、失眠和月经紊乱。

【注意事项】

(1)脱水或已有肝、肾功能不全者、孕妇、哺乳期需慎用。

(2)严重免疫功能缺陷者长期或多次应用本品治疗后可能引起单纯疱疹病毒和带状疱疹病毒对本品耐药。如单纯疱疹患者应用阿昔洛韦后皮损不见改善者应测试单纯疱疹病毒对本品的敏感性。

(3)随访检查:由于生殖器疱疹患者大多易患子宫颈癌,因此患者至少应每年检查1次,以早期发现。

(4)一旦疱疹症状与体征出现,应尽早给药。

(5)给药期间应给予患者充足的水,防止本品在肾小管内沉淀。

(6)一次血液透析可使血药浓度减低60%,因此血液透析后应补给一次剂量。

(7)对单纯疱疹病毒的潜伏感染和复发无明显效果,不能根除病毒。

(8)与齐多夫定合用可引起肾毒性,表现为深度昏睡和疲劳。

(9)与丙磺舒竞争性抑制有机酸分泌,合并用丙磺舒可使本品的排泄减慢,半衰期延长,体内药物量蓄积。

\section{湿疹}

湿疹是由多种复杂的内、外因素引起的一种具有明显渗出倾向的皮肤炎症反应。湿疹发病原因复杂,且外在因子和内在因子相互影响,相互作用。外在因子如生活环境、气候因素等均可影响湿疹的发生,日光、炎热、多汗、化妆品、动植物等均可诱发湿疹。内在因子如精神紧张、过度疲劳、失眠、情绪变化、新陈代谢障碍和内分泌失调等均可产生或加重湿疹病情。

从发病机制看,湿疹主要由复杂的内外激发因子引起的一种迟发型变态反应,临床表现呈多样性,按皮损表现特点分为急性、亚急性和慢性湿疹三种。①急性湿疹:为多数粟粒大红色丘疹、丘疱疹或水疱,尚有明显点状或小片状糜烂、渗液、结痂,损害境界不清,合并感染时可出现脓疱、脓性渗出及痂屑等。②亚急性湿疹:常因急性期损害处理不当迁延而来,皮损以红色丘疹、斑丘疹、鳞屑或结痂为主,兼有少数丘疱疹或水疱及糜烂渗液。③慢性湿疹:多有急性、亚急性湿疹反复不愈转化而来,皮损为暗红或棕红色斑或斑丘疹,常融合增厚呈苔藓样变,表面有鳞屑、抓痕和血痂,周围散在少数丘疹、斑丘疹等。皮损在一定诱因下可急性发作。皮疹往往对称性分布,瘙痒剧烈,可发生在任何部位,但以外露部位及屈侧多见,常见特定部位的湿疹有耳湿疹、手足湿疹、乳房湿疹、肛门外生殖器湿疹、小腿湿疹等。

\subsection{一般防治原则}

应尽可能寻找患者发病或诱发加重的原因,详细了解工作环境、生活习惯、饮食、嗜好及思想情绪等;尽可能避免外界不良刺激,如热水洗烫、剧烈搔抓等;尽量不穿化纤贴身内衣、皮毛制品;避免食用易致敏和刺激性食物,如海鲜、辣椒、酒、咖啡等;保持皮肤清洁,避免过劳、保持乐观稳定的情绪。

\subsection{系统治疗}

选用抗组胺药止痒,如西替利嗪每日1次,每次口服10mg;氯雷他定每日1次,每次口服10mg。因湿疹多在晚间瘙痒剧烈,故建议晚餐后或临睡前服用。

\subsection{局部治疗}

\subsubsection{急性湿疹}

无渗出时,炉甘石洗剂,每日4~6次外用。瘙痒明显时,酌加糖皮质激素乳膏外用,如0.1%曲安奈德乳膏或0.1%糠酸莫米松霜等,每日1或2次外用。有渗出时,首先用3%硼酸溶液作湿敷。

\subsubsection{亚急性湿疹}

可选用糊剂如氧化锌糊膏。

\subsubsection{慢性湿疹}

可选糖皮质激素软膏剂及配用焦油类制剂效果较好。常用糖皮质激素类软膏剂为1%氢化可的松乳膏、0.1%曲安奈德乳膏、0.1%糠酸莫米松乳膏或复方卤米松霜等,每日2或3次外用;焦油类制剂常用10%鱼石脂软膏等,每日2或3次外用。

常用药物如下。

\textbf{氯雷他定:}

【药理作用】 本品为高效、作用持久的三环类抗组胺药,为选择性外周H{1}
受体拮抗剂,可缓解过敏反应引起的各种症状。

【适应证】 用于缓解过敏性鼻炎有关的症状,如喷嚏、流涕、鼻痒、鼻塞以及眼部痒及烧灼感。口服药物后,鼻和眼部症状及体征得以迅速缓解。亦适用于缓解慢性荨麻疹、瘙痒性皮肤病及其他过敏性皮肤病的症状及体征。

【用法用量】 口服。成人及12岁以上儿童:每日1次,每次10mg;2~11岁儿童(体重>30kg):每日1次,每次10mg;2~11岁儿童(体重≤30kg):每日1次,每次5mg。

【不良反应】 在每日10mg的推荐剂量下,本品未见明显的镇静作用。常见不良反应有乏力、头痛、嗜睡、口干、胃肠道不适,包括恶心、胃炎以及皮疹等。罕见不良反应有脱发、过敏反应、肝功能异常及心动过速及心悸等。

【注意事项】 严重肝功能不全的患者慎用,妊娠期及哺乳期妇女慎用。在作皮试前约48h应中止使用本品,因抗组胺药能阻止或降低皮试的阳性反应发生。同时服用酮康唑、大环内酯类抗生素、西咪替丁、茶碱等药物,会提高氯雷他定在血浆中的浓度,应慎用。其他已知能抑制肝脏代谢的药物,在未明确与氯雷他定相互作用前应慎用。

\textbf{西替利嗪:}

【药理作用】 本品为选择性组胺H{1}
受体拮抗剂。无明显抗胆碱和抗5-羟色胺作用,中枢抑制作用较小。

【适应证】 适用于季节性鼻炎、常年性过敏性鼻炎、过敏性结膜炎及过敏引起的瘙痒和荨麻疹的对症治疗。

【用法用量】 口服:成人或12岁以上儿童,每次10mg,每日1次。如出现不良反应,可改为早晚各5mg。6~11岁儿童,根据症状的严重程度不同,推荐起始剂量为5mg或10mg,每日1次。2~5岁儿童,推荐起始剂量为2.5mg,每日1次;最大剂量可增至5mg,每日1次,或2.5mg每12h
1次。

【不良反应】 偶有报告患者有轻微和短暂不良反应,如头痛、头晕、嗜睡、激动不安、口干、腹部不适。

【注意事项】 对羟嗪过敏者、严重肾功能损害患者、妊娠前3个月及哺乳期妇女等禁用。服用本品时应谨慎饮酒,服药期间不得驾驶机、车、船、从事高空作业、机械作业及操作精密仪器。

\textbf{复方曲安奈德乳膏:}

【主要成分】 曲安奈德、制霉菌素、硫酸新霉素和短杆菌肽。

【适应证】 曲安奈德为肾上腺皮质激素类药,局部应用有抗炎、抗瘙痒和收缩血管等作用;制霉菌素具有抗假丝酵母菌活性;新霉素与短杆菌肽具有抗细菌活性。适用于假丝酵母菌感染的皮肤病、伴有假丝酵母菌或细菌感染的异位湿疹样皮炎、壅滞性皮炎、钱币形皮炎、接触性皮炎、脂溢性皮炎、神经性皮炎和中毒性皮炎,以及婴儿湿疹、单纯性苔癣等症、肛门及外阴瘙痒症。

【用法用量】 外用。每日2或3次,涂擦患处。

【不良反应】 长期使用可引起局部皮肤萎缩、毛细血管扩张、痤疮样皮炎、毛囊炎、色素沉着及继发感染。

【注意事项】 牛痘、水痘等病毒性皮肤瘸患者,假丝酵母菌以外的其他真菌性皮肤病患者,眼科及鼓膜穿孔的患者及对本品过敏患者禁用。本品含有新霉素能引起肾中毒和耳中毒,对于大面积烧伤、营养性溃疡等患者应慎用。有明显循环系统疾病的患者慎用。避免全身大面积使用及长期使用,一般用药不宜超过4~6周。涂药处如有灼烧、瘙痒、红肿时,应停止用药。

\textbf{卤米松乳膏:}

【适应证】 具有抗炎、抗过敏、血管收缩和抗增生作用。最适用于渗出不多的急性或亚急性期湿疹性皮炎和皮肤病,也适用于皮脂溢出性或对脂肪过敏的患者。

【用法用量】 以薄层涂于患处,依症状每日1或2次,并缓和地摩擦。如有需要,可用多孔绷带包扎患处。药效欠佳或较顽固者,可改用短时的密封包扎以增强疗效。对于慢性皮肤疾患(如银屑病慢性湿疹),使用本品时不应突然停用,应交替换用润肤剂或药效较弱的另一种皮质类固醇,逐渐减少卤米松乳膏用药剂量。

【不良反应】 偶发用药部位刺激性症状,如烧灼感、瘙痒。罕见皮肤干燥、红斑、皮肤萎缩及毛囊炎痤疮脓疱等。

【注意事项】

(1)细菌和病毒性皮肤病、真菌性皮肤病、梅毒性皮肤病变、皮肤结核病、玫瑰痤疮、口周皮炎及寻常痤疮患者禁用。

(2)无论患者的年龄,均应避免长期连续使用。密封性包扎应限于短期和小面积皮肤。如特殊需要大剂量使用本品,或应用于大面积皮肤、使用密封性包扎或长期使用,应对患者进行定时的医疗检查。

(3)本品应慎用于面部或擦烂的部位,且只能短期使用。尚未见全身性不良反应的报道,例如对于肾上腺皮质功能的作用。

(4)在以下条件使用本品时,如大面积皮肤上使用密封包扎时,如果用药皮肤发生了感染,应立即加以合适的抗菌药物治疗。本品不能与眼结膜或黏膜接触。

\textbf{丁酸氢化可的松乳膏:}

【适应证】 本品为糖皮质激素类药物,外用具有抗炎、抗过敏、止痒及减少渗出作用。用于过敏性皮炎、脂溢性皮炎、过敏性湿疹及苔藓样瘙痒症。

【用法用量】 局部外用。取适量本品涂于患处,每日2次。

【不良反应】 长期使用可致皮肤萎缩、毛细血管扩张、色素沉着以及继发感染。偶见过敏反应。

【注意事项】 感染性皮肤病禁用,不得用于皮肤破溃处,避免接触眼睛和其他黏膜(如口、鼻等)。用药部位如有烧灼感、红肿等情况应停药,并将局部药物洗净。不宜大面积、长期使用,过敏体质者、儿童、孕妇和哺乳期妇女慎用。

\textbf{复方炉甘石洗剂:}

【主要成分】 炉甘石、氧化锌、甘油等。

【适应证】 炉甘石和氧化锌具有收敛、保护作用。用于各种皮疹,如荨麻疹、痱子、湿疹。

【用法与用量】 局部外用,用时摇匀,取适量涂于患处,每日3或4次。

【注意事项】 用时摇匀。避免接触眼睛和其他黏膜(如口、鼻等)。本品不宜用于有渗液的皮肤。

\section{手足皲裂}

手足皲裂是皮肤因缺少滋润而干裂并有疼痛。手足掌跖部皮肤角质层厚,无毛囊和皮脂腺,在寒冷干燥时由于无皮脂保护,皮肤易于损伤,好发于冬季并多见于经常接触脂溶性或碱性物质的工作人员。皲裂惯发于足跟、足跖外侧缘、手掌、手指屈侧等处。损伤深浅不一,某些皮肤病如手足癣、手足慢性湿疹、鱼鳞病、掌跖角化症等也可继发皲裂,或两者并存。根据皲裂的深浅程度,一般可分为Ⅰ~Ⅲ度。

本病治疗原则以预防为主,在冬季未发生皲裂前保持手足清洁、滋润,尽量少接触酸、碱、有机溶媒等。

对Ⅰ、Ⅱ度皲裂可局部搽尿素乳膏、10%硫磺水杨酸软膏及紫归治裂膏等,每日2或3次。对Ⅲ度皲裂,用药前先用热水浸泡患处,使角质软化,并用刀片将角质削薄,拭干后涂搽上述软膏。

常用药物有尿素乳膏。

\textbf{尿素乳膏:}

【适应证】 尿素具有抗菌、使蛋白质溶解、变性、增加蛋白质水合作用,故本品具有增加皮肤角质层水合作用。用于鱼鳞病、手、足皲裂及皮肤干燥等。

【用法用量】 涂擦于洗净的患处,每日1~3次。

【不良反应】 偶见皮肤刺激和过敏反应。

【注意事项】 使用时如有灼烧感、瘙痒、红肿等,应停止使用。

\section{夏季皮炎}

夏季皮炎是由于因夏季炎热引起的季节性炎症性皮肤病。夏季皮炎是夏天的多发病、常见病,系持续的高温、高湿度的外环境,加上皮肤出汗多又没有及时清洗所致的皮肤炎症。夏季皮炎皮损表现为密集针头至栗粒大的红斑、丘疹或丘疱疹。瘙痒明显,并伴有轻度灼热感。由于瘙痒、搔抓,局部可有结痂、抓痕和色素沉着。皮疹好发于躯干和四肢伸侧,尤以小腿伸侧为甚。

夏季室内保持良好通风和散热,使室内温度不宜过高。衣服宜宽大轻薄,保持皮肤清洁及干燥。宜用温水洗净后用毛巾揩干,外用炉甘石洗剂、薄荷脑醑剂等,剧痒时可口服抗组胺药治疗。

常用药物如薄荷脑醑。

\textbf{薄荷脑醑:}

【适应证】 局部应用时有消炎、止痒、止痛等作用。用于夏季皮炎。

【用法与用量】 涂搽患处,每日2或3次。